The numerical simulation of physical problems modeled by means of hyperbolic partial differential equations involves the solution of possibly discontinuous waves.
%
The precise tracking of those waves is of major importance in solid mechanics, especially for history-dependent constitutive equations, as it allows to properly assess the residual states.
\review{More specifically, high-speed forming techniques such that electromagnetic forming \cite{Formage} also requires the tracking of solid interfaces.}   
% \note{(a) Examples of applications that involve large deformations}
% \note{Discontinuities + Large deformations!!}
%
Nevertheless, the accurate simulation of such problems can be prevented by several reasons.
%% Mesh-based Lagrangian
First, Lagrangian formulations of mesh-based approaches such as the widely used Finite Element (FEM) \cite{Belytschko} and Finite Volume (FVM) \cite{Leveque} methods become less accurate for very large deformations due to sever distortions of the mesh.
%% Mesh-based Eulerian and ALE
Although those instabilities can be avoided by resorting to Eulerian or Arbitrary Lagrangian-Eulerian formulations, other difficulties arise owing to additional diffusive projections of fields.
%% Time integration
Second, it is well-known that explicit finite element schemes can exhibit numerical noise near sharp solutions that can be hard to remove with artificial viscosity without loss of accuracy.
Such oscillations can nevertheless be removed from FVM solutions due to numerical fluxes involved in the formulation, allowing the building of Total Variation Diminishing (TVD) schemes \cite{Harten}.
%% DG
The Discontinuous Galerkin approximation in space \cite{NeutronDG}, combined with FEM (DGFEM), enables to take advantage of similar interface fluxes so as to construct Total Variation Diminishing in the Means (TVDM) finite element procedures \cite{Cockburn}.
%% Time integration in DGFEM -> STDG
While the introduction of the DG approximation within FEM schemes enables to avoid non-physical oscillations, providing the use of suitable limiters \cite{vanLeer_Limiters}, these approaches are constrained by a restrictive CFL condition \cite{DGFEM_CFL} and hence, suffer from numerical diffusion.
Space-time DGFEM formulations \cite{ST_DGFEM1} enable to circumvent this drawback but are nonetheless subject to mesh entanglement.
%\review{Moreover, the satisfaction of the compatibility condition between reference and updated configurations, through the Piola identity, is not a straightforward undertaking in Lagrangian formulations \cite{Vilar_PiolaIdentity,LagrangianDG_thesis}, which is also the case for finite volumes \cite{Haider_FVM}.
%In other words, updating the geometry based on discontinuous velocities and deformation gradients across element faces may lead to discontinuities in the displacement vector.}


%% MPM
One possibility to avoid mesh tangling instabilities while providing a material description of the motion is to use mesh-free approaches such as those of the Particle-In-Cell (PIC) family \cite{PIC} and, in particular, the Material Point Method \cite{Sulsky94}.
The MPM is based on a dual discretization of a domain made of a collection of material points lying in an arbitrary grid.
A discrete system is solved on the grid, whereas the loading history is stored at particles during the motion so that field projections, which introduce some freedom into the scheme, are required.
%
Indeed, the updated velocity at the grid level can be directly interpolated to the particles according to the original PIC procedure.
Alternatively, the particle velocities can be updated by interpolating the nodal acceleration resulting from the solution of the discrete system, as introduced in PIC by the FLuid Implicit Particle method (FLIP) \cite{FLIP}.
The latter allows to reduce numerical diffusion at the cost of spurious oscillations \cite{PIC_Nishiguchi}.
Recently, a tunable mapping procedure, based on a parameter $m$ that completely eliminates the noise in MPM solutions, has been proposed in the Extended PIC of order $m$ (XPIC(m)) \cite{XPIC}.
A classical interpolation is selected for $m=1$ whereas the mapping tends to FLIP one as $m\rightarrow \infty$.
Nevertheless, the numerical diffusion still prevents from capturing (discontinuous) waves.

%% DGMPM 
\review{
  %A different point of view, which is followed in the Discontinuous Galerkin Material Point Method (DGMPM) \cite{DGMPM,Thesis}, is that the numerical diffusion exhibited by the PIC can be limited by reducing the domain of influence of nodes rather than modifying the projections themselves.
  Note also that the numerical diffusion exhibited by the PIC can be limited by reducing the domain of influence of nodes rather than modifying the projections themselves.
  This approach is followed in the Discontinuous Galerkin Material Point Method (DGMPM) \cite{DGMPM,Thesis}.
  The introduction of the DG approximation within the MPM, combined with the use of the PIC projection, thus aims at providing non-oscillating discontinuous solutions with low numerical diffusion due to the support of the shape functions that reduces to one cell.
  Therefore, the DGMPM enables to take advantage of space-DGFEM and MPM in order to accurately follow waves in finite-deforming media.
}
%
In that method, the weak form of a system of conservation laws is written on an arbitrary grid and numerical fluxes arising from the DG approximation are computed at cell faces by means of an approximate Riemann solver.
Those intercell fluxes allow to take into account the characteristic structure of hyperbolic problems, and in particular the transverse propagation of waves through the use of the Corner Transport Upwind method (CTU) \cite{Colella_CTU} developed for finite volumes.
The CTU is however reformulated in order to fit the DGMPM approximation, in which fields are edge-wise constant in Riemann problems rather than cell-wise constant as it is the case for FVM.
%% Arbitrary grid
Furthermore, as in MPM, all the fields are stored at material points moving in the arbitrary grid, the mapping between nodes and particles being made with PIC projection so that non-oscillating solutions are provided.
\review{Hence, the geometry is updated at the particles level based on a single-valued velocity field. %, in such a way that the difficulties related to the Piola identity are expected to vanish, though it has not been shown so far.
  %% Total Lagrangian
  As a first development step, the DGMPM has been constructed upon a total Lagrangian formulation.
  %
  The numerical results provided by the method for a plane wave problem in a finite-deforming hyperelastic material showed excellent agreement with the exact solution consisting of either a shock or a rarefaction wave \cite{DGMPM}.
  It is worth noticing that similar results can be obtained with the FVM written in the reference configuration, the key point being however that the update of the geometry is straightforward in the DGMPM.
  %
  Furthermore, an interesting feature of the method consists in allowing the employment of mesh adaption strategies so that waves can be accurately captured in the current configuration.
  On the other hand, only linear shape functions leading to a first-order accuracy \cite{Thesis} have been considered, the extension of the method to higher-order approximations being the object of ongoing works.
  The low-order approximation is however not seen as a shortcoming for now since the development of the method has been focused so far on capturing discontinuous solutions for which the accuracy of any numerical scheme falls to one \cite{Leveque}.
}
%Although the equations of solid mechanics are solved in the reference configuration according to a total Lagrangian formulation, an interesting feature of the method consists in allowing the employment of mesh adaption strategies so that waves can be accurately captured in the current configuration.
%% CFL depending on the space discretization
Nonetheless, the stability of the DGMPM highly depends on the distribution of particles inside the computational grid.
Indeed, the stability analysis of the one-dimensional DGMPM scheme coupled to the forward Euler time integration \cite{DGMPM} yields a stability condition that depends on the space discretization and the CFL number.
%
Conversely, one can ensure the stability for a given distribution of material points by finding the CFL number satisfying the aforementioned relation.
%
Such a condition, which does not exist for the MPM and other DGMPM discretizations, is crucial since it allows to: (i) ensure the stability of the scheme while minimizing the numerical diffusion; (ii) adapt the Courant number when the grid is reconstructed; (iii) adapt the grid so that a given CFL condition is satisfied.
It is the purpose of this paper to provide the stability conditions for the one-dimensional DGMPM scheme combined with the two-stage second-order Runge-Kutta (RK2) time discretization and for the two-dimensional DGMPM coupled to the forward Euler algorithm.
\review{
  Although the solution of linear equations is considered here, the results presented must be put into the context of the problems aimed by the method, involving large deformation, and for which the DGMPM enables plenty of perspectives. 
  }

%% Organization of the paper
In the following, the DGMPM discrete system for the multi-dimensional scalar linear advection equation is derived and the computation of interface fluxes, as well as the solution scheme, are recalled in section \ref{sec:dgmpm}.
In particular, we shall see that the adaptation of the Corner Transport Upwind method (CTU) \cite{Colella_CTU} to DGMPM leads to the same corrections of intercell fluxes as for finite volumes.
Second, the system resulting from the combination of DGMPM and RK2 discretization is specialized to one-dimensional problems in section \ref{sec:1d_stab} so that the von Neumann linear stability analysis is carried out.
At last, the same approach is followed in section \ref{sec:2d_stab} for the DGMPM scheme coupled to the forward Euler time integration applied to the two-dimensional problem.


%%% Local Variables:
%%% mode: latex
%%% TeX-master: "manuscript"
%%% End:
