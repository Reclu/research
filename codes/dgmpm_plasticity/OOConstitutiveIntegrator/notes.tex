\documentclass[11pt]{article}
\usepackage{graphicx}
\usepackage[font=normal]{caption}
\usepackage[font=normal]{subcaption}
\usepackage[tensorialbold]{userCommands}
\usepackage{geometry}
\usepackage[utf8]{inputenc}
\usepackage[francais]{babel}
\geometry{hmargin=2.cm,vmargin=3.5cm}
\captionsetup{margin={1.5cm,1.5cm}}
\captionsetup[subfigure]{position=top}

\usepackage{tikz} % geometrical figures
\usetikzlibrary{calc,trees,positioning,arrows,chains,shapes.geometric,decorations.pathreplacing,decorations.pathmorphing,shapes,
    matrix,shapes.symbols,shapes.multipart,patterns,shapes,snakes,pgfplots.groupplots,spy,backgrounds}
\usetikzlibrary{external}
\tikzexternalize[prefix=external/]
\usetikzlibrary{spy,backgrounds}
\usetikzlibrary{decorations.markings}
\usetikzlibrary{arrows.meta}

\usepackage{pgfplots} % pdf picture declaration

\definecolor{Purple}{RGB}{120,28,129}
\definecolor{Orange}{RGB}{231,133,50}
\definecolor{Blue}{RGB}{63,96,174}
\definecolor{Red}{RGB}{217,33,32}
\definecolor{Duck}{RGB}{83,158,182}
\definecolor{Green}{RGB}{109,179,136}
\definecolor{Yellow}{RGB}{202,184,67}

\usepackage{array}
\newcolumntype{M}[1]{>{\centering\arraybackslash}m{#1}}
\newcolumntype{N}{@{}m{0pt}@{}}

\begin{document}

\section{{\'E}crouissage isotrope}
\label{sec:ecrouissage-isotrope}

\subsection{Variables d'{\'e}tats}
\label{sec:description}

\begin{itemize}
\item Variables internes: $\tens{\eps}^p, r$
\item Forces thermodynamiques associées: $\tens{s}, -R = k p^{\frac{1}{m}}$
\end{itemize}

\subsection{Surface de charge}
\label{sec:surface-de-charge}

Pour un solide de Von Mises:
\begin{equation}
  \label{eq:isotropic_hardening_yield_surface}
  f = \sqrt{\frac{3}{2}}\norm{\tens{s}} - \sigma^y - R(p) \equiv 0
\end{equation}

\subsection{Normalit{\'e}}
\label{sec:normalite}

\begin{equation}
  d\tens{\eps}^p = \drond{f}{\tens{\sigma}} d\lambda =\sqrt{\frac{3}{2}}\frac{\tens{s}}{\norm{\tens{s}}} d\lambda= \sqrt{\frac{3}{2}}\tens{N} d\lambda
  %& dp = \drond{f}{-R} d\lambda = d\lambda
\end{equation}

\subsection{Conditions d'évolution}

Kuhn-Tucker + plasticité associées...
\begin{align}
  & F(\tens{s},R) = f \\
  &d\tens{\eps}^p = \drond{F}{\tens{s}} d\lambda =\sqrt{\frac{3}{2}} \tens{N} d\lambda\\
  & dp = \drond{F}{-R} d\lambda = d\lambda
\end{align}

\subsection{Consistance}

\begin{equation}
  \label{eq:2}
  df = \drond{f}{\tens{s}} d\tens{s} - \drond{f}{R}dR = 0 \Rightarrow dR = \sqrt{\frac{3}{2}}\tens{N}:d\tens{s}
\end{equation}


\subsection{Retour radial}

On connait $\tens{\eps}^p_{n},p_n,\tens{\sigma}_n$ et $\tens{\eps}_{n+1}$.
\begin{itemize}
\item[1-] Prédiction élastique: $\tens{\sigma}^{\text{trial}}=2\mu (\tens{\eps}_{n+1}-\tens{\eps}^p_{n}) + \lambda \tr{\tens{\eps}_{n+1}}\tens{I} $
\item[2-] Critère de plasticité: $f^{\text{trial}} = \sqrt{\frac{3}{2}}\norm{\tens{s}^{\text{trial}}} - \sigma^y - R(p_n)$
\item[3-] Si $f^{\text{trial}}\leq 0$, l'évolution est élastique donc $\tens{\sigma}_{n+1}=\tens{\sigma}{\text{trial}}$. Sinon, on fait la correction élastique:
  \begin{itemize}
  \item[(a)] $\tens{s}_{n+1}=\tens{s}^{\text{trial}} - 2\mu \sqrt{\frac{3}{2}} \tens{N} dp$ ???
  \item[(b)] $(a):\tens{N} \Rightarrow \norm{\tens{s}_{n+1}}=\norm{\tens{s}^{\text{trial}}} - 2\mu \sqrt{\frac{3}{2}} dp$ 
  \item[(c)] On cherche à vérifier de critère à $t^{n+1} \: \Rightarrow \sqrt{\frac{3}{2}}\norm{\tens{s}_{n+1}} - \sigma^y -R(p+dp) = 0 $
  \item[(d)] Avec (c), (b) devient:
    \begin{align}
      & \norm{\tens{s}^{\text{trial}}} - 2\mu \sqrt{\frac{3}{2}} dp - \sqrt{\frac{2}{3}}(\sigma^y  + R(p+dp)) =0 \\
      & \sqrt{\frac{3}{2}} \norm{\tens{s}^{\text{trial}}} - 3\mu  dp - (\sigma^y  + R(p+dp)) =0 \\
      & \sqrt{\frac{3}{2}} \norm{\tens{s}^{\text{trial}}} - 3\mu  dp - \sigma^y  - R(p+dp) +R(p) -R(p) =0 \\
      & f^{\text{trial}}  - 3\mu  dp   - R(p+dp) +R(p) =0 \\
      & f^{\text{trial}}  - 3\mu  dp   - \Delta R =0 \\
      & f^{\text{trial}}  - 3\mu  dp   - R'(p)dp =0 \\
      & (3\mu +R'(p))dp -f^{\text{trial}}=0
    \end{align}
  \item[(e)] Une fois résolue l'équation nonlinéaire ci-dessus pour $dp$, on peut actualiser toutes les quantités:
    \begin{align}
      & \tens{\eps}_{n+1}^p = \tens{\eps}_{n}^p + \sqrt{\frac{3}{2}}\tens{N}dp\\
      & \tens{\sigma}_{n+1}=2\mu (\tens{\eps}_{n+1}-\tens{\eps}^p_{n+1}) + \lambda \tr{\tens{\eps}_{n+1}}\tens{I} 
    \end{align}
  \end{itemize}
\end{itemize}

\section{{\'E}crouissage cinématique (lin{\'e}aire)}
\label{sec:ecrouissage-isotrope}

\subsection{Variables d'{\'e}tats}


\begin{itemize}
\item Variable interne: $\tens{\eps}^p$
\item Force thermodynamique associée: $\tens{s}-\tens{X}$, avec $\tens{X}=\frac{2}{3}C\tens{\eps}^p$
\end{itemize}

\subsection{Surface de charge}


Pour un solide de Von Mises:
\begin{equation}
  \label{eq:kinematic_hardening_yield_surface}
  f = \sqrt{\frac{3}{2}}\norm{\tens{s}-\tens{X}} - \sigma^y  \equiv 0
\end{equation}

\subsection{Normalit{\'e}}


\begin{equation}
  d\tens{\eps}^p = \drond{f}{\tens{\sigma}} d\lambda %= \frac{\tens{s}-\tens{X}}{\norm{\tens{s}-\tens{X}}} d\lambda= \tens{N} d\lambda
\end{equation}

\end{document}
%%% Local Variables:
%%% mode: latex
%%% TeX-master: t
%%% End:

%%% Local Variables:
%%% mode: latex
%%% TeX-master: t
%%% End:
