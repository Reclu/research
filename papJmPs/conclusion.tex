
\subsection{Summary of the important contributions}
\label{sec:summ-import-results}


%% Theoretical part: generic formulation
In this paper, the characteristic structure of the solution of hyperbolic problems in elastic-plastic solids in two space dimensions has been highlighted.
First, a thermodynamically-consistent formulation leading to the writing of a hyperbolic system involving the fourth-order elastic-plastic stiffness tensor as been proposed.
The aforementioned tensor can be easily specialized to plane stress and plane strain problems in such a way that the quasi-linear form derived provides a generic framework for the study of all mechanical problems in two space dimensions.

%% Characteristic analysis
Second, the characteristic analysis of the hyperbolic plane strain and plane stress problems has been carried out.
As already emphasized for simpler two-dimensional problems in prior works \cite{Rakhmatulin,CRISTESCU19591605} the solutions involve slow and fast simple waves. 
The characteristic equations governing the evolution of the system inside the simple waves have then been derived as a set of ODEs by applying the method of characteristics.

%% Mathematical properties of the loading paths
Third, some mathematical properties of these characteristic equations have been highlighted for plane strain and plane stress, despite the complexity of the equations.
As an interesting result of this work, it has been shown that the loading paths followed inside slow and fast waves are perpendicular in the stress space.
Although this feature has been already emphasized in \cite{Clifton} for a combined longitudinal and torsional loading, the property is in fact due to the symmetry of the acoustic tensor and is therefore valid for all two-dimensional problems.

%% Numerical results
At last, to overcome the mathematical complexity of the characteristic equations, numerical investigations have been proposed.
The loading paths depicted in the stress space or in the deviatoric plane then enable the identification of symmetry properties that are not proofed mathematically.
Moreover, the integral curves holding inside fast waves are restricted to the initial yield surface for both plane stress and plane strain situations.  
In the former case, the paths end as soon as a direction of pure shear is reached in the deviatoric plane, whereas in the latter one, the paths appears to be radial once  a direction of pure tension/compression is achieved.
On the other hand, the loading through the simple waves exhibit rough changes regardless of the kinematic considered for low hardening moduli (\textit{i.e. $C=\mathcal{O}(10^8)$}). 
It has moreover been shown numerically that this inflection corresponds, for plane stress, to the reaching of the maximum shear stress on the current yield surface for a given longitudinal stress.
For plane strain, a similar response is also seen but before the \textit{maximum-shear-stress} condition is achieved.
% Similar conclusions can be drawn for plane strain.
Nevertheless, increasing the hardening modulus leads to loading paths whose direction changes before the \textit{maximum-shear-stress} condition is achieved for plane stress, and which reach a direction of pure shear at the same time as the maximum shear stress.


\subsection{Concluding remarks}
\label{sec:concludingRemarks}

%% Perspectives
The results of the present paper allow a better understanding of the physical response of linearly hardening elastic-plastic solids to dynamic loadings.
However, the singularities that have been highlighted numerically still need to be identified mathematically.
%It would also be interesting to consider different hardenings (\textit{i.e. kinematic, non-linear etc.}).  
Notice that kinematic hardening should yield identical results for the monotonic loadings considered here, but would greatly influence the response for unloading or reverse plastic loading.
  These waves must also be the object of an analysis for two-dimensional problems in order to construct the solution of the Riemann problem. 
As a more long-term perspective, the elementary loading paths studied here could be used in order to enrich numerical methods based on the use of Riemann solvers.
In fact, following the idea of \textsc{Lin} and \textsc{Ballman} \cite{Lin_et_Ballman}, a numerical procedure that accounts for both elastic and plastic characteristics can be developed in order to improve the tracking of waves in elastic-plastic solids.
% Nonetheless, the hyperbolic problems in elastoplastic media may involve not only simple waves but also shocks so that the effects of the latter must be investigated as well. 


%%% Local Variables:
%%% mode: latex
%%% TeX-master: "manuscript"
%%% End:
