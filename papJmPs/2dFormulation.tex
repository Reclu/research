%\subsection{Problems in two space dimensions}
%The formulation developed in the previous section differs from those of \textsc{Bleich} \cite{Bleich}, \textsc{Clifton} \cite{Clifton}, and hence from these of \textsc{Ting} and \textsc{Nan} \cite{Ting68} and \textsc{Ting} \cite{Ting69}, in that equation  \eqref{eq:quasilinear_normal} is based on the elastoplastic stiffnesses rather than softnesses.
%As a consequence, it will be seen in what follows that the equations can be easily specialized to plane strain and plane stress cases.

We now focus on the solid domain $(x_1, x_2, x_3) \in [0,\infty[ \times [-h,h] \times [-e,e]$ in a Cartesian coordinate system, where $e$ and $h$ are arbitrary lengths.
%The solid is subject in the plane $x_1=0$ to a traction force $\vect{T}^1$ restricted to the $(\vect{e}_1,\vect{e}_2)$ plane, that is $T_3=0$.
It is assumed that all quantities depend solely on $x_1$ and $x_2$ except the velocity component $v_3$ that may depend on $x_3$.
In particular, this is the case for $e \ll h$.
% $
Furthermore, only plastic simple waves are considered from now on.

The solid is under the plane strain condition $\tens{\eps}\cdot\vect{e}_3=\vect{0}$ if the velocity $\vect{v}$ does not depend on $x_3$ and if $v_3$ vanishes.
Thus, combining the additive partition of the infinitesimal strain tensor: $\tens{\eps}=\tens{\eps}^e+\tens{\eps}^p$, with the elastic law \eqref{eq:elastic_inverse} and the kinematic condition $\eps_{33}=0$, one gets for isotropic materials:
\begin{equation}
  \label{eq:plane_strain_stress33}
  \sigma_{33}=\nu\(\sigma_{11}+\sigma_{22}\) - E\eps^p_{33}
\end{equation}
Hence, the quasi-linear form \eqref{eq:quasilinear_normal} reduces for plane strain problems to a system of dimension $5$ for the unknowns $v_1,v_2, \sigma_{11},\sigma_{12}$, and $\sigma_{22}$.


The plane stress state $\tens{\sigma}\cdot\vect{e}_3=\vect{0}$ is assumed if the planes $x_3=\pm h$ are traction free and $e\ll h$.
In that case the stress component $\sigma_{33}$ is removed from system \eqref{eq:quasilinear_normal}.
Nevertheless, the tangent modulus must account for the vanishing out-of-plane stress component by specializing equation \eqref{eq:elastoplastic_tangent} to $\sigma_{33}$:
\begin{equation*}
  \dot{\sigma}_{33}=C^{ep}_{33ij} \dot{\eps}_{ij} =0
\end{equation*}
and therefore:
\begin{equation*}
  C^{ep}_{3333} \dot{\eps}_{33} = - C^{ep}_{33ij}\dot{\eps}_{ij} \quad i,j=\{1,2\}
\end{equation*}
since $\eps_{13}=\eps_{23}=0$ also in plane stress.
Hence, the constitutive equations are rewritten by means of a two-dimensional tangent modulus $\widetilde{\Cbb}^{ep}$:
\begin{equation}
  %\label{eq:CP_constitutive}
  \dot{\sigma}_{ij}=\(C^{ep}_{ijkl} - \frac{C^{ep}_{ij33}C^{ep}_{33kl}}{C^{ep}_{3333}}\) \dot{\eps}_{kl}= \widetilde{C}^{ep}_{ijkl} \dot{\eps}_{kl}\quad i,j,k,l=\{1,2\}
\end{equation}
The characteristic structure of the problem is then governed by the associated acoustic tensor $\tens{\widetilde{A}}^{ep}=\vect{n}\cdot\widetilde{\Cbb}^{ep}\cdot \vect{n}$.

The removal of $\sigma_{33}$ from system \eqref{eq:quasilinear_normal} for both plane strain and plane stress situations allows solving the problem in a two-dimensional setting.
Then, generically denoting the acoustic tensor by $\tens{A}$, the characteristic structures are given by the eigenvalues:
\begin{align}
    \label{eq:eigenAcc1}
  &\omega_1 = \frac{A_{11}+A_{22} + \sqrt{(A_{11}-A_{22})^2+{4A_{12}}^2}}{2} \\
  \label{eq:eigenAcc2}
  &\omega_2 = \frac{A_{11}+A_{22} - \sqrt{(A_{11}-A_{22})^2+{4A_{12}}^2}}{2}     
\end{align}
and the associated left eigenvectors:
\begin{equation}
  \label{eq:eigenvectAcc}
  \vect{l}^1=\[ A_{22}-  \omega_1 \:,\: -A_{12}\] \: ;\:  \vect{l}^2=\[ -A_{12} \:,\:A_{11}- \omega_2 \]
\end{equation}
From equation \eqref{eq:left_eigenfields}, we see that two families of waves with celerities $c_f=\pm \sqrt{\omega_1/\rho}$ and $c_s = \pm \sqrt{\omega_2/\rho}$ may travel in the domain.
These waves are respectively referred to as fast and slow waves.
One easily shows that property \ref{pr:mandel_inequality} can be particularized to two-dimensional problems and yields: $c_1\geq c_f\geq c_2 \geq c_s$, where $c_1$ and $c_2$ are the speeds of elastic pressure and shear waves respectively.
% Note that subtracting equations \eqref{eq:eigenAcc1} and \eqref{eq:eigenAcc2} leads to:
% \begin{equation}
%   \label{eq:diff_celerities}
%   \rho c_f^2 - \rho c_s^2 = \sqrt{(A_{11}-A_{22})^2+{4A_{12}}^2} \geq 0
% \end{equation}
% Hence, the characteristic speed associated with fast waves is always greater than or equal to that of slow waves.

\begin{remark}
  % Due to the non-linearity of $\Cbb^{ep}$, the solution may contain shock and/or simple waves.
  % However, we restrict here to the latter situation by assuming that:
  Given the non-linearity of $\Cbb^{ep}$ and the mathematical complexity of equations \eqref{eq:eigenAcc1} and \eqref{eq:eigenAcc2}, the following assumptions are made in what follows:
  \begin{itemize}
  %\item[(i)] the characteristic speeds satisfy $c_1 \geq c_f \geq c_2 \geq c_s $, where $c_1$ and $c_2$ are the speeds of elastic pressure and shear waves respectively \review{first Mandel's inequality ! \cite{mandel_book}: valid since the focus is here on simple wave solutions.}
  \item[(i)] $c_f$ and $c_s$ monotonically decrease with the hardening of the material,
  \item[(ii)] the computational domain is in an initial natural, plastic strain free state. %, since the hardening has an influence on the wave speeds.
  \end{itemize}
\end{remark}
% \review{Note also that property \ref{pr:mandel_inequality} is still valid for $\tens{\widetilde{A}}^{ep}$.}

Given the eigenvalues \eqref{eq:eigenAcc1} and \eqref{eq:eigenAcc2}, the four left eigenfields of the Jacobian matrix read:
\begin{align}
  \label{eq:Jac_eigenfield_fast}
  &\left\lbrace \pm c_f ; \quad \Lcb^{c_f^\pm}=\[\: \pm \rho c_f \vect{l}^1 , -\vect{l}^1\otimes \vect{n} \:\]  \right\rbrace \\
  \label{eq:Jac_eigenfield_slow}
  &\left\lbrace \pm c_s ; \quad \Lcb^{c_s^\pm}=\[\: \pm \rho c_s \vect{l}^2 , -\vect{l}^2\otimes \vect{n} \:\]  \right\rbrace
\end{align}
where $\Lcb^{c_f^+}$ and $\Lcb^{c_f^-}$ are associated with the right-going and left-going fast waves respectively.
The same goes for $\Lcb^{c_s^+}$ and $\Lcb^{c_s^-}$.
Furthermore, one stationary wave associated with the zero eigenvalue of the Jacobian matrix, and whose left eigenvector satisfies equation \eqref{eq:left_null_eigenvectors}, has to be added:
\begin{equation}
  \label{eq:null_left_eigen}
  {\Lcb^0}^T =  \matrice{0 \\[5.pt] 0 \\[5.pt] \(C_{121i}C_{222j}-C_{221i}C_{122j}\)n_in_j \\[5.pt] \(C_{111i}C_{122j}-C_{112i}C_{121j}\)n_in_j \\[5.pt] \(C_{112i}C_{221j}-C_{111i}C_{222j}\)\frac{n_in_j}{2}} = \matrice{0 \\ 0 \\ \alpha_{11} \\ \alpha_{22} \\ \alpha_{12} }
\end{equation}
with $\Cbb=\Cbb^{ep}$ for plane strain and $\Cbb=\widetilde{\Cbb}^{ep}$ for plane stress situations.
The above eigenfields are used in the next section in order to derive the simple wave solutions by means of the method of characteristics \cite{Courant}.


%%% Local Variables:
%%% mode: latex
%%% TeX-master: "manuscript"
%%% End:
