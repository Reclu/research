The extension of the material point method to the DG approximation has been motivated in the previous section and is developed hereinafter. After a brief historical review of DG methods, the Discontinuous Galerkin Material Point Method (DGMPM) is derived within the large strain framework. It will then be seen that this new numerical approach uses the approximate-state Riemann solver developed in section \ref{sec:riemann_solvers} to compute terms at elements interfaces which purpose is to connect elements together. At last, the DGMPM solution scheme will be provided for hyperbolic problems.

\subsection{The discontinuous Galerkin approximation}
The DG approximation was first introduced in the context of the finite element method for the solution of the neutron transport equation \cite{NeutronDG}. This hyperbolic equation describes the advection of the angular flux which quantifies the amount of neutrons at a given location. Since neutrons can lie in a cell of a finite element mesh while its neighbors are empty, the need of describing discontinuities of the primal field across elements interfaces within a FEM context arised. Hence, an approximate solution was seeked by the Galerkin method, in a domain discretized with triangular elements by means of Lagrange polynomials that can be discontinuous across the cells. This approach amounts to duplicate the nodes of the mesh so that the support of each shape function reduces to one finite element. Those early works have launched a series of developed of the Discontinuous Galerkin Finite Element Method (DGFEM) for parabolic \cite{Arnold_IPM}, elliptic \cite{Hansbo_DGsolid,Noel_HEDG}, and hyperbolic problems \cite{Cockburn}. Indeed, the DGFEM gained more and more popularity since the 80's due to its ability to locally handle high-order approximation and its highly parallelizable nature. 
%However, the increase of the dimension of the discrete it implies led more recently to the formulation of Hibridizable Discontinuous Galerkin methods (HDG) \cite{Cockburn_HDG0,Cockburn_HDG1,Cockburn_HDG2}.
For application to hyperbolic problems, of particular interest here, researches enabled the introduction of numerical tools developed for the Finite Volume Method (FVM) within finite element schemes. Namely, the use of suitable \textit{slope limiters} based on the \textit{total variation} \cite{vanLeer_Limiters} allowed the formulation of flexible numerical methods in which high resolution of discontinuities, without destroying the accuracy in smooth regions, is possible.

Ensuite, parler des tentatives d'application à la PIC: 
A similar approach based on the particle-in-cell method has already been developed in \cite{DGPIC_maxwell}. However in the latter work, the particles are not viewed as quadrature points that carry all the fields and this method can not be considered as a fully Lagrangian approach.

Although this approach can easily handle mesh-adaption strategies due to the relaxation of fields continuity, the mesh tangling problems do not vanish. We hope that the following developments will be the starting point of a DG-method enabling the solution of problems involving finite deformation in solid dynamics. 


% %%%%%%%%%%%%%%%%%%%%%%%%%%%%%%%%%%%%%%%%%%%%%%%%%%%%%%%%%%%%%
% % Hyperbolic
% The \textit{discontinuous Galerkin (DG)} approximation enables to build numerical schemes that benefit from both finite element and finite volume methods. 
% Parler des limiteurs \cite{vanLeer_Limiters} pour atteindre la notion de schéma TVB et TVDM (TVD \cite{Harten_TVD}). Extension to RK so that the scheme can reach locally high order accuracy. In addition, the same order of accuracy is reached for velocity and gradients within a finite element framework when the weak form is based on a conservation laws system. 

% % In parallel, for parabolic or elliptic
% \cite[parabolic+penalties]{Arnold_IPM},\cite{Hansbo_DGsolid},\cite[elliptic]{Noel_HEDG}: Three field Hu-Washizu variational formulation ; assumed form of deformation gradient ; Total Lagrangian ; penalization of displacement jumps; Advantages--Drawbacks in note books

% % Introduction of HDG
% \cite{Cockburn_HDG0},\cite{Cockburn_HDG1},\cite{Cockburn_HDG2}: HDG for elliptic problems (differrence between LDG-H and HDG ?)

% % Recently extended to hyperbolic problems in solid dynamics
% \cite{NGuyen_HDG} for application to solid mechanics and extension to hyperbolic problems. Solved for displacement with enforced continuity across elements interfaces. Does not use the characteristic structure as what is done in finite volumes or original DGFEM

% However, all those approaches based on the primal field do not use the characterisitc structure. Dans l'état, ça ne peut pas être appliqué à la méca des solides puisque l'on est obligé d'imposer la continuité du champ de déplacement et donc de formuler le problème en u.
% \begin{itemize}
% \item \cite{Chavent_Salzano,Chavent_Cockburn,Cockburn_Shu,DGFEM_CFL,Cockburn}:
% \item \cite{Chavent_Salzano}--\cite{Cockburn} developments of DGFEM (mainly for fluid mechanics ?)
% \item \cite[parabolic+penalties]{Arnold_IPM},\cite{Hansbo_DGsolid},\cite[elliptic]{Noel_HEDG}: Three field Hu-Washizu variational formulation ; assumed form of deformation gradient ; Total Lagrangian ; penalization of displacement jumps; Advantages--Drawbacks in note books
% \item \cite{Cockburn_HDG0},\cite{Cockburn_HDG1},\cite{Cockburn_HDG2}: HDG for elliptic problems (differrence between LDG-H and HDG ?)
% \item \cite{NGuyen_HDG} for application to solid mechanics and extension to hyperbolic problems. Solved for displacement with enforced continuity across elements interfaces. Does not use the characteristic structure as what is done in finite volumes or original DGFEM
% \item \cite{DGPIC,DGPIC_maxwell}: application to PIC
% \end{itemize}


% RKDG - limiters - TVD - CFL - HDG (steady convection-diffusion problem (elliptic problem)) - HE DG
% Applied to steady solid mechanics problems for the ability of local high order approximation ?
% Continuity of displacement enforced through Lagrange multipliers. Benefits from superconvergence properties. A priori on utilise plus du HDG en méca du solide \cite[ALE]{NGuyen_HDG} + postprocessing pour le champ de déplacement + implicit en dynamique. Originally developed in \cite{Cockburn_HDG0}: "we may define more generally as a hybrid method any finite element method based on a formulation where one unknown is a function, or some of its derivatives, on the set $\Omega$, and the other unknown is the trace of some of its derivatives of the same function, or the trace of the function itself, along the boundaries of the set K". Numerical traces \cite{Cockburn_HDG1}


\subsection{The DGMPM discretization}
As for MPM, a continuum body $\Omega_t$ is discretized into a set of $N_p$ material points in an arbitrary Cartesian grid made of  $N_n$ nodes and $E$ non-overlapping cells of volume $\Omega^e$, with the boundary of the domain defined by boundary material points (recall the two-dimensional example in figure \ref{fig:domain}).

The DGMPM is also based on a weak formulation. Indeed, we seek an approximate solution of a Lagrangian system of conservation laws in conservative form written with a Cartesian coordinates system within the time interval $\tau$:
\begin{equation}
  \label{eq:conservative_form}
  \drond{\Qcb}{t} + \sum_{\alpha=1}^D \drond{\Fcb\cdot \vect{e}_\alpha}{X_\alpha} = \Scb \quad \forall \vect{X},t \in \Omega_0 \times \tau
\end{equation}
The key idea of DG methods is to allow jump of fields across mesh elements faces by using broken polynomials spaces \cite{DiPietro,Hesthaven,DG_survey}, for a first order approximation:
\begin{equation}
\Vscr^1 = \{ \Vcb \in H^1(\Omega^e) \} \quad ;\quad \Vscr_h^1 = \{\Vcb \in \Pscr^1(\Omega^e) \} \subset \Vscr^1
\end{equation}
with $H^1(\Omega^e)$, the Sobolev space and $\Pscr^1(\Omega^e)$, the linear (Lagrange) polynomials space in $\Omega^e$. Those broken polynomials spaces yield a weak form of equation \eqref{eq:conservative_form} written element per element. After integration by part:
\begin{equation}
  \label{eq:DGMPM_weak_form}
  \begin{aligned}
    &\text{Find $\Qcb \in \Vscr_h^1$ such that} \\
    &\int_{\Omega^e} \drond{\Qcb}{t} \vect{\Vc} \: d\Omega - \int_{\Omega^e} \Fcb_\alpha  \drond{\vect{\Vc}}{X_\alpha} \: d\Omega   + \int_{\partial \Omega^e} \(\Fcb\cdot \vect{N}\)  \vect{\Vc} \: d\Gamma = 0 \quad \forall \: \vect{\Vc} \in \Vscr_h^1, \forall\: e \in \[1,E\] , \forall\: t \in \tau
  \end{aligned}
\end{equation}
where $\partial \Omega^e$ is the boundary of the $e$th element with outward normal vector $\vect{N}$. 
%Those numerical fluxes can be computed from the solution of an approximate Riemann solver \cite{Trangenstein} (see section \ref{}). In particular, the stationary solution of Riemann problems defined at element boundaries $\Gamma_i$ such that $\cup_i \Gamma_i = \Gamma_e $  gives the well-known \textit{Godunov's method}.
As in MPM, the use of the delta Dirac characteristic function is made for material points so that specific quantities are introduced:
\begin{align}
& \Qcb = \rho_0 \bar{\Qcb} \quad ; \quad \Fcb_\alpha = \rho_0 \bar{\Fcb}_\alpha \\
& \rho_0\(\vect{X}\) =  \sum_{p=1}^{N_p} m_p \delta\(\vect{X}^p - \vect{X}\)
\end{align}
where $m_p$ is the mass of the $p$th material point. Such a discretization of the reference mass density combined with the writing of Lagrangian conservation laws \eqref{eq:conservative_form} yields a total Lagrangian formulation, and equation \eqref{eq:DGMPM_weak_form} thus reads:
\begin{equation} 
  \label{eq:DGMPM_discrete_weak}
  \sum_{p=1}^{N_p} m_p\[\drond{\bar{\Qcb}}{t}  \vect{\Vc} - \bar{\Fcb}_{\alpha} \drond{\vect{\Vc}}{X_\alpha} \]_{|\vect{X}=\vect{X}^p} + \int_{\partial \Omega^e} \(\Fcb\cdot\vect{N}\)  \vect{\Vc} \: d\Gamma = 0 \quad \forall \: \vect{\Vc} \in \Vsf_h^1, \forall\: e \in \[1,E\], \forall\: t \in \tau  
\end{equation}

Material points are viewed as interpolation points so that the fields of the weak form \eqref{eq:DGMPM_discrete_weak} are evaluated from nodal values with the shape functions $S(\vect{X})$:
\begin{equation}
  \label{eq:DGMPM_node2points}
  \bar{\Qcb}(\vect{X}^p) = \bar{\Qcb}^p =\sum^{N_n}_{i=1} S_{i}(\vect{X}^p)\bar{\Qcb}^i = \sum^{N_n}_{i=1} S_{ip}\bar{\Qcb}^i 
\end{equation}
$\bar{\Qcb}^i$ hence denotes the specific vector of conserved quantities at node $i$. In the following, the same convention than before, consisting in denoting particles values by index $p$ and nodal ones by $i$ or $j$, is used. 

Combination of equations \eqref{eq:DGMPM_discrete_weak}-\eqref{eq:DGMPMnode2points} and arbitrariness of the test field yield the semi-discrete system that must be solved on the grid:
\begin{equation}
  \label{eq:DGMPM_semi_discrete}
  \sum_{p=1}^{N_p}\[ S_{ip} m_p S_{jp} \drond{\bar{\Qcb}_j}{t}  - \drond{S_{ip}}{X_\alpha} m_p S_{jp} \bar{\Fcb}_{\alpha,j}\] + \int_{\Gamma_e} S_i(\vect{X}) \(\Fcb\cdot\vect{N}\)  \: d\Gamma =  0  \quad \forall\: e \in \[1,E\] , \forall\: t \in \tau
\end{equation}
or, in matrix form:
\begin{equation}
  \label{eq:DGMPM_semi_discrete_matrix}
   M_{ij} \drond{\bar{\Qcb}_j}{t} - K^\alpha_{ij} \bar{\Fcb}^j_{\alpha} + \vect{\hat{\Fc}}^i = \vect{0}  
\end{equation}
In system \eqref{eq:DGMPM_semi_discrete_matrix}, the consistent mass matrix and the \textit{pseudo-stiffness} matrix are respectively defined as:
\begin{subequations}
  \begin{alignat}{1}
    \label{eq:DGMPM_mass_matrix}
    & M_{ij}= \sum_{p=1}^{N_p} m_p  S_{jp} S_{ip} \\
    \label{eq:DGMPM_pseudo-stiffness}
    & K^\alpha_{ij} = \sum_{p=1}^{N_p} m_p\drond{S_{ip}}{X_\alpha} S_{jp} \bar{\Fcb}_{\alpha,j}
  \end{alignat}
\end{subequations}
Note that the consistent mass matrix $M_{ij}$ may also be singular when only one material point lies in an element due to reduced integration. Hence, the diagonally lumped mass matrix $M^L_i$ is used.

Finally, the explicit forward Euler time discretization is perfomed, leading to the discrete system:
\begin{equation}
  \label{eq:DGMPM_discrete}
  M^L_i \frac{\bar{\Qcb}^{i,n+1} - \bar{\Qcb}^{i,n}}{\Delta t^{n} } = K^\alpha_{ij} \bar{\Fcb}_{\alpha}^{j,n}- \vect{\hat{\Fc}}^{i,n}  
\end{equation}
where the superscripts denotes the time step at which a field is evaluated.


\subsection{Numerical fluxes}
\subsubsection*{The Godunov's fluxes}

Discontinuity of stationary solution across stationary waves so, discontinuity of flux so, two fluxes are used (discontinuity of flux means loss of information. Does it correspond to transverse correction ?).
\subsubsection*{Transverse corrections}
The method derived above for the computation of normal fluxes can be viewed as the \textit{Donor-Cell Upwind (DCU)} method \cite{Leveque} meaning that only contributions from the upwind cells sharing an edge (in two dimensions) with the current one are considered. For multidimensional problems waves can travel in several directions such that contributions coming from corner cells must be taken into account in order to improve accuracy and stability of the numerical scheme. The \textit{Corner Transport Upwind (CTU)} method \cite{Colella_CTU} consists in considering contributions coming from upwind cells sharing only a node (in two dimensions) with the considered grid cell and propagating in bias. This approach allows to improve the Courant condition especially for solid mechanics problems for which strain components are coupled through Poisson's effect. One can define at each cell interface left-going and right-going fluctuations defined as:
\begin{equation}
  \Acb^-(\Delta \Qcb) = \Fcb(\Qcb^*) - \Fcb(\Qcb^{\xi^-}) \qquad ;  \qquad \Acb^+(\Delta \Qcb) = \Fcb(\Qcb^{\xi^+})-\Fcb(\Qcb^*) 
\end{equation}
\begin{figure}[h!]
  \centering
  \definecolor{Purple}{RGB}{120,28,129}
\definecolor{Orange}{RGB}{231,133,50}
\definecolor{Blue}{RGB}{63,96,174}
\definecolor{Red}{RGB}{217,33,32}
\definecolor{Duck}{RGB}{83,158,182}
\definecolor{Green}{RGB}{109,179,136}
\definecolor{Yellow}{RGB}{202,184,67}
\begin{tikzpicture}[scale=1.8]
  \draw (0.,0.) -- (0.,2.); 
  \draw (-2.,2.) -- (2.,2.);\draw (-2.,0.) -- (2.,0.);
  \draw[->,thick,Red] (-0.25,1.4) -- (0.25,1.4) node [right] {$\Acb^{i,+}(\Delta \Wcb)$};
  \draw[->,thick,Red] (0.25,0.6) -- (-0.25,.6) node [left] {$\Acb^{i,-}(\Delta \Wcb)$};
  \node[left] at (0.,1.) {$(i)$};
  \draw[->] (0.,1.) -- (0.25,1.) node [right] {$\vect{N}^i$};
  \node[above] at (-1.5,0.)  {(k)};\node[above] at (1.5,0.)  {(m)};
  \node[below] at (-1.5,2.)  {(j)};\node[below] at (1.5,2.)  {(l)};
  \draw[->] (-0.5,0.) -- (-0.5,-0.25) node [below] {$\vect{N}^k$};
  \draw[->] (-0.5,2.) -- (-0.5,2.25) node [above] {$\vect{N}^j$};
  \draw[->] (0.5,0) -- (0.5,-0.25) node [below] {$\vect{N}^m$};
  \draw[->] (0.5,2) -- (0.5,2.25) node [above] {$\vect{N}^l$};
  \draw[->,thick,Blue] (-1.25,0.25) -- (-1.25,-0.5) node [below] {$\Bcb^{k,+} \Acb^{i,-}(\Delta \Wcb)$};
  \draw[->,thick,Blue] (-1.25,1.75) -- (-1.25,2.5) node [above] {$\Bcb^{j,+} \Acb^{i,-}(\Delta \Wcb)$};
  \draw[->,thick,Blue] (1.25,0.25) -- (1.25,-0.5) node [below] {$\Bcb^{m,+} \Acb^{i,+}(\Delta \Wcb)$};
  \draw[->,thick,Blue] (1.25,1.75) -- (1.25,2.5) node [above] {$\Bcb^{l,+} \Acb^{i,+}(\Delta \Wcb)$};
  \node[left] at (-1.,1.) {$\text{L}$} ;\node[right] at (1.,1.) {$\text{R}$} ;
  \node[above] at (-0.9,2.2) {$\text{T}$} ;
\end{tikzpicture}

%%% Local Variables: 
%%% mode: latex
%%% TeX-master: "../../mainManuscript"
%%% End:

  \caption{Normal and transverse fluctuations defined from edge $i$.}
  \label{fig:CTU}
\end{figure}
Let's consider the patch of grid cells shown in figure \ref{fig:CTU}, and focus on the edge denoted $(i)$ which local normal vector $\vect{N}^i$ is shown. The Riemann problem defined at this edge gives rise to normal fluctuations $\Acb^{i,-}(\Delta \Qcb)$ and $\Acb^{i,+}(\Delta \Qcb)$ contributing to cells L et R respectively. These terms lead to the computation of transverse fluctuations giving contribution to neighboring cells across edges $(j)$ and $(k)$ for cell L, and across edges $(m)$ and $(l)$ for cell R. Transverse fluctuations are computed by projecting normal fluctuations on the characteristic basis associated to the Riemann problem \eqref{Riemann_problem} defined on the adjacent edge, hence the name transverse Riemann solver. Spectral analysis of corresponding Jacobian matrix is carried out in \cite{Kluth}. The negative normal fluctuation is, for instance, decomposed on the characteristic basis associated to edge $(j)$ as:
\begin{equation}
\Acb^{i,-}(\Delta \Qcb) = \sum_{m=1}^{M} \beta_m \vect{\Rc}^{j,m}
\end{equation}
where $\vect{\Rc}^{j,m}$ is based on the normal vector $\vect{N}^j$ but also on different tangent moduli between grid cells L and its neighbor T. Since only waves with positive characteristic speeds with respect to the orientation defined by outward normal vector to considered edge will contribute to the transverse fluctuation, only the positive operator $\Bcb^+$ is used:
\begin{equation}
\Bcb^{j,+} \Acb^{i,-}(\Delta \Qcb) = \sum_{\underset{\lambda_m >0}{m=1}}^{M} c_m \beta_m \vect{\Rc}^{j,m}
\end{equation}
An additional numerical flux defined at edges is hence built from these transverse fluctuations:
\begin{equation}
\Fcb^{j,\text{tran}} = \frac{\Delta t}{2 \Delta X^j} \Bcb^+ \Acb^-(\Delta \Qcb)
\end{equation}
which contributes to the grid cell T ($\Delta X^j$ being the length of edge $(j)$). With duplicated nodes introduced by the DG approximation this contribution must be counted negatively for nodes belonging to cell L (outgoing fluctuation) and positively for nodes belonging to cell T (incoming fluctuation).
\subsubsection*{Enforcement of boundary conditions}
Quasi-linear form so that the integration of constitutive equations is not required providing that the auxiliary vector is projected onto the grid with the vector of conserved quantities.

\subsection{DGMPM solution scheme}
Let us assume that the vector of specific conserved quantities $\bar{\Qcb}^n$ is known at every material points that discretize a continuum body $\Omega$ in a grid made of $N_{nodes}$ nodes at a given time $t^n$. We can now derive the computational steps to obtain the DGMPM solution, which requires to (i) solve the hyperbolic system \eqref{} in $\Omega$ with associated boundary conditions and (ii) update the vector $\bar{\Qcb}^n$ on material points.

The scheme has been established with a total Lagrangian formulation therefore, the weak form \eqref{} is written on the reference configuration. Hence, the lumped mass matrix $M^L_i$ and the \textit{pseudo-stiffness} matrices $K^\alpha_{ij}$ are computed once and for all at the beginning of the calculation:
\begin{equation}
M^L_i = \sum^{N_p}_{p=1} S_{ip} m_p  \quad ; \quad  K^\alpha_{ij} = \sum^{N_n}_{p=1} \drond{S_{ip}}{X_\alpha} m_p S_{jp} \quad i,j = 1,...,N_{nodes}
\end{equation}
Then, the procedure follows these steps:
\begin{itemize}
\item[] $\bar{\Qcb}$ known at material points
\item[(a)] Convective phase: the discrete equation \eqref{} and Riemann problems \eqref{} require a projection of fields onto the grid:
  \begin{equation}
    \label{eq:DGMPM_points2nodes}
    M^L_i \bar{\Qcb}^i = \sum_{p=1}^{N_p} S_{ip} m_p \bar{\Qcb}^p \qquad \text{and} \qquad M^L_i \vect{\Qc}^i = \sum_{p=1}^{N_p} S_{ip} m_p \vect{\Qc}^p 
  \end{equation}
  to be solved for each $\bar{\Qcb}^i$ and $\vect{\Qc}^i$ respectively.
  This weighted least squares interpolation is similar the one introduced in FLIP (\textit{Fluid Implicit Particle method}) to avoid a diffusion-like error \cite{Sulsky94,Mass_Flip,FLIP}. %The same projection is made for the auxiliary vector. 
\item[(b)] Compute time step: in the general case the tangent modulus depends on the deformation gradient as well as the waves speeds. It is therefore needed to compute the time step so that the fastest wave can at most cross the smallest cell of the mesh according to Courant condition.
\item[(c)] The specific flux vectors $\bar{\Fcb}^i_{\alpha}$ involved in the discrete equation \eqref{} are computed from $\vect{\Qc}^i$ knowing $\rho_0$, thus avoiding the computation of constitutive equations.
\item[(d)] Enforce boundary conditions on mesh nodes (see section \ref{}).
\item[(e)] Computation of interface fluxes: 
  \begin{itemize}
  \item[1-] Build $\vect{\Qc}^{\xi^\pm}$ states based on $\vect{\Qc}^{i,n}$ where $i$ denotes the nodes belonging to the face on side $L$ or $R$.
  \item[2-] Compute the stationary solution by means of the approximate Riemann solver.
  \item[3-] Calculate the corresponding normal flux.
  \end{itemize} 
\item[(f)] Advance solution in time by solving the discrete equation \eqref{} at each node.
\item[(g)] Map the updated solution to material points with a classical interpolation:
  \begin{equation}
    \bar{\Qcb}^{n+1}_p = \sum_{i=1}^{N} S_{ip}\bar{\Qcb}^{i,n+1}
  \end{equation}
  At this point the mesh has virtually moved, but since fields have been transferred back to particles, the underlying grid can be discarded and rebuilt for the next time step for computational convenience, thus involving the convective phase (a) at the next time step. Notice that adaptive grid algorithms can be applied in the reference configuration, so that to improve wave front tracking in the current one.
\item[(h)] Material point kinematics and constitutive model: The new solution $\bar{\Qcb}^{p,n+1}$ allows to increment the deformation $\vect{\varphi}(\vect{X},t)$ and to update stress components which will be used in the auxiliary vector for the next time step, through hyperelastic constitutive equations:
  \begin{align}
    & \vect{\varphi}^{p,n+1}_p = \vect{X}^p + \Delta t \vect{v}^{p,n+1} \\
    & \tens{\Pi}^{p,n+1} =  \drond{\Psi}{\tens{F}}(\tens{F}^{p,n+1})
  \end{align}
\end{itemize}

This summary highlights significant differences with respect to the MPM. First, while the DGMPM uses a classical interpolation in order to map back the updated solution to material points (step f), FLIP method and MPM require an additional time integration on material points. Indeed, in the original schemes the changes in grid values (i.e: accelerations and strain rates) are mapped back to particles rather than the new value in order to avoid numerical dissipation \cite{FLIP}. Furthermore, the use of conservation laws \eqref{} instead of momentum equation in the weak form implies that both velocity and gradients are solved at nodes making this new approach close to finite volume methods. Finally, the deformation gradient is no longer calculated with shape functions gradients so that the task of choosing between \textit{Update Stress First} (USF) and \textit{Update Stress Last} (USL) algorithms \cite{USF_USL,Book_MPM} vanishes. The USL formulation consists in integrating the constitutive equations at material points with the updated nodal velocity field given by the resolution of the MPM discrete equation thanks to shape functions gradients. Within the USF algorithm, the material points stress tensors are computed with the nodal velocity field resulting from the convective phase at the beginning of each time step. Since the stress tensors at particles are used in the discrete equation through the internal forces vector, these two formulations modify the accuracy of the MPM. In that sense the DGMPM scheme is simpler though additional cost is linked to Riemann solutions. Fortunately, the use of discontinuous Galerkin approximation makes this numerical method highly parallelizable \cite{Cockburn}.


%%% Local Variables: 
%%% mode: latex
%%% TeX-master: "../mainManuscript"
%%% End:
