%% MPM "survey"
MPM = collocation method ?
\begin{itemize}
\item \cite{PIC}: Analogy with FVM since mean values of fields are defined in cells
\item \cite{FLIP}: mapping in momentum when particles cross a cell (see \cite{PIC_Nishiguchi}) ; finite volume for each particle (i.e. not a dirac) ; Make use of grid velocity and not material point velocity to update particles kinematic by direct interpolation to model \textit{collisional fluids} $\vect{v}_p=S_{ip}\vect{v}_i$ \textbf{two different velocity fields}
\item \cite{Mass_Flip}: "coarse-grained" (see \cite{Brackbill_PIC}), better to have a \textit{sub-grid-scale motion} and no additional diffusion ; mapp changes of fields instead of fields 
\item \cite{DDMP0}: use another form of gradient of shape functions.
\item \cite{DDMP}: shock capturing techniques ?
  \item \cite[Ch.8?,p.54]{MPMB_BSpline1}: oscillations due to quadrature rule (pas évident).
\item \cite{BsplineMPM}: ?
\end{itemize}
Le mapping des variations est évoqué dans \cite[p.26]{Love} qui cite lui-même \cite{FLIP} soit-disant que ça limite la dissipation quand on utilise une matrice lumpée. En réalité, c'est plutôt dit dans \cite{FLIP0} qui cite \cite{PIC}
Dire qu'un champ est défini dans les cellules grâce aux fonctions de forme (notamment dans la forme faible) $q(x)=N_iq_i(x)$, ce qui sous-entend que le mapping inverse doit être fait ainsi.

%%% Local Variables: 
%%% mode: latex
%%% TeX-master: "../mainManuscript"
%%% End:
