\chapter{Numerical Results}
\section*{Introduction}
The Discontinuous Galerkin Material Point Method developed in the previous chapter is now illustrated and compared to other numerical schemes (MPM, FVM, FEM) or analytical solutions. 
Firstly, one and two-dimensional problems that fall within the infinitesimal theory will be considered for elastic, elastic-viscoplastic and elastoplastic materials in sections \ref{sec:1dhpp_simulations} and \ref{sec:2dhpp_simulations}. The simulations performed in that context show significant improvements enabled by the introduction of DG approximation in the MPM. Second, numerical solutions of problems involving waves travelling in finite deforming solids, for which the dependence on history of constitutive models is neglected, are analyzed in sections \ref{sec:1dhe_simulations} and \ref{sec:2dhe_simulations}.

%%Study the influence of the number of material points per cell in simulations + comparison MPM

\section{Bar problems -- Linearized geometrical framework}
\label{sec:1dhpp_simulations}
\input{chapter4/1dhpp}

\section{Two-dimensional small strain problems}
\label{sec:2dhpp_simulations}
\subsection{Two-dimensional elasticity}
ceci est un test permettant de voir si la longueur prise par la figure estcohérente avec la longueur de ligne du texte. 
ceci est un test permettant de voir si la longueur prise par la figure estcohérente avec la longueur de ligne du texte. 
ceci est un test permettant de voir si la longueur prise par la figure estcohérente avec la longueur de ligne du texte. 
\begin{figure}[h!]
  \centering
  

\begin{figure}\centering
  \begin{tikzpicture}
    \begin{groupplot}[group style={group size=3 by 2,
        ylabels at=edge left, yticklabels at=edge left,
        horizontal sep=.5ex,
        vertical sep=2ex,},
      enlargelimits=0,
      xmin=0.,xmax=1., ymin=-0.,ymax=1.
      ,axis on top,scale only axis,xtick=\empty,ytick=\empty,width=0.25\linewidth,
      colorbar style={
        title style={
          font=\scriptsize,
          at={(1,.5)},
          anchor=north west
        },yticklabel style={font=\scriptsize}
      ,at={(current axis.south east)},anchor=south west
      }]
      %% FIRST ROW (time 1 = 3.5e-4s)
      %%% RANGE -1.6e7 -- 2.9e8
      \nextgroupplot[ylabel={$t=3.5\times 10^{-4} \:s.$},title={FEM}]\addplot graphics[xmin=0.,xmax=1., ymin=-0.,ymax=1.] {chapter4/pngFigures/fem_stress_78.png};
      \nextgroupplot[title={DGMPM 1ppc}]\addplot graphics[xmin=-0.,xmax=1., ymin=-0.,ymax=1.] {chapter4/pngFigures/dgmpm1ppc_stress_78.png};
      \nextgroupplot[title={DGMPM 4ppc},
      colorbar,colorbar style={
        title= {$\sigma_{11}\: (GPa)$},
        ytick={-1.6e-2,2.9e-1},
        %yticklabels={0,0.5,1},
      }]
      \addplot[scatter,scatter src=y,mark size=0.pt] coordinates {(0.,-1.6e-2) (0.,2.9e-1)};% Fake extreme values to fix scale
      \addplot graphics[xmin=-0.,xmax=1., ymin=-0.,ymax=1.] {chapter4/pngFigures/dgmpm4ppc_stress_78.png};

      %% SECOND ROW (time 2 =1.e-3s)
      %%% RANGE -2.9e7 -- 3.5e8
      \nextgroupplot[ylabel={$t=1.0\times 10^{-3} \:s.$}]\addplot graphics[xmin=0.,xmax=1., ymin=-0.,ymax=1.] {chapter4/pngFigures/fem_stress_231.png};
      \nextgroupplot[]\addplot graphics[xmin=-0.,xmax=1., ymin=-0.,ymax=1.] {chapter4/pngFigures/dgmpm1ppc_stress_231.png};
      \nextgroupplot[colorbar,colorbar style={
        title= {$\sigma_{11}\: (GPa)$},
        ytick={-2.9e-2,3.5e-1},
        %yticklabels={0,0.5,1},
      }]
      \addplot[scatter,scatter src=y,mark size=0.pt] coordinates {(0.,-2.9e-2) (0.,3.5e-1)};% Fake extreme values to fix scale
      \addplot graphics[xmin=-0.,xmax=1., ymin=-0.,ymax=1.] {chapter4/pngFigures/dgmpm4ppc_stress_231.png};
      
    \end{groupplot}
  \end{tikzpicture}
  \caption{time $t=3.5e-4s$}
  \label{fig:2delast_comparison1}
\end{figure}



%%% Local Variables:
%%% mode: latex
%%% TeX-master: "../mainManuscript"
%%% End:

  \caption{Longitudinal stress $\sigma_{11}$ isovalues in a linear elastic plate.}
  \label{fig:2delast_comparison1}
\end{figure}


\subsection{Two-dimensional elastoplasticity}
Citer le papier sur le immersed boundary de basiliev pour les oscillations sur la MPM

\begin{figure}[h!]
  \centering
  \begin{tikzpicture}
  \begin{groupplot}[group style={group size=3 by 3,
      ylabels at=edge left, yticklabels at=edge left,
      horizontal sep=1.ex,
      vertical sep=5ex,},
    enlargelimits=0,
    xmin=0.,xmax=1., ymin=-0.,ymax=1.
    ,axis on top,scale only axis,xtick=\empty,ytick=\empty,width=0.27\linewidth,
    colorbar style={
      title style={
        font=\scriptsize,
        at={(1,.5)},
        anchor=north west
      },yticklabel style={font=\scriptsize}
      ,at={(current axis.south east)},anchor=south west
    }]
    %% FIRST ROW (first stress sigma11)
    %%% RANGE -4.8e7 -- 1.1e9
    \nextgroupplot[title={(a) FEM}]\addplot graphics[xmin=0.,xmax=1., ymin=-0.,ymax=1.] {chapter4/pngFigures/ep_fem_stress77.png};
    \nextgroupplot[title={(b) DGMPM 1ppc}]\addplot graphics[xmin=-0.,xmax=1., ymin=-0.,ymax=1.] {chapter4/pngFigures/ep_dgmpm1ppc_stress77.png};
    \nextgroupplot[title={(c) DGMPM 4ppc},
    colorbar,colorbar style={
      title= {$\sigma_{11}\: (MPa)$},
      ytick={-0.048,1.1},
      yticklabels={-48,1.1e3},
    }]
    \addplot[scatter,scatter src=y,mark size=0.pt] coordinates {(0.,-0.048) (0.,1.1)};% Fake extreme values to fix scale
    \addplot graphics[xmin=-0.,xmax=1., ymin=-0.,ymax=1.] {chapter4/pngFigures/ep_dgmpm4ppc_stress77.png};
    
    %% SECOND ROW (first plastic strain epsp11)
    %%% RANGE -4.3e-5 -- 3.9e-3
    \nextgroupplot[]\addplot graphics[xmin=0.,xmax=1., ymin=-0.,ymax=1.] {chapter4/pngFigures/ep_fem_epsp77.png};
    \nextgroupplot[]\addplot graphics[xmin=-0.,xmax=1., ymin=-0.,ymax=1.] {chapter4/pngFigures/ep_dgmpm1ppc_epsp77.png};
    \nextgroupplot[colorbar,colorbar style={
      title= {$\eps^p_{11}$},
      ytick={-0.043,3.9},
      yticklabels={-4.3e-5,3.9e-3},
    }]
    \addplot[scatter,scatter src=y,mark size=0.pt] coordinates {(0.,-0.043) (0.,3.9)};% Fake extreme values to fix scale
    \addplot graphics[xmin=-0.,xmax=1., ymin=-0.,ymax=1.] {chapter4/pngFigures/ep_dgmpm4ppc_epsp77.png};
    
    %% THIRD ROW (equivalent plastic strain)
    %%% RANGE 0 -- 4.3e-3
    \nextgroupplot[]\addplot graphics[xmin=0.,xmax=1., ymin=-0.,ymax=1.] {chapter4/pngFigures/ep_fem_p77.png};
    \nextgroupplot[]\addplot graphics[xmin=-0.,xmax=1., ymin=-0.,ymax=1.] {chapter4/pngFigures/ep_dgmpm1ppc_p77.png};
    \nextgroupplot[colorbar,colorbar style={
      title= {$p$},
      ytick={0.,4.3},
      yticklabels={0,4.3e-3},
    }]
    \addplot[scatter,scatter src=y,mark size=0.pt] coordinates {(0.,0.) (0.,4.3)};% Fake extreme values to fix scale
    \addplot graphics[xmin=-0.,xmax=1., ymin=-0.,ymax=1.] {chapter4/pngFigures/ep_dgmpm4ppc_p77.png};
  \end{groupplot}
\end{tikzpicture}



%%% Local Variables:
%%% mode: latex
%%% TeX-master: "../mainManuscript"
%%% End:
  \caption{Longitudinal stress $\sigma_{11}$ isovalues at time $t=3.5e-4s$ in an elastic-plastic plate made of a linear isotropic hardening material}
  \label{fig:2dEP_comparison1}
\end{figure}
% %\input{chapter4/pngFigs}

\begin{figure}[h!]
  \centering
  \begin{tikzpicture}
  \begin{groupplot}[group style={group size=3 by 2,
      ylabels at=edge left, yticklabels at=edge left,
      horizontal sep=1.ex,
      vertical sep=2.ex,},
    enlargelimits=0,
    xmin=0.,xmax=1., ymin=-0.,ymax=1.
    ,axis on top,scale only axis,xtick=\empty,ytick=\empty,width=0.27\linewidth,
    colorbar style={
      title style={
        font=\scriptsize,
        at={(1,.5)},
        anchor=north west
      },yticklabel style={font=\scriptsize}
      ,at={(current axis.south east)},anchor=south west
    }]
    %% FIRST ROW (first stress sigma11)
    %%% RANGE -7.2e7 -- 1.1e9
    \nextgroupplot[title={(a) FEM}]\addplot graphics[xmin=0.,xmax=1., ymin=-0.,ymax=1.] {chapter4/pngFigures/ep_fem_stress338.png};
    \nextgroupplot[title={(b) DGMPM 1ppc}]\addplot graphics[xmin=-0.,xmax=1., ymin=-0.,ymax=1.] {chapter4/pngFigures/ep_dgmpm1ppc_stress338.png};
    \nextgroupplot[title={(c) DGMPM 4ppc},
    colorbar,colorbar style={
      title= {$\sigma_{11}\: (MPa)$},
      ytick={-0.73,8,8.9},
      yticklabels={-73,800,870},
    }]
    \addplot[scatter,scatter src=y,mark size=0.pt] coordinates {(0.,-0.73) (0.,8.9)};% Fake extreme values to fix scale
    \addplot graphics[xmin=-0.,xmax=1., ymin=-0.,ymax=1.] {chapter4/pngFigures/ep_dgmpm4ppc_stress338.png};
    
    %% SECOND ROW (first plastic strain epsp11)
    %%% RANGE -3.2e-5 -- 1.7e-2
    % \nextgroupplot[]\addplot graphics[xmin=0.,xmax=1., ymin=-0.,ymax=1.] {chapter4/pngFigures/ep_fem_epsp338.png};
    % \nextgroupplot[]\addplot graphics[xmin=-0.,xmax=1., ymin=-0.,ymax=1.] {chapter4/pngFigures/ep_dgmpm1ppc_epsp338.png};
    % \nextgroupplot[colorbar,colorbar style={
    %   title= {$\eps^p_{11}\cdot 10^{-2}$},
    %   ytick={-0.0032,1.7},
    %   yticklabels={-3.2e-5,1.7e-2},
    % }]
    % \addplot[scatter,scatter src=y,mark size=0.pt] coordinates {(0.,-0.0032) (0.,1.7)};% Fake extreme values to fix scale
    % \addplot graphics[xmin=-0.,xmax=1., ymin=-0.,ymax=1.] {chapter4/pngFigures/ep_dgmpm4ppc_epsp338.png};
      
    %% THIRD ROW (equivalent plastic strain)
    %%% RANGE 0 -- 2.e-2
    \nextgroupplot[]\addplot graphics[xmin=0.,xmax=1., ymin=-0.,ymax=1.] {chapter4/pngFigures/ep_fem_p338.png};
    \nextgroupplot[]\addplot graphics[xmin=-0.,xmax=1., ymin=-0.,ymax=1.] {chapter4/pngFigures/ep_dgmpm1ppc_p338.png};
    \nextgroupplot[colorbar,colorbar style={
      title= {$p \: (\%)$},
      ytick={0.,2.},
      yticklabels={0,2},
    }]
    \addplot[scatter,scatter src=y,mark size=0.pt] coordinates {(0.,0.) (0.,2.)};% Fake extreme values to fix scale
    \addplot graphics[xmin=-0.,xmax=1., ymin=-0.,ymax=1.] {chapter4/pngFigures/ep_dgmpm4ppc_p338.png};
  \end{groupplot}
\end{tikzpicture}



%%% Local Variables:
%%% mode: latex
%%% TeX-master: "../mainManuscript"
%%% End:
  \caption{Longitudinal stress $\sigma_{11}$ isovalues at time $t=1.0e-3ss$ in an elastic-plastic plate made of a linear isotropic hardening material}
  \label{fig:2dEP_comparison2}
\end{figure}
%\begin{tikzpicture}
  \begin{groupplot}[group style={group size=3 by 2,
      ylabels at=edge left, yticklabels at=edge left,
      horizontal sep=1.ex,
      vertical sep=2.ex,},
    enlargelimits=0,
    xmin=0.,xmax=1., ymin=-0.,ymax=1.
    ,axis on top,scale only axis,xtick=\empty,ytick=\empty,width=0.27\linewidth,
    colorbar style={
      title style={
        font=\scriptsize,
        at={(1,.5)},
        anchor=north west
      },yticklabel style={font=\scriptsize}
      ,at={(current axis.south east)},anchor=south west
    }]
    %% FIRST ROW (first stress sigma11)
    %%% RANGE -7.2e7 -- 1.1e9
    \nextgroupplot[title={(a) FEM}]\addplot graphics[xmin=0.,xmax=1., ymin=-0.,ymax=1.] {chapter4/pngFigures/ep_fem_stress338.png};
    \nextgroupplot[title={(b) DGMPM 1ppc}]\addplot graphics[xmin=-0.,xmax=1., ymin=-0.,ymax=1.] {chapter4/pngFigures/ep_dgmpm1ppc_stress338.png};
    \nextgroupplot[title={(c) DGMPM 4ppc},
    colorbar,colorbar style={
      title= {$\sigma_{11}\: (MPa)$},
      ytick={-0.73,8,8.9},
      yticklabels={-73,800,870},
    }]
    \addplot[scatter,scatter src=y,mark size=0.pt] coordinates {(0.,-0.73) (0.,8.9)};% Fake extreme values to fix scale
    \addplot graphics[xmin=-0.,xmax=1., ymin=-0.,ymax=1.] {chapter4/pngFigures/ep_dgmpm4ppc_stress338.png};
    
    %% SECOND ROW (first plastic strain epsp11)
    %%% RANGE -3.2e-5 -- 1.7e-2
    % \nextgroupplot[]\addplot graphics[xmin=0.,xmax=1., ymin=-0.,ymax=1.] {chapter4/pngFigures/ep_fem_epsp338.png};
    % \nextgroupplot[]\addplot graphics[xmin=-0.,xmax=1., ymin=-0.,ymax=1.] {chapter4/pngFigures/ep_dgmpm1ppc_epsp338.png};
    % \nextgroupplot[colorbar,colorbar style={
    %   title= {$\eps^p_{11}\cdot 10^{-2}$},
    %   ytick={-0.0032,1.7},
    %   yticklabels={-3.2e-5,1.7e-2},
    % }]
    % \addplot[scatter,scatter src=y,mark size=0.pt] coordinates {(0.,-0.0032) (0.,1.7)};% Fake extreme values to fix scale
    % \addplot graphics[xmin=-0.,xmax=1., ymin=-0.,ymax=1.] {chapter4/pngFigures/ep_dgmpm4ppc_epsp338.png};
      
    %% THIRD ROW (equivalent plastic strain)
    %%% RANGE 0 -- 2.e-2
    \nextgroupplot[]\addplot graphics[xmin=0.,xmax=1., ymin=-0.,ymax=1.] {chapter4/pngFigures/ep_fem_p338.png};
    \nextgroupplot[]\addplot graphics[xmin=-0.,xmax=1., ymin=-0.,ymax=1.] {chapter4/pngFigures/ep_dgmpm1ppc_p338.png};
    \nextgroupplot[colorbar,colorbar style={
      title= {$p \: (\%)$},
      ytick={0.,2.},
      yticklabels={0,2},
    }]
    \addplot[scatter,scatter src=y,mark size=0.pt] coordinates {(0.,0.) (0.,2.)};% Fake extreme values to fix scale
    \addplot graphics[xmin=-0.,xmax=1., ymin=-0.,ymax=1.] {chapter4/pngFigures/ep_dgmpm4ppc_p338.png};
  \end{groupplot}
\end{tikzpicture}



%%% Local Variables:
%%% mode: latex
%%% TeX-master: "../mainManuscript"
%%% End:

\begin{figure}[h!]
  \centering
  \input{chapter4/2dEP_mpm}
  \caption{Longitudinal stress $\sigma_{11}$ isovalues at time $t=3.5e-4s$ in an elastic-plastic plate made of a linear isotropic hardening material}
  \label{fig:2dEP_comparison1}
\end{figure}
%%% Local Variables:
%%% mode: latex
%%% TeX-master: "../mainManuscript"
%%% End:

% \section{Hypoelastic solids}
% Run again the test case with an updated CFL if many ppcs. Also look at the mass matrix that is probably zero for first node, the grid must be moved or something like that.

\section{Plane wave problems -- Large strains framework}
\label{sec:1dhe_simulations}
\subsection{Compression impact on a SVK medium}
\subsection{Tensile impact on a SVK medium}
\subsection{Comparison with Neo-Hookean model}

%%% Local Variables:
%%% mode: latex
%%% TeX-master: "../mainManuscript"
%%% End:

\section{Two-dimensional large strains problem}
\label{sec:2dhe_simulations}
\subsection{Compression impact on a SVK medium}
\subsection{Tensile impact on a SVK medium}
\subsection{Comparison with Neo-Hookean model}

%%% Local Variables:
%%% mode: latex
%%% TeX-master: "../mainManuscript"
%%% End:

\section{Conclusion}
%%% Local Variables:
%%% mode: latex
%%% TeX-master: "../mainManuscript"
%%% End:
