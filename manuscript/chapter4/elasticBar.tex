To begin with, let us focus on the problem that has been used in section \ref{sec:MPM} to illustrate shortcomings of the MPM and that motivated the derivation of the DGMPM.
We thus consider an infinite medium in directions $\vect{e}_2$ and $\vect{e}_3$ and of length $L=6\:m$ in direction $\vect{e}_1$. The Cauchy stress and infinitesimal strain tensor are of the form:
\begin{align*}
  & \tens{\sigma} = \sigma \: \vect{e}_1\otimes \vect{e}_1 \\
  & \tens{\eps} = \eps \: \vect{e}_1\otimes \vect{e}_1
\end{align*}
so that the bar assumption holds. Riemann-type initial conditions on the horizontal velocity $v=\vect{v}\cdot\vect{e}_1$ are prescribed in the bar, that is, $v=v_0>0$ for $x_1\in\[0,L/2\]$ and $v=-v_0$ for $x_1\in \:]L/2,L]$. In addition, both ends of the domain are traction free. The bar is assumed elastic with density $\rho=7800 \: kg.m^{-3}$, Poisson ration $\nu=0.3$ and Young's modulus $E=2\times 10^{11}\:Pa$.
The exact solution of this problem \cite[Ch.1]{Wang} has been recalled in section \ref{subsec:charac_Linear_problems} and consists of two elastic discontinuities propagating left and rightward in the bar at constant speeds $c=\pm\sqrt{E/\rho}$. 

The domain is discretized with material points in a grid made of $50$ regular cells which contain either one centered particle or two material points symmetrically placed with respect to element centers. The problem is solved with the DGMPM coupled with both Euler and RK2 time integration, and MPM formulations. Since more numerical noise appears in the USF formulation of the MPM (see section \ref{sec:MPM}), only the USL implementation is used. In addition, the projection of the updated velocity from nodes to particles is made by means of both PIC and FLIP back mappings for the 1 Particle Per Cell (ppc) discretization, and of FLIP projection only for the 2ppc discretization. The Courant number is set to $1/2$ for MPM implementations while that of DGMPM schemes satisfy the stability condition \eqref{eq:stability} derived in section \ref{sec:DGMPM_analysis}.

Figure \ref{fig:elastic_stress} numerical solutions at two different times in terms of stress and velocity, compared to exact ones. 
\begin{figure}[h!]
  \centering
  {\phantomsubcaption \label{subfig:rp_elastic1}}
  {\phantomsubcaption \label{subfig:rp_elastic2}}
  % \begin{tikzpicture}[scale=0.9]
  \begin{groupplot}[group style={group size=2 by 1,
      ylabels at=edge left, yticklabels at=edge left,
      horizontal sep=4.ex,
      vertical sep=2ex,},
    ymajorgrids=true,xmajorgrids=true,
    enlargelimits=0,
    xmin=0.,xmax=6.,
    ylabel=$\sigma (Pa)$,
    xlabel=$x (m)$,
    axis on top,scale only axis,width=0.45\linewidth%,xtick=\empty,ytick=\empty,width=0.25\linewidth,
    ]
    %% FIRST PLOT
    \nextgroupplot[title={(a) time $t = 1.21\times 10^{-4} $ s.}]
    \addplot[Red,very thick,mark=none,dashed,mark size=3pt] coordinates {(0.0,-1.7182156787438973e-08) (0.12244897959183673,4.295539196859743e-08) (0.24489795918367346,1.718215678743897e-08) (0.36734693877551017,8.591078393719486e-09) (0.4897959183673469,-1.7182156787438973e-08) (0.6122448979591837,-3.4364313574877946e-08) (0.7346938775510203,-2.5773235181158458e-08) (0.8571428571428571,0.0) (0.9795918367346939,0.0) (1.1020408163265305,-1.7182156787438973e-08) (1.2244897959183674,1.7182156787438973e-08) (1.346938775510204,0.0) (1.4693877551020407,0.0) (1.5918367346938775,0.0) (1.7142857142857142,0.0) (1.836734693877551,-10490.417480468344) (1.9591836734693877,-180244.44580078288) (2.0816326530612246,-1426887.5122070103) (2.204081632653061,-6905937.1948242355) (2.326530612244898,-22447967.529296968) (2.4489795918367347,-53339385.9863282) (2.571428571428571,-87870025.6347658) (2.693877551020408,-120363616.94335955) (2.816326530612245,-112137222.29003876) (2.9387755102040813,-95318222.04589848) (3.061224489795918,-95318222.04589835) (3.183673469387755,-112137222.29003933) (3.306122448979592,-120363616.94335933) (3.4285714285714284,-87870025.63476545) (3.5510204081632653,-53339385.98632792) (3.673469387755102,-22447967.529296756) (3.7959183673469385,-6905937.194824192) (3.9183673469387754,-1426887.512207019) (4.040816326530612,-180244.44580077427) (4.163265306122449,-10490.417480468344) (4.285714285714286,0.0) (4.408163265306122,0.0) (4.530612244897959,0.0) (4.653061224489796,0.0) (4.775510204081632,1.7182156787438973e-08) (4.8979591836734695,-1.7182156787438973e-08) (5.020408163265306,0.0) (5.142857142857142,0.0) (5.26530612244898,0.0) (5.387755102040816,0.0) (5.5102040816326525,0.0) (5.63265306122449,0.0) (5.755102040816326,0.0) (5.877551020408163,0.0) (6.0,0.0) };
\addplot[Duck,thin,mark=*,solid,mark size=3pt] coordinates {(0.0,8.591078393719486e-09) (0.12244897959183673,7.731970554347537e-08) (0.24489795918367346,9.450186233091434e-08) (0.36734693877551017,2.5773235181158448e-08) (0.4897959183673469,-2.5773235181158458e-08) (0.6122448979591837,-7.731970554347535e-08) (0.7346938775510203,-9.450186233091434e-08) (0.8571428571428571,-5.1546470362316915e-08) (0.9795918367346939,-2.5773235181158458e-08) (1.1020408163265305,-1.7182156787438973e-08) (1.2244897959183674,1.7182156787438973e-08) (1.346938775510204,0.0) (1.4693877551020407,0.0) (1.5918367346938775,0.0) (1.7142857142857142,0.0) (1.836734693877551,-97656.24999999622) (1.9591836734693877,-1074218.75000001) (2.0816326530612246,-5468750.0000000205) (2.204081632653061,-17187500.00000005) (2.326530612244898,-37695312.50000009) (2.4489795918367347,-62304687.500000015) (2.571428571428571,-82812500.00000007) (2.693877551020408,-94531250.00000004) (2.816326530612245,-98925781.24999997) (2.9387755102040813,-99902343.74999999) (3.061224489795918,-99902343.75000001) (3.183673469387755,-98925781.24999999) (3.306122448979592,-94531250.0) (3.4285714285714284,-82812499.99999993) (3.5510204081632653,-62304687.49999988) (3.673469387755102,-37695312.499999896) (3.7959183673469385,-17187499.999999933) (3.9183673469387754,-5468749.99999996) (4.040816326530612,-1074218.7500000014) (4.163265306122449,-97656.24999999622) (4.285714285714286,0.0) (4.408163265306122,0.0) (4.530612244897959,0.0) (4.653061224489796,0.0) (4.775510204081632,1.7182156787438973e-08) (4.8979591836734695,-1.7182156787438973e-08) (5.020408163265306,0.0) (5.142857142857142,0.0) (5.26530612244898,0.0) (5.387755102040816,0.0) (5.5102040816326525,0.0) (5.63265306122449,0.0) (5.755102040816326,0.0) (5.877551020408163,0.0) (6.0,0.0) };
\addplot[Orange,very thick,mark=none,densely dotted,mark size=3pt] coordinates {(0.0,0.0) (0.06060606060606061,0.0) (0.12121212121212122,0.0) (0.18181818181818182,0.0) (0.24242424242424243,0.0) (0.30303030303030304,0.0) (0.36363636363636365,0.0) (0.42424242424242425,0.0) (0.48484848484848486,0.0) (0.5454545454545454,0.0) (0.6060606060606061,8.504299824085957e-09) (0.6666666666666667,8.504299824085957e-09) (0.7272727272727273,-1.7008599648171914e-08) (0.7878787878787878,-1.7008599648171914e-08) (0.8484848484848485,0.0) (0.9090909090909092,0.0) (0.9696969696969697,-2.551289947225787e-08) (1.0303030303030303,-2.551289947225787e-08) (1.0909090909090908,0.0) (1.1515151515151516,0.0) (1.2121212121212122,1.7008599648171914e-08) (1.2727272727272727,1.7008599648171914e-08) (1.3333333333333335,-8.504299824085957e-09) (1.393939393939394,-8.504299824085957e-09) (1.4545454545454546,-8.504299824085957e-09) (1.5151515151515151,-8.504299824085957e-09) (1.5757575757575757,8.504299824085957e-09) (1.6363636363636365,8.504299824085957e-09) (1.696969696969697,1.7008599648171914e-08) (1.7575757575757576,1.7008599648171914e-08) (1.8181818181818183,-5499.273538584786) (1.878787878787879,-5499.273538584786) (1.9393939393939394,-115492.194890958) (2.0,-115492.194890958) (2.0606060606060606,-1086898.8931179084) (2.121212121212121,-1086898.8931179084) (2.1818181818181817,-6038592.01073648) (2.2424242424242427,-6038592.01073648) (2.303030303030303,-21882501.244545024) (2.3636363636363638,-21882501.244545024) (2.4242424242424243,-54484376.31130224) (2.484848484848485,-54484376.31130224) (2.5454545454545454,-94001361.72771455) (2.606060606060606,-94001361.72771455) (2.666666666666667,-117371180.65357204) (2.7272727272727275,-117371180.65357204) (2.787878787878788,-111648218.33372112) (2.8484848484848486,-111648218.33372112) (2.909090909090909,-93365879.35686114) (2.9696969696969697,-93365879.35686114) (3.0303030303030303,-93365879.35686119) (3.090909090909091,-93365879.35686119) (3.1515151515151514,-111648218.33372112) (3.2121212121212124,-111648218.33372112) (3.272727272727273,-117371180.6535721) (3.3333333333333335,-117371180.6535721) (3.393939393939394,-94001361.72771455) (3.4545454545454546,-94001361.72771455) (3.515151515151515,-54484376.3113022) (3.5757575757575757,-54484376.3113022) (3.6363636363636367,-21882501.24454498) (3.6969696969696972,-21882501.24454498) (3.757575757575758,-6038592.01073648) (3.8181818181818183,-6038592.01073648) (3.878787878787879,-1086898.8931178998) (3.9393939393939394,-1086898.8931178998) (4.0,-115492.19489097499) (4.0606060606060606,-115492.19489097499) (4.121212121212121,-5499.273538601796) (4.181818181818182,-5499.273538601796) (4.242424242424242,-8.504299824085957e-09) (4.303030303030303,-8.504299824085957e-09) (4.363636363636363,1.7008599648171914e-08) (4.424242424242425,1.7008599648171914e-08) (4.484848484848485,-8.504299824085957e-09) (4.545454545454546,-8.504299824085957e-09) (4.606060606060606,0.0) (4.666666666666667,0.0) (4.7272727272727275,0.0) (4.787878787878788,0.0) (4.848484848484849,0.0) (4.909090909090909,0.0) (4.96969696969697,0.0) (5.03030303030303,0.0) (5.090909090909091,0.0) (5.151515151515151,0.0) (5.212121212121212,0.0) (5.2727272727272725,0.0) (5.333333333333334,0.0) (5.3939393939393945,0.0) (5.454545454545455,0.0) (5.515151515151516,0.0) (5.575757575757576,0.0) (5.636363636363637,0.0) (5.696969696969697,0.0) (5.757575757575758,0.0) (5.818181818181818,0.0) (5.878787878787879,0.0) (5.9393939393939394,0.0) (6.0,0.0) };
\addplot[Blue,very thick,mark=none,solid,mark size=3pt] coordinates {(0.0,2.205276437763443e-23) (0.12244897959183673,-4.821391029817259e-08) (0.24489795918367346,-3.616043272362948e-08) (0.36734693877551017,2.4106955149086293e-08) (0.4897959183673469,-3.616043272362948e-08) (0.6122448979591837,-2.4106955149086323e-08) (0.7346938775510203,1.2053477574543153e-08) (0.8571428571428571,-1.205347757454316e-08) (0.9795918367346939,1.205347757454315e-08) (1.1020408163265305,1.2053477574543135e-08) (1.2244897959183674,2.4106955149086313e-08) (1.346938775510204,2.41069551490863e-08) (1.4693877551020407,-6.3007898221812655e-24) (1.5918367346938775,-6.3007898221812655e-24) (1.7142857142857142,1.2053477574543152e-08) (1.836734693877551,-2.4106955149086303e-08) (1.9591836734693877,0.0) (2.0816326530612246,0.0) (2.204081632653061,1.5751974555453164e-23) (2.326530612244898,1.5751974555453164e-23) (2.4489795918367347,-99999999.99999999) (2.571428571428571,-100000000.00000003) (2.693877551020408,-100000000.0) (2.816326530612245,-99999999.99999999) (2.9387755102040813,-100000000.0) (3.061224489795918,-99999999.99999997) (3.183673469387755,-100000000.0) (3.306122448979592,-99999999.99999999) (3.4285714285714284,-99999999.99999999) (3.5510204081632653,-99999999.99999999) (3.673469387755102,-1.2601579644362531e-23) (3.7959183673469385,1.2053477574543145e-08) (3.9183673469387754,1.2053477574543147e-08) (4.040816326530612,-3.6160432723629465e-08) (4.163265306122449,1.4176777099907847e-23) (4.285714285714286,-1.4176777099907847e-23) (4.408163265306122,-2.410695514908632e-08) (4.530612244897959,-1.2053477574543148e-08) (4.653061224489796,1.2053477574543138e-08) (4.775510204081632,-1.8902369466543796e-23) (4.8979591836734695,-1.2053477574543148e-08) (5.020408163265306,-2.4106955149086316e-08) (5.142857142857142,3.616043272362944e-08) (5.26530612244898,-1.2053477574543148e-08) (5.387755102040816,-1.8902369466543796e-23) (5.5102040816326525,2.410695514908629e-08) (5.63265306122449,-1.8902369466543796e-23) (5.755102040816326,-1.2053477574543173e-08) (5.877551020408163,2.410695514908629e-08) (6.0,-1.5751974555453164e-23) };
\addplot[Purple,very thick,mark=|,solid,mark size=3pt] coordinates {(0.0,1.1954895665669321e-11) (0.06060606060606061,-1.1954895665669323e-11) (0.12121212121212122,-9.040062200539418e-09) (0.18181818181818182,-1.5066892948546882e-08) (0.24242424242424243,-2.0148953581298875e-08) (0.30303030303030304,-4.011843429141689e-08) (0.36363636363636365,-3.9801572080816714e-08) (0.42424242424242425,-4.457277094098534e-08) (0.48484848484848486,-3.051726216575409e-08) (0.5454545454545454,-2.975012570696168e-08) (0.6060606060606061,-1.895467099939049e-08) (0.6666666666666667,-5.152284149695791e-09) (0.7272727272727273,-1.179037790916254e-08) (0.7878787878787878,-1.2316577239923773e-08) (0.8484848484848485,-2.5473583645144106e-08) (0.9090909090909092,1.3666284960577943e-09) (0.9696969696969697,3.645346560854217e-08) (1.0303030303030303,2.381392226417357e-08) (1.0909090909090908,2.4282772581015557e-08) (1.1515151515151516,2.3931137717157056e-08) (1.2121212121212122,1.2243410979431472e-08) (1.2727272727272727,1.1863544169654833e-08) (1.3333333333333335,1.2583171413268185e-08) (1.393939393939394,1.1523783735818112e-08) (1.4545454545454546,2.7371193430229355e-08) (1.5151515151515151,2.0842716867943245e-08) (1.5757575757575757,9.624966966074275e-09) (1.6363636363636365,1.448198818301203e-08) (1.696969696969697,2.104809967646104e-08) (1.7575757575757576,1.511233304716842e-08) (1.8181818181818183,-3666.244447226061) (1.878787878787879,-10998.733341690284) (1.9393939393939394,-100618.04205178331) (2.0,-262747.51871824573) (2.0606060606060606,-1140066.2362575496) (2.121212121212121,-2551162.9879474747) (2.1818181818181817,-6952185.183763518) (2.2424242424242427,-13110550.493001966) (2.303030303030303,-25132333.114743244) (2.3636363636363638,-39370855.31651974) (2.4242424242424243,-56894854.8287153) (2.484848484848485,-73865127.93600555) (2.5454545454545454,-86159824.57995412) (2.606060606060606,-95498106.62865634) (2.666666666666667,-98513982.44500154) (2.7272727272727275,-100261419.26646227) (2.787878787878788,-100094961.18873355) (2.8484848484848486,-100070640.63102004) (2.909090909090909,-100007508.13633199) (2.9696969696969697,-99998390.4883265) (3.0303030303030303,-99998390.48832652) (3.090909090909091,-100007508.136332) (3.1515151515151514,-100070640.63102002) (3.2121212121212124,-100094961.18873355) (3.272727272727273,-100261419.2664623) (3.3333333333333335,-98513982.44500159) (3.393939393939394,-95498106.62865636) (3.4545454545454546,-86159824.57995415) (3.515151515151515,-73865127.93600555) (3.5757575757575757,-56894854.8287153) (3.6363636363636367,-39370855.31651971) (3.6969696969696972,-25132333.11474323) (3.757575757575758,-13110550.493001949) (3.8181818181818183,-6952185.183763516) (3.878787878787879,-2551162.987947458) (3.9393939393939394,-1140066.2362575415) (4.0,-262747.51871824026) (4.0606060606060606,-100618.04205180079) (4.121212121212121,-10998.73334169629) (4.181818181818182,-3666.2444472321054) (4.242424242424242,6.026554865799796e-09) (4.303030303030303,6.026922708743356e-09) (4.363636363636363,2.0086063933044116e-09) (4.424242424242425,-1.4062083967847561e-08) (4.484848484848485,0.0) (4.545454545454546,0.0) (4.606060606060606,0.0) (4.666666666666667,0.0) (4.7272727272727275,0.0) (4.787878787878788,0.0) (4.848484848484849,0.0) (4.909090909090909,0.0) (4.96969696969697,0.0) (5.03030303030303,0.0) (5.090909090909091,0.0) (5.151515151515151,0.0) (5.212121212121212,0.0) (5.2727272727272725,0.0) (5.333333333333334,0.0) (5.3939393939393945,0.0) (5.454545454545455,0.0) (5.515151515151516,0.0) (5.575757575757576,0.0) (5.636363636363637,0.0) (5.696969696969697,0.0) (5.757575757575758,0.0) (5.818181818181818,0.0) (5.878787878787879,0.0) (5.9393939393939394,-6.0267272921795895e-09) (6.0,-6.026750282363563e-09) };
\addplot[Green,thick,mark=x,only marks,mark size=3pt] coordinates {(0.0,-2.1093585755450518e-08) (0.06060606060606061,-3.0133693936357875e-09) (0.12121212121212122,-2.504863308459749e-08) (0.18181818181818182,-2.316527721357512e-08) (0.24242424242424243,-1.40780851358922e-08) (0.30303030303030304,-1.0028870013194108e-08) (0.36363636363636365,-4.3717398156106705e-08) (0.42424242424242425,-2.8603467291152215e-08) (0.48484848484848486,7.156752309884969e-09) (0.5454545454545454,1.6950202839201318e-08) (0.6060606060606061,-4.3693856207718916e-08) (0.6666666666666667,-5.2733964388626276e-08) (0.7272727272727273,3.3523734504198126e-08) (0.7878787878787878,3.87971309430608e-08) (0.8484848484848485,-1.4878511381076717e-08) (0.9090909090909092,-9.228443768009598e-09) (0.9696969696969697,5.57002498854865e-08) (1.0303030303030303,8.894148100903133e-08) (1.0909090909090908,-5.925508409204129e-08) (1.1515151515151516,-3.7172736504303915e-08) (1.2121212121212122,2.895659651696907e-09) (1.2727272727272727,2.1211295497389412e-08) (1.3333333333333335,3.926796991081633e-08) (1.393939393939394,8.945940387356272e-09) (1.4545454545454546,-2.0104823923163776e-08) (1.5151515151515151,-2.810908637500884e-08) (1.5757575757575757,5.155686696923746e-09) (1.6363636363636365,-5.155686696923745e-09) (1.696969696969697,1.909252014248924e-08) (1.7575757575757576,5.014435006597071e-09) (1.8181818181818183,-1.0546792877725262e-08) (1.878787878787879,-1.3560162271361046e-08) (1.9393939393939394,-4.8967252646581575e-09) (2.0,-1.9210229884428146e-08) (2.0606060606060606,6.191532425986014e-09) (2.121212121212121,-6.191532425986031e-09) (2.1818181818181817,-1.7774171032773557e-08) (2.2424242424242427,-3.043973926539904e-08) (2.303030303030303,1.7609377394059133e-08) (2.3636363636363638,6.4975777550271785e-09) (2.4242424242424243,-99999999.99999994) (2.484848484848485,-100000000.00000006) (2.5454545454545454,-99999999.99999997) (2.606060606060606,-99999999.99999996) (2.666666666666667,-99999999.99999994) (2.7272727272727275,-99999999.99999994) (2.787878787878788,-100000000.00000001) (2.8484848484848486,-100000000.00000001) (2.909090909090909,-99999999.99999997) (2.9696969696969697,-99999999.99999999) (3.0303030303030303,-99999999.99999996) (3.090909090909091,-99999999.99999997) (3.1515151515151514,-99999999.99999996) (3.2121212121212124,-99999999.99999997) (3.272727272727273,-99999999.99999996) (3.3333333333333335,-99999999.99999999) (3.393939393939394,-99999999.99999999) (3.4545454545454546,-100000000.0) (3.515151515151515,-99999999.99999993) (3.5757575757575757,-100000000.00000009) (3.6363636363636367,0.0) (3.6969696969696972,0.0) (3.757575757575758,0.0) (3.8181818181818183,0.0) (3.878787878787879,0.0) (3.9393939393939394,0.0) (4.0,-6.026738787271577e-09) (4.0606060606060606,-1.8080216361814732e-08) (4.121212121212121,1.8080216361814716e-08) (4.181818181818182,6.0267387872715865e-09) (4.242424242424242,-1.6479363871445125e-10) (4.303030303030303,-2.3942161510371846e-08) (4.363636363636363,2.2105889536125074e-08) (4.424242424242425,2.610802076204755e-08) (4.484848484848485,-1.101763184548084e-08) (4.545454545454546,-1.308932330360547e-08) (4.606060606060606,2.4106955149086287e-08) (4.666666666666667,2.4106955149086323e-08) (4.7272727272727275,-2.4106955149086296e-08) (4.787878787878788,-2.410695514908632e-08) (4.848484848484849,0.0) (4.909090909090909,0.0) (4.96969696969697,0.0) (5.03030303030303,0.0) (5.090909090909091,0.0) (5.151515151515151,0.0) (5.212121212121212,0.0) (5.2727272727272725,0.0) (5.333333333333334,0.0) (5.3939393939393945,0.0) (5.454545454545455,0.0) (5.515151515151516,0.0) (5.575757575757576,0.0) (5.636363636363637,0.0) (5.696969696969697,0.0) (5.757575757575758,0.0) (5.818181818181818,0.0) (5.878787878787879,0.0) (5.9393939393939394,0.0) (6.0,0.0) };
\addplot[black,thin,mark=none,solid,mark size=3pt] coordinates {(0.0,-0.0) (0.12244897959183673,-0.0) (0.24489795918367346,-0.0) (0.36734693877551017,-0.0) (0.4897959183673469,-0.0) (0.6122448979591837,-0.0) (0.7346938775510203,-0.0) (0.8571428571428571,-0.0) (0.9795918367346939,-0.0) (1.1020408163265305,-0.0) (1.2244897959183674,-0.0) (1.346938775510204,-0.0) (1.4693877551020407,-0.0) (1.5918367346938775,-0.0) (1.7142857142857142,-0.0) (1.836734693877551,-0.0) (1.9591836734693877,-0.0) (2.0816326530612246,-0.0) (2.204081632653061,-0.0) (2.326530612244898,-0.0) (2.4489795918367347,-100000000.0) (2.571428571428571,-100000000.0) (2.693877551020408,-100000000.0) (2.816326530612245,-100000000.0) (2.9387755102040813,-100000000.0) (3.061224489795918,-100000000.0) (3.183673469387755,-100000000.0) (3.306122448979592,-100000000.0) (3.4285714285714284,-100000000.0) (3.5510204081632653,-100000000.0) (3.673469387755102,-0.0) (3.7959183673469385,-0.0) (3.9183673469387754,-0.0) (4.040816326530612,-0.0) (4.163265306122449,-0.0) (4.285714285714286,-0.0) (4.408163265306122,-0.0) (4.530612244897959,-0.0) (4.653061224489796,-0.0) (4.775510204081632,-0.0) (4.8979591836734695,-0.0) (5.020408163265306,-0.0) (5.142857142857142,-0.0) (5.26530612244898,-0.0) (5.387755102040816,-0.0) (5.5102040816326525,-0.0) (5.63265306122449,-0.0) (5.755102040816326,-0.0) (5.877551020408163,-0.0) (6.0,-0.0) };
    %% SECOND PLOT
    \nextgroupplot[title={(b) time $t = 4.84\times 10^{-4} $ s.},legend style={at={($(0.62,-0.35)+(0.9cm,1cm)$)},legend columns=4}]
    \addplot[Red,very thick,mark=none,dashed,mark size=3pt] coordinates {(0.0,-745947.6317356245) (0.12244897959183673,-2825505.704681147) (0.24489795918367346,-6801870.379054734) (0.36734693877551017,-14249320.469123462) (0.4897959183673469,-26743878.38137685) (0.6122448979591837,-45064188.07482756) (0.7346938775510203,-68064481.52123177) (0.8571428571428571,-91948308.59367745) (0.9795918367346939,-110948018.79598898) (1.1020408163265305,-119819288.51732583) (1.2244897959183674,-117150564.11810811) (1.346938775510204,-106891678.01826178) (1.4693877551020407,-96652794.70257169) (1.5918367346938775,-92267743.75879696) (1.7142857142857142,-95070577.34692116) (1.836734693877551,-99847820.1972167) (1.9591836734693877,-103232485.94759263) (2.0816326530612246,-101626491.39890483) (2.204081632653061,-100355242.65368447) (2.326530612244898,-98244216.75952996) (2.4489795918367347,-100067772.41751443) (2.571428571428571,-99702418.44172558) (2.693877551020408,-100893884.04862353) (2.816326530612245,-99751345.3825965) (2.9387755102040813,-99889896.90888087) (3.061224489795918,-99889896.90888077) (3.183673469387755,-99751345.38259669) (3.306122448979592,-100893884.04862344) (3.4285714285714284,-99702418.44172575) (3.5510204081632653,-100067772.41751438) (3.673469387755102,-98244216.75953004) (3.7959183673469385,-100355242.65368438) (3.9183673469387754,-101626491.39890482) (4.040816326530612,-103232485.94759259) (4.163265306122449,-99847820.1972167) (4.285714285714286,-95070577.34692109) (4.408163265306122,-92267743.7587969) (4.530612244897959,-96652794.70257166) (4.653061224489796,-106891678.01826191) (4.775510204081632,-117150564.11810826) (4.8979591836734695,-119819288.51732588) (5.020408163265306,-110948018.79598887) (5.142857142857142,-91948308.59367728) (5.26530612244898,-68064481.52123177) (5.387755102040816,-45064188.074827455) (5.5102040816326525,-26743878.38137681) (5.63265306122449,-14249320.469123438) (5.755102040816326,-6801870.379054725) (5.877551020408163,-2825505.7046811893) (6.0,-745947.6317356417) };
\addplot[Duck,thin,mark=*,solid,mark size=3pt] coordinates {(0.0,-3658473.820541984) (0.12244897959183673,-11485497.138528435) (0.24489795918367346,-20650075.223420583) (0.36734693877551017,-31470071.326657515) (0.4897959183673469,-43620393.930359595) (0.6122448979591837,-56234556.96822935) (0.7346938775510203,-68199491.57707022) (0.8571428571428571,-78518360.75669788) (0.9795918367346939,-86590219.21803293) (1.1020408163265305,-92306933.66125712) (1.2244897959183674,-95965467.32727806) (1.346938775510204,-98076133.61253974) (1.4693877551020407,-99170549.7565818) (1.5918367346938775,-99678671.19148654) (1.7142857142857142,-99888928.31124762) (1.836734693877551,-99966022.58725037) (1.9591836734693877,-99990891.70851223) (2.0816326530612246,-99997886.14886712) (2.204081632653061,-99999581.77077132) (2.326530612244898,-99999930.86939865) (2.4489795918367347,-99999990.71487762) (2.571428571428571,-99999999.0267497) (2.693877551020408,-99999999.92533047) (2.816326530612245,-99999999.99627104) (2.9387755102040813,-99999999.99990901) (3.061224489795918,-99999999.99990901) (3.183673469387755,-99999999.99627106) (3.306122448979592,-99999999.9253305) (3.4285714285714284,-99999999.02674973) (3.5510204081632653,-99999990.71487764) (3.673469387755102,-99999930.86939867) (3.7959183673469385,-99999581.77077134) (3.9183673469387754,-99997886.14886712) (4.040816326530612,-99990891.70851222) (4.163265306122449,-99966022.58725034) (4.285714285714286,-99888928.31124759) (4.408163265306122,-99678671.19148651) (4.530612244897959,-99170549.75658175) (4.653061224489796,-98076133.61253972) (4.775510204081632,-95965467.32727805) (4.8979591836734695,-92306933.66125712) (5.020408163265306,-86590219.21803287) (5.142857142857142,-78518360.7566978) (5.26530612244898,-68199491.5770702) (5.387755102040816,-56234556.968229264) (5.5102040816326525,-43620393.930359535) (5.63265306122449,-31470071.32665747) (5.755102040816326,-20650075.223420557) (5.877551020408163,-11485497.138528453) (6.0,-3658473.8205419495) };
\addplot[Orange,very thick,mark=none,densely dotted,mark size=3pt] coordinates {(0.0,-569245.6271305095) (0.06060606060606061,-569245.6271305095) (0.12121212121212122,-2277785.4230159773) (0.18181818181818182,-2277785.4230159773) (0.24242424242424243,-5901161.70250096) (0.30303030303030304,-5901161.70250096) (0.36363636363636365,-13233001.909618562) (0.42424242424242425,-13233001.909618562) (0.48484848484848486,-26238300.23104623) (0.5454545454545454,-26238300.23104623) (0.6060606060606061,-45981632.22002476) (0.6666666666666667,-45981632.22002476) (0.7272727272727273,-70997511.41827326) (0.7878787878787878,-70997511.41827326) (0.8484848484848485,-96258666.79270092) (0.9090909090909092,-96258666.79270092) (0.9696969696969697,-114407544.96327573) (1.0303030303030303,-114407544.96327573) (1.0909090909090908,-119769201.2337222) (1.1515151515151516,-119769201.2337222) (1.2121212121212122,-112838974.87666531) (1.2727272727272727,-112838974.87666531) (1.3333333333333335,-100993168.67842056) (1.393939393939394,-100993168.67842056) (1.4545454545454546,-93414330.53328812) (1.5151515151515151,-93414330.53328812) (1.5757575757575757,-93992004.01740366) (1.6363636363636365,-93992004.01740366) (1.696969696969697,-99087098.74325639) (1.7575757575757576,-99087098.74325639) (1.8181818181818183,-102489655.2966868) (1.878787878787879,-102489655.2966868) (1.9393939393939394,-101890791.3937958) (2.0,-101890791.3937958) (2.0606060606060606,-99782074.12880751) (2.121212121212121,-99782074.12880751) (2.1818181818181817,-99055570.19076754) (2.2424242424242427,-99055570.19076754) (2.303030303030303,-99748095.76886229) (2.3636363636363638,-99748095.76886229) (2.4242424242424243,-100313223.73164806) (2.484848484848485,-100313223.73164806) (2.5454545454545454,-100175907.04998955) (2.606060606060606,-100175907.04998955) (2.666666666666667,-99908114.85126807) (2.7272727272727275,-99908114.85126807) (2.787878787878788,-99919571.77694954) (2.8484848484848486,-99919571.77694954) (2.909090909090909,-100047873.551725) (2.9696969696969697,-100047873.551725) (3.0303030303030303,-100047873.55172497) (3.090909090909091,-100047873.55172497) (3.1515151515151514,-99919571.7769495) (3.2121212121212124,-99919571.7769495) (3.272727272727273,-99908114.851268) (3.3333333333333335,-99908114.851268) (3.393939393939394,-100175907.0499895) (3.4545454545454546,-100175907.0499895) (3.515151515151515,-100313223.73164806) (3.5757575757575757,-100313223.73164806) (3.6363636363636367,-99748095.7688623) (3.6969696969696972,-99748095.7688623) (3.757575757575758,-99055570.19076759) (3.8181818181818183,-99055570.19076759) (3.878787878787879,-99782074.12880753) (3.9393939393939394,-99782074.12880753) (4.0,-101890791.39379583) (4.0606060606060606,-101890791.39379583) (4.121212121212121,-102489655.29668684) (4.181818181818182,-102489655.29668684) (4.242424242424242,-99087098.74325638) (4.303030303030303,-99087098.74325638) (4.363636363636363,-93992004.01740365) (4.424242424242425,-93992004.01740365) (4.484848484848485,-93414330.53328809) (4.545454545454546,-93414330.53328809) (4.606060606060606,-100993168.67842054) (4.666666666666667,-100993168.67842054) (4.7272727272727275,-112838974.87666531) (4.787878787878788,-112838974.87666531) (4.848484848484849,-119769201.2337222) (4.909090909090909,-119769201.2337222) (4.96969696969697,-114407544.96327575) (5.03030303030303,-114407544.96327575) (5.090909090909091,-96258666.79270092) (5.151515151515151,-96258666.79270092) (5.212121212121212,-70997511.41827326) (5.2727272727272725,-70997511.41827326) (5.333333333333334,-45981632.22002478) (5.3939393939393945,-45981632.22002478) (5.454545454545455,-26238300.23104622) (5.515151515151516,-26238300.23104622) (5.575757575757576,-13233001.90961853) (5.636363636363637,-13233001.90961853) (5.696969696969697,-5901161.702500943) (5.757575757575758,-5901161.702500943) (5.818181818181818,-2277785.4230159433) (5.878787878787879,-2277785.4230159433) (5.9393939393939394,-569245.627130518) (6.0,-569245.627130518) };
\addplot[Blue,very thick,mark=none,solid,mark size=3pt] coordinates {(0.0,6.026738787271584e-08) (0.12244897959183673,-1.2053477574543037e-08) (0.24489795918367346,2.410695514908625e-08) (0.36734693877551017,2.4106955149086283e-08) (0.4897959183673469,6.026738787271564e-08) (0.6122448979591837,-99999999.99999984) (0.7346938775510203,-100000000.00000025) (0.8571428571428571,-99999999.9999998) (0.9795918367346939,-100000000.00000009) (1.1020408163265305,-99999999.99999985) (1.2244897959183674,-100000000.0) (1.346938775510204,-99999999.99999991) (1.4693877551020407,-100000000.0) (1.5918367346938775,-99999999.99999979) (1.7142857142857142,-100000000.00000003) (1.836734693877551,-99999999.99999988) (1.9591836734693877,-99999999.99999994) (2.0816326530612246,-99999999.99999994) (2.204081632653061,-99999999.99999999) (2.326530612244898,-99999999.99999991) (2.4489795918367347,-99999999.99999991) (2.571428571428571,-99999999.99999997) (2.693877551020408,-99999999.99999991) (2.816326530612245,-99999999.99999994) (2.9387755102040813,-99999999.99999997) (3.061224489795918,-99999999.99999988) (3.183673469387755,-99999999.99999997) (3.306122448979592,-99999999.99999988) (3.4285714285714284,-100000000.00000003) (3.5510204081632653,-99999999.99999991) (3.673469387755102,-100000000.0) (3.7959183673469385,-100000000.00000003) (3.9183673469387754,-99999999.99999987) (4.040816326530612,-99999999.99999997) (4.163265306122449,-99999999.99999994) (4.285714285714286,-99999999.99999997) (4.408163265306122,-99999999.99999997) (4.530612244897959,-100000000.00000009) (4.653061224489796,-99999999.99999988) (4.775510204081632,-100000000.00000013) (4.8979591836734695,-99999999.99999987) (5.020408163265306,-100000000.00000012) (5.142857142857142,-99999999.99999991) (5.26530612244898,-100000000.00000007) (5.387755102040816,-99999999.99999999) (5.5102040816326525,-2.410695514908632e-08) (5.63265306122449,2.410695514908628e-08) (5.755102040816326,-7.232086544725893e-08) (5.877551020408163,-1.205347757454315e-08) (6.0,-2.4106955149086326e-08) };
\addplot[Purple,very thick,mark=|,solid,mark size=3pt] coordinates {(0.0,-901250.8727450209) (0.06060606060606061,-2481132.001666541) (0.12121212121212122,-4826425.776536643) (0.18181818181818182,-7379655.632326556) (0.24242424242424243,-11256070.988140773) (0.30303030303030304,-15619284.087205578) (0.36363636363636365,-21632650.461414523) (0.42424242424242425,-28194347.738026302) (0.48484848484848486,-36184793.045001954) (0.5454545454545454,-44512745.90296996) (0.6060606060606061,-53388027.81990887) (0.6666666666666667,-62184684.90343942) (0.7272727272727273,-70312839.86926234) (0.7878787878787878,-77956322.11918701) (0.8484848484848485,-83999795.65014975) (0.9090909090909092,-89383013.54381928) (0.9696969696969697,-92952697.00923085) (1.0303030303030303,-95962268.87367085) (1.0909090909090908,-97581224.04592441) (1.1515151515151516,-98874662.75556055) (1.2121212121212122,-99404486.29729274) (1.2727272727272727,-99808341.57283215) (1.3333333333333335,-99915762.38856114) (1.393939393939394,-99996248.63365263) (1.4545454545454546,-100001427.378495) (1.5151515151515151,-100006998.37736051) (1.5757575757575757,-100003237.88768458) (1.6363636363636365,-100001492.41546689) (1.696969696969697,-100000520.10664989) (1.7575757575757576,-100000028.36063254) (1.8181818181818183,-99999995.05295742) (1.878787878787879,-99999970.09524688) (1.9393939393939394,-99999990.32856694) (2.0,-99999997.18169779) (2.0606060606060606,-99999999.42352416) (2.121212121212121,-100000000.25132437) (2.1818181818181817,-100000000.07258461) (2.2424242424242427,-100000000.03564933) (2.303030303030303,-100000000.00601049) (2.3636363636363638,-99999999.99877489) (2.4242424242424243,-99999999.99971391) (2.484848484848485,-99999999.99982226) (2.5454545454545454,-99999999.99998015) (2.606060606060606,-100000000.00000417) (2.666666666666667,-100000000.00000048) (2.7272727272727275,-100000000.0000002) (2.787878787878788,-99999999.99999994) (2.8484848484848486,-99999999.99999994) (2.909090909090909,-99999999.99999994) (2.9696969696969697,-99999999.99999994) (3.0303030303030303,-99999999.99999994) (3.090909090909091,-99999999.99999994) (3.1515151515151514,-99999999.99999996) (3.2121212121212124,-99999999.99999997) (3.272727272727273,-100000000.00000024) (3.3333333333333335,-100000000.00000054) (3.393939393939394,-100000000.0000042) (3.4545454545454546,-99999999.99998017) (3.515151515151515,-99999999.99982227) (3.5757575757575757,-99999999.99971391) (3.6363636363636367,-99999999.99877487) (3.6969696969696972,-100000000.00601052) (3.757575757575758,-100000000.03564937) (3.8181818181818183,-100000000.07258466) (3.878787878787879,-100000000.2513244) (3.9393939393939394,-99999999.42352419) (4.0,-99999997.18169783) (4.0606060606060606,-99999990.328567) (4.121212121212121,-99999970.09524693) (4.181818181818182,-99999995.05295746) (4.242424242424242,-100000028.36063258) (4.303030303030303,-100000520.10664994) (4.363636363636363,-100001492.41546696) (4.424242424242425,-100003237.88768464) (4.484848484848485,-100006998.37736058) (4.545454545454546,-100001427.37849504) (4.606060606060606,-99996248.63365267) (4.666666666666667,-99915762.38856122) (4.7272727272727275,-99808341.57283223) (4.787878787878788,-99404486.29729284) (4.848484848484849,-98874662.75556065) (4.909090909090909,-97581224.04592451) (4.96969696969697,-95962268.87367095) (5.03030303030303,-92952697.00923099) (5.090909090909091,-89383013.5438194) (5.151515151515151,-83999795.65014987) (5.212121212121212,-77956322.11918712) (5.2727272727272725,-70312839.86926243) (5.333333333333334,-62184684.90343948) (5.3939393939393945,-53388027.81990896) (5.454545454545455,-44512745.90297001) (5.515151515151516,-36184793.04500202) (5.575757575757576,-28194347.73802632) (5.636363636363637,-21632650.461414546) (5.696969696969697,-15619284.087205617) (5.757575757575758,-11256070.988140827) (5.818181818181818,-7379655.632326601) (5.878787878787879,-4826425.7765366705) (5.9393939393939394,-2481132.001666579) (6.0,-901250.8727450437) };
\addplot[Green,thick,mark=x,only marks,mark size=3pt] coordinates {(0.0,-6.52473551238276e-08) (0.06060606060606061,-5.528742062160391e-08) (0.12121212121212122,-5.4360491549496524e-08) (0.18181818181818182,-4.206732904684871e-08) (0.24242424242424243,2.6389399231237735e-10) (0.30303030303030304,2.3843061156773917e-08) (0.36363636363636365,1.6444762021031894e-08) (0.42424242424242425,7.662193128054401e-09) (0.48484848484848486,-2.6065356036709213e-08) (0.5454545454545454,-4.625550941054968e-08) (0.6060606060606061,-99999999.99999985) (0.6666666666666667,-99999999.99999993) (0.7272727272727273,-99999999.99999994) (0.7878787878787878,-99999999.99999994) (0.8484848484848485,-99999999.99999994) (0.9090909090909092,-99999999.99999994) (0.9696969696969697,-100000000.0) (1.0303030303030303,-100000000.0) (1.0909090909090908,-99999999.99999988) (1.1515151515151516,-99999999.9999999) (1.2121212121212122,-99999999.99999991) (1.2727272727272727,-99999999.99999993) (1.3333333333333335,-99999999.99999996) (1.393939393939394,-99999999.99999997) (1.4545454545454546,-99999999.99999988) (1.5151515151515151,-99999999.99999988) (1.5757575757575757,-99999999.99999991) (1.6363636363636365,-99999999.99999993) (1.696969696969697,-99999999.99999997) (1.7575757575757576,-99999999.99999999) (1.8181818181818183,-99999999.99999994) (1.878787878787879,-99999999.99999994) (1.9393939393939394,-99999999.99999994) (2.0,-99999999.99999994) (2.0606060606060606,-99999999.99999996) (2.121212121212121,-99999999.99999999) (2.1818181818181817,-99999999.99999996) (2.2424242424242427,-99999999.99999997) (2.303030303030303,-99999999.99999994) (2.3636363636363638,-99999999.99999994) (2.4242424242424243,-99999999.99999994) (2.484848484848485,-99999999.99999994) (2.5454545454545454,-99999999.9999999) (2.606060606060606,-99999999.99999993) (2.666666666666667,-99999999.99999994) (2.7272727272727275,-99999999.99999994) (2.787878787878788,-99999999.99999996) (2.8484848484848486,-99999999.99999999) (2.909090909090909,-99999999.99999991) (2.9696969696969697,-99999999.99999993) (3.0303030303030303,-99999999.99999994) (3.090909090909091,-99999999.99999994) (3.1515151515151514,-99999999.99999994) (3.2121212121212124,-99999999.99999994) (3.272727272727273,-100000000.0) (3.3333333333333335,-100000000.0) (3.393939393939394,-99999999.99999991) (3.4545454545454546,-99999999.99999993) (3.515151515151515,-99999999.99999999) (3.5757575757575757,-100000000.0) (3.6363636363636367,-99999999.99999991) (3.6969696969696972,-99999999.99999993) (3.757575757575758,-99999999.99999993) (3.8181818181818183,-99999999.99999994) (3.878787878787879,-100000000.00000004) (3.9393939393939394,-100000000.00000003) (4.0,-99999999.99999991) (4.0606060606060606,-99999999.99999991) (4.121212121212121,-100000000.00000001) (4.181818181818182,-100000000.00000003) (4.242424242424242,-99999999.99999993) (4.303030303030303,-99999999.99999994) (4.363636363636363,-99999999.99999997) (4.424242424242425,-99999999.99999997) (4.484848484848485,-100000000.00000006) (4.545454545454546,-100000000.00000006) (4.606060606060606,-99999999.99999991) (4.666666666666667,-99999999.99999993) (4.7272727272727275,-99999999.99999996) (4.787878787878788,-99999999.99999997) (4.848484848484849,-99999999.99999996) (4.909090909090909,-99999999.99999999) (4.96969696969697,-99999999.99999997) (5.03030303030303,-99999999.99999996) (5.090909090909091,-99999999.99999997) (5.151515151515151,-99999999.99999999) (5.212121212121212,-99999999.99999996) (5.2727272727272725,-99999999.99999997) (5.333333333333334,-99999999.99999991) (5.3939393939393945,-100000000.00000009) (5.454545454545455,-6.171875818689624e-09) (5.515151515151516,6.171875818689572e-09) (5.575757575757576,-3.413288241873195e-08) (5.636363636363637,-6.229493817761331e-08) (5.696969696969697,2.564453865316311e-08) (5.757575757575758,-1.5375835040768292e-09) (5.818181818181818,-5.420092204753976e-08) (5.878787878787879,-4.222689854880543e-08) (5.9393939393939394,3.843223074305013e-09) (6.0,-3.8432230743049925e-09) };
\addplot[black,thin,mark=none,solid,mark size=3pt] coordinates {(0.0,-0.0) (0.12244897959183673,-0.0) (0.24489795918367346,-0.0) (0.36734693877551017,-0.0) (0.4897959183673469,-0.0) (0.6122448979591837,-100000000.0) (0.7346938775510203,-100000000.0) (0.8571428571428571,-100000000.0) (0.9795918367346939,-100000000.0) (1.1020408163265305,-100000000.0) (1.2244897959183674,-100000000.0) (1.346938775510204,-100000000.0) (1.4693877551020407,-100000000.0) (1.5918367346938775,-100000000.0) (1.7142857142857142,-100000000.0) (1.836734693877551,-100000000.0) (1.9591836734693877,-100000000.0) (2.0816326530612246,-100000000.0) (2.204081632653061,-100000000.0) (2.326530612244898,-100000000.0) (2.4489795918367347,-100000000.0) (2.571428571428571,-100000000.0) (2.693877551020408,-100000000.0) (2.816326530612245,-100000000.0) (2.9387755102040813,-100000000.0) (3.061224489795918,-100000000.0) (3.183673469387755,-100000000.0) (3.306122448979592,-100000000.0) (3.4285714285714284,-100000000.0) (3.5510204081632653,-100000000.0) (3.673469387755102,-100000000.0) (3.7959183673469385,-100000000.0) (3.9183673469387754,-100000000.0) (4.040816326530612,-100000000.0) (4.163265306122449,-100000000.0) (4.285714285714286,-100000000.0) (4.408163265306122,-100000000.0) (4.530612244897959,-100000000.0) (4.653061224489796,-100000000.0) (4.775510204081632,-100000000.0) (4.8979591836734695,-100000000.0) (5.020408163265306,-100000000.0) (5.142857142857142,-100000000.0) (5.26530612244898,-100000000.0) (5.387755102040816,-100000000.0) (5.5102040816326525,-0.0) (5.63265306122449,-0.0) (5.755102040816326,-0.0) (5.877551020408163,-0.0) (6.0,-0.0) };
\addlegendentry{usl 1ppc}
\addlegendentry{usl-pic 1ppc}
\addlegendentry{usl 2ppc}
\addlegendentry{dgmpm 1ppc}
\addlegendentry{dgmpm 2ppc}
\addlegendentry{dgmpm 2ppc (RK2)}
\addlegendentry{exact}
  \end{groupplot}
\end{tikzpicture}
%%% Local Variables:
%%% mode: latex
%%% TeX-master: "../../mainManuscript"
%%% End:



  \begin{tikzpicture}[scale=.9]
\begin{groupplot}[group style={group size=2 by 2,
ylabels at=edge left, yticklabels at=edge left,horizontal sep=4.ex,
vertical sep=2ex,xticklabels at=edge bottom,xlabels at=edge bottom},
ymajorgrids=true,xmajorgrids=true,enlargelimits=0,xmin=0.,xmax=6.,xlabel=x (m),
axis on top,scale only axis,width=0.45\linewidth
]
\nextgroupplot[title={(a) time $t = 1.21\times 10^{-4} $ s.},ylabel=$\sigma (Pa)$,]
\addplot[Red,dashed,mark=none,very thick,mark size=3pt] coordinates{(0.0,-1.71821567874e-08) (0.122448979592,4.29553919686e-08) (0.244897959184,1.71821567874e-08) (0.367346938776,8.59107839372e-09) (0.489795918367,-1.71821567874e-08) (0.612244897959,-3.43643135749e-08) (0.734693877551,-2.57732351812e-08) (0.857142857143,0.0) (0.979591836735,0.0) (1.10204081633,-1.71821567874e-08) (1.22448979592,1.71821567874e-08) (1.34693877551,0.0) (1.4693877551,0.0) (1.59183673469,0.0) (1.71428571429,0.0) (1.83673469388,-10490.4174805) (1.95918367347,-180244.445801) (2.08163265306,-1426887.51221) (2.20408163265,-6905937.19482) (2.32653061224,-22447967.5293) (2.44897959184,-53339385.9863) (2.57142857143,-87870025.6348) (2.69387755102,-120363616.943) (2.81632653061,-112137222.29) (2.9387755102,-95318222.0459) (3.0612244898,-95318222.0459) (3.18367346939,-112137222.29) (3.30612244898,-120363616.943) (3.42857142857,-87870025.6348) (3.55102040816,-53339385.9863) (3.67346938776,-22447967.5293) (3.79591836735,-6905937.19482) (3.91836734694,-1426887.51221) (4.04081632653,-180244.445801) (4.16326530612,-10490.4174805) (4.28571428571,0.0) (4.40816326531,0.0) (4.5306122449,0.0) (4.65306122449,0.0) (4.77551020408,1.71821567874e-08) (4.89795918367,-1.71821567874e-08) (5.02040816327,0.0) (5.14285714286,0.0) (5.26530612245,0.0) (5.38775510204,0.0) (5.51020408163,0.0) (5.63265306122,0.0) (5.75510204082,0.0) (5.87755102041,0.0) (6.0,0.0) };
\addplot[Orange,densely dotted,mark=none,very thick,mark size=3pt] coordinates{(0.0,0.0) (0.0606060606061,0.0) (0.121212121212,0.0) (0.181818181818,0.0) (0.242424242424,0.0) (0.30303030303,0.0) (0.363636363636,0.0) (0.424242424242,0.0) (0.484848484848,0.0) (0.545454545455,0.0) (0.606060606061,-8.50429982409e-09) (0.666666666667,-8.50429982409e-09) (0.727272727273,0.0) (0.787878787879,0.0) (0.848484848485,0.0) (0.909090909091,0.0) (0.969696969697,8.50429982409e-09) (1.0303030303,8.50429982409e-09) (1.09090909091,0.0) (1.15151515152,0.0) (1.21212121212,0.0) (1.27272727273,0.0) (1.33333333333,0.0) (1.39393939394,0.0) (1.45454545455,0.0) (1.51515151515,0.0) (1.57575757576,0.0) (1.63636363636,0.0) (1.69696969697,0.0) (1.75757575758,0.0) (1.81818181818,-5499.27353859) (1.87878787879,-5499.27353859) (1.93939393939,-115492.194891) (2.0,-115492.194891) (2.06060606061,-1086898.89312) (2.12121212121,-1086898.89312) (2.18181818182,-6038592.01074) (2.24242424242,-6038592.01074) (2.30303030303,-21882501.2445) (2.36363636364,-21882501.2445) (2.42424242424,-54484376.3113) (2.48484848485,-54484376.3113) (2.54545454545,-94001361.7277) (2.60606060606,-94001361.7277) (2.66666666667,-117371180.654) (2.72727272727,-117371180.654) (2.78787878788,-111648218.334) (2.84848484848,-111648218.334) (2.90909090909,-93365879.3569) (2.9696969697,-93365879.3569) (3.0303030303,-93365879.3569) (3.09090909091,-93365879.3569) (3.15151515152,-111648218.334) (3.21212121212,-111648218.334) (3.27272727273,-117371180.654) (3.33333333333,-117371180.654) (3.39393939394,-94001361.7277) (3.45454545455,-94001361.7277) (3.51515151515,-54484376.3113) (3.57575757576,-54484376.3113) (3.63636363636,-21882501.2445) (3.69696969697,-21882501.2445) (3.75757575758,-6038592.01074) (3.81818181818,-6038592.01074) (3.87878787879,-1086898.89312) (3.93939393939,-1086898.89312) (4.0,-115492.194891) (4.06060606061,-115492.194891) (4.12121212121,-5499.2735386) (4.18181818182,-5499.2735386) (4.24242424242,-8.50429982409e-09) (4.30303030303,-8.50429982409e-09) (4.36363636364,1.70085996482e-08) (4.42424242424,1.70085996482e-08) (4.48484848485,-8.50429982409e-09) (4.54545454545,-8.50429982409e-09) (4.60606060606,0.0) (4.66666666667,0.0) (4.72727272727,0.0) (4.78787878788,0.0) (4.84848484848,0.0) (4.90909090909,0.0) (4.9696969697,0.0) (5.0303030303,0.0) (5.09090909091,0.0) (5.15151515152,0.0) (5.21212121212,0.0) (5.27272727273,0.0) (5.33333333333,0.0) (5.39393939394,0.0) (5.45454545455,0.0) (5.51515151515,0.0) (5.57575757576,0.0) (5.63636363636,0.0) (5.69696969697,0.0) (5.75757575758,0.0) (5.81818181818,0.0) (5.87878787879,0.0) (5.93939393939,0.0) (6.0,0.0) };
\addplot[Duck,solid,mark=*,thick,mark size=2pt] coordinates{(0.0,0.0) (0.0606060606061,0.0) (0.121212121212,0.0) (0.181818181818,0.0) (0.242424242424,0.0) (0.30303030303,0.0) (0.363636363636,-2.55128994723e-08) (0.424242424242,-2.55128994723e-08) (0.484848484848,-2.55128994723e-08) (0.545454545455,-2.55128994723e-08) (0.606060606061,-8.50429982409e-09) (0.666666666667,-8.50429982409e-09) (0.727272727273,-1.70085996482e-08) (0.787878787879,-1.70085996482e-08) (0.848484848485,1.70085996482e-08) (0.909090909091,1.70085996482e-08) (0.969696969697,1.70085996482e-08) (1.0303030303,1.70085996482e-08) (1.09090909091,8.50429982409e-09) (1.15151515152,8.50429982409e-09) (1.21212121212,2.55128994723e-08) (1.27272727273,2.55128994723e-08) (1.33333333333,8.50429982409e-09) (1.39393939394,8.50429982409e-09) (1.45454545455,0.0) (1.51515151515,0.0) (1.57575757576,-2.55128994723e-08) (1.63636363636,-2.55128994723e-08) (1.69696969697,-3.40171992963e-08) (1.75757575758,-3.40171992963e-08) (1.81818181818,-29361.1134112) (1.87878787879,-29361.1134112) (1.93939393939,-478765.91052) (2.0,-478765.91052) (2.06060606061,-3385772.94096) (2.12121212121,-3385772.94096) (2.18181818182,-13710104.1379) (2.24242424242,-13710104.1379) (2.30303030303,-35621767.9306) (2.36363636364,-35621767.9306) (2.42424242424,-64051116.4238) (2.48484848485,-64051116.4238) (2.54545454545,-86391746.7796) (2.60606060606,-86391746.7796) (2.66666666667,-96780453.3284) (2.72727272727,-96780453.3284) (2.78787878788,-99574824.0166) (2.84848484848,-99574824.0166) (2.90909090909,-99976087.4183) (2.9696969697,-99976087.4183) (3.0303030303,-99976087.4183) (3.09090909091,-99976087.4183) (3.15151515152,-99574824.0166) (3.21212121212,-99574824.0166) (3.27272727273,-96780453.3284) (3.33333333333,-96780453.3284) (3.39393939394,-86391746.7796) (3.45454545455,-86391746.7796) (3.51515151515,-64051116.4238) (3.57575757576,-64051116.4238) (3.63636363636,-35621767.9306) (3.69696969697,-35621767.9306) (3.75757575758,-13710104.1379) (3.81818181818,-13710104.1379) (3.87878787879,-3385772.94096) (3.93939393939,-3385772.94096) (4.0,-478765.91052) (4.06060606061,-478765.91052) (4.12121212121,-29361.1134111) (4.18181818182,-29361.1134111) (4.24242424242,0.0) (4.30303030303,0.0) (4.36363636364,3.40171992963e-08) (4.42424242424,3.40171992963e-08) (4.48484848485,-8.50429982409e-09) (4.54545454545,-8.50429982409e-09) (4.60606060606,0.0) (4.66666666667,0.0) (4.72727272727,0.0) (4.78787878788,0.0) (4.84848484848,0.0) (4.90909090909,0.0) (4.9696969697,0.0) (5.0303030303,0.0) (5.09090909091,0.0) (5.15151515152,0.0) (5.21212121212,0.0) (5.27272727273,0.0) (5.33333333333,0.0) (5.39393939394,0.0) (5.45454545455,0.0) (5.51515151515,0.0) (5.57575757576,0.0) (5.63636363636,0.0) (5.69696969697,0.0) (5.75757575758,0.0) (5.81818181818,0.0) (5.87878787879,0.0) (5.93939393939,0.0) (6.0,0.0) };
\addplot[Blue,solid,mark=none,very thick,mark size=3pt] coordinates{(0.0,2.20527643776e-23) (0.122448979592,-4.82139102982e-08) (0.244897959184,-3.61604327236e-08) (0.367346938776,2.41069551491e-08) (0.489795918367,-3.61604327236e-08) (0.612244897959,-2.41069551491e-08) (0.734693877551,1.20534775745e-08) (0.857142857143,-1.20534775745e-08) (0.979591836735,1.20534775745e-08) (1.10204081633,1.20534775745e-08) (1.22448979592,2.41069551491e-08) (1.34693877551,2.41069551491e-08) (1.4693877551,-6.30078982218e-24) (1.59183673469,-6.30078982218e-24) (1.71428571429,1.20534775745e-08) (1.83673469388,-2.41069551491e-08) (1.95918367347,0.0) (2.08163265306,0.0) (2.20408163265,1.57519745555e-23) (2.32653061224,1.57519745555e-23) (2.44897959184,-100000000.0) (2.57142857143,-100000000.0) (2.69387755102,-100000000.0) (2.81632653061,-100000000.0) (2.9387755102,-100000000.0) (3.0612244898,-100000000.0) (3.18367346939,-100000000.0) (3.30612244898,-100000000.0) (3.42857142857,-100000000.0) (3.55102040816,-100000000.0) (3.67346938776,-1.26015796444e-23) (3.79591836735,1.20534775745e-08) (3.91836734694,1.20534775745e-08) (4.04081632653,-3.61604327236e-08) (4.16326530612,1.41767770999e-23) (4.28571428571,-1.41767770999e-23) (4.40816326531,-2.41069551491e-08) (4.5306122449,-1.20534775745e-08) (4.65306122449,1.20534775745e-08) (4.77551020408,-1.89023694665e-23) (4.89795918367,-1.20534775745e-08) (5.02040816327,-2.41069551491e-08) (5.14285714286,3.61604327236e-08) (5.26530612245,-1.20534775745e-08) (5.38775510204,-1.89023694665e-23) (5.51020408163,2.41069551491e-08) (5.63265306122,-1.89023694665e-23) (5.75510204082,-1.20534775745e-08) (5.87755102041,2.41069551491e-08) (6.0,-1.57519745555e-23) };
\addplot[Purple,solid,mark=|,very thick,mark size=3pt] coordinates{(0.0,1.19548956657e-11) (0.0606060606061,-1.19548956657e-11) (0.121212121212,-9.04006220054e-09) (0.181818181818,-1.50668929485e-08) (0.242424242424,-2.01489535813e-08) (0.30303030303,-4.01184342914e-08) (0.363636363636,-3.98015720808e-08) (0.424242424242,-4.4572770941e-08) (0.484848484848,-3.05172621658e-08) (0.545454545455,-2.9750125707e-08) (0.606060606061,-1.89546709994e-08) (0.666666666667,-5.1522841497e-09) (0.727272727273,-1.17903779092e-08) (0.787878787879,-1.23165772399e-08) (0.848484848485,-2.54735836451e-08) (0.909090909091,1.36662849606e-09) (0.969696969697,3.64534656085e-08) (1.0303030303,2.38139222642e-08) (1.09090909091,2.4282772581e-08) (1.15151515152,2.39311377172e-08) (1.21212121212,1.22434109794e-08) (1.27272727273,1.18635441697e-08) (1.33333333333,1.25831714133e-08) (1.39393939394,1.15237837358e-08) (1.45454545455,2.73711934302e-08) (1.51515151515,2.08427168679e-08) (1.57575757576,9.62496696607e-09) (1.63636363636,1.4481988183e-08) (1.69696969697,2.10480996765e-08) (1.75757575758,1.51123330472e-08) (1.81818181818,-3666.24444723) (1.87878787879,-10998.7333417) (1.93939393939,-100618.042052) (2.0,-262747.518718) (2.06060606061,-1140066.23626) (2.12121212121,-2551162.98795) (2.18181818182,-6952185.18376) (2.24242424242,-13110550.493) (2.30303030303,-25132333.1147) (2.36363636364,-39370855.3165) (2.42424242424,-56894854.8287) (2.48484848485,-73865127.936) (2.54545454545,-86159824.58) (2.60606060606,-95498106.6287) (2.66666666667,-98513982.445) (2.72727272727,-100261419.266) (2.78787878788,-100094961.189) (2.84848484848,-100070640.631) (2.90909090909,-100007508.136) (2.9696969697,-99998390.4883) (3.0303030303,-99998390.4883) (3.09090909091,-100007508.136) (3.15151515152,-100070640.631) (3.21212121212,-100094961.189) (3.27272727273,-100261419.266) (3.33333333333,-98513982.445) (3.39393939394,-95498106.6287) (3.45454545455,-86159824.58) (3.51515151515,-73865127.936) (3.57575757576,-56894854.8287) (3.63636363636,-39370855.3165) (3.69696969697,-25132333.1147) (3.75757575758,-13110550.493) (3.81818181818,-6952185.18376) (3.87878787879,-2551162.98795) (3.93939393939,-1140066.23626) (4.0,-262747.518718) (4.06060606061,-100618.042052) (4.12121212121,-10998.7333417) (4.18181818182,-3666.24444723) (4.24242424242,6.0265548658e-09) (4.30303030303,6.02692270874e-09) (4.36363636364,2.0086063933e-09) (4.42424242424,-1.40620839678e-08) (4.48484848485,0.0) (4.54545454545,0.0) (4.60606060606,0.0) (4.66666666667,0.0) (4.72727272727,0.0) (4.78787878788,0.0) (4.84848484848,0.0) (4.90909090909,0.0) (4.9696969697,0.0) (5.0303030303,0.0) (5.09090909091,0.0) (5.15151515152,0.0) (5.21212121212,0.0) (5.27272727273,0.0) (5.33333333333,0.0) (5.39393939394,0.0) (5.45454545455,0.0) (5.51515151515,0.0) (5.57575757576,0.0) (5.63636363636,0.0) (5.69696969697,0.0) (5.75757575758,0.0) (5.81818181818,0.0) (5.87878787879,0.0) (5.93939393939,-6.02672729218e-09) (6.0,-6.02675028236e-09) };
\addplot[Green,only marks,mark=x,thick,mark size=3pt] coordinates{(0.0,-2.10935857555e-08) (0.0606060606061,-3.01336939364e-09) (0.121212121212,-2.50486330846e-08) (0.181818181818,-2.31652772136e-08) (0.242424242424,-1.40780851359e-08) (0.30303030303,-1.00288700132e-08) (0.363636363636,-4.37173981561e-08) (0.424242424242,-2.86034672912e-08) (0.484848484848,7.15675230988e-09) (0.545454545455,1.69502028392e-08) (0.606060606061,-4.36938562077e-08) (0.666666666667,-5.27339643886e-08) (0.727272727273,3.35237345042e-08) (0.787878787879,3.87971309431e-08) (0.848484848485,-1.48785113811e-08) (0.909090909091,-9.22844376801e-09) (0.969696969697,5.57002498855e-08) (1.0303030303,8.8941481009e-08) (1.09090909091,-5.9255084092e-08) (1.15151515152,-3.71727365043e-08) (1.21212121212,2.8956596517e-09) (1.27272727273,2.12112954974e-08) (1.33333333333,3.92679699108e-08) (1.39393939394,8.94594038736e-09) (1.45454545455,-2.01048239232e-08) (1.51515151515,-2.8109086375e-08) (1.57575757576,5.15568669692e-09) (1.63636363636,-5.15568669692e-09) (1.69696969697,1.90925201425e-08) (1.75757575758,5.0144350066e-09) (1.81818181818,-1.05467928777e-08) (1.87878787879,-1.35601622714e-08) (1.93939393939,-4.89672526466e-09) (2.0,-1.92102298844e-08) (2.06060606061,6.19153242599e-09) (2.12121212121,-6.19153242599e-09) (2.18181818182,-1.77741710328e-08) (2.24242424242,-3.04397392654e-08) (2.30303030303,1.76093773941e-08) (2.36363636364,6.49757775503e-09) (2.42424242424,-100000000.0) (2.48484848485,-100000000.0) (2.54545454545,-100000000.0) (2.60606060606,-100000000.0) (2.66666666667,-100000000.0) (2.72727272727,-100000000.0) (2.78787878788,-100000000.0) (2.84848484848,-100000000.0) (2.90909090909,-100000000.0) (2.9696969697,-100000000.0) (3.0303030303,-100000000.0) (3.09090909091,-100000000.0) (3.15151515152,-100000000.0) (3.21212121212,-100000000.0) (3.27272727273,-100000000.0) (3.33333333333,-100000000.0) (3.39393939394,-100000000.0) (3.45454545455,-100000000.0) (3.51515151515,-100000000.0) (3.57575757576,-100000000.0) (3.63636363636,0.0) (3.69696969697,0.0) (3.75757575758,0.0) (3.81818181818,0.0) (3.87878787879,0.0) (3.93939393939,0.0) (4.0,-6.02673878727e-09) (4.06060606061,-1.80802163618e-08) (4.12121212121,1.80802163618e-08) (4.18181818182,6.02673878727e-09) (4.24242424242,-1.64793638714e-10) (4.30303030303,-2.39421615104e-08) (4.36363636364,2.21058895361e-08) (4.42424242424,2.6108020762e-08) (4.48484848485,-1.10176318455e-08) (4.54545454545,-1.30893233036e-08) (4.60606060606,2.41069551491e-08) (4.66666666667,2.41069551491e-08) (4.72727272727,-2.41069551491e-08) (4.78787878788,-2.41069551491e-08) (4.84848484848,0.0) (4.90909090909,0.0) (4.9696969697,0.0) (5.0303030303,0.0) (5.09090909091,0.0) (5.15151515152,0.0) (5.21212121212,0.0) (5.27272727273,0.0) (5.33333333333,0.0) (5.39393939394,0.0) (5.45454545455,0.0) (5.51515151515,0.0) (5.57575757576,0.0) (5.63636363636,0.0) (5.69696969697,0.0) (5.75757575758,0.0) (5.81818181818,0.0) (5.87878787879,0.0) (5.93939393939,0.0) (6.0,0.0) };
\addplot[black,solid,mark=pentagone*,thin,mark size=3pt] coordinates{(0.0,-0.0) (0.122448979592,-0.0) (0.244897959184,-0.0) (0.367346938776,-0.0) (0.489795918367,-0.0) (0.612244897959,-0.0) (0.734693877551,-0.0) (0.857142857143,-0.0) (0.979591836735,-0.0) (1.10204081633,-0.0) (1.22448979592,-0.0) (1.34693877551,-0.0) (1.4693877551,-0.0) (1.59183673469,-0.0) (1.71428571429,-0.0) (1.83673469388,-0.0) (1.95918367347,-0.0) (2.08163265306,-0.0) (2.20408163265,-0.0) (2.32653061224,-0.0) (2.44897959184,-100000000.0) (2.57142857143,-100000000.0) (2.69387755102,-100000000.0) (2.81632653061,-100000000.0) (2.9387755102,-100000000.0) (3.0612244898,-100000000.0) (3.18367346939,-100000000.0) (3.30612244898,-100000000.0) (3.42857142857,-100000000.0) (3.55102040816,-100000000.0) (3.67346938776,-0.0) (3.79591836735,-0.0) (3.91836734694,-0.0) (4.04081632653,-0.0) (4.16326530612,-0.0) (4.28571428571,-0.0) (4.40816326531,-0.0) (4.5306122449,-0.0) (4.65306122449,-0.0) (4.77551020408,-0.0) (4.89795918367,-0.0) (5.02040816327,-0.0) (5.14285714286,-0.0) (5.26530612245,-0.0) (5.38775510204,-0.0) (5.51020408163,-0.0) (5.63265306122,-0.0) (5.75510204082,-0.0) (5.87755102041,-0.0) (6.0,-0.0) };
\nextgroupplot[title={(b) time $t = 4.84\times 10^{-4} $ s.},]
\addplot[Red,dashed,mark=none,very thick,mark size=3pt] coordinates{(0.0,-8.59107839372e-09) (0.122448979592,4.29553919686e-08) (0.244897959184,2.57732351812e-08) (0.367346938776,8.59107839372e-09) (0.489795918367,-8.59107839372e-09) (0.612244897959,-1.23054636115) (0.734693877551,-40.5516402702) (0.857142857143,-640.213875183) (0.979591836735,-6438.55873936) (1.10204081633,-46257.8011138) (1.22448979592,-252333.895514) (1.34693877551,-1084103.88038) (1.4693877551,-3755013.24106) (1.59183673469,-10642636.9901) (1.71428571429,-24916947.9583) (1.83673469388,-48387367.3518) (1.95918367347,-78275857.836) (2.08163265306,-105019368.728) (2.20408163265,-119223316.597) (2.32653061224,-113889153.316) (2.44897959184,-103439986.404) (2.57142857143,-92963963.4978) (2.69387755102,-96202736.2231) (2.81632653061,-98859035.7487) (2.9387755102,-103034799.977) (3.0612244898,-103034799.977) (3.18367346939,-98859035.7487) (3.30612244898,-96202736.2231) (3.42857142857,-92963963.4978) (3.55102040816,-103439986.404) (3.67346938776,-113889153.316) (3.79591836735,-119223316.597) (3.91836734694,-105019368.728) (4.04081632653,-78275857.836) (4.16326530612,-48387367.3518) (4.28571428571,-24916947.9583) (4.40816326531,-10642636.9901) (4.5306122449,-3755013.24106) (4.65306122449,-1084103.88038) (4.77551020408,-252333.895514) (4.89795918367,-46257.8011138) (5.02040816327,-6438.55873932) (5.14285714286,-640.213875175) (5.26530612245,-40.5516402702) (5.38775510204,-1.23054632679) (5.51020408163,0.0) (5.63265306122,0.0) (5.75510204082,0.0) (5.87755102041,0.0) (6.0,0.0) };
\addplot[Orange,densely dotted,mark=none,very thick,mark size=3pt] coordinates{(0.0,0.0) (0.0606060606061,0.0) (0.121212121212,0.0) (0.181818181818,0.0) (0.242424242424,0.0) (0.30303030303,0.0) (0.363636363636,0.0) (0.424242424242,0.0) (0.484848484848,0.0) (0.545454545455,0.0) (0.606060606061,-0.302430335559) (0.666666666667,-0.302430335559) (0.727272727273,-12.3996438685) (0.787878787879,-12.3996438685) (0.848484848485,-240.230497765) (0.909090909091,-240.230497765) (0.969696969697,-2921.77951372) (1.0303030303,-2921.77951372) (1.09090909091,-24993.4810877) (1.15151515152,-24993.4810877) (1.21212121212,-159637.896304) (1.27272727273,-159637.896304) (1.33333333333,-788716.389772) (1.39393939394,-788716.389772) (1.45454545455,-3080624.98423) (1.51515151515,-3080624.98423) (1.57575757576,-9638116.18636) (1.63636363636,-9638116.18636) (1.69696969697,-24323655.3377) (1.75757575758,-24323655.3377) (1.81818181818,-49636559.9635) (1.87878787879,-49636559.9635) (1.93939393939,-81867011.1376) (2.0,-81867011.1376) (2.06060606061,-109111634.24) (2.12121212121,-109111634.24) (2.18181818182,-118687950.99) (2.24242424242,-118687950.99) (2.30303030303,-110105828.793) (2.36363636364,-110105828.793) (2.42424242424,-97435608.9991) (2.48484848485,-97435608.9991) (2.54545454545,-94244104.916) (2.60606060606,-94244104.916) (2.66666666667,-99081654.2889) (2.72727272727,-99081654.2889) (2.78787878788,-101740412.886) (2.84848484848,-101740412.886) (2.90909090909,-100070314.798) (2.9696969697,-100070314.798) (3.0303030303,-100070314.798) (3.09090909091,-100070314.798) (3.15151515152,-101740412.886) (3.21212121212,-101740412.886) (3.27272727273,-99081654.2889) (3.33333333333,-99081654.2889) (3.39393939394,-94244104.916) (3.45454545455,-94244104.916) (3.51515151515,-97435608.9991) (3.57575757576,-97435608.9991) (3.63636363636,-110105828.793) (3.69696969697,-110105828.793) (3.75757575758,-118687950.99) (3.81818181818,-118687950.99) (3.87878787879,-109111634.24) (3.93939393939,-109111634.24) (4.0,-81867011.1376) (4.06060606061,-81867011.1376) (4.12121212121,-49636559.9635) (4.18181818182,-49636559.9635) (4.24242424242,-24323655.3377) (4.30303030303,-24323655.3377) (4.36363636364,-9638116.18636) (4.42424242424,-9638116.18636) (4.48484848485,-3080624.98423) (4.54545454545,-3080624.98423) (4.60606060606,-788716.389772) (4.66666666667,-788716.389772) (4.72727272727,-159637.896304) (4.78787878788,-159637.896304) (4.84848484848,-24993.4810877) (4.90909090909,-24993.4810877) (4.9696969697,-2921.7795137) (5.0303030303,-2921.7795137) (5.09090909091,-240.230497756) (5.15151515152,-240.230497756) (5.21212121212,-12.3996438685) (5.27272727273,-12.3996438685) (5.33333333333,-0.302430344063) (5.39393939394,-0.302430344063) (5.45454545455,0.0) (5.51515151515,0.0) (5.57575757576,0.0) (5.63636363636,0.0) (5.69696969697,0.0) (5.75757575758,0.0) (5.81818181818,0.0) (5.87878787879,0.0) (5.93939393939,0.0) (6.0,0.0) };
\addplot[Duck,solid,mark=*,thick,mark size=2pt] coordinates{(0.0,-8.50429982409e-09) (0.0606060606061,-8.50429982409e-09) (0.121212121212,-2.55128994723e-08) (0.181818181818,-2.55128994723e-08) (0.242424242424,-1.70085996482e-08) (0.30303030303,-1.70085996482e-08) (0.363636363636,-5.10257989445e-08) (0.424242424242,-5.10257989445e-08) (0.484848484848,-3.40171992963e-08) (0.545454545455,-3.40171992963e-08) (0.606060606061,-7.54315610416) (0.666666666667,-7.54315610416) (0.727272727273,-243.074356172) (0.787878787879,-243.074356172) (0.848484848485,-3634.66708743) (0.909090909091,-3634.66708743) (0.969696969697,-33492.7067527) (1.0303030303,-33492.7067527) (1.09090909091,-213117.914955) (1.15151515152,-213117.914955) (1.21212121212,-995090.547212) (1.27272727273,-995090.547212) (1.33333333333,-3540220.94427) (1.39393939394,-3540220.94427) (1.45454545455,-9852354.5249) (1.51515151515,-9852354.5249) (1.57575757576,-21906072.5461) (1.63636363636,-21906072.5461) (1.69696969697,-39711055.8831) (1.75757575758,-39711055.8831) (1.81818181818,-60065449.8718) (1.87878787879,-60065449.8718) (1.93939393939,-78032993.3013) (2.0,-78032993.3013) (2.06060606061,-90231435.0995) (2.12121212121,-90231435.0995) (2.18181818182,-96567889.7944) (2.24242424242,-96567889.7944) (2.30303030303,-99068409.7758) (2.36363636364,-99068409.7758) (2.42424242424,-99809735.7536) (2.48484848485,-99809735.7536) (2.54545454545,-99971826.9969) (2.60606060606,-99971826.9969) (2.66666666667,-99997149.6439) (2.72727272727,-99997149.6439) (2.78787878788,-99999824.5348) (2.84848484848,-99999824.5348) (2.90909090909,-99999994.876) (2.9696969697,-99999994.876) (3.0303030303,-99999994.876) (3.09090909091,-99999994.876) (3.15151515152,-99999824.5348) (3.21212121212,-99999824.5348) (3.27272727273,-99997149.6439) (3.33333333333,-99997149.6439) (3.39393939394,-99971826.9969) (3.45454545455,-99971826.9969) (3.51515151515,-99809735.7536) (3.57575757576,-99809735.7536) (3.63636363636,-99068409.7758) (3.69696969697,-99068409.7758) (3.75757575758,-96567889.7944) (3.81818181818,-96567889.7944) (3.87878787879,-90231435.0995) (3.93939393939,-90231435.0995) (4.0,-78032993.3013) (4.06060606061,-78032993.3013) (4.12121212121,-60065449.8718) (4.18181818182,-60065449.8718) (4.24242424242,-39711055.8831) (4.30303030303,-39711055.8831) (4.36363636364,-21906072.5461) (4.42424242424,-21906072.5461) (4.48484848485,-9852354.5249) (4.54545454545,-9852354.5249) (4.60606060606,-3540220.94427) (4.66666666667,-3540220.94427) (4.72727272727,-995090.547212) (4.78787878788,-995090.547212) (4.84848484848,-213117.914955) (4.90909090909,-213117.914955) (4.9696969697,-33492.7067527) (5.0303030303,-33492.7067527) (5.09090909091,-3634.66708747) (5.15151515152,-3634.66708747) (5.21212121212,-243.074356181) (5.27272727273,-243.074356181) (5.33333333333,-7.54315609566) (5.39393939394,-7.54315609566) (5.45454545455,0.0) (5.51515151515,0.0) (5.57575757576,0.0) (5.63636363636,0.0) (5.69696969697,0.0) (5.75757575758,0.0) (5.81818181818,0.0) (5.87878787879,0.0) (5.93939393939,0.0) (6.0,0.0) };
\addplot[Blue,solid,mark=none,very thick,mark size=3pt] coordinates{(0.0,1.89023694665e-23) (0.122448979592,6.30078982218e-23) (0.244897959184,-2.20527643776e-23) (0.367346938776,1.08481298171e-07) (0.489795918367,-6.02673878727e-08) (0.612244897959,-3.78047389331e-23) (0.734693877551,-2.41069551491e-08) (0.857142857143,-1.20534775745e-08) (0.979591836735,3.61604327236e-08) (1.10204081633,-1.20534775745e-08) (1.22448979592,3.61604327236e-08) (1.34693877551,-3.15039491109e-24) (1.4693877551,1.20534775745e-08) (1.59183673469,-6.30078982218e-24) (1.71428571429,2.41069551491e-08) (1.83673469388,-100000000.0) (1.95918367347,-100000000.0) (2.08163265306,-100000000.0) (2.20408163265,-100000000.0) (2.32653061224,-100000000.0) (2.44897959184,-100000000.0) (2.57142857143,-100000000.0) (2.69387755102,-100000000.0) (2.81632653061,-100000000.0) (2.9387755102,-100000000.0) (3.0612244898,-100000000.0) (3.18367346939,-100000000.0) (3.30612244898,-100000000.0) (3.42857142857,-100000000.0) (3.55102040816,-100000000.0) (3.67346938776,-100000000.0) (3.79591836735,-100000000.0) (3.91836734694,-100000000.0) (4.04081632653,-100000000.0) (4.16326530612,-100000000.0) (4.28571428571,1.20534775745e-08) (4.40816326531,-3.61604327236e-08) (4.5306122449,-6.30078982218e-24) (4.65306122449,-2.41069551491e-08) (4.77551020408,-1.20534775745e-08) (4.89795918367,-4.82139102982e-08) (5.02040816327,-3.30791465665e-23) (5.14285714286,2.41069551491e-08) (5.26530612245,-4.82139102982e-08) (5.38775510204,2.41069551491e-08) (5.51020408163,-3.15039491109e-23) (5.63265306122,-3.61604327236e-08) (5.75510204082,2.41069551491e-08) (5.87755102041,3.61604327236e-08) (6.0,-1.20534775745e-08) };
\addplot[Purple,solid,mark=|,very thick,mark size=3pt] coordinates{(0.0,-6.62623071053e-10) (0.0606060606061,-1.13908545035e-08) (0.121212121212,-1.03251991157e-08) (0.181818181818,-2.58352336079e-08) (0.242424242424,-5.14381821174e-08) (0.30303030303,-4.4989638479e-08) (0.363636363636,-5.34390135805e-08) (0.424242424242,-5.50422845904e-08) (0.484848484848,-2.21025908105e-08) (0.545454545455,-3.81647970622e-08) (0.606060606061,-0.201620272159) (0.666666666667,-0.604860699987) (0.727272727273,-10.6858719382) (0.787878787879,-29.9070000639) (0.848484848485,-256.986134699) (0.909090909091,-668.206763393) (0.969696969697,-3715.07114041) (1.0303030303,-8933.82287809) (1.09090909091,-36066.6707163) (1.15151515152,-79819.6254771) (1.21212121212,-248965.8763) (1.27272727273,-504422.520742) (1.33333333333,-1263261.9825) (1.39393939394,-2330406.84983) (1.45454545455,-4810891.93605) (1.51515151515,-8037941.25365) (1.57575757576,-13953612.4785) (1.63636363636,-21022005.3552) (1.69696969697,-31228050.0112) (1.75757575758,-42338748.4528) (1.81818181818,-54831068.3252) (1.87878787879,-67114743.6608) (1.93939393939,-77590809.6787) (2.0,-86813439.423) (2.06060606061,-92463242.0658) (2.12121212121,-96885420.8686) (2.18181818182,-98592164.6111) (2.24242424242,-99785509.2556) (2.30303030303,-99950031.6158) (2.36363636364,-100067351.927) (2.42424242424,-100025492.446) (2.48484848485,-100011084.154) (2.54545454545,-100002494.767) (2.60606060606,-99999559.9552) (2.66666666667,-99999874.1387) (2.72727272727,-99999912.159) (2.78787878788,-99999990.2886) (2.84848484848,-100000001.919) (2.90909090909,-100000000.164) (2.9696969697,-100000000.078) (3.0303030303,-100000000.078) (3.09090909091,-100000000.164) (3.15151515152,-100000001.919) (3.21212121212,-99999990.2886) (3.27272727273,-99999912.159) (3.33333333333,-99999874.1387) (3.39393939394,-99999559.9552) (3.45454545455,-100002494.767) (3.51515151515,-100011084.154) (3.57575757576,-100025492.446) (3.63636363636,-100067351.927) (3.69696969697,-99950031.6158) (3.75757575758,-99785509.2556) (3.81818181818,-98592164.6111) (3.87878787879,-96885420.8686) (3.93939393939,-92463242.0658) (4.0,-86813439.423) (4.06060606061,-77590809.6787) (4.12121212121,-67114743.6608) (4.18181818182,-54831068.3252) (4.24242424242,-42338748.4528) (4.30303030303,-31228050.0112) (4.36363636364,-21022005.3552) (4.42424242424,-13953612.4785) (4.48484848485,-8037941.25365) (4.54545454545,-4810891.93605) (4.60606060606,-2330406.84983) (4.66666666667,-1263261.9825) (4.72727272727,-504422.520742) (4.78787878788,-248965.8763) (4.84848484848,-79819.6254771) (4.90909090909,-36066.6707163) (4.9696969697,-8933.82287811) (5.0303030303,-3715.07114043) (5.09090909091,-668.206763402) (5.15151515152,-256.986134702) (5.21212121212,-29.9070000726) (5.27272727273,-10.6858719415) (5.33333333333,-0.604860674869) (5.39393939394,-0.201620224956) (5.45454545455,0.0) (5.51515151515,0.0) (5.57575757576,0.0) (5.63636363636,0.0) (5.69696969697,0.0) (5.75757575758,0.0) (5.81818181818,0.0) (5.87878787879,0.0) (5.93939393939,-6.02673878726e-09) (6.0,-6.02673878728e-09) };
\addplot[Green,only marks,mark=x,thick,mark size=3pt] coordinates{(0.0,-4.24785031222e-08) (0.0606060606061,-5.39493174741e-08) (0.121212121212,4.17860387809e-08) (0.181818181818,3.05348266663e-08) (0.242424242424,-1.08312734142e-08) (0.30303030303,1.08312734142e-08) (0.363636363636,-1.37558777075e-08) (0.424242424242,-1.03510774416e-08) (0.484848484848,-6.55691082182e-08) (0.545454545455,-5.49656675272e-08) (0.606060606061,-4.3242145073e-08) (0.666666666667,-2.90787203742e-08) (0.727272727273,-3.72459372501e-08) (0.787878787879,-3.50749281972e-08) (0.848484848485,-3.29863159637e-09) (0.909090909091,-2.08083235527e-08) (0.969696969697,1.35203892531e-08) (1.0303030303,1.0586565896e-08) (1.09090909091,1.42527645537e-09) (1.15151515152,-1.42527645537e-09) (1.21212121212,-4.952752343e-09) (1.27272727273,4.952752343e-09) (1.33333333333,4.19430157571e-08) (1.39393939394,3.03778496901e-08) (1.45454545455,3.00669764225e-08) (1.51515151515,4.22538890248e-08) (1.57575757576,-1.21426105178e-08) (1.63636363636,-1.19643446313e-08) (1.69696969697,1.70747866461e-08) (1.75757575758,3.1139123652e-08) (1.81818181818,-100000000.0) (1.87878787879,-100000000.0) (1.93939393939,-100000000.0) (2.0,-100000000.0) (2.06060606061,-100000000.0) (2.12121212121,-100000000.0) (2.18181818182,-100000000.0) (2.24242424242,-100000000.0) (2.30303030303,-100000000.0) (2.36363636364,-100000000.0) (2.42424242424,-100000000.0) (2.48484848485,-100000000.0) (2.54545454545,-100000000.0) (2.60606060606,-100000000.0) (2.66666666667,-100000000.0) (2.72727272727,-100000000.0) (2.78787878788,-100000000.0) (2.84848484848,-100000000.0) (2.90909090909,-100000000.0) (2.9696969697,-100000000.0) (3.0303030303,-100000000.0) (3.09090909091,-100000000.0) (3.15151515152,-100000000.0) (3.21212121212,-100000000.0) (3.27272727273,-100000000.0) (3.33333333333,-100000000.0) (3.39393939394,-100000000.0) (3.45454545455,-100000000.0) (3.51515151515,-100000000.0) (3.57575757576,-100000000.0) (3.63636363636,-100000000.0) (3.69696969697,-100000000.0) (3.75757575758,-100000000.0) (3.81818181818,-100000000.0) (3.87878787879,-100000000.0) (3.93939393939,-100000000.0) (4.0,-100000000.0) (4.06060606061,-100000000.0) (4.12121212121,-100000000.0) (4.18181818182,-100000000.0) (4.24242424242,-4.81448937659e-09) (4.30303030303,4.81448937659e-09) (4.36363636364,-1.4144710689e-09) (4.42424242424,1.4144710689e-09) (4.48484848485,1.2195740833e-08) (4.54545454545,1.19112143161e-08) (4.60606060606,-3.0110151988e-08) (4.66666666667,-4.22107134593e-08) (4.72727272727,-6.28570021954e-09) (4.78787878788,-1.78212549295e-08) (4.84848484848,8.92239843897e-09) (4.90909090909,1.51845567101e-08) (4.9696969697,-4.2375507098e-09) (5.0303030303,4.2375507098e-09) (5.09090909091,-1.35601622714e-08) (5.15151515152,-1.05467928777e-08) (5.21212121212,1.20534775745e-08) (5.27272727273,3.61604327236e-08) (5.33333333333,-4.21871715109e-08) (5.39393939394,-3.01336939364e-08) (5.45454545455,0.0) (5.51515151515,0.0) (5.57575757576,0.0) (5.63636363636,0.0) (5.69696969697,0.0) (5.75757575758,0.0) (5.81818181818,0.0) (5.87878787879,0.0) (5.93939393939,0.0) (6.0,0.0) };
\addplot[black,solid,mark=pentagone*,thin,mark size=3pt] coordinates{(0.0,-0.0) (0.122448979592,-0.0) (0.244897959184,-0.0) (0.367346938776,-0.0) (0.489795918367,-0.0) (0.612244897959,-0.0) (0.734693877551,-0.0) (0.857142857143,-0.0) (0.979591836735,-0.0) (1.10204081633,-0.0) (1.22448979592,-0.0) (1.34693877551,-0.0) (1.4693877551,-0.0) (1.59183673469,-0.0) (1.71428571429,-0.0) (1.83673469388,-100000000.0) (1.95918367347,-100000000.0) (2.08163265306,-100000000.0) (2.20408163265,-100000000.0) (2.32653061224,-100000000.0) (2.44897959184,-100000000.0) (2.57142857143,-100000000.0) (2.69387755102,-100000000.0) (2.81632653061,-100000000.0) (2.9387755102,-100000000.0) (3.0612244898,-100000000.0) (3.18367346939,-100000000.0) (3.30612244898,-100000000.0) (3.42857142857,-100000000.0) (3.55102040816,-100000000.0) (3.67346938776,-100000000.0) (3.79591836735,-100000000.0) (3.91836734694,-100000000.0) (4.04081632653,-100000000.0) (4.16326530612,-100000000.0) (4.28571428571,-0.0) (4.40816326531,-0.0) (4.5306122449,-0.0) (4.65306122449,-0.0) (4.77551020408,-0.0) (4.89795918367,-0.0) (5.02040816327,-0.0) (5.14285714286,-0.0) (5.26530612245,-0.0) (5.38775510204,-0.0) (5.51020408163,-0.0) (5.63265306122,-0.0) (5.75510204082,-0.0) (5.87755102041,-0.0) (6.0,-0.0) };
\nextgroupplot[ylabel=v (m/s),]
\addplot[Red,dashed,mark=none,very thick,mark size=3pt] coordinates{(0.0,2.53184841771) (0.122448979592,2.53184841771) (0.244897959184,2.53184841771) (0.367346938776,2.53184841771) (0.489795918367,2.53184841771) (0.612244897959,2.53184841771) (0.734693877551,2.53184841771) (0.857142857143,2.53184841771) (0.979591836735,2.53184841771) (1.10204081633,2.53184841771) (1.22448979592,2.53184841771) (1.34693877551,2.53184841771) (1.4693877551,2.53184841771) (1.59183673469,2.53184841771) (1.71428571429,2.53184841771) (1.83673469388,2.53168422771) (1.95918367347,2.52888333949) (2.08163265306,2.50699777845) (2.20408163265,2.40545109318) (2.32653061224,2.08436306275) (2.44897959184,1.45719589947) (2.57142857143,0.368133294256) (2.69387755102,0.0385943080031) (2.81632653061,-0.906077680431) (2.9387755102,0.154976042597) (3.0612244898,-0.154976042597) (3.18367346939,0.906077680431) (3.30612244898,-0.0385943080031) (3.42857142857,-0.368133294256) (3.55102040816,-1.45719589947) (3.67346938776,-2.08436306275) (3.79591836735,-2.40545109318) (3.91836734694,-2.50699777845) (4.04081632653,-2.52888333949) (4.16326530612,-2.53168422771) (4.28571428571,-2.53184841771) (4.40816326531,-2.53184841771) (4.5306122449,-2.53184841771) (4.65306122449,-2.53184841771) (4.77551020408,-2.53184841771) (4.89795918367,-2.53184841771) (5.02040816327,-2.53184841771) (5.14285714286,-2.53184841771) (5.26530612245,-2.53184841771) (5.38775510204,-2.53184841771) (5.51020408163,-2.53184841771) (5.63265306122,-2.53184841771) (5.75510204082,-2.53184841771) (5.87755102041,-2.53184841771) (6.0,-2.53184841771) };
\addplot[Orange,densely dotted,mark=none,very thick,mark size=3pt] coordinates{(0.0,2.53184841771) (0.0606060606061,2.53184841771) (0.121212121212,2.53184841771) (0.181818181818,2.53184841771) (0.242424242424,2.53184841771) (0.30303030303,2.53184841771) (0.363636363636,2.53184841771) (0.424242424242,2.53184841771) (0.484848484848,2.53184841771) (0.545454545455,2.53184841771) (0.606060606061,2.53184841771) (0.666666666667,2.53184841771) (0.727272727273,2.53184841771) (0.787878787879,2.53184841771) (0.848484848485,2.53184841771) (0.909090909091,2.53184841771) (0.969696969697,2.53184841771) (1.0303030303,2.53184841771) (1.09090909091,2.53184841771) (1.15151515152,2.53184841771) (1.21212121212,2.53184841771) (1.27272727273,2.53184841771) (1.33333333333,2.53184841771) (1.39393939394,2.53184841771) (1.45454545455,2.53184841771) (1.51515151515,2.53184841771) (1.57575757576,2.53184841771) (1.63636363636,2.53184841771) (1.69696969697,2.53184841771) (1.75757575758,2.53184841771) (1.81818181818,2.53180200348) (1.87878787879,2.53170917501) (1.93939393939,2.53075014243) (2.0,2.52892490574) (2.06060606061,2.52011291678) (2.12121212121,2.50431417554) (2.18181818182,2.45734572485) (2.24242424242,2.37920756473) (2.30303030303,2.21852587251) (2.36363636364,1.97530064819) (2.42424242424,1.62562122327) (2.48484848485,1.16948759776) (2.54545454545,0.652158035596) (2.60606060606,0.0736325367928) (2.66666666667,-0.209593400584) (2.72727272727,-0.197519776535) (2.78787878788,-0.41340730943) (2.84848484848,-0.857255999272) (2.90909090909,-0.176423153717) (2.9696969697,1.62909122723) (3.0303030303,-1.62909122723) (3.09090909091,0.176423153717) (3.15151515152,0.857255999272) (3.21212121212,0.41340730943) (3.27272727273,0.197519776535) (3.33333333333,0.209593400584) (3.39393939394,-0.0736325367928) (3.45454545455,-0.652158035596) (3.51515151515,-1.16948759776) (3.57575757576,-1.62562122327) (3.63636363636,-1.97530064819) (3.69696969697,-2.21852587251) (3.75757575758,-2.37920756473) (3.81818181818,-2.45734572485) (3.87878787879,-2.50431417554) (3.93939393939,-2.52011291678) (4.0,-2.52892490574) (4.06060606061,-2.53075014243) (4.12121212121,-2.53170917501) (4.18181818182,-2.53180200348) (4.24242424242,-2.53184841771) (4.30303030303,-2.53184841771) (4.36363636364,-2.53184841771) (4.42424242424,-2.53184841771) (4.48484848485,-2.53184841771) (4.54545454545,-2.53184841771) (4.60606060606,-2.53184841771) (4.66666666667,-2.53184841771) (4.72727272727,-2.53184841771) (4.78787878788,-2.53184841771) (4.84848484848,-2.53184841771) (4.90909090909,-2.53184841771) (4.9696969697,-2.53184841771) (5.0303030303,-2.53184841771) (5.09090909091,-2.53184841771) (5.15151515152,-2.53184841771) (5.21212121212,-2.53184841771) (5.27272727273,-2.53184841771) (5.33333333333,-2.53184841771) (5.39393939394,-2.53184841771) (5.45454545455,-2.53184841771) (5.51515151515,-2.53184841771) (5.57575757576,-2.53184841771) (5.63636363636,-2.53184841771) (5.69696969697,-2.53184841771) (5.75757575758,-2.53184841771) (5.81818181818,-2.53184841771) (5.87878787879,-2.53184841771) (5.93939393939,-2.53184841771) (6.0,-2.53184841771) };
\addplot[Duck,solid,mark=*,thick,mark size=2pt] coordinates{(0.0,2.53184841771) (0.0606060606061,2.53184841771) (0.121212121212,2.53184841771) (0.181818181818,2.53184841771) (0.242424242424,2.53184841771) (0.30303030303,2.53184841771) (0.363636363636,2.53184841771) (0.424242424242,2.53184841771) (0.484848484848,2.53184841771) (0.545454545455,2.53184841771) (0.606060606061,2.53184841771) (0.666666666667,2.53184841771) (0.727272727273,2.53184841771) (0.787878787879,2.53184841771) (0.848484848485,2.53184841771) (0.909090909091,2.53184841771) (0.969696969697,2.53184841771) (1.0303030303,2.53184841771) (1.09090909091,2.53184841771) (1.15151515152,2.53184841771) (1.21212121212,2.53184841771) (1.27272727273,2.53184841771) (1.33333333333,2.53184841771) (1.39393939394,2.53184841771) (1.45454545455,2.53184841771) (1.51515151515,2.53184841771) (1.57575757576,2.53184841771) (1.63636363636,2.53184841771) (1.69696969697,2.53184841771) (1.75757575758,2.53184841771) (1.81818181818,2.53147672827) (1.87878787879,2.53073334938) (1.93939393939,2.52515042224) (2.0,2.51472794685) (2.06060606061,2.47915629191) (2.12121212121,2.41843545742) (2.18181818182,2.29318461062) (2.24242424242,2.10340375152) (2.30303030303,1.83698560404) (2.36363636364,1.49393016818) (2.42424242424,1.14155493551) (2.48484848485,0.779859906029) (2.54545454545,0.490899655057) (2.60606060606,0.274674182594) (2.66666666667,0.131308307876) (2.72727272727,0.0608020309012) (2.78787878788,0.0195497010067) (2.84848484848,0.00755131819221) (2.90909090909,0.00116409508873) (2.9696969697,0.000388031696242) (3.0303030303,-0.000388031696245) (3.09090909091,-0.00116409508873) (3.15151515152,-0.00755131819221) (3.21212121212,-0.0195497010067) (3.27272727273,-0.0608020309012) (3.33333333333,-0.131308307876) (3.39393939394,-0.274674182594) (3.45454545455,-0.490899655057) (3.51515151515,-0.779859906029) (3.57575757576,-1.14155493551) (3.63636363636,-1.49393016818) (3.69696969697,-1.83698560404) (3.75757575758,-2.10340375152) (3.81818181818,-2.29318461062) (3.87878787879,-2.41843545742) (3.93939393939,-2.47915629191) (4.0,-2.51472794685) (4.06060606061,-2.52515042224) (4.12121212121,-2.53073334938) (4.18181818182,-2.53147672827) (4.24242424242,-2.53184841771) (4.30303030303,-2.53184841771) (4.36363636364,-2.53184841771) (4.42424242424,-2.53184841771) (4.48484848485,-2.53184841771) (4.54545454545,-2.53184841771) (4.60606060606,-2.53184841771) (4.66666666667,-2.53184841771) (4.72727272727,-2.53184841771) (4.78787878788,-2.53184841771) (4.84848484848,-2.53184841771) (4.90909090909,-2.53184841771) (4.9696969697,-2.53184841771) (5.0303030303,-2.53184841771) (5.09090909091,-2.53184841771) (5.15151515152,-2.53184841771) (5.21212121212,-2.53184841771) (5.27272727273,-2.53184841771) (5.33333333333,-2.53184841771) (5.39393939394,-2.53184841771) (5.45454545455,-2.53184841771) (5.51515151515,-2.53184841771) (5.57575757576,-2.53184841771) (5.63636363636,-2.53184841771) (5.69696969697,-2.53184841771) (5.75757575758,-2.53184841771) (5.81818181818,-2.53184841771) (5.87878787879,-2.53184841771) (5.93939393939,-2.53184841771) (6.0,-2.53184841771) };
\addplot[Blue,solid,mark=none,very thick,mark size=3pt] coordinates{(0.0,2.53184841771) (0.122448979592,2.53184841771) (0.244897959184,2.53184841771) (0.367346938776,2.53184841771) (0.489795918367,2.53184841771) (0.612244897959,2.53184841771) (0.734693877551,2.53184841771) (0.857142857143,2.53184841771) (0.979591836735,2.53184841771) (1.10204081633,2.53184841771) (1.22448979592,2.53184841771) (1.34693877551,2.53184841771) (1.4693877551,2.53184841771) (1.59183673469,2.53184841771) (1.71428571429,2.53184841771) (1.83673469388,2.53184841771) (1.95918367347,2.53184841771) (2.08163265306,2.53184841771) (2.20408163265,2.53184841771) (2.32653061224,2.53184841771) (2.44897959184,0.0) (2.57142857143,-1.13182444172e-15) (2.69387755102,3.77274813907e-16) (2.81632653061,-3.94430452611e-31) (2.9387755102,7.54549627813e-16) (3.0612244898,-1.13182444172e-15) (3.18367346939,3.77274813907e-16) (3.30612244898,3.77274813907e-16) (3.42857142857,7.54549627813e-16) (3.55102040816,2.22044604925e-16) (3.67346938776,-2.53184841771) (3.79591836735,-2.53184841771) (3.91836734694,-2.53184841771) (4.04081632653,-2.53184841771) (4.16326530612,-2.53184841771) (4.28571428571,-2.53184841771) (4.40816326531,-2.53184841771) (4.5306122449,-2.53184841771) (4.65306122449,-2.53184841771) (4.77551020408,-2.53184841771) (4.89795918367,-2.53184841771) (5.02040816327,-2.53184841771) (5.14285714286,-2.53184841771) (5.26530612245,-2.53184841771) (5.38775510204,-2.53184841771) (5.51020408163,-2.53184841771) (5.63265306122,-2.53184841771) (5.75510204082,-2.53184841771) (5.87755102041,-2.53184841771) (6.0,-2.53184841771) };
\addplot[Purple,solid,mark=|,very thick,mark size=3pt] coordinates{(0.0,2.53184841771) (0.0606060606061,2.53184841771) (0.121212121212,2.53184841771) (0.181818181818,2.53184841771) (0.242424242424,2.53184841771) (0.30303030303,2.53184841771) (0.363636363636,2.53184841771) (0.424242424242,2.53184841771) (0.484848484848,2.53184841771) (0.545454545455,2.53184841771) (0.606060606061,2.53184841771) (0.666666666667,2.53184841771) (0.727272727273,2.53184841771) (0.787878787879,2.53184841771) (0.848484848485,2.53184841771) (0.909090909091,2.53184841771) (0.969696969697,2.53184841771) (1.0303030303,2.53184841771) (1.09090909091,2.53184841771) (1.15151515152,2.53184841771) (1.21212121212,2.53184841771) (1.27272727273,2.53184841771) (1.33333333333,2.53184841771) (1.39393939394,2.53184841771) (1.45454545455,2.53184841771) (1.51515151515,2.53184841771) (1.57575757576,2.53184841771) (1.63636363636,2.53184841771) (1.69696969697,2.53184841771) (1.75757575758,2.53184841771) (1.81818181818,2.53175559396) (1.87878787879,2.53156994645) (1.93939393939,2.5293009214) (2.0,2.52519604881) (2.06060606061,2.50298366875) (2.12121212121,2.46725683797) (2.18181818182,2.35582962714) (2.24242424242,2.1999091525) (2.30303030303,1.89553583941) (2.36363636364,1.53503804034) (2.42424242424,1.09135693597) (2.48484848485,0.661695344823) (2.54545454545,0.350412262381) (2.60606060606,0.113981116089) (2.66666666667,0.0376237119531) (2.72727272727,-0.00661873956151) (2.78787878788,-0.00240427335439) (2.84848484848,-0.00178851369874) (2.90909090909,-0.000190094630922) (2.9696969697,4.07503958371e-05) (3.0303030303,-4.07503958369e-05) (3.09090909091,0.000190094630922) (3.15151515152,0.00178851369874) (3.21212121212,0.00240427335439) (3.27272727273,0.00661873956151) (3.33333333333,-0.0376237119531) (3.39393939394,-0.113981116089) (3.45454545455,-0.350412262381) (3.51515151515,-0.661695344823) (3.57575757576,-1.09135693597) (3.63636363636,-1.53503804034) (3.69696969697,-1.89553583941) (3.75757575758,-2.1999091525) (3.81818181818,-2.35582962714) (3.87878787879,-2.46725683797) (3.93939393939,-2.50298366875) (4.0,-2.52519604881) (4.06060606061,-2.5293009214) (4.12121212121,-2.53156994645) (4.18181818182,-2.53175559396) (4.24242424242,-2.53184841771) (4.30303030303,-2.53184841771) (4.36363636364,-2.53184841771) (4.42424242424,-2.53184841771) (4.48484848485,-2.53184841771) (4.54545454545,-2.53184841771) (4.60606060606,-2.53184841771) (4.66666666667,-2.53184841771) (4.72727272727,-2.53184841771) (4.78787878788,-2.53184841771) (4.84848484848,-2.53184841771) (4.90909090909,-2.53184841771) (4.9696969697,-2.53184841771) (5.0303030303,-2.53184841771) (5.09090909091,-2.53184841771) (5.15151515152,-2.53184841771) (5.21212121212,-2.53184841771) (5.27272727273,-2.53184841771) (5.33333333333,-2.53184841771) (5.39393939394,-2.53184841771) (5.45454545455,-2.53184841771) (5.51515151515,-2.53184841771) (5.57575757576,-2.53184841771) (5.63636363636,-2.53184841771) (5.69696969697,-2.53184841771) (5.75757575758,-2.53184841771) (5.81818181818,-2.53184841771) (5.87878787879,-2.53184841771) (5.93939393939,-2.53184841771) (6.0,-2.53184841771) };
\addplot[Green,only marks,mark=x,thick,mark size=3pt] coordinates{(0.0,2.53184841771) (0.0606060606061,2.53184841771) (0.121212121212,2.53184841771) (0.181818181818,2.53184841771) (0.242424242424,2.53184841771) (0.30303030303,2.53184841771) (0.363636363636,2.53184841771) (0.424242424242,2.53184841771) (0.484848484848,2.53184841771) (0.545454545455,2.53184841771) (0.606060606061,2.53184841771) (0.666666666667,2.53184841771) (0.727272727273,2.53184841771) (0.787878787879,2.53184841771) (0.848484848485,2.53184841771) (0.909090909091,2.53184841771) (0.969696969697,2.53184841771) (1.0303030303,2.53184841771) (1.09090909091,2.53184841771) (1.15151515152,2.53184841771) (1.21212121212,2.53184841771) (1.27272727273,2.53184841771) (1.33333333333,2.53184841771) (1.39393939394,2.53184841771) (1.45454545455,2.53184841771) (1.51515151515,2.53184841771) (1.57575757576,2.53184841771) (1.63636363636,2.53184841771) (1.69696969697,2.53184841771) (1.75757575758,2.53184841771) (1.81818181818,2.53184841771) (1.87878787879,2.53184841771) (1.93939393939,2.53184841771) (2.0,2.53184841771) (2.06060606061,2.53184841771) (2.12121212121,2.53184841771) (2.18181818182,2.53184841771) (2.24242424242,2.53184841771) (2.30303030303,2.53184841771) (2.36363636364,2.53184841771) (2.42424242424,7.77156117238e-16) (2.48484848485,-2.10942374679e-15) (2.54545454545,-2.49800180541e-16) (2.60606060606,-1.08246744901e-15) (2.66666666667,8.52730145934e-16) (2.72727272727,8.3320073577e-16) (2.78787878788,-6.63692638505e-16) (2.84848484848,-6.68574991046e-16) (2.90909090909,-9.66638403637e-17) (2.9696969697,2.7349546644e-16) (3.0303030303,-9.78979608712e-17) (3.09090909091,9.86076380571e-16) (3.15151515152,-5.85064728901e-16) (3.21212121212,-3.58561070263e-17) (3.27272727273,8.51821686005e-17) (3.33333333333,-1.28385002791e-16) (3.39393939394,1.77836698406e-16) (3.45454545455,8.94208853675e-17) (3.51515151515,-9.99200722163e-16) (3.57575757576,2.33146835171e-15) (3.63636363636,-2.53184841771) (3.69696969697,-2.53184841771) (3.75757575758,-2.53184841771) (3.81818181818,-2.53184841771) (3.87878787879,-2.53184841771) (3.93939393939,-2.53184841771) (4.0,-2.53184841771) (4.06060606061,-2.53184841771) (4.12121212121,-2.53184841771) (4.18181818182,-2.53184841771) (4.24242424242,-2.53184841771) (4.30303030303,-2.53184841771) (4.36363636364,-2.53184841771) (4.42424242424,-2.53184841771) (4.48484848485,-2.53184841771) (4.54545454545,-2.53184841771) (4.60606060606,-2.53184841771) (4.66666666667,-2.53184841771) (4.72727272727,-2.53184841771) (4.78787878788,-2.53184841771) (4.84848484848,-2.53184841771) (4.90909090909,-2.53184841771) (4.9696969697,-2.53184841771) (5.0303030303,-2.53184841771) (5.09090909091,-2.53184841771) (5.15151515152,-2.53184841771) (5.21212121212,-2.53184841771) (5.27272727273,-2.53184841771) (5.33333333333,-2.53184841771) (5.39393939394,-2.53184841771) (5.45454545455,-2.53184841771) (5.51515151515,-2.53184841771) (5.57575757576,-2.53184841771) (5.63636363636,-2.53184841771) (5.69696969697,-2.53184841771) (5.75757575758,-2.53184841771) (5.81818181818,-2.53184841771) (5.87878787879,-2.53184841771) (5.93939393939,-2.53184841771) (6.0,-2.53184841771) };
\addplot[black,solid,mark=pentagone*,thin,mark size=3pt] coordinates{(0.0,2.53184841771) (0.122448979592,2.53184841771) (0.244897959184,2.53184841771) (0.367346938776,2.53184841771) (0.489795918367,2.53184841771) (0.612244897959,2.53184841771) (0.734693877551,2.53184841771) (0.857142857143,2.53184841771) (0.979591836735,2.53184841771) (1.10204081633,2.53184841771) (1.22448979592,2.53184841771) (1.34693877551,2.53184841771) (1.4693877551,2.53184841771) (1.59183673469,2.53184841771) (1.71428571429,2.53184841771) (1.83673469388,2.53184841771) (1.95918367347,2.53184841771) (2.08163265306,2.53184841771) (2.20408163265,2.53184841771) (2.32653061224,2.53184841771) (2.44897959184,0.0) (2.57142857143,0.0) (2.69387755102,0.0) (2.81632653061,0.0) (2.9387755102,0.0) (3.0612244898,-0.0) (3.18367346939,-0.0) (3.30612244898,-0.0) (3.42857142857,-0.0) (3.55102040816,-0.0) (3.67346938776,-2.53184841771) (3.79591836735,-2.53184841771) (3.91836734694,-2.53184841771) (4.04081632653,-2.53184841771) (4.16326530612,-2.53184841771) (4.28571428571,-2.53184841771) (4.40816326531,-2.53184841771) (4.5306122449,-2.53184841771) (4.65306122449,-2.53184841771) (4.77551020408,-2.53184841771) (4.89795918367,-2.53184841771) (5.02040816327,-2.53184841771) (5.14285714286,-2.53184841771) (5.26530612245,-2.53184841771) (5.38775510204,-2.53184841771) (5.51020408163,-2.53184841771) (5.63265306122,-2.53184841771) (5.75510204082,-2.53184841771) (5.87755102041,-2.53184841771) (6.0,-2.53184841771) };
\nextgroupplot[legend style={at={($(0.62,-0.35)+(0.9cm,1cm)$)},legend columns=4}]
\addplot[Red,dashed,mark=none,very thick,mark size=3pt] coordinates{(0.0,2.53184841771) (0.122448979592,2.53184841771) (0.244897959184,2.53184841771) (0.367346938776,2.53184841771) (0.489795918367,2.53184841771) (0.612244897959,2.53184839845) (0.734693877551,2.53184776595) (0.857142857143,2.53183783837) (0.979591836735,2.53173890982) (1.10204081633,2.5310376551) (1.22448979592,2.52728460584) (1.34693877551,2.51158299538) (1.4693877551,2.45917663656) (1.59183673469,2.31811687514) (1.71428571429,2.01175812394) (1.83673469388,1.47729725871) (1.95918367347,0.755482211344) (2.08163265306,0.0106841656561) (2.20408163265,-0.393966883973) (2.32653061224,-0.518911254007) (2.44897959184,-0.0207115140802) (2.57142857143,-0.0492533894619) (2.69387755102,0.377229252824) (2.81632653061,-0.284866824279) (2.9387755102,0.279534610826) (3.0612244898,-0.279534610826) (3.18367346939,0.284866824279) (3.30612244898,-0.377229252824) (3.42857142857,0.0492533894619) (3.55102040816,0.0207115140802) (3.67346938776,0.518911254007) (3.79591836735,0.393966883973) (3.91836734694,-0.0106841656561) (4.04081632653,-0.755482211344) (4.16326530612,-1.47729725871) (4.28571428571,-2.01175812394) (4.40816326531,-2.31811687514) (4.5306122449,-2.45917663656) (4.65306122449,-2.51158299538) (4.77551020408,-2.52728460584) (4.89795918367,-2.5310376551) (5.02040816327,-2.53173890982) (5.14285714286,-2.53183783837) (5.26530612245,-2.53184776595) (5.38775510204,-2.53184839845) (5.51020408163,-2.53184841771) (5.63265306122,-2.53184841771) (5.75510204082,-2.53184841771) (5.87755102041,-2.53184841771) (6.0,-2.53184841771) };
\addplot[Orange,densely dotted,mark=none,very thick,mark size=3pt] coordinates{(0.0,2.53184841771) (0.0606060606061,2.53184841771) (0.121212121212,2.53184841771) (0.181818181818,2.53184841771) (0.242424242424,2.53184841771) (0.30303030303,2.53184841771) (0.363636363636,2.53184841771) (0.424242424242,2.53184841771) (0.484848484848,2.53184841771) (0.545454545455,2.53184841771) (0.606060606061,2.53184841516) (0.666666666667,2.53184841005) (0.727272727273,2.53184830626) (0.787878787879,2.53184810377) (0.848484848485,2.5318461135) (0.909090909091,2.53184233544) (0.969696969697,2.53181844517) (1.0303030303,2.53177444268) (1.09090909091,2.53157350116) (1.15151515152,2.5312156206) (1.21212121212,2.5299599966) (1.27272727273,2.52780662917) (1.33333333333,2.52177973021) (1.39393939394,2.51187929971) (1.45454545455,2.48923666538) (1.51515151515,2.45385182722) (1.57575757576,2.38671442175) (1.63636363636,2.28782444897) (1.69696969697,2.13092661749) (1.75757575758,1.9160209273) (1.81818181818,1.63073383023) (1.87878787879,1.27506532628) (1.93939393939,0.884612464987) (2.0,0.45937524635) (2.06060606061,0.0872776390367) (2.12121212121,-0.231680356951) (2.18181818182,-0.417548370979) (2.24242424242,-0.470326403048) (2.30303030303,-0.418756373042) (2.36363636364,-0.262838280961) (2.42424242424,-0.0954921221874) (2.48484848485,0.0832821032784) (2.54545454545,0.144072347907) (2.60606060606,0.0868786116988) (2.66666666667,0.11137418505) (2.72727272727,0.21755906796) (2.78787878788,-0.0280313601172) (2.84848484848,-0.625397099181) (2.90909090909,-0.0600978721068) (2.9696969697,1.6678663211) (3.0303030303,-1.6678663211) (3.09090909091,0.0600978721068) (3.15151515152,0.625397099181) (3.21212121212,0.0280313601172) (3.27272727273,-0.21755906796) (3.33333333333,-0.11137418505) (3.39393939394,-0.0868786116988) (3.45454545455,-0.144072347907) (3.51515151515,-0.0832821032784) (3.57575757576,0.0954921221874) (3.63636363636,0.262838280961) (3.69696969697,0.418756373042) (3.75757575758,0.470326403048) (3.81818181818,0.417548370979) (3.87878787879,0.231680356951) (3.93939393939,-0.0872776390367) (4.0,-0.45937524635) (4.06060606061,-0.884612464987) (4.12121212121,-1.27506532628) (4.18181818182,-1.63073383023) (4.24242424242,-1.9160209273) (4.30303030303,-2.13092661749) (4.36363636364,-2.28782444897) (4.42424242424,-2.38671442175) (4.48484848485,-2.45385182722) (4.54545454545,-2.48923666538) (4.60606060606,-2.51187929971) (4.66666666667,-2.52177973021) (4.72727272727,-2.52780662917) (4.78787878788,-2.5299599966) (4.84848484848,-2.5312156206) (4.90909090909,-2.53157350116) (4.9696969697,-2.53177444268) (5.0303030303,-2.53181844517) (5.09090909091,-2.53184233544) (5.15151515152,-2.5318461135) (5.21212121212,-2.53184810377) (5.27272727273,-2.53184830626) (5.33333333333,-2.53184841005) (5.39393939394,-2.53184841516) (5.45454545455,-2.53184841771) (5.51515151515,-2.53184841771) (5.57575757576,-2.53184841771) (5.63636363636,-2.53184841771) (5.69696969697,-2.53184841771) (5.75757575758,-2.53184841771) (5.81818181818,-2.53184841771) (5.87878787879,-2.53184841771) (5.93939393939,-2.53184841771) (6.0,-2.53184841771) };
\addplot[Duck,solid,mark=*,thick,mark size=2pt] coordinates{(0.0,2.53184841771) (0.0606060606061,2.53184841771) (0.121212121212,2.53184841771) (0.181818181818,2.53184841771) (0.242424242424,2.53184841771) (0.30303030303,2.53184841771) (0.363636363636,2.53184841771) (0.424242424242,2.53184841771) (0.484848484848,2.53184841771) (0.545454545455,2.53184841771) (0.606060606061,2.53184832222) (0.666666666667,2.53184813124) (0.727272727273,2.53184517687) (0.787878787879,2.53183945913) (0.848484848485,2.53179727414) (0.909090909091,2.53171862191) (0.969696969697,2.53134998514) (1.0303030303,2.53069136382) (1.09090909091,2.52848704147) (1.15151515152,2.5247370181) (1.21212121212,2.51518190702) (1.27272727273,2.49982170822) (1.33333333333,2.46878496941) (1.39393939394,2.42207169057) (1.45454545455,2.34503241073) (1.51515151515,2.23766712989) (1.57575757576,2.08990106819) (1.63636363636,1.90173422563) (1.69696969697,1.68155632428) (1.75757575758,1.42936736414) (1.81818181818,1.17421216421) (1.87878787879,0.916090724484) (1.93939393939,0.68656479914) (2.0,0.485634388181) (2.06060606061,0.326032520553) (2.12121212121,0.207759196256) (2.18181818182,0.122490738411) (2.24242424242,0.0702271470191) (2.30303030303,0.0355177061145) (2.36363636364,0.0183624156976) (2.42424242424,0.00772971069831) (2.48484848485,0.00361959111676) (2.54545454545,0.00121583464632) (2.60606060606,0.000518441286976) (2.66666666667,0.000130114164112) (2.72727272727,5.08532777256e-05) (2.78787878788,8.49674864507e-06) (2.84848484848,3.04457687026e-06) (2.90909090909,2.38868236939e-07) (2.9696969697,7.9622745095e-08) (3.0303030303,-7.96227466051e-08) (3.09090909091,-2.38868238161e-07) (3.15151515152,-3.0445768714e-06) (3.21212121212,-8.49674864633e-06) (3.27272727273,-5.08532777268e-05) (3.33333333333,-0.000130114164113) (3.39393939394,-0.000518441286976) (3.45454545455,-0.00121583464632) (3.51515151515,-0.00361959111676) (3.57575757576,-0.00772971069831) (3.63636363636,-0.0183624156976) (3.69696969697,-0.0355177061145) (3.75757575758,-0.0702271470191) (3.81818181818,-0.122490738411) (3.87878787879,-0.207759196256) (3.93939393939,-0.326032520553) (4.0,-0.485634388181) (4.06060606061,-0.68656479914) (4.12121212121,-0.916090724484) (4.18181818182,-1.17421216421) (4.24242424242,-1.42936736414) (4.30303030303,-1.68155632428) (4.36363636364,-1.90173422563) (4.42424242424,-2.08990106819) (4.48484848485,-2.23766712989) (4.54545454545,-2.34503241073) (4.60606060606,-2.42207169057) (4.66666666667,-2.46878496941) (4.72727272727,-2.49982170822) (4.78787878788,-2.51518190702) (4.84848484848,-2.5247370181) (4.90909090909,-2.52848704147) (4.9696969697,-2.53069136382) (5.0303030303,-2.53134998514) (5.09090909091,-2.53171862191) (5.15151515152,-2.53179727414) (5.21212121212,-2.53183945913) (5.27272727273,-2.53184517687) (5.33333333333,-2.53184813124) (5.39393939394,-2.53184832222) (5.45454545455,-2.53184841771) (5.51515151515,-2.53184841771) (5.57575757576,-2.53184841771) (5.63636363636,-2.53184841771) (5.69696969697,-2.53184841771) (5.75757575758,-2.53184841771) (5.81818181818,-2.53184841771) (5.87878787879,-2.53184841771) (5.93939393939,-2.53184841771) (6.0,-2.53184841771) };
\addplot[Blue,solid,mark=none,very thick,mark size=3pt] coordinates{(0.0,2.53184841771) (0.122448979592,2.53184841771) (0.244897959184,2.53184841771) (0.367346938776,2.53184841771) (0.489795918367,2.53184841771) (0.612244897959,2.53184841771) (0.734693877551,2.53184841771) (0.857142857143,2.53184841771) (0.979591836735,2.53184841771) (1.10204081633,2.53184841771) (1.22448979592,2.53184841771) (1.34693877551,2.53184841771) (1.4693877551,2.53184841771) (1.59183673469,2.53184841771) (1.71428571429,2.53184841771) (1.83673469388,-6.66133814775e-16) (1.95918367347,-2.48569348837e-15) (2.08163265306,6.8773523187e-16) (2.20408163265,-5.91645678916e-31) (2.32653061224,-3.77274813907e-16) (2.44897959184,3.94430452611e-31) (2.57142857143,1.13182444172e-15) (2.69387755102,-5.91645678916e-31) (2.81632653061,1.13182444172e-15) (2.9387755102,-3.94430452611e-31) (3.0612244898,-7.88860905221e-31) (3.18367346939,-3.77274813907e-16) (3.30612244898,2.26364888344e-15) (3.42857142857,-1.88637406953e-15) (3.55102040816,2.10841867446e-15) (3.67346938776,-9.98195649833e-16) (3.79591836735,3.10460417963e-16) (3.91836734694,-6.20920835927e-16) (4.04081632653,1.06501004578e-15) (4.16326530612,-1.55431223448e-15) (4.28571428571,-2.53184841771) (4.40816326531,-2.53184841771) (4.5306122449,-2.53184841771) (4.65306122449,-2.53184841771) (4.77551020408,-2.53184841771) (4.89795918367,-2.53184841771) (5.02040816327,-2.53184841771) (5.14285714286,-2.53184841771) (5.26530612245,-2.53184841771) (5.38775510204,-2.53184841771) (5.51020408163,-2.53184841771) (5.63265306122,-2.53184841771) (5.75510204082,-2.53184841771) (5.87755102041,-2.53184841771) (6.0,-2.53184841771) };
\addplot[Purple,solid,mark=|,very thick,mark size=3pt] coordinates{(0.0,2.53184841771) (0.0606060606061,2.53184841771) (0.121212121212,2.53184841771) (0.181818181818,2.53184841771) (0.242424242424,2.53184841771) (0.30303030303,2.53184841771) (0.363636363636,2.53184841771) (0.424242424242,2.53184841771) (0.484848484848,2.53184841771) (0.545454545455,2.53184841771) (0.606060606061,2.5318484126) (0.666666666667,2.5318484024) (0.727272727273,2.53184814716) (0.787878787879,2.53184766051) (0.848484848485,2.53184191121) (0.909090909091,2.53183149973) (0.969696969697,2.53175435774) (1.0303030303,2.53162222686) (1.09090909091,2.53093526428) (1.15151515152,2.52982750578) (1.21212121212,2.52554497911) (1.27272727273,2.5190772041) (1.33333333333,2.49986453919) (1.39393939394,2.47284604876) (1.45454545455,2.41004392635) (1.51515151515,2.32833992926) (1.57575757576,2.17856410096) (1.63636363636,1.99960310775) (1.69696969697,1.74120152762) (1.75757575758,1.45989548493) (1.81818181818,1.1436088819) (1.87878787879,0.832604842284) (1.93939393939,0.56736673057) (2.0,0.333863725319) (2.06060606061,0.190819286504) (2.12121212121,0.0788564224576) (2.18181818182,0.0356442580175) (2.24242424242,0.00543058051751) (2.30303030303,0.00126512374563) (2.36363636364,-0.0017052487095) (2.42424242424,-0.000645430085645) (2.48484848485,-0.000280633982574) (2.54545454545,-6.3163709523e-05) (2.60606060606,1.11412675525e-05) (2.66666666667,3.18661702975e-06) (2.72727272727,2.22400078625e-06) (2.78787878788,2.45878827279e-07) (2.84848484848,-4.85856639077e-08) (2.90909090909,-4.14411861477e-09) (2.9696969697,-1.97507262108e-09) (3.0303030303,1.97507197983e-09) (3.09090909091,4.14411817152e-09) (3.15151515152,4.85856640375e-08) (3.21212121212,-2.45878827371e-07) (3.27272727273,-2.2240007863e-06) (3.33333333333,-3.18661702944e-06) (3.39393939394,-1.11412675523e-05) (3.45454545455,6.31637095228e-05) (3.51515151515,0.000280633982573) (3.57575757576,0.000645430085645) (3.63636363636,0.0017052487095) (3.69696969697,-0.00126512374563) (3.75757575758,-0.00543058051751) (3.81818181818,-0.0356442580175) (3.87878787879,-0.0788564224576) (3.93939393939,-0.190819286504) (4.0,-0.333863725319) (4.06060606061,-0.56736673057) (4.12121212121,-0.832604842284) (4.18181818182,-1.1436088819) (4.24242424242,-1.45989548493) (4.30303030303,-1.74120152762) (4.36363636364,-1.99960310775) (4.42424242424,-2.17856410096) (4.48484848485,-2.32833992926) (4.54545454545,-2.41004392635) (4.60606060606,-2.47284604876) (4.66666666667,-2.49986453919) (4.72727272727,-2.5190772041) (4.78787878788,-2.52554497911) (4.84848484848,-2.52982750578) (4.90909090909,-2.53093526428) (4.9696969697,-2.53162222686) (5.0303030303,-2.53175435774) (5.09090909091,-2.53183149973) (5.15151515152,-2.53184191121) (5.21212121212,-2.53184766051) (5.27272727273,-2.53184814716) (5.33333333333,-2.5318484024) (5.39393939394,-2.5318484126) (5.45454545455,-2.53184841771) (5.51515151515,-2.53184841771) (5.57575757576,-2.53184841771) (5.63636363636,-2.53184841771) (5.69696969697,-2.53184841771) (5.75757575758,-2.53184841771) (5.81818181818,-2.53184841771) (5.87878787879,-2.53184841771) (5.93939393939,-2.53184841771) (6.0,-2.53184841771) };
\addplot[Green,only marks,mark=x,thick,mark size=3pt] coordinates{(0.0,2.53184841771) (0.0606060606061,2.53184841771) (0.121212121212,2.53184841771) (0.181818181818,2.53184841771) (0.242424242424,2.53184841771) (0.30303030303,2.53184841771) (0.363636363636,2.53184841771) (0.424242424242,2.53184841771) (0.484848484848,2.53184841771) (0.545454545455,2.53184841771) (0.606060606061,2.53184841771) (0.666666666667,2.53184841771) (0.727272727273,2.53184841771) (0.787878787879,2.53184841771) (0.848484848485,2.53184841771) (0.909090909091,2.53184841771) (0.969696969697,2.53184841771) (1.0303030303,2.53184841771) (1.09090909091,2.53184841771) (1.15151515152,2.53184841771) (1.21212121212,2.53184841771) (1.27272727273,2.53184841771) (1.33333333333,2.53184841771) (1.39393939394,2.53184841771) (1.45454545455,2.53184841771) (1.51515151515,2.53184841771) (1.57575757576,2.53184841771) (1.63636363636,2.53184841771) (1.69696969697,2.53184841771) (1.75757575758,2.53184841771) (1.81818181818,7.77156117238e-16) (1.87878787879,-2.10942374679e-15) (1.93939393939,-1.24284674418e-15) (2.0,-1.15443093114e-15) (2.06060606061,1.36969320734e-15) (2.12121212121,1.24761892825e-15) (2.18181818182,-2.44190935805e-15) (2.24242424242,-2.08538840883e-15) (2.30303030303,1.29436233137e-17) (2.36363636364,-1.465724152e-16) (2.42424242424,1.11111998695e-15) (2.48484848485,5.74810894758e-16) (2.54545454545,-1.82343883642e-16) (2.60606060606,-3.95374118094e-16) (2.66666666667,4.31804378728e-16) (2.72727272727,4.99576875162e-16) (2.78787878788,5.15062957257e-16) (2.84848484848,4.16318296633e-16) (2.90909090909,-5.30731420376e-16) (2.9696969697,-4.69865813603e-17) (3.0303030303,3.48504357156e-16) (3.09090909091,-2.14875565269e-16) (3.15151515152,7.26777152008e-16) (3.21212121212,9.15950895506e-16) (3.27272727273,-3.42770219967e-16) (3.33333333333,-4.54982242036e-16) (3.39393939394,-5.71760378957e-17) (3.45454545455,-4.30116006144e-16) (3.51515151515,2.73232265062e-16) (3.57575757576,-4.50063891139e-16) (3.63636363636,-3.95561893405e-16) (3.69696969697,-5.35819360485e-16) (3.75757575758,9.93808100673e-16) (3.81818181818,1.09300915669e-15) (3.87878787879,-7.51598182594e-16) (3.93939393939,-6.23872281146e-16) (4.0,3.77274813907e-16) (4.06060606061,5.10903605793e-16) (4.12121212121,-1.22124532709e-15) (4.18181818182,2.55351295664e-15) (4.24242424242,-2.53184841771) (4.30303030303,-2.53184841771) (4.36363636364,-2.53184841771) (4.42424242424,-2.53184841771) (4.48484848485,-2.53184841771) (4.54545454545,-2.53184841771) (4.60606060606,-2.53184841771) (4.66666666667,-2.53184841771) (4.72727272727,-2.53184841771) (4.78787878788,-2.53184841771) (4.84848484848,-2.53184841771) (4.90909090909,-2.53184841771) (4.9696969697,-2.53184841771) (5.0303030303,-2.53184841771) (5.09090909091,-2.53184841771) (5.15151515152,-2.53184841771) (5.21212121212,-2.53184841771) (5.27272727273,-2.53184841771) (5.33333333333,-2.53184841771) (5.39393939394,-2.53184841771) (5.45454545455,-2.53184841771) (5.51515151515,-2.53184841771) (5.57575757576,-2.53184841771) (5.63636363636,-2.53184841771) (5.69696969697,-2.53184841771) (5.75757575758,-2.53184841771) (5.81818181818,-2.53184841771) (5.87878787879,-2.53184841771) (5.93939393939,-2.53184841771) (6.0,-2.53184841771) };
\addplot[black,solid,mark=pentagone*,thin,mark size=3pt] coordinates{(0.0,2.53184841771) (0.122448979592,2.53184841771) (0.244897959184,2.53184841771) (0.367346938776,2.53184841771) (0.489795918367,2.53184841771) (0.612244897959,2.53184841771) (0.734693877551,2.53184841771) (0.857142857143,2.53184841771) (0.979591836735,2.53184841771) (1.10204081633,2.53184841771) (1.22448979592,2.53184841771) (1.34693877551,2.53184841771) (1.4693877551,2.53184841771) (1.59183673469,2.53184841771) (1.71428571429,2.53184841771) (1.83673469388,0.0) (1.95918367347,0.0) (2.08163265306,0.0) (2.20408163265,0.0) (2.32653061224,0.0) (2.44897959184,0.0) (2.57142857143,0.0) (2.69387755102,0.0) (2.81632653061,0.0) (2.9387755102,0.0) (3.0612244898,-0.0) (3.18367346939,-0.0) (3.30612244898,-0.0) (3.42857142857,-0.0) (3.55102040816,-0.0) (3.67346938776,-0.0) (3.79591836735,-0.0) (3.91836734694,-0.0) (4.04081632653,-0.0) (4.16326530612,-0.0) (4.28571428571,-2.53184841771) (4.40816326531,-2.53184841771) (4.5306122449,-2.53184841771) (4.65306122449,-2.53184841771) (4.77551020408,-2.53184841771) (4.89795918367,-2.53184841771) (5.02040816327,-2.53184841771) (5.14285714286,-2.53184841771) (5.26530612245,-2.53184841771) (5.38775510204,-2.53184841771) (5.51020408163,-2.53184841771) (5.63265306122,-2.53184841771) (5.75510204082,-2.53184841771) (5.87755102041,-2.53184841771) (6.0,-2.53184841771) };
\addlegendentry{usl 1ppc}
\addlegendentry{usl 2ppc}
\addlegendentry{usl-pic 2ppc}
\addlegendentry{dgmpm 1ppc}
\addlegendentry{dgmpm 2ppc}
\addlegendentry{dgmpm 2ppc (RK2)}
\addlegendentry{exact}

\end{groupplot}
\end{tikzpicture}
%%% Local Variables:
%%% mode: latex
%%% TeX-master: "../../mainManuscript"
%%% End:

  \caption{Stress (first row) and velocity (second row) solutions of the Riemann problem in an isotropic elastic at two different times (columns \subref{subfig:rp_elastic1} and \subref{subfig:rp_elastic2}). Comparison between DGMPM coupled with Euler or RK2 time integration, MPM-USL formulation using either PIC and FLIP mapping of the updated velocity, and the exact solution for an initial velocity set to $v_0=\frac{c}{200}$.}
  \label{fig:elastic_stress}
\end{figure}
First, since Courant number can be set at one for the DGMPM-Euler when 1ppc is used, the method allows to capture the discontinuities and yield solutions fiting perfectly the analytical ones. The same property holds for the DGMPM-RK2 with 2ppc while in that case the DGMPM-Euler exhibits a more restrictive stability condition that prevents the accurate resolution of waves. Nevertheless, as expected by the PIC mapping it uses, DGMPM formulations do not suffer from oscillations. 
%%
Second, this projection of updated fields from nodes to particles that allows to avoid the locking of velocity in the central region in USL solution also enables the correct assessment of the velocity with DGMPM schemes.
\begin{figure}[h!]
  \centering
  \begin{tikzpicture}[scale=0.9]
\begin{axis}[xlabel=$time (s)$,ylabel=$\frac{e}{e_{max}}$,ymajorgrids=true,xmajorgrids=true,legend pos=outer north east]%,title={(c) evolution of total energy $e$}]
\addplot[Red,very thick,mark=none,dashed,mark size=3pt] coordinates {(0.0,0.9876543209876545) (1.2090867953958061e-05,0.997530864197531) (2.4181735907916122e-05,1.0) (3.627260386187418e-05,0.9959104938271608) (4.8363471815832244e-05,0.9894000771604939) (6.045433976979031e-05,0.9840934847608025) (7.254520772374836e-05,0.9813625382788388) (8.463607567770642e-05,0.9805725733439128) (9.672694363166447e-05,0.9803534996362381) (0.00010881781158562253,0.9797373358436207) (0.00012090867953958059,0.97855223305006) (0.00013299954749353864,0.9771562700386213) (0.0001450904154474967,0.9759583476912668) (0.00015718128340145475,0.9751165956200767) (0.0001692721513554128,0.9745300840652308) (0.00018136301930937087,0.9740055540525709) (0.00019345388726332892,0.9734177949795755) (0.00020554475521728698,0.9727609779118863) (0.00021763562317124503,0.9721011582864625) (0.0002297264911252031,0.971500815494294) (0.00024181735907916115,0.9709763284318171) (0.0002539082270331192,0.9705033540228475) (0.0002659990949870773,0.9700475116254563) (0.00027808996294103537,0.9695899456257216) (0.00029018083089499345,0.9691327035661251) (0.00030227169884895153,0.96868819213753) (0.0003143625668029096,0.9682659231600772) (0.0003264534347568677,0.9678664411889574) (0.0003385443027108258,0.9674837019503967) (0.00035063517066478387,0.9671110350826551) (0.00036272603861874195,0.9667453037400304) (0.00037481690657270003,0.9663871533998759) (0.0003869077745266581,0.9660386728300239) (0.0003989986424806162,0.9657010208032414) (0.0004110895104345743,0.9653736208998025) (0.00042318037838853236,0.9650548630015646) (0.00043527124634249045,0.9647432582700906) (0.00044736211429644853,0.9644380781342287) (0.0004594529822504066,0.964139187671229) (0.0004715438502043647,0.9638463458344536) (0.0004836347181583228,0.9635583008484584) (0.0004957255861122808,0.9632716356128425) (0.0005078164540662388,0.9629788582366817) (0.0005199073220201969,0.9626650352466273) (0.0005319981899741549,0.9623025558244344) (0.0005440890579281129,0.9618444966372697) (0.000556179925882071,0.9612184911067746) (0.000568270793836029,0.9603246926454502) (0.000580361661789987,0.959042628998641) (0.000592452529743945,0.9572513419900455) };
\addplot[Duck,thin,mark=*,solid,mark size=2pt] coordinates {(0.0,1.0) (1.2090867953958061e-05,0.9799999999999998) (2.4181735907916122e-05,0.97) (3.627260386187418e-05,0.9624999999999999) (4.8363471815832244e-05,0.9562499999999999) (6.045433976979031e-05,0.9507812499999999) (7.254520772374836e-05,0.9458593750000001) (8.463607567770642e-05,0.94134765625) (9.672694363166447e-05,0.937158203125) (0.00010881781158562253,0.9332305908203123) (0.00012090867953958059,0.9295211791992186) (0.00013299954749353864,0.9259972381591796) (0.0001450904154474967,0.9226334762573243) (0.00015718128340145475,0.9194098711013794) (0.0001692721513554128,0.9163102507591249) (0.00018136301930937087,0.9133213311433792) (0.00019345388726332892,0.9104320421814918) (0.00020554475521728698,0.9076330434996636) (0.00021763562317124503,0.9049163683084772) (0.0002297264911252031,0.902275156317046) (0.00024181735907916115,0.8997034499043367) (0.0002539082270331192,0.8971960361519449) (0.0002659990949870773,0.8947483227269913) (0.00027808996294103537,0.8923562391526049) (0.00029018083089499345,0.8900161573950528) (0.00030227169884895153,0.887724827340783) (0.0003143625668029096,0.8854793238875988) (0.0003264534347568677,0.8832770031931295) (0.0003385443027108258,0.8811154662152247) (0.00035063517066478387,0.8789925281119251) (0.00036272603861874195,0.8769061923897169) (0.00037481690657270003,0.8748546289295456) (0.0003869077745266581,0.8728361552026028) (0.0003989986424806162,0.8708492201276435) (0.0004110895104345743,0.8688923901295775) (0.00042318037838853236,0.8669643370432477) (0.00043527124634249045,0.865063827572437) (0.00044736211429644853,0.8631897140664987) (0.0004594529822504066,0.8613409264187486) (0.0004715438502043647,0.8595164649242585) (0.0004836347181583228,0.8577153939617489) (0.0004957255861122808,0.8559368363862708) (0.0005078164540662388,0.8541799685373229) (0.0005199073220201969,0.8524440157818148) (0.0005319981899741549,0.8507282485234637) (0.0005440890579281129,0.8490319786203213) (0.000556179925882071,0.8473545561605471) (0.000568270793836029,0.8456953665535965) (0.000580361661789987,0.8440538278999112) (0.000592452529743945,0.842429388607202) };
\addplot[Orange,very thick,mark=none,densely dotted,mark size=3pt] coordinates {(0.0,0.9873495834618945) (1.1968737974625153e-05,0.9972230792965134) (2.3937475949250306e-05,1.0) (3.590621392387546e-05,0.996543312249306) (4.787495189850061e-05,0.9916724016989357) (5.9843689873125764e-05,0.9886458260672434) (7.181242784775092e-05,0.9877380525503068) (8.378116582237607e-05,0.9876316182265924) (9.574990379700122e-05,0.9872072095327306) (0.00010771864177162638,0.9862586211196859) (0.00011968737974625153,0.9851624959469047) (0.00013165611772087668,0.9842780087964154) (0.00014362485569550183,0.9836648835656183) (0.000155593593670127,0.9831763602119745) (0.00016756233164475214,0.9826698995825812) (0.0001795310696193773,0.9821135512367487) (0.00019149980759400245,0.9815547588976113) (0.0002034685455686276,0.9810419341151472) (0.00021543728354325275,0.9805830603182993) (0.0002274060215178779,0.9801564013202002) (0.00023937475949250306,0.9797401327914591) (0.0002513434974671282,0.979328388660163) (0.00026331223544175336,0.9789272663692082) (0.0002752809734163785,0.9785432962021172) (0.00028724971139100367,0.9781770579069166) (0.0002992184493656288,0.9778246343106916) (0.000311187187340254,0.9774820931955118) (0.0003231559253148791,0.9771479959874065) (0.0003351246632895043,0.9768228205666122) (0.00034709340126412943,0.9765071704950676) (0.0003590621392387546,0.9762007565377634) (0.00037103087721337974,0.97590260746931) (0.0003829996151880049,0.9756117726801273) (0.00039496835316263004,0.9753277223895899) (0.0004069370911372552,0.9750502574593737) (0.00041890582911188035,0.9747792209216646) (0.0004308745670865055,0.9745143288096212) (0.00044284330506113066,0.9742551920745642) (0.0004548120430357558,0.9740013948706343) (0.00046678078101038096,0.9737524481523462) (0.0004787495189850061,0.9735074554284332) (0.0004907182569596313,0.9732642558608436) (0.0005026869949342565,0.9730175918911446) (0.0005146557329088817,0.972755592971648) (0.0005266244708835069,0.9724538550537384) (0.0005385932088581321,0.9720670387858211) (0.0005505619468327574,0.971519585751107) (0.0005625306848073826,0.9706998361345011) (0.0005744994227820078,0.9694646058920318) (0.000586468160756633,0.9676621151764617) };
\addplot[Blue,very thick,mark=none,solid,mark size=3pt] coordinates {(0.0,0.9999999999999998) (2.4181735907916125e-05,0.9999999999999998) (4.836347181583225e-05,0.9999999999999998) (7.254520772374838e-05,0.9999999999999998) (9.67269436316645e-05,0.9999999999999998) (0.00012090867953958063,0.9999999999999998) (0.00014509041544749675,1.0) (0.0001692721513554129,0.9999999999999998) (0.000193453887263329,0.9999999999999998) (0.00021763562317124512,0.9999999999999998) (0.00024181735907916123,0.9999999999999998) (0.00026599909498707734,0.9999999999999997) (0.00029018083089499345,0.9999999999999997) (0.00031436256680290956,0.9999999999999997) (0.0003385443027108257,0.9999999999999997) (0.0003627260386187418,0.9999999999999997) (0.0003869077745266579,0.9999999999999997) (0.000411089510434574,0.9999999999999997) (0.0004352712463424901,0.9999999999999994) (0.00045945298225040623,0.9999999999999997) (0.00048363471815832235,0.9999999999999994) (0.0005078164540662385,0.9999999999999992) (0.0005319981899741547,0.9999999999999992) (0.0005561799258820708,0.9999999999999992) (0.000580361661789987,0.9999999999999992) (0.0006045433976979032,0.9999999999999992) };
\addplot[Purple,very thick,mark=|,solid,mark size=3pt] coordinates {(0.0,1.0) (1.1968737974625153e-05,0.9849999999999999) (2.3937475949250306e-05,0.9782812499999997) (3.590621392387546e-05,0.9728686523437499) (4.787495189850061e-05,0.9685175323486328) (5.9843689873125764e-05,0.9646764224767683) (7.181242784775092e-05,0.9612245353125033) (8.378116582237607e-05,0.9580584139595156) (9.574990379700122e-05,0.9551170218177228) (0.00010771864177162638,0.9523583400291113) (0.00011968737974625153,0.9497518470633621) (0.00013165611772087668,0.9472747764554426) (0.00014362485569550183,0.9449095127565772) (0.000155593593670127,0.942642119512832) (0.00016756233164475214,0.9404613389338139) (0.0001795310696193773,0.9383579222991295) (0.00019149980759400245,0.9363241607011475) (0.0002034685455686276,0.9343535480062032) (0.00021543728354325275,0.9324405332414724) (0.0002274060215178779,0.9305803349197901) (0.00023937475949250306,0.928768799378865) (0.0002513434974671282,0.9270022910038743) (0.00026331223544175336,0.9252776059761404) (0.0002752809734163785,0.9235919036615065) (0.00028724971139100367,0.9219426514179895) (0.0002992184493656288,0.9203275797465361) (0.000311187187340254,0.9187446455091373) (0.0003231559253148791,0.9171920015078542) (0.0003351246632895043,0.915667971129286) (0.00034709340126412943,0.914171027059833) (0.0003590621392387546,0.9126997733000616) (0.00037103087721337974,0.9112529298736883) (0.0003829996151880049,0.9098293197534216) (0.00039496835316263004,0.9084278576229242) (0.0004069370911372552,0.9070475401691336) (0.00041890582911188035,0.9056874376575983) (0.0004308745670865055,0.9043466865894064) (0.00044284330506113066,0.9030244832746345) (0.0004548120430357558,0.9017200781862146) (0.00046678078101038096,0.9004327709813899) (0.0004787495189850061,0.8991619060967203) (0.0004907182569596313,0.897906868837865) (0.0005026869949342565,0.8966670818978505) (0.0005146557329088817,0.8954420022477787) (0.0005266244708835069,0.8942311183524017) (0.0005385932088581321,0.8930339476700012) (0.0005505619468327574,0.8918500344018717) (0.0005625306848073826,0.8906789474616031) (0.0005744994227820078,0.8895202786384733) (0.000586468160756633,0.8883736409327396) (0.0005984368987312582,0.8872386670435604) };
\addplot[Green,thick,mark=x,only marks,mark size=3pt] coordinates {(0.0,1.0) (2.3937475949250306e-05,0.9999999999999997) (4.787495189850061e-05,0.9999999999999998) (7.181242784775092e-05,0.9999999999999998) (9.574990379700122e-05,0.9999999999999998) (0.00011968737974625153,0.9999999999999998) (0.00014362485569550183,0.9999999999999998) (0.00016756233164475214,0.9999999999999997) (0.00019149980759400245,0.9999999999999997) (0.00021543728354325275,0.9999999999999994) (0.00023937475949250306,0.9999999999999994) (0.00026331223544175336,0.9999999999999992) (0.00028724971139100367,0.999999999999999) (0.000311187187340254,0.999999999999999) (0.0003351246632895043,0.999999999999999) (0.0003590621392387546,0.999999999999999) (0.0003829996151880049,0.999999999999999) (0.0004069370911372552,0.999999999999999) (0.0004308745670865055,0.999999999999999) (0.0004548120430357558,0.999999999999999) (0.0004787495189850061,0.9999999999999989) (0.0005026869949342564,0.999999999999999) (0.0005266244708835067,0.999999999999999) (0.000550561946832757,0.9999999999999989) (0.0005744994227820073,0.9999999999999989) (0.0005984368987312576,0.999999999999999) };
\addplot[black,thin,mark=none,solid,mark size=3pt] coordinates {(0.0,1.0) (1e-08,1.0) };
\legend{usl 1ppc,usl-pic 1ppc,usl 2ppc,dgmpm 1ppc,dgmpm 2ppc,dgmpm 2ppc (RK2),exact}
\end{axis}
\end{tikzpicture}
%%% Local Variables:
%%% mode: latex
%%% TeX-master: "../../mainManuscript"
%%% End:

  \caption{Evolution of total energy $e$ for DGMPM and MPM-USL solutions on the Riemann problem in an elastic bar.}
  \label{fig:energy_elastic_RP}
\end{figure}
%%
At last, the introduction of DG approximation within the USL-PIC leads to a reduction of numerical diffusion, though less significant than that permitted by using FLIP mapping as originally proposed for the MPM. This can be seen in figure \ref{fig:energy_elastic_RP} in which the evolution of total energy is depicted for every methods.
One can also see in this figure that the situations for which the CFL number is set to unity for DGMPM formulations, leads to an exact conservation of the total energy during the computation.
%%% Local Variables:
%%% mode: latex
%%% TeX-master: "../mainManuscript"
%%% End:
