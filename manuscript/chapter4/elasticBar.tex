To begin with, let us focus on the problem that has been used in section \ref{sec:MPM} to illustrate shortcomings of the MPM and that motivated the derivation of the DGMPM.
We thus consider an infinite medium in directions $\vect{e}_2$ and $\vect{e}_3$ and of length $L=6\:m$ in direction $\vect{e}_1$. The Cauchy stress and infinitesimal strain tensor are of the form:
\begin{align*}
  & \tens{\sigma} = \sigma \: \vect{e}_1\otimes \vect{e}_1 \\
  & \tens{\eps} = \eps \: \vect{e}_1\otimes \vect{e}_1
\end{align*}
so that the bar assumption holds. Riemann-type initial conditions on the horizontal velocity $v=\vect{v}\cdot\vect{e}_1$ are prescribed in the bar, that is, $v=v_0>0$ for $x_1\in\[0,L/2\]$ and $v=-v_0$ for $x_1\in \:]L/2,L]$. In addition, both ends of the domain are traction free. The bar is assumed elastic with density $\rho=7800 \: kg.m^{-3}$, Poisson ration $\nu=0.3$ and Young's modulus $E=2\times 10^{11}\:Pa$.
The exact solution of this problem \cite[Ch.1]{Wang} has been recalled in section \ref{subsec:charac_Linear_problems} and consists of two elastic discontinuities propagating left and rightward in the bar at constant speeds $c=\pm\sqrt{E/\rho}$, and reflect on the free ends of the bar. 

The domain is discretized with material points in a grid made of $50$ regular cells which contain either one centered particle or two material points symmetrically placed with respect to element centers. Figure \ref{fig:elastic_stress} shows a comparison of stress profiles provided by DGMPM schemes and MPM-USL formulations with the exact solution for an initial velocity set to $v_0=\frac{c}{200}$. 
\begin{figure}[h!]
  \centering
  {\phantomsubcaption \label{subfig:rp_stress1}}
  {\phantomsubcaption \label{subfig:rp_stress2}}
  \begin{tikzpicture}[scale=0.9]
  \begin{groupplot}[group style={group size=2 by 1,
      ylabels at=edge left, yticklabels at=edge left,
      horizontal sep=4.ex,
      vertical sep=2ex,},
    ymajorgrids=true,xmajorgrids=true,
    enlargelimits=0,
    xmin=0.,xmax=6.,
    ylabel=$\sigma (Pa)$,
    xlabel=$x (m)$,
    axis on top,scale only axis,width=0.45\linewidth%,xtick=\empty,ytick=\empty,width=0.25\linewidth,
    ]
    %% FIRST PLOT
    \nextgroupplot[title={(a) time $t = 1.21\times 10^{-4} $ s.}]
    \addplot[Red,very thick,mark=none,dashed,mark size=3pt] coordinates {(0.0,-1.7182156787438973e-08) (0.12244897959183673,4.295539196859743e-08) (0.24489795918367346,1.718215678743897e-08) (0.36734693877551017,8.591078393719486e-09) (0.4897959183673469,-1.7182156787438973e-08) (0.6122448979591837,-3.4364313574877946e-08) (0.7346938775510203,-2.5773235181158458e-08) (0.8571428571428571,0.0) (0.9795918367346939,0.0) (1.1020408163265305,-1.7182156787438973e-08) (1.2244897959183674,1.7182156787438973e-08) (1.346938775510204,0.0) (1.4693877551020407,0.0) (1.5918367346938775,0.0) (1.7142857142857142,0.0) (1.836734693877551,-10490.417480468344) (1.9591836734693877,-180244.44580078288) (2.0816326530612246,-1426887.5122070103) (2.204081632653061,-6905937.1948242355) (2.326530612244898,-22447967.529296968) (2.4489795918367347,-53339385.9863282) (2.571428571428571,-87870025.6347658) (2.693877551020408,-120363616.94335955) (2.816326530612245,-112137222.29003876) (2.9387755102040813,-95318222.04589848) (3.061224489795918,-95318222.04589835) (3.183673469387755,-112137222.29003933) (3.306122448979592,-120363616.94335933) (3.4285714285714284,-87870025.63476545) (3.5510204081632653,-53339385.98632792) (3.673469387755102,-22447967.529296756) (3.7959183673469385,-6905937.194824192) (3.9183673469387754,-1426887.512207019) (4.040816326530612,-180244.44580077427) (4.163265306122449,-10490.417480468344) (4.285714285714286,0.0) (4.408163265306122,0.0) (4.530612244897959,0.0) (4.653061224489796,0.0) (4.775510204081632,1.7182156787438973e-08) (4.8979591836734695,-1.7182156787438973e-08) (5.020408163265306,0.0) (5.142857142857142,0.0) (5.26530612244898,0.0) (5.387755102040816,0.0) (5.5102040816326525,0.0) (5.63265306122449,0.0) (5.755102040816326,0.0) (5.877551020408163,0.0) (6.0,0.0) };
\addplot[Duck,thin,mark=*,solid,mark size=3pt] coordinates {(0.0,8.591078393719486e-09) (0.12244897959183673,7.731970554347537e-08) (0.24489795918367346,9.450186233091434e-08) (0.36734693877551017,2.5773235181158448e-08) (0.4897959183673469,-2.5773235181158458e-08) (0.6122448979591837,-7.731970554347535e-08) (0.7346938775510203,-9.450186233091434e-08) (0.8571428571428571,-5.1546470362316915e-08) (0.9795918367346939,-2.5773235181158458e-08) (1.1020408163265305,-1.7182156787438973e-08) (1.2244897959183674,1.7182156787438973e-08) (1.346938775510204,0.0) (1.4693877551020407,0.0) (1.5918367346938775,0.0) (1.7142857142857142,0.0) (1.836734693877551,-97656.24999999622) (1.9591836734693877,-1074218.75000001) (2.0816326530612246,-5468750.0000000205) (2.204081632653061,-17187500.00000005) (2.326530612244898,-37695312.50000009) (2.4489795918367347,-62304687.500000015) (2.571428571428571,-82812500.00000007) (2.693877551020408,-94531250.00000004) (2.816326530612245,-98925781.24999997) (2.9387755102040813,-99902343.74999999) (3.061224489795918,-99902343.75000001) (3.183673469387755,-98925781.24999999) (3.306122448979592,-94531250.0) (3.4285714285714284,-82812499.99999993) (3.5510204081632653,-62304687.49999988) (3.673469387755102,-37695312.499999896) (3.7959183673469385,-17187499.999999933) (3.9183673469387754,-5468749.99999996) (4.040816326530612,-1074218.7500000014) (4.163265306122449,-97656.24999999622) (4.285714285714286,0.0) (4.408163265306122,0.0) (4.530612244897959,0.0) (4.653061224489796,0.0) (4.775510204081632,1.7182156787438973e-08) (4.8979591836734695,-1.7182156787438973e-08) (5.020408163265306,0.0) (5.142857142857142,0.0) (5.26530612244898,0.0) (5.387755102040816,0.0) (5.5102040816326525,0.0) (5.63265306122449,0.0) (5.755102040816326,0.0) (5.877551020408163,0.0) (6.0,0.0) };
\addplot[Orange,very thick,mark=none,densely dotted,mark size=3pt] coordinates {(0.0,0.0) (0.06060606060606061,0.0) (0.12121212121212122,0.0) (0.18181818181818182,0.0) (0.24242424242424243,0.0) (0.30303030303030304,0.0) (0.36363636363636365,0.0) (0.42424242424242425,0.0) (0.48484848484848486,0.0) (0.5454545454545454,0.0) (0.6060606060606061,8.504299824085957e-09) (0.6666666666666667,8.504299824085957e-09) (0.7272727272727273,-1.7008599648171914e-08) (0.7878787878787878,-1.7008599648171914e-08) (0.8484848484848485,0.0) (0.9090909090909092,0.0) (0.9696969696969697,-2.551289947225787e-08) (1.0303030303030303,-2.551289947225787e-08) (1.0909090909090908,0.0) (1.1515151515151516,0.0) (1.2121212121212122,1.7008599648171914e-08) (1.2727272727272727,1.7008599648171914e-08) (1.3333333333333335,-8.504299824085957e-09) (1.393939393939394,-8.504299824085957e-09) (1.4545454545454546,-8.504299824085957e-09) (1.5151515151515151,-8.504299824085957e-09) (1.5757575757575757,8.504299824085957e-09) (1.6363636363636365,8.504299824085957e-09) (1.696969696969697,1.7008599648171914e-08) (1.7575757575757576,1.7008599648171914e-08) (1.8181818181818183,-5499.273538584786) (1.878787878787879,-5499.273538584786) (1.9393939393939394,-115492.194890958) (2.0,-115492.194890958) (2.0606060606060606,-1086898.8931179084) (2.121212121212121,-1086898.8931179084) (2.1818181818181817,-6038592.01073648) (2.2424242424242427,-6038592.01073648) (2.303030303030303,-21882501.244545024) (2.3636363636363638,-21882501.244545024) (2.4242424242424243,-54484376.31130224) (2.484848484848485,-54484376.31130224) (2.5454545454545454,-94001361.72771455) (2.606060606060606,-94001361.72771455) (2.666666666666667,-117371180.65357204) (2.7272727272727275,-117371180.65357204) (2.787878787878788,-111648218.33372112) (2.8484848484848486,-111648218.33372112) (2.909090909090909,-93365879.35686114) (2.9696969696969697,-93365879.35686114) (3.0303030303030303,-93365879.35686119) (3.090909090909091,-93365879.35686119) (3.1515151515151514,-111648218.33372112) (3.2121212121212124,-111648218.33372112) (3.272727272727273,-117371180.6535721) (3.3333333333333335,-117371180.6535721) (3.393939393939394,-94001361.72771455) (3.4545454545454546,-94001361.72771455) (3.515151515151515,-54484376.3113022) (3.5757575757575757,-54484376.3113022) (3.6363636363636367,-21882501.24454498) (3.6969696969696972,-21882501.24454498) (3.757575757575758,-6038592.01073648) (3.8181818181818183,-6038592.01073648) (3.878787878787879,-1086898.8931178998) (3.9393939393939394,-1086898.8931178998) (4.0,-115492.19489097499) (4.0606060606060606,-115492.19489097499) (4.121212121212121,-5499.273538601796) (4.181818181818182,-5499.273538601796) (4.242424242424242,-8.504299824085957e-09) (4.303030303030303,-8.504299824085957e-09) (4.363636363636363,1.7008599648171914e-08) (4.424242424242425,1.7008599648171914e-08) (4.484848484848485,-8.504299824085957e-09) (4.545454545454546,-8.504299824085957e-09) (4.606060606060606,0.0) (4.666666666666667,0.0) (4.7272727272727275,0.0) (4.787878787878788,0.0) (4.848484848484849,0.0) (4.909090909090909,0.0) (4.96969696969697,0.0) (5.03030303030303,0.0) (5.090909090909091,0.0) (5.151515151515151,0.0) (5.212121212121212,0.0) (5.2727272727272725,0.0) (5.333333333333334,0.0) (5.3939393939393945,0.0) (5.454545454545455,0.0) (5.515151515151516,0.0) (5.575757575757576,0.0) (5.636363636363637,0.0) (5.696969696969697,0.0) (5.757575757575758,0.0) (5.818181818181818,0.0) (5.878787878787879,0.0) (5.9393939393939394,0.0) (6.0,0.0) };
\addplot[Blue,very thick,mark=none,solid,mark size=3pt] coordinates {(0.0,2.205276437763443e-23) (0.12244897959183673,-4.821391029817259e-08) (0.24489795918367346,-3.616043272362948e-08) (0.36734693877551017,2.4106955149086293e-08) (0.4897959183673469,-3.616043272362948e-08) (0.6122448979591837,-2.4106955149086323e-08) (0.7346938775510203,1.2053477574543153e-08) (0.8571428571428571,-1.205347757454316e-08) (0.9795918367346939,1.205347757454315e-08) (1.1020408163265305,1.2053477574543135e-08) (1.2244897959183674,2.4106955149086313e-08) (1.346938775510204,2.41069551490863e-08) (1.4693877551020407,-6.3007898221812655e-24) (1.5918367346938775,-6.3007898221812655e-24) (1.7142857142857142,1.2053477574543152e-08) (1.836734693877551,-2.4106955149086303e-08) (1.9591836734693877,0.0) (2.0816326530612246,0.0) (2.204081632653061,1.5751974555453164e-23) (2.326530612244898,1.5751974555453164e-23) (2.4489795918367347,-99999999.99999999) (2.571428571428571,-100000000.00000003) (2.693877551020408,-100000000.0) (2.816326530612245,-99999999.99999999) (2.9387755102040813,-100000000.0) (3.061224489795918,-99999999.99999997) (3.183673469387755,-100000000.0) (3.306122448979592,-99999999.99999999) (3.4285714285714284,-99999999.99999999) (3.5510204081632653,-99999999.99999999) (3.673469387755102,-1.2601579644362531e-23) (3.7959183673469385,1.2053477574543145e-08) (3.9183673469387754,1.2053477574543147e-08) (4.040816326530612,-3.6160432723629465e-08) (4.163265306122449,1.4176777099907847e-23) (4.285714285714286,-1.4176777099907847e-23) (4.408163265306122,-2.410695514908632e-08) (4.530612244897959,-1.2053477574543148e-08) (4.653061224489796,1.2053477574543138e-08) (4.775510204081632,-1.8902369466543796e-23) (4.8979591836734695,-1.2053477574543148e-08) (5.020408163265306,-2.4106955149086316e-08) (5.142857142857142,3.616043272362944e-08) (5.26530612244898,-1.2053477574543148e-08) (5.387755102040816,-1.8902369466543796e-23) (5.5102040816326525,2.410695514908629e-08) (5.63265306122449,-1.8902369466543796e-23) (5.755102040816326,-1.2053477574543173e-08) (5.877551020408163,2.410695514908629e-08) (6.0,-1.5751974555453164e-23) };
\addplot[Purple,very thick,mark=|,solid,mark size=3pt] coordinates {(0.0,1.1954895665669321e-11) (0.06060606060606061,-1.1954895665669323e-11) (0.12121212121212122,-9.040062200539418e-09) (0.18181818181818182,-1.5066892948546882e-08) (0.24242424242424243,-2.0148953581298875e-08) (0.30303030303030304,-4.011843429141689e-08) (0.36363636363636365,-3.9801572080816714e-08) (0.42424242424242425,-4.457277094098534e-08) (0.48484848484848486,-3.051726216575409e-08) (0.5454545454545454,-2.975012570696168e-08) (0.6060606060606061,-1.895467099939049e-08) (0.6666666666666667,-5.152284149695791e-09) (0.7272727272727273,-1.179037790916254e-08) (0.7878787878787878,-1.2316577239923773e-08) (0.8484848484848485,-2.5473583645144106e-08) (0.9090909090909092,1.3666284960577943e-09) (0.9696969696969697,3.645346560854217e-08) (1.0303030303030303,2.381392226417357e-08) (1.0909090909090908,2.4282772581015557e-08) (1.1515151515151516,2.3931137717157056e-08) (1.2121212121212122,1.2243410979431472e-08) (1.2727272727272727,1.1863544169654833e-08) (1.3333333333333335,1.2583171413268185e-08) (1.393939393939394,1.1523783735818112e-08) (1.4545454545454546,2.7371193430229355e-08) (1.5151515151515151,2.0842716867943245e-08) (1.5757575757575757,9.624966966074275e-09) (1.6363636363636365,1.448198818301203e-08) (1.696969696969697,2.104809967646104e-08) (1.7575757575757576,1.511233304716842e-08) (1.8181818181818183,-3666.244447226061) (1.878787878787879,-10998.733341690284) (1.9393939393939394,-100618.04205178331) (2.0,-262747.51871824573) (2.0606060606060606,-1140066.2362575496) (2.121212121212121,-2551162.9879474747) (2.1818181818181817,-6952185.183763518) (2.2424242424242427,-13110550.493001966) (2.303030303030303,-25132333.114743244) (2.3636363636363638,-39370855.31651974) (2.4242424242424243,-56894854.8287153) (2.484848484848485,-73865127.93600555) (2.5454545454545454,-86159824.57995412) (2.606060606060606,-95498106.62865634) (2.666666666666667,-98513982.44500154) (2.7272727272727275,-100261419.26646227) (2.787878787878788,-100094961.18873355) (2.8484848484848486,-100070640.63102004) (2.909090909090909,-100007508.13633199) (2.9696969696969697,-99998390.4883265) (3.0303030303030303,-99998390.48832652) (3.090909090909091,-100007508.136332) (3.1515151515151514,-100070640.63102002) (3.2121212121212124,-100094961.18873355) (3.272727272727273,-100261419.2664623) (3.3333333333333335,-98513982.44500159) (3.393939393939394,-95498106.62865636) (3.4545454545454546,-86159824.57995415) (3.515151515151515,-73865127.93600555) (3.5757575757575757,-56894854.8287153) (3.6363636363636367,-39370855.31651971) (3.6969696969696972,-25132333.11474323) (3.757575757575758,-13110550.493001949) (3.8181818181818183,-6952185.183763516) (3.878787878787879,-2551162.987947458) (3.9393939393939394,-1140066.2362575415) (4.0,-262747.51871824026) (4.0606060606060606,-100618.04205180079) (4.121212121212121,-10998.73334169629) (4.181818181818182,-3666.2444472321054) (4.242424242424242,6.026554865799796e-09) (4.303030303030303,6.026922708743356e-09) (4.363636363636363,2.0086063933044116e-09) (4.424242424242425,-1.4062083967847561e-08) (4.484848484848485,0.0) (4.545454545454546,0.0) (4.606060606060606,0.0) (4.666666666666667,0.0) (4.7272727272727275,0.0) (4.787878787878788,0.0) (4.848484848484849,0.0) (4.909090909090909,0.0) (4.96969696969697,0.0) (5.03030303030303,0.0) (5.090909090909091,0.0) (5.151515151515151,0.0) (5.212121212121212,0.0) (5.2727272727272725,0.0) (5.333333333333334,0.0) (5.3939393939393945,0.0) (5.454545454545455,0.0) (5.515151515151516,0.0) (5.575757575757576,0.0) (5.636363636363637,0.0) (5.696969696969697,0.0) (5.757575757575758,0.0) (5.818181818181818,0.0) (5.878787878787879,0.0) (5.9393939393939394,-6.0267272921795895e-09) (6.0,-6.026750282363563e-09) };
\addplot[Green,thick,mark=x,only marks,mark size=3pt] coordinates {(0.0,-2.1093585755450518e-08) (0.06060606060606061,-3.0133693936357875e-09) (0.12121212121212122,-2.504863308459749e-08) (0.18181818181818182,-2.316527721357512e-08) (0.24242424242424243,-1.40780851358922e-08) (0.30303030303030304,-1.0028870013194108e-08) (0.36363636363636365,-4.3717398156106705e-08) (0.42424242424242425,-2.8603467291152215e-08) (0.48484848484848486,7.156752309884969e-09) (0.5454545454545454,1.6950202839201318e-08) (0.6060606060606061,-4.3693856207718916e-08) (0.6666666666666667,-5.2733964388626276e-08) (0.7272727272727273,3.3523734504198126e-08) (0.7878787878787878,3.87971309430608e-08) (0.8484848484848485,-1.4878511381076717e-08) (0.9090909090909092,-9.228443768009598e-09) (0.9696969696969697,5.57002498854865e-08) (1.0303030303030303,8.894148100903133e-08) (1.0909090909090908,-5.925508409204129e-08) (1.1515151515151516,-3.7172736504303915e-08) (1.2121212121212122,2.895659651696907e-09) (1.2727272727272727,2.1211295497389412e-08) (1.3333333333333335,3.926796991081633e-08) (1.393939393939394,8.945940387356272e-09) (1.4545454545454546,-2.0104823923163776e-08) (1.5151515151515151,-2.810908637500884e-08) (1.5757575757575757,5.155686696923746e-09) (1.6363636363636365,-5.155686696923745e-09) (1.696969696969697,1.909252014248924e-08) (1.7575757575757576,5.014435006597071e-09) (1.8181818181818183,-1.0546792877725262e-08) (1.878787878787879,-1.3560162271361046e-08) (1.9393939393939394,-4.8967252646581575e-09) (2.0,-1.9210229884428146e-08) (2.0606060606060606,6.191532425986014e-09) (2.121212121212121,-6.191532425986031e-09) (2.1818181818181817,-1.7774171032773557e-08) (2.2424242424242427,-3.043973926539904e-08) (2.303030303030303,1.7609377394059133e-08) (2.3636363636363638,6.4975777550271785e-09) (2.4242424242424243,-99999999.99999994) (2.484848484848485,-100000000.00000006) (2.5454545454545454,-99999999.99999997) (2.606060606060606,-99999999.99999996) (2.666666666666667,-99999999.99999994) (2.7272727272727275,-99999999.99999994) (2.787878787878788,-100000000.00000001) (2.8484848484848486,-100000000.00000001) (2.909090909090909,-99999999.99999997) (2.9696969696969697,-99999999.99999999) (3.0303030303030303,-99999999.99999996) (3.090909090909091,-99999999.99999997) (3.1515151515151514,-99999999.99999996) (3.2121212121212124,-99999999.99999997) (3.272727272727273,-99999999.99999996) (3.3333333333333335,-99999999.99999999) (3.393939393939394,-99999999.99999999) (3.4545454545454546,-100000000.0) (3.515151515151515,-99999999.99999993) (3.5757575757575757,-100000000.00000009) (3.6363636363636367,0.0) (3.6969696969696972,0.0) (3.757575757575758,0.0) (3.8181818181818183,0.0) (3.878787878787879,0.0) (3.9393939393939394,0.0) (4.0,-6.026738787271577e-09) (4.0606060606060606,-1.8080216361814732e-08) (4.121212121212121,1.8080216361814716e-08) (4.181818181818182,6.0267387872715865e-09) (4.242424242424242,-1.6479363871445125e-10) (4.303030303030303,-2.3942161510371846e-08) (4.363636363636363,2.2105889536125074e-08) (4.424242424242425,2.610802076204755e-08) (4.484848484848485,-1.101763184548084e-08) (4.545454545454546,-1.308932330360547e-08) (4.606060606060606,2.4106955149086287e-08) (4.666666666666667,2.4106955149086323e-08) (4.7272727272727275,-2.4106955149086296e-08) (4.787878787878788,-2.410695514908632e-08) (4.848484848484849,0.0) (4.909090909090909,0.0) (4.96969696969697,0.0) (5.03030303030303,0.0) (5.090909090909091,0.0) (5.151515151515151,0.0) (5.212121212121212,0.0) (5.2727272727272725,0.0) (5.333333333333334,0.0) (5.3939393939393945,0.0) (5.454545454545455,0.0) (5.515151515151516,0.0) (5.575757575757576,0.0) (5.636363636363637,0.0) (5.696969696969697,0.0) (5.757575757575758,0.0) (5.818181818181818,0.0) (5.878787878787879,0.0) (5.9393939393939394,0.0) (6.0,0.0) };
\addplot[black,thin,mark=none,solid,mark size=3pt] coordinates {(0.0,-0.0) (0.12244897959183673,-0.0) (0.24489795918367346,-0.0) (0.36734693877551017,-0.0) (0.4897959183673469,-0.0) (0.6122448979591837,-0.0) (0.7346938775510203,-0.0) (0.8571428571428571,-0.0) (0.9795918367346939,-0.0) (1.1020408163265305,-0.0) (1.2244897959183674,-0.0) (1.346938775510204,-0.0) (1.4693877551020407,-0.0) (1.5918367346938775,-0.0) (1.7142857142857142,-0.0) (1.836734693877551,-0.0) (1.9591836734693877,-0.0) (2.0816326530612246,-0.0) (2.204081632653061,-0.0) (2.326530612244898,-0.0) (2.4489795918367347,-100000000.0) (2.571428571428571,-100000000.0) (2.693877551020408,-100000000.0) (2.816326530612245,-100000000.0) (2.9387755102040813,-100000000.0) (3.061224489795918,-100000000.0) (3.183673469387755,-100000000.0) (3.306122448979592,-100000000.0) (3.4285714285714284,-100000000.0) (3.5510204081632653,-100000000.0) (3.673469387755102,-0.0) (3.7959183673469385,-0.0) (3.9183673469387754,-0.0) (4.040816326530612,-0.0) (4.163265306122449,-0.0) (4.285714285714286,-0.0) (4.408163265306122,-0.0) (4.530612244897959,-0.0) (4.653061224489796,-0.0) (4.775510204081632,-0.0) (4.8979591836734695,-0.0) (5.020408163265306,-0.0) (5.142857142857142,-0.0) (5.26530612244898,-0.0) (5.387755102040816,-0.0) (5.5102040816326525,-0.0) (5.63265306122449,-0.0) (5.755102040816326,-0.0) (5.877551020408163,-0.0) (6.0,-0.0) };
    %% SECOND PLOT
    \nextgroupplot[title={(b) time $t = 4.84\times 10^{-4} $ s.},legend style={at={($(0.62,-0.35)+(0.9cm,1cm)$)},legend columns=4}]
    \addplot[Red,very thick,mark=none,dashed,mark size=3pt] coordinates {(0.0,-745947.6317356245) (0.12244897959183673,-2825505.704681147) (0.24489795918367346,-6801870.379054734) (0.36734693877551017,-14249320.469123462) (0.4897959183673469,-26743878.38137685) (0.6122448979591837,-45064188.07482756) (0.7346938775510203,-68064481.52123177) (0.8571428571428571,-91948308.59367745) (0.9795918367346939,-110948018.79598898) (1.1020408163265305,-119819288.51732583) (1.2244897959183674,-117150564.11810811) (1.346938775510204,-106891678.01826178) (1.4693877551020407,-96652794.70257169) (1.5918367346938775,-92267743.75879696) (1.7142857142857142,-95070577.34692116) (1.836734693877551,-99847820.1972167) (1.9591836734693877,-103232485.94759263) (2.0816326530612246,-101626491.39890483) (2.204081632653061,-100355242.65368447) (2.326530612244898,-98244216.75952996) (2.4489795918367347,-100067772.41751443) (2.571428571428571,-99702418.44172558) (2.693877551020408,-100893884.04862353) (2.816326530612245,-99751345.3825965) (2.9387755102040813,-99889896.90888087) (3.061224489795918,-99889896.90888077) (3.183673469387755,-99751345.38259669) (3.306122448979592,-100893884.04862344) (3.4285714285714284,-99702418.44172575) (3.5510204081632653,-100067772.41751438) (3.673469387755102,-98244216.75953004) (3.7959183673469385,-100355242.65368438) (3.9183673469387754,-101626491.39890482) (4.040816326530612,-103232485.94759259) (4.163265306122449,-99847820.1972167) (4.285714285714286,-95070577.34692109) (4.408163265306122,-92267743.7587969) (4.530612244897959,-96652794.70257166) (4.653061224489796,-106891678.01826191) (4.775510204081632,-117150564.11810826) (4.8979591836734695,-119819288.51732588) (5.020408163265306,-110948018.79598887) (5.142857142857142,-91948308.59367728) (5.26530612244898,-68064481.52123177) (5.387755102040816,-45064188.074827455) (5.5102040816326525,-26743878.38137681) (5.63265306122449,-14249320.469123438) (5.755102040816326,-6801870.379054725) (5.877551020408163,-2825505.7046811893) (6.0,-745947.6317356417) };
\addplot[Duck,thin,mark=*,solid,mark size=3pt] coordinates {(0.0,-3658473.820541984) (0.12244897959183673,-11485497.138528435) (0.24489795918367346,-20650075.223420583) (0.36734693877551017,-31470071.326657515) (0.4897959183673469,-43620393.930359595) (0.6122448979591837,-56234556.96822935) (0.7346938775510203,-68199491.57707022) (0.8571428571428571,-78518360.75669788) (0.9795918367346939,-86590219.21803293) (1.1020408163265305,-92306933.66125712) (1.2244897959183674,-95965467.32727806) (1.346938775510204,-98076133.61253974) (1.4693877551020407,-99170549.7565818) (1.5918367346938775,-99678671.19148654) (1.7142857142857142,-99888928.31124762) (1.836734693877551,-99966022.58725037) (1.9591836734693877,-99990891.70851223) (2.0816326530612246,-99997886.14886712) (2.204081632653061,-99999581.77077132) (2.326530612244898,-99999930.86939865) (2.4489795918367347,-99999990.71487762) (2.571428571428571,-99999999.0267497) (2.693877551020408,-99999999.92533047) (2.816326530612245,-99999999.99627104) (2.9387755102040813,-99999999.99990901) (3.061224489795918,-99999999.99990901) (3.183673469387755,-99999999.99627106) (3.306122448979592,-99999999.9253305) (3.4285714285714284,-99999999.02674973) (3.5510204081632653,-99999990.71487764) (3.673469387755102,-99999930.86939867) (3.7959183673469385,-99999581.77077134) (3.9183673469387754,-99997886.14886712) (4.040816326530612,-99990891.70851222) (4.163265306122449,-99966022.58725034) (4.285714285714286,-99888928.31124759) (4.408163265306122,-99678671.19148651) (4.530612244897959,-99170549.75658175) (4.653061224489796,-98076133.61253972) (4.775510204081632,-95965467.32727805) (4.8979591836734695,-92306933.66125712) (5.020408163265306,-86590219.21803287) (5.142857142857142,-78518360.7566978) (5.26530612244898,-68199491.5770702) (5.387755102040816,-56234556.968229264) (5.5102040816326525,-43620393.930359535) (5.63265306122449,-31470071.32665747) (5.755102040816326,-20650075.223420557) (5.877551020408163,-11485497.138528453) (6.0,-3658473.8205419495) };
\addplot[Orange,very thick,mark=none,densely dotted,mark size=3pt] coordinates {(0.0,-569245.6271305095) (0.06060606060606061,-569245.6271305095) (0.12121212121212122,-2277785.4230159773) (0.18181818181818182,-2277785.4230159773) (0.24242424242424243,-5901161.70250096) (0.30303030303030304,-5901161.70250096) (0.36363636363636365,-13233001.909618562) (0.42424242424242425,-13233001.909618562) (0.48484848484848486,-26238300.23104623) (0.5454545454545454,-26238300.23104623) (0.6060606060606061,-45981632.22002476) (0.6666666666666667,-45981632.22002476) (0.7272727272727273,-70997511.41827326) (0.7878787878787878,-70997511.41827326) (0.8484848484848485,-96258666.79270092) (0.9090909090909092,-96258666.79270092) (0.9696969696969697,-114407544.96327573) (1.0303030303030303,-114407544.96327573) (1.0909090909090908,-119769201.2337222) (1.1515151515151516,-119769201.2337222) (1.2121212121212122,-112838974.87666531) (1.2727272727272727,-112838974.87666531) (1.3333333333333335,-100993168.67842056) (1.393939393939394,-100993168.67842056) (1.4545454545454546,-93414330.53328812) (1.5151515151515151,-93414330.53328812) (1.5757575757575757,-93992004.01740366) (1.6363636363636365,-93992004.01740366) (1.696969696969697,-99087098.74325639) (1.7575757575757576,-99087098.74325639) (1.8181818181818183,-102489655.2966868) (1.878787878787879,-102489655.2966868) (1.9393939393939394,-101890791.3937958) (2.0,-101890791.3937958) (2.0606060606060606,-99782074.12880751) (2.121212121212121,-99782074.12880751) (2.1818181818181817,-99055570.19076754) (2.2424242424242427,-99055570.19076754) (2.303030303030303,-99748095.76886229) (2.3636363636363638,-99748095.76886229) (2.4242424242424243,-100313223.73164806) (2.484848484848485,-100313223.73164806) (2.5454545454545454,-100175907.04998955) (2.606060606060606,-100175907.04998955) (2.666666666666667,-99908114.85126807) (2.7272727272727275,-99908114.85126807) (2.787878787878788,-99919571.77694954) (2.8484848484848486,-99919571.77694954) (2.909090909090909,-100047873.551725) (2.9696969696969697,-100047873.551725) (3.0303030303030303,-100047873.55172497) (3.090909090909091,-100047873.55172497) (3.1515151515151514,-99919571.7769495) (3.2121212121212124,-99919571.7769495) (3.272727272727273,-99908114.851268) (3.3333333333333335,-99908114.851268) (3.393939393939394,-100175907.0499895) (3.4545454545454546,-100175907.0499895) (3.515151515151515,-100313223.73164806) (3.5757575757575757,-100313223.73164806) (3.6363636363636367,-99748095.7688623) (3.6969696969696972,-99748095.7688623) (3.757575757575758,-99055570.19076759) (3.8181818181818183,-99055570.19076759) (3.878787878787879,-99782074.12880753) (3.9393939393939394,-99782074.12880753) (4.0,-101890791.39379583) (4.0606060606060606,-101890791.39379583) (4.121212121212121,-102489655.29668684) (4.181818181818182,-102489655.29668684) (4.242424242424242,-99087098.74325638) (4.303030303030303,-99087098.74325638) (4.363636363636363,-93992004.01740365) (4.424242424242425,-93992004.01740365) (4.484848484848485,-93414330.53328809) (4.545454545454546,-93414330.53328809) (4.606060606060606,-100993168.67842054) (4.666666666666667,-100993168.67842054) (4.7272727272727275,-112838974.87666531) (4.787878787878788,-112838974.87666531) (4.848484848484849,-119769201.2337222) (4.909090909090909,-119769201.2337222) (4.96969696969697,-114407544.96327575) (5.03030303030303,-114407544.96327575) (5.090909090909091,-96258666.79270092) (5.151515151515151,-96258666.79270092) (5.212121212121212,-70997511.41827326) (5.2727272727272725,-70997511.41827326) (5.333333333333334,-45981632.22002478) (5.3939393939393945,-45981632.22002478) (5.454545454545455,-26238300.23104622) (5.515151515151516,-26238300.23104622) (5.575757575757576,-13233001.90961853) (5.636363636363637,-13233001.90961853) (5.696969696969697,-5901161.702500943) (5.757575757575758,-5901161.702500943) (5.818181818181818,-2277785.4230159433) (5.878787878787879,-2277785.4230159433) (5.9393939393939394,-569245.627130518) (6.0,-569245.627130518) };
\addplot[Blue,very thick,mark=none,solid,mark size=3pt] coordinates {(0.0,6.026738787271584e-08) (0.12244897959183673,-1.2053477574543037e-08) (0.24489795918367346,2.410695514908625e-08) (0.36734693877551017,2.4106955149086283e-08) (0.4897959183673469,6.026738787271564e-08) (0.6122448979591837,-99999999.99999984) (0.7346938775510203,-100000000.00000025) (0.8571428571428571,-99999999.9999998) (0.9795918367346939,-100000000.00000009) (1.1020408163265305,-99999999.99999985) (1.2244897959183674,-100000000.0) (1.346938775510204,-99999999.99999991) (1.4693877551020407,-100000000.0) (1.5918367346938775,-99999999.99999979) (1.7142857142857142,-100000000.00000003) (1.836734693877551,-99999999.99999988) (1.9591836734693877,-99999999.99999994) (2.0816326530612246,-99999999.99999994) (2.204081632653061,-99999999.99999999) (2.326530612244898,-99999999.99999991) (2.4489795918367347,-99999999.99999991) (2.571428571428571,-99999999.99999997) (2.693877551020408,-99999999.99999991) (2.816326530612245,-99999999.99999994) (2.9387755102040813,-99999999.99999997) (3.061224489795918,-99999999.99999988) (3.183673469387755,-99999999.99999997) (3.306122448979592,-99999999.99999988) (3.4285714285714284,-100000000.00000003) (3.5510204081632653,-99999999.99999991) (3.673469387755102,-100000000.0) (3.7959183673469385,-100000000.00000003) (3.9183673469387754,-99999999.99999987) (4.040816326530612,-99999999.99999997) (4.163265306122449,-99999999.99999994) (4.285714285714286,-99999999.99999997) (4.408163265306122,-99999999.99999997) (4.530612244897959,-100000000.00000009) (4.653061224489796,-99999999.99999988) (4.775510204081632,-100000000.00000013) (4.8979591836734695,-99999999.99999987) (5.020408163265306,-100000000.00000012) (5.142857142857142,-99999999.99999991) (5.26530612244898,-100000000.00000007) (5.387755102040816,-99999999.99999999) (5.5102040816326525,-2.410695514908632e-08) (5.63265306122449,2.410695514908628e-08) (5.755102040816326,-7.232086544725893e-08) (5.877551020408163,-1.205347757454315e-08) (6.0,-2.4106955149086326e-08) };
\addplot[Purple,very thick,mark=|,solid,mark size=3pt] coordinates {(0.0,-901250.8727450209) (0.06060606060606061,-2481132.001666541) (0.12121212121212122,-4826425.776536643) (0.18181818181818182,-7379655.632326556) (0.24242424242424243,-11256070.988140773) (0.30303030303030304,-15619284.087205578) (0.36363636363636365,-21632650.461414523) (0.42424242424242425,-28194347.738026302) (0.48484848484848486,-36184793.045001954) (0.5454545454545454,-44512745.90296996) (0.6060606060606061,-53388027.81990887) (0.6666666666666667,-62184684.90343942) (0.7272727272727273,-70312839.86926234) (0.7878787878787878,-77956322.11918701) (0.8484848484848485,-83999795.65014975) (0.9090909090909092,-89383013.54381928) (0.9696969696969697,-92952697.00923085) (1.0303030303030303,-95962268.87367085) (1.0909090909090908,-97581224.04592441) (1.1515151515151516,-98874662.75556055) (1.2121212121212122,-99404486.29729274) (1.2727272727272727,-99808341.57283215) (1.3333333333333335,-99915762.38856114) (1.393939393939394,-99996248.63365263) (1.4545454545454546,-100001427.378495) (1.5151515151515151,-100006998.37736051) (1.5757575757575757,-100003237.88768458) (1.6363636363636365,-100001492.41546689) (1.696969696969697,-100000520.10664989) (1.7575757575757576,-100000028.36063254) (1.8181818181818183,-99999995.05295742) (1.878787878787879,-99999970.09524688) (1.9393939393939394,-99999990.32856694) (2.0,-99999997.18169779) (2.0606060606060606,-99999999.42352416) (2.121212121212121,-100000000.25132437) (2.1818181818181817,-100000000.07258461) (2.2424242424242427,-100000000.03564933) (2.303030303030303,-100000000.00601049) (2.3636363636363638,-99999999.99877489) (2.4242424242424243,-99999999.99971391) (2.484848484848485,-99999999.99982226) (2.5454545454545454,-99999999.99998015) (2.606060606060606,-100000000.00000417) (2.666666666666667,-100000000.00000048) (2.7272727272727275,-100000000.0000002) (2.787878787878788,-99999999.99999994) (2.8484848484848486,-99999999.99999994) (2.909090909090909,-99999999.99999994) (2.9696969696969697,-99999999.99999994) (3.0303030303030303,-99999999.99999994) (3.090909090909091,-99999999.99999994) (3.1515151515151514,-99999999.99999996) (3.2121212121212124,-99999999.99999997) (3.272727272727273,-100000000.00000024) (3.3333333333333335,-100000000.00000054) (3.393939393939394,-100000000.0000042) (3.4545454545454546,-99999999.99998017) (3.515151515151515,-99999999.99982227) (3.5757575757575757,-99999999.99971391) (3.6363636363636367,-99999999.99877487) (3.6969696969696972,-100000000.00601052) (3.757575757575758,-100000000.03564937) (3.8181818181818183,-100000000.07258466) (3.878787878787879,-100000000.2513244) (3.9393939393939394,-99999999.42352419) (4.0,-99999997.18169783) (4.0606060606060606,-99999990.328567) (4.121212121212121,-99999970.09524693) (4.181818181818182,-99999995.05295746) (4.242424242424242,-100000028.36063258) (4.303030303030303,-100000520.10664994) (4.363636363636363,-100001492.41546696) (4.424242424242425,-100003237.88768464) (4.484848484848485,-100006998.37736058) (4.545454545454546,-100001427.37849504) (4.606060606060606,-99996248.63365267) (4.666666666666667,-99915762.38856122) (4.7272727272727275,-99808341.57283223) (4.787878787878788,-99404486.29729284) (4.848484848484849,-98874662.75556065) (4.909090909090909,-97581224.04592451) (4.96969696969697,-95962268.87367095) (5.03030303030303,-92952697.00923099) (5.090909090909091,-89383013.5438194) (5.151515151515151,-83999795.65014987) (5.212121212121212,-77956322.11918712) (5.2727272727272725,-70312839.86926243) (5.333333333333334,-62184684.90343948) (5.3939393939393945,-53388027.81990896) (5.454545454545455,-44512745.90297001) (5.515151515151516,-36184793.04500202) (5.575757575757576,-28194347.73802632) (5.636363636363637,-21632650.461414546) (5.696969696969697,-15619284.087205617) (5.757575757575758,-11256070.988140827) (5.818181818181818,-7379655.632326601) (5.878787878787879,-4826425.7765366705) (5.9393939393939394,-2481132.001666579) (6.0,-901250.8727450437) };
\addplot[Green,thick,mark=x,only marks,mark size=3pt] coordinates {(0.0,-6.52473551238276e-08) (0.06060606060606061,-5.528742062160391e-08) (0.12121212121212122,-5.4360491549496524e-08) (0.18181818181818182,-4.206732904684871e-08) (0.24242424242424243,2.6389399231237735e-10) (0.30303030303030304,2.3843061156773917e-08) (0.36363636363636365,1.6444762021031894e-08) (0.42424242424242425,7.662193128054401e-09) (0.48484848484848486,-2.6065356036709213e-08) (0.5454545454545454,-4.625550941054968e-08) (0.6060606060606061,-99999999.99999985) (0.6666666666666667,-99999999.99999993) (0.7272727272727273,-99999999.99999994) (0.7878787878787878,-99999999.99999994) (0.8484848484848485,-99999999.99999994) (0.9090909090909092,-99999999.99999994) (0.9696969696969697,-100000000.0) (1.0303030303030303,-100000000.0) (1.0909090909090908,-99999999.99999988) (1.1515151515151516,-99999999.9999999) (1.2121212121212122,-99999999.99999991) (1.2727272727272727,-99999999.99999993) (1.3333333333333335,-99999999.99999996) (1.393939393939394,-99999999.99999997) (1.4545454545454546,-99999999.99999988) (1.5151515151515151,-99999999.99999988) (1.5757575757575757,-99999999.99999991) (1.6363636363636365,-99999999.99999993) (1.696969696969697,-99999999.99999997) (1.7575757575757576,-99999999.99999999) (1.8181818181818183,-99999999.99999994) (1.878787878787879,-99999999.99999994) (1.9393939393939394,-99999999.99999994) (2.0,-99999999.99999994) (2.0606060606060606,-99999999.99999996) (2.121212121212121,-99999999.99999999) (2.1818181818181817,-99999999.99999996) (2.2424242424242427,-99999999.99999997) (2.303030303030303,-99999999.99999994) (2.3636363636363638,-99999999.99999994) (2.4242424242424243,-99999999.99999994) (2.484848484848485,-99999999.99999994) (2.5454545454545454,-99999999.9999999) (2.606060606060606,-99999999.99999993) (2.666666666666667,-99999999.99999994) (2.7272727272727275,-99999999.99999994) (2.787878787878788,-99999999.99999996) (2.8484848484848486,-99999999.99999999) (2.909090909090909,-99999999.99999991) (2.9696969696969697,-99999999.99999993) (3.0303030303030303,-99999999.99999994) (3.090909090909091,-99999999.99999994) (3.1515151515151514,-99999999.99999994) (3.2121212121212124,-99999999.99999994) (3.272727272727273,-100000000.0) (3.3333333333333335,-100000000.0) (3.393939393939394,-99999999.99999991) (3.4545454545454546,-99999999.99999993) (3.515151515151515,-99999999.99999999) (3.5757575757575757,-100000000.0) (3.6363636363636367,-99999999.99999991) (3.6969696969696972,-99999999.99999993) (3.757575757575758,-99999999.99999993) (3.8181818181818183,-99999999.99999994) (3.878787878787879,-100000000.00000004) (3.9393939393939394,-100000000.00000003) (4.0,-99999999.99999991) (4.0606060606060606,-99999999.99999991) (4.121212121212121,-100000000.00000001) (4.181818181818182,-100000000.00000003) (4.242424242424242,-99999999.99999993) (4.303030303030303,-99999999.99999994) (4.363636363636363,-99999999.99999997) (4.424242424242425,-99999999.99999997) (4.484848484848485,-100000000.00000006) (4.545454545454546,-100000000.00000006) (4.606060606060606,-99999999.99999991) (4.666666666666667,-99999999.99999993) (4.7272727272727275,-99999999.99999996) (4.787878787878788,-99999999.99999997) (4.848484848484849,-99999999.99999996) (4.909090909090909,-99999999.99999999) (4.96969696969697,-99999999.99999997) (5.03030303030303,-99999999.99999996) (5.090909090909091,-99999999.99999997) (5.151515151515151,-99999999.99999999) (5.212121212121212,-99999999.99999996) (5.2727272727272725,-99999999.99999997) (5.333333333333334,-99999999.99999991) (5.3939393939393945,-100000000.00000009) (5.454545454545455,-6.171875818689624e-09) (5.515151515151516,6.171875818689572e-09) (5.575757575757576,-3.413288241873195e-08) (5.636363636363637,-6.229493817761331e-08) (5.696969696969697,2.564453865316311e-08) (5.757575757575758,-1.5375835040768292e-09) (5.818181818181818,-5.420092204753976e-08) (5.878787878787879,-4.222689854880543e-08) (5.9393939393939394,3.843223074305013e-09) (6.0,-3.8432230743049925e-09) };
\addplot[black,thin,mark=none,solid,mark size=3pt] coordinates {(0.0,-0.0) (0.12244897959183673,-0.0) (0.24489795918367346,-0.0) (0.36734693877551017,-0.0) (0.4897959183673469,-0.0) (0.6122448979591837,-100000000.0) (0.7346938775510203,-100000000.0) (0.8571428571428571,-100000000.0) (0.9795918367346939,-100000000.0) (1.1020408163265305,-100000000.0) (1.2244897959183674,-100000000.0) (1.346938775510204,-100000000.0) (1.4693877551020407,-100000000.0) (1.5918367346938775,-100000000.0) (1.7142857142857142,-100000000.0) (1.836734693877551,-100000000.0) (1.9591836734693877,-100000000.0) (2.0816326530612246,-100000000.0) (2.204081632653061,-100000000.0) (2.326530612244898,-100000000.0) (2.4489795918367347,-100000000.0) (2.571428571428571,-100000000.0) (2.693877551020408,-100000000.0) (2.816326530612245,-100000000.0) (2.9387755102040813,-100000000.0) (3.061224489795918,-100000000.0) (3.183673469387755,-100000000.0) (3.306122448979592,-100000000.0) (3.4285714285714284,-100000000.0) (3.5510204081632653,-100000000.0) (3.673469387755102,-100000000.0) (3.7959183673469385,-100000000.0) (3.9183673469387754,-100000000.0) (4.040816326530612,-100000000.0) (4.163265306122449,-100000000.0) (4.285714285714286,-100000000.0) (4.408163265306122,-100000000.0) (4.530612244897959,-100000000.0) (4.653061224489796,-100000000.0) (4.775510204081632,-100000000.0) (4.8979591836734695,-100000000.0) (5.020408163265306,-100000000.0) (5.142857142857142,-100000000.0) (5.26530612244898,-100000000.0) (5.387755102040816,-100000000.0) (5.5102040816326525,-0.0) (5.63265306122449,-0.0) (5.755102040816326,-0.0) (5.877551020408163,-0.0) (6.0,-0.0) };
\addlegendentry{usl 1ppc}
\addlegendentry{usl-pic 1ppc}
\addlegendentry{usl 2ppc}
\addlegendentry{dgmpm 1ppc}
\addlegendentry{dgmpm 2ppc}
\addlegendentry{dgmpm 2ppc (RK2)}
\addlegendentry{exact}
  \end{groupplot}
\end{tikzpicture}
%%% Local Variables:
%%% mode: latex
%%% TeX-master: "../../mainManuscript"
%%% End:



  \caption{Stress solutions of the Riemann problem in an isotropic elastic bar before and after reflection of incident waves on the free boundaries of the bar. Comparison between  }
  \label{fig:elastic_stress}
\end{figure}
We see that no oscillations, diffusion introduced for 2ppc dgmpm while mpm does not change. However RK2 allows to perfectly fit the exact solution \ref{fig:elastic_stress}\subref{subfig:rp_stress1} \ref{fig:elastic_stress}\subref{subfig:rp_stress2}.

\begin{figure}[h!]
  \centering
  \begin{tikzpicture}[scale=.675]
\begin{axis}[xlabel=$time \: (s)$,ylabel=$\frac{e}{e_{max}}$,ymajorgrids=true,xmajorgrids=true,legend pos=outer north east,title={Energy plots}]
\addplot[Red,very thick,mark=none,dashed,mark size=2pt] coordinates {(0.0,0.987654320988) (1.2090867954e-05,0.997530864198) (2.41817359079e-05,1.0) (3.62726038619e-05,0.995910493827) (4.83634718158e-05,0.98940007716) (6.04543397698e-05,0.984093484761) (7.25452077237e-05,0.981362538279) (8.46360756777e-05,0.980572573344) (9.67269436317e-05,0.980353499636) (0.000108817811586,0.979737335844) (0.00012090867954,0.97855223305) (0.000132999547494,0.977156270039) (0.000145090415447,0.975958347691) (0.000157181283401,0.97511659562) (0.000169272151355,0.974530084065) (0.000181363019309,0.974005554053) (0.000193453887263,0.97341779498) (0.000205544755217,0.972760977912) (0.000217635623171,0.972101158286) (0.000229726491125,0.971500815494) (0.000241817359079,0.970976328432) (0.000253908227033,0.970503354023) (0.000265999094987,0.970047511625) (0.000278089962941,0.969589945626) (0.000290180830895,0.969132703566) (0.000302271698849,0.968688192138) (0.000314362566803,0.96826592316) (0.000326453434757,0.967866441189) (0.000338544302711,0.96748370195) (0.000350635170665,0.967111035083) (0.000362726038619,0.96674530374) (0.000374816906573,0.9663871534) (0.000386907774527,0.96603867283) (0.000398998642481,0.965701020803) (0.000411089510435,0.9653736209) (0.000423180378389,0.965054863002) (0.000435271246342,0.96474325827) (0.000447362114296,0.964438078134) (0.00045945298225,0.964139187671) (0.000471543850204,0.963846345834) (0.000483634718158,0.963558300848) (0.000495725586112,0.963271635613) (0.000507816454066,0.962978858237) (0.00051990732202,0.962665035247) (0.000531998189974,0.962302555824) (0.000544089057928,0.961844496637) (0.000556179925882,0.961218491107) (0.000568270793836,0.960324692645) (0.00058036166179,0.959042628999) (0.000592452529744,0.95725134199) };
\addplot[Orange,very thick,mark=none,densely dotted,mark size=3pt] coordinates {(0.0,0.987349583462) (1.19687379746e-05,0.997223079297) (2.39374759493e-05,1.0) (3.59062139239e-05,0.996543312249) (4.78749518985e-05,0.991672401699) (5.98436898731e-05,0.988645826067) (7.18124278478e-05,0.98773805255) (8.37811658224e-05,0.987631618227) (9.5749903797e-05,0.987207209533) (0.000107718641772,0.98625862112) (0.000119687379746,0.985162495947) (0.000131656117721,0.984278008796) (0.000143624855696,0.983664883566) (0.00015559359367,0.983176360212) (0.000167562331645,0.982669899583) (0.000179531069619,0.982113551237) (0.000191499807594,0.981554758898) (0.000203468545569,0.981041934115) (0.000215437283543,0.980583060318) (0.000227406021518,0.98015640132) (0.000239374759493,0.979740132791) (0.000251343497467,0.97932838866) (0.000263312235442,0.978927266369) (0.000275280973416,0.978543296202) (0.000287249711391,0.978177057907) (0.000299218449366,0.977824634311) (0.00031118718734,0.977482093196) (0.000323155925315,0.977147995987) (0.00033512466329,0.976822820567) (0.000347093401264,0.976507170495) (0.000359062139239,0.976200756538) (0.000371030877213,0.975902607469) (0.000382999615188,0.97561177268) (0.000394968353163,0.97532772239) (0.000406937091137,0.975050257459) (0.000418905829112,0.974779220922) (0.000430874567087,0.97451432881) (0.000442843305061,0.974255192075) (0.000454812043036,0.974001394871) (0.00046678078101,0.973752448152) (0.000478749518985,0.973507455428) (0.00049071825696,0.973264255861) (0.000502686994934,0.973017591891) (0.000514655732909,0.972755592972) (0.000526624470884,0.972453855054) (0.000538593208858,0.972067038786) (0.000550561946833,0.971519585751) (0.000562530684807,0.970699836135) (0.000574499422782,0.969464605892) (0.000586468160757,0.967662115176) };
\addplot[Duck,thick,mark=*,solid,mark size=2pt] coordinates {(0.0,1.0) (1.19687379746e-05,0.9825) (2.39374759493e-05,0.97517578125) (3.59062139239e-05,0.969031906128) (4.78749518985e-05,0.96380058825) (5.98436898731e-05,0.959171754479) (7.18124278478e-05,0.954982516842) (8.37811658224e-05,0.951129379832) (9.5749903797e-05,0.947543340455) (0.000107718641772,0.944175962079) (0.000119687379746,0.940991760326) (0.000131656117721,0.93796387047) (0.000143624855696,0.935071387013) (0.00015559359367,0.932297671085) (0.000167562331645,0.929629226561) (0.000179531069619,0.927054929147) (0.000191499807594,0.924565482142) (0.000203468545569,0.922153022591) (0.000215437283543,0.91981083008) (0.000227406021518,0.917533107309) (0.000239374759493,0.91531481198) (0.000251343497467,0.913151526055) (0.000263312235442,0.911039352738) (0.000275280973416,0.908974834313) (0.000287249711391,0.906954885898) (0.000299218449366,0.904976741526) (0.00031118718734,0.903037909836) (0.000323155925315,0.901136137378) (0.00033512466329,0.89926937799) (0.000347093401264,0.897435767059) (0.000359062139239,0.895633599746) (0.000371030877213,0.893861312443) (0.000382999615188,0.892117466744) (0.000394968353163,0.890400734736) (0.000406937091137,0.888709882017) (0.000418905829112,0.88704373739) (0.000430874567087,0.88540112132) (0.000442843305061,0.883780678009) (0.000454812043036,0.882180531538) (0.00046678078101,0.880597698856) (0.000478749518985,0.879027284148) (0.00049071825696,0.877461660668) (0.000502686994934,0.875890051367) (0.000514655732909,0.874299006831) (0.000526624470884,0.87267411056) (0.000538593208858,0.871002801233) (0.000550561946833,0.869277656037) (0.000562530684807,0.867499110287) (0.000574499422782,0.865676626234) (0.000586468160757,0.86382779387) };
\addplot[Blue,very thick,mark=none,solid,mark size=3pt] coordinates {(0.0,1.0) (2.41817359079e-05,1.0) (4.83634718158e-05,1.0) (7.25452077237e-05,1.0) (9.67269436317e-05,1.0) (0.00012090867954,1.0) (0.000145090415447,1.0) (0.000169272151355,1.0) (0.000193453887263,1.0) (0.000217635623171,1.0) (0.000241817359079,1.0) (0.000265999094987,1.0) (0.000290180830895,1.0) (0.000314362566803,1.0) (0.000338544302711,1.0) (0.000362726038619,1.0) (0.000386907774527,1.0) (0.000411089510435,1.0) (0.000435271246342,1.0) (0.00045945298225,1.0) (0.000483634718158,1.0) (0.000507816454066,1.0) (0.000531998189974,1.0) (0.000556179925882,1.0) (0.00058036166179,1.0) (0.000604543397698,1.0) };
\addplot[Purple,very thick,mark=+,solid,mark size=3pt] coordinates {(0.0,1.0) (1.19687379746e-05,0.985) (2.39374759493e-05,0.97828125) (3.59062139239e-05,0.972868652344) (4.78749518985e-05,0.968517532349) (5.98436898731e-05,0.964676422477) (7.18124278478e-05,0.961224535313) (8.37811658224e-05,0.95805841396) (9.5749903797e-05,0.955117021818) (0.000107718641772,0.952358340029) (0.000119687379746,0.949751847063) (0.000131656117721,0.947274776455) (0.000143624855696,0.944909512757) (0.00015559359367,0.942642119513) (0.000167562331645,0.940461338934) (0.000179531069619,0.938357922299) (0.000191499807594,0.936324160701) (0.000203468545569,0.934353548006) (0.000215437283543,0.932440533241) (0.000227406021518,0.93058033492) (0.000239374759493,0.928768799379) (0.000251343497467,0.927002291004) (0.000263312235442,0.925277605976) (0.000275280973416,0.923591903662) (0.000287249711391,0.921942651418) (0.000299218449366,0.920327579747) (0.00031118718734,0.918744645509) (0.000323155925315,0.917192001508) (0.00033512466329,0.915667971129) (0.000347093401264,0.91417102706) (0.000359062139239,0.9126997733) (0.000371030877213,0.911252929874) (0.000382999615188,0.909829319753) (0.000394968353163,0.908427857623) (0.000406937091137,0.907047540169) (0.000418905829112,0.905687437658) (0.000430874567087,0.904346686589) (0.000442843305061,0.903024483275) (0.000454812043036,0.901720078186) (0.00046678078101,0.900432770981) (0.000478749518985,0.899161906097) (0.00049071825696,0.897906868838) (0.000502686994934,0.896667081898) (0.000514655732909,0.895442002248) (0.000526624470884,0.894231118352) (0.000538593208858,0.89303394767) (0.000550561946833,0.891850034402) (0.000562530684807,0.890678947462) (0.000574499422782,0.889520278638) (0.000586468160757,0.888373640933) (0.000598436898731,0.887238667044) };
\addplot[Green,thick,mark=x,only marks,mark size=3pt] coordinates {(0.0,1.0) (2.39374759493e-05,1.0) (4.78749518985e-05,1.0) (7.18124278478e-05,1.0) (9.5749903797e-05,1.0) (0.000119687379746,1.0) (0.000143624855696,1.0) (0.000167562331645,1.0) (0.000191499807594,1.0) (0.000215437283543,1.0) (0.000239374759493,1.0) (0.000263312235442,1.0) (0.000287249711391,1.0) (0.00031118718734,1.0) (0.00033512466329,1.0) (0.000359062139239,1.0) (0.000382999615188,1.0) (0.000406937091137,1.0) (0.000430874567087,1.0) (0.000454812043036,1.0) (0.000478749518985,1.0) (0.000502686994934,1.0) (0.000526624470884,1.0) (0.000550561946833,1.0) (0.000574499422782,1.0) (0.000598436898731,1.0) };
\legend{mpm-flip 1ppc -- CFL=0.5,mpm-flip 2ppc -- CFL=0.5,mpm-pic 2ppc -- CFL=0.5,dgmpm 1ppc -- CFL=1,dgmpm 2ppc -- CFL=0.5,dgmpm 2ppc (RK2) -- CFL=1}
\end{axis}
\end{tikzpicture}
%%% Local Variables:
%%% mode: latex
%%% TeX-master: "../../presentation"
%%% End:

  \caption{elastic RP stress and energies}
  \label{fig:energy_elastic_RP}
\end{figure}


\begin{figure}[h!]
  \centering
  {\begin{tikzpicture}[scale=0.9]
  \begin{groupplot}[group style={group size=2 by 1,
      ylabels at=edge left, yticklabels at=edge left,
      horizontal sep=4.ex,
      vertical sep=2ex,},
    ymajorgrids=true,xmajorgrids=true,
    enlargelimits=0,
    xmin=0.,xmax=6.,
    ylabel=$v (m/s)$,
    xlabel=$x (m)$,
    axis on top,scale only axis,width=0.45\linewidth%,xtick=\empty,ytick=\empty,width=0.25\linewidth,
    ]
    %% FIRST PLOT
    \nextgroupplot[title={(a) time $t = 1.21\times 10^{-4} $ s.}]
    \addplot[Red,very thick,mark=none,dashed,mark size=3pt] coordinates {(0.0,2.5318484177091665) (0.12244897959183673,2.5318484177091665) (0.24489795918367346,2.531848417709166) (0.36734693877551017,2.531848417709166) (0.4897959183673469,2.5318484177091656) (0.6122448979591837,2.5318484177091665) (0.7346938775510203,2.531848417709167) (0.8571428571428571,2.5318484177091665) (0.9795918367346939,2.5318484177091665) (1.1020408163265305,2.5318484177091665) (1.2244897959183674,2.5318484177091665) (1.346938775510204,2.5318484177091665) (1.4693877551020407,2.5318484177091665) (1.5918367346938775,2.5318484177091665) (1.7142857142857142,2.5318484177091665) (1.836734693877551,2.531684227710154) (1.9591836734693877,2.528883339491711) (2.0816326530612246,2.5069977784469075) (2.204081632653061,2.405451093175476) (2.326530612244898,2.0843630627542513) (2.4489795918367347,1.4571958994685654) (2.571428571428571,0.3681332942558399) (2.693877551020408,0.03859430800310576) (2.816326530612245,-0.9060776804312415) (2.9387755102040813,0.15497604259705183) (3.061224489795918,-0.15497604259704956) (3.183673469387755,0.9060776804312402) (3.306122448979592,-0.038594308003106204) (3.4285714285714284,-0.3681332942558391) (3.5510204081632653,-1.4571958994685674) (3.673469387755102,-2.084363062754252) (3.7959183673469385,-2.4054510931754756) (3.9183673469387754,-2.506997778446907) (4.040816326530612,-2.5288833394917107) (4.163265306122449,-2.531684227710154) (4.285714285714286,-2.5318484177091665) (4.408163265306122,-2.5318484177091665) (4.530612244897959,-2.5318484177091665) (4.653061224489796,-2.5318484177091665) (4.775510204081632,-2.5318484177091665) (4.8979591836734695,-2.5318484177091665) (5.020408163265306,-2.5318484177091665) (5.142857142857142,-2.5318484177091665) (5.26530612244898,-2.5318484177091665) (5.387755102040816,-2.5318484177091665) (5.5102040816326525,-2.5318484177091665) (5.63265306122449,-2.5318484177091665) (5.755102040816326,-2.5318484177091665) (5.877551020408163,-2.5318484177091665) (6.0,-2.5318484177091665) };
\addplot[Duck,thin,mark=*,solid,mark size=3pt] coordinates {(0.0,2.531848417709167) (0.12244897959183673,2.531848417709168) (0.24489795918367346,2.531848417709168) (0.36734693877551017,2.531848417709168) (0.4897959183673469,2.531848417709167) (0.6122448979591837,2.531848417709167) (0.7346938775510203,2.531848417709167) (0.8571428571428571,2.531848417709167) (0.9795918367346939,2.531848417709166) (1.1020408163265305,2.531848417709166) (1.2244897959183674,2.531848417709166) (1.346938775510204,2.531848417709166) (1.4693877551020407,2.531848417709166) (1.5918367346938775,2.531848417709166) (1.7142857142857142,2.531848417709166) (1.836734693877551,2.529375909488747) (1.9591836734693877,2.5046508272845562) (2.0816326530612246,2.3933879573656958) (2.204081632653061,2.0966869709154032) (2.326530612244898,1.5774602446273907) (2.4489795918367347,0.9543881730817732) (2.571428571428571,0.4351614467937638) (2.693877551020408,0.1384604603434717) (2.816326530612245,0.02719759042461175) (2.9387755102040813,0.002472508220420069) (3.061224489795918,-0.002472508220419442) (3.183673469387755,-0.027197590424610934) (3.306122448979592,-0.13846046034347081) (3.4285714285714284,-0.4351614467937628) (3.5510204081632653,-0.9543881730817754) (3.673469387755102,-1.5774602446273907) (3.7959183673469385,-2.0966869709154032) (3.9183673469387754,-2.3933879573656966) (4.040816326530612,-2.5046508272845562) (4.163265306122449,-2.5293759094887474) (4.285714285714286,-2.5318484177091665) (4.408163265306122,-2.5318484177091665) (4.530612244897959,-2.5318484177091665) (4.653061224489796,-2.5318484177091665) (4.775510204081632,-2.5318484177091665) (4.8979591836734695,-2.5318484177091665) (5.020408163265306,-2.5318484177091665) (5.142857142857142,-2.5318484177091665) (5.26530612244898,-2.5318484177091665) (5.387755102040816,-2.5318484177091665) (5.5102040816326525,-2.5318484177091665) (5.63265306122449,-2.5318484177091665) (5.755102040816326,-2.5318484177091665) (5.877551020408163,-2.5318484177091665) (6.0,-2.5318484177091665) };
\addplot[Orange,very thick,mark=none,densely dotted,mark size=3pt] coordinates {(0.0,2.5318484177091665) (0.06060606060606061,2.5318484177091665) (0.12121212121212122,2.5318484177091665) (0.18181818181818182,2.5318484177091665) (0.24242424242424243,2.5318484177091665) (0.30303030303030304,2.5318484177091665) (0.36363636363636365,2.5318484177091665) (0.42424242424242425,2.5318484177091665) (0.48484848484848486,2.5318484177091665) (0.5454545454545454,2.5318484177091665) (0.6060606060606061,2.5318484177091665) (0.6666666666666667,2.531848417709167) (0.7272727272727273,2.531848417709167) (0.7878787878787878,2.5318484177091665) (0.8484848484848485,2.5318484177091665) (0.9090909090909092,2.5318484177091674) (0.9696969696969697,2.5318484177091665) (1.0303030303030303,2.5318484177091665) (1.0909090909090908,2.5318484177091656) (1.1515151515151516,2.5318484177091665) (1.2121212121212122,2.5318484177091665) (1.2727272727272727,2.5318484177091665) (1.3333333333333335,2.5318484177091656) (1.393939393939394,2.531848417709166) (1.4545454545454546,2.5318484177091665) (1.5151515151515151,2.5318484177091656) (1.5757575757575757,2.5318484177091665) (1.6363636363636365,2.531848417709167) (1.696969696969697,2.5318484177091665) (1.7575757575757576,2.5318484177091665) (1.8181818181818183,2.5318020034751854) (1.878787878787879,2.5317091750072236) (1.9393939393939394,2.5307501424302727) (2.0,2.5289249057443333) (2.0606060606060606,2.5201129167796545) (2.121212121212121,2.504314175536237) (2.1818181818181817,2.4573457248537327) (2.2424242424242427,2.379207564732142) (2.303030303030303,2.2185258725100665) (2.3636363636363638,1.975300648187503) (2.4242424242424243,1.625621223269172) (2.484848484848485,1.1694875977550743) (2.5454545454545454,0.6521580355962723) (2.606060606060606,0.07363253679276927) (2.666666666666667,-0.20959340058418097) (2.7272727272727275,-0.1975197765345734) (2.787878787878788,-0.4134073094303823) (2.8484848484848486,-0.8572559992716083) (2.909090909090909,-0.17642315371687442) (2.9696969696969697,1.6290912272338196) (3.0303030303030303,-1.6290912272338194) (3.090909090909091,0.17642315371686795) (3.1515151515151514,0.8572559992716065) (3.2121212121212124,0.41340730943038484) (3.272727272727273,0.19751977653457933) (3.3333333333333335,0.20959340058418124) (3.393939393939394,-0.07363253679276877) (3.4545454545454546,-0.6521580355962722) (3.515151515151515,-1.1694875977550738) (3.5757575757575757,-1.6256212232691716) (3.6363636363636367,-1.9753006481875028) (3.6969696969696972,-2.2185258725100656) (3.757575757575758,-2.3792075647321425) (3.8181818181818183,-2.457345724853733) (3.878787878787879,-2.5043141755362366) (3.9393939393939394,-2.520112916779654) (4.0,-2.528924905744333) (4.0606060606060606,-2.5307501424302727) (4.121212121212121,-2.5317091750072236) (4.181818181818182,-2.531802003475186) (4.242424242424242,-2.5318484177091665) (4.303030303030303,-2.5318484177091665) (4.363636363636363,-2.5318484177091665) (4.424242424242425,-2.5318484177091665) (4.484848484848485,-2.5318484177091665) (4.545454545454546,-2.5318484177091665) (4.606060606060606,-2.5318484177091665) (4.666666666666667,-2.5318484177091665) (4.7272727272727275,-2.5318484177091665) (4.787878787878788,-2.5318484177091665) (4.848484848484849,-2.5318484177091665) (4.909090909090909,-2.5318484177091665) (4.96969696969697,-2.5318484177091665) (5.03030303030303,-2.5318484177091665) (5.090909090909091,-2.5318484177091665) (5.151515151515151,-2.5318484177091665) (5.212121212121212,-2.5318484177091665) (5.2727272727272725,-2.5318484177091665) (5.333333333333334,-2.5318484177091665) (5.3939393939393945,-2.5318484177091665) (5.454545454545455,-2.5318484177091665) (5.515151515151516,-2.5318484177091665) (5.575757575757576,-2.5318484177091665) (5.636363636363637,-2.5318484177091665) (5.696969696969697,-2.5318484177091665) (5.757575757575758,-2.5318484177091665) (5.818181818181818,-2.5318484177091665) (5.878787878787879,-2.5318484177091665) (5.9393939393939394,-2.5318484177091665) (6.0,-2.5318484177091665) };
\addplot[Blue,very thick,mark=none,solid,mark size=3pt] coordinates {(0.0,2.531848417709167) (0.12244897959183673,2.5318484177091656) (0.24489795918367346,2.531848417709166) (0.36734693877551017,2.5318484177091665) (0.4897959183673469,2.531848417709166) (0.6122448979591837,2.531848417709166) (0.7346938775510203,2.531848417709166) (0.8571428571428571,2.531848417709166) (0.9795918367346939,2.531848417709166) (1.1020408163265305,2.531848417709166) (1.2244897959183674,2.5318484177091656) (1.346938775510204,2.531848417709166) (1.4693877551020407,2.531848417709166) (1.5918367346938775,2.5318484177091665) (1.7142857142857142,2.531848417709166) (1.836734693877551,2.531848417709167) (1.9591836734693877,2.5318484177091665) (2.0816326530612246,2.5318484177091665) (2.204081632653061,2.5318484177091665) (2.326530612244898,2.5318484177091665) (2.4489795918367347,0.0) (2.571428571428571,-1.131824441720173e-15) (2.693877551020408,3.772748139067242e-16) (2.816326530612245,-3.944304526105059e-31) (2.9387755102040813,7.545496278134482e-16) (3.061224489795918,-1.131824441720173e-15) (3.183673469387755,3.772748139067241e-16) (3.306122448979592,3.772748139067244e-16) (3.4285714285714284,7.545496278134485e-16) (3.5510204081632653,2.220446049250313e-16) (3.673469387755102,-2.5318484177091665) (3.7959183673469385,-2.531848417709166) (3.9183673469387754,-2.531848417709166) (4.040816326530612,-2.531848417709167) (4.163265306122449,-2.531848417709166) (4.285714285714286,-2.5318484177091656) (4.408163265306122,-2.531848417709167) (4.530612244897959,-2.5318484177091665) (4.653061224489796,-2.531848417709166) (4.775510204081632,-2.531848417709167) (4.8979591836734695,-2.5318484177091665) (5.020408163265306,-2.531848417709167) (5.142857142857142,-2.531848417709167) (5.26530612244898,-2.531848417709167) (5.387755102040816,-2.531848417709167) (5.5102040816326525,-2.531848417709167) (5.63265306122449,-2.5318484177091665) (5.755102040816326,-2.531848417709166) (5.877551020408163,-2.531848417709167) (6.0,-2.5318484177091665) };
\addplot[Purple,very thick,mark=|,solid,mark size=3pt] coordinates {(0.0,2.531848417709166) (0.06060606060606061,2.531848417709166) (0.12121212121212122,2.531848417709166) (0.18181818181818182,2.531848417709166) (0.24242424242424243,2.5318484177091656) (0.30303030303030304,2.5318484177091656) (0.36363636363636365,2.5318484177091656) (0.42424242424242425,2.5318484177091656) (0.48484848484848486,2.5318484177091656) (0.5454545454545454,2.531848417709165) (0.6060606060606061,2.531848417709165) (0.6666666666666667,2.531848417709165) (0.7272727272727273,2.531848417709165) (0.7878787878787878,2.531848417709165) (0.8484848484848485,2.531848417709165) (0.9090909090909092,2.5318484177091647) (0.9696969696969697,2.5318484177091647) (1.0303030303030303,2.5318484177091647) (1.0909090909090908,2.5318484177091647) (1.1515151515151516,2.531848417709165) (1.2121212121212122,2.531848417709165) (1.2727272727272727,2.531848417709165) (1.3333333333333335,2.531848417709165) (1.393939393939394,2.531848417709165) (1.4545454545454546,2.531848417709165) (1.5151515151515151,2.5318484177091656) (1.5757575757575757,2.5318484177091656) (1.6363636363636365,2.531848417709166) (1.696969696969697,2.531848417709166) (1.7575757575757576,2.531848417709166) (1.8181818181818183,2.5317555939571394) (1.878787878787879,2.5315699464530863) (1.9393939393939394,2.529300921403548) (2.0,2.5251960488139282) (2.0606060606060606,2.502983668745643) (2.121212121212121,2.4672568379656363) (2.1818181818181817,2.3558296271378385) (2.2424242424242427,2.199909152499134) (2.303030303030303,1.8955358394101423) (2.3636363636363638,1.5350380403392951) (2.4242424242424243,1.0913569359704094) (2.484848484848485,0.6616953448225573) (2.5454545454545454,0.35041226238060424) (2.606060606060606,0.11398111608931762) (2.666666666666667,0.037623711953108305) (2.7272727272727275,-0.006618739561512254) (2.787878787878788,-0.0024042733543885486) (2.8484848484848486,-0.0017885136987410197) (2.909090909090909,-0.0001900946309221662) (2.9696969696969697,4.075039583706578e-05) (3.0303030303030303,-4.075039583690026e-05) (3.090909090909091,0.000190094630922048) (3.1515151515151514,0.001788513698741276) (3.2121212121212124,0.0024042733543890842) (3.272727272727273,0.006618739561513227) (3.3333333333333335,-0.03762371195310714) (3.393939393939394,-0.11398111608931744) (3.4545454545454546,-0.35041226238060363) (3.515151515151515,-0.6616953448225579) (3.5757575757575757,-1.09135693597041) (3.6363636363636367,-1.535038040339297) (3.6969696969696972,-1.8955358394101425) (3.757575757575758,-2.199909152499135) (3.8181818181818183,-2.3558296271378385) (3.878787878787879,-2.4672568379656363) (3.9393939393939394,-2.502983668745643) (4.0,-2.5251960488139282) (4.0606060606060606,-2.5293009214035473) (4.121212121212121,-2.5315699464530863) (4.181818181818182,-2.5317555939571394) (4.242424242424242,-2.531848417709166) (4.303030303030303,-2.531848417709166) (4.363636363636363,-2.531848417709166) (4.424242424242425,-2.531848417709166) (4.484848484848485,-2.531848417709166) (4.545454545454546,-2.531848417709166) (4.606060606060606,-2.531848417709166) (4.666666666666667,-2.531848417709166) (4.7272727272727275,-2.531848417709166) (4.787878787878788,-2.531848417709166) (4.848484848484849,-2.531848417709166) (4.909090909090909,-2.531848417709166) (4.96969696969697,-2.531848417709166) (5.03030303030303,-2.531848417709166) (5.090909090909091,-2.531848417709166) (5.151515151515151,-2.531848417709166) (5.212121212121212,-2.531848417709166) (5.2727272727272725,-2.531848417709166) (5.333333333333334,-2.531848417709166) (5.3939393939393945,-2.531848417709166) (5.454545454545455,-2.531848417709166) (5.515151515151516,-2.531848417709166) (5.575757575757576,-2.531848417709166) (5.636363636363637,-2.531848417709166) (5.696969696969697,-2.531848417709166) (5.757575757575758,-2.531848417709166) (5.818181818181818,-2.531848417709166) (5.878787878787879,-2.531848417709166) (5.9393939393939394,-2.531848417709166) (6.0,-2.5318484177091665) };
\addplot[Green,thick,mark=x,only marks,mark size=3pt] coordinates {(0.0,2.531848417709166) (0.06060606060606061,2.531848417709166) (0.12121212121212122,2.531848417709166) (0.18181818181818182,2.531848417709166) (0.24242424242424243,2.5318484177091665) (0.30303030303030304,2.5318484177091665) (0.36363636363636365,2.5318484177091656) (0.42424242424242425,2.5318484177091656) (0.48484848484848486,2.5318484177091665) (0.5454545454545454,2.5318484177091665) (0.6060606060606061,2.531848417709165) (0.6666666666666667,2.5318484177091656) (0.7272727272727273,2.531848417709166) (0.7878787878787878,2.531848417709166) (0.8484848484848485,2.5318484177091656) (0.9090909090909092,2.5318484177091656) (0.9696969696969697,2.5318484177091665) (1.0303030303030303,2.5318484177091665) (1.0909090909090908,2.5318484177091647) (1.1515151515151516,2.531848417709165) (1.2121212121212122,2.531848417709166) (1.2727272727272727,2.531848417709166) (1.3333333333333335,2.5318484177091647) (1.393939393939394,2.5318484177091647) (1.4545454545454546,2.5318484177091665) (1.5151515151515151,2.5318484177091665) (1.5757575757575757,2.531848417709166) (1.6363636363636365,2.531848417709166) (1.696969696969697,2.531848417709166) (1.7575757575757576,2.531848417709166) (1.8181818181818183,2.531848417709166) (1.878787878787879,2.531848417709166) (1.9393939393939394,2.531848417709166) (2.0,2.531848417709166) (2.0606060606060606,2.5318484177091656) (2.121212121212121,2.531848417709166) (2.1818181818181817,2.5318484177091665) (2.2424242424242427,2.531848417709166) (2.303030303030303,2.5318484177091665) (2.3636363636363638,2.5318484177091665) (2.4242424242424243,7.771561172376108e-16) (2.484848484848485,-2.1094237467877935e-15) (2.5454545454545454,-2.4980018054065835e-16) (2.606060606060606,-1.0824674490095235e-15) (2.666666666666667,8.52730145933733e-16) (2.7272727272727275,8.332007357698796e-16) (2.787878787878788,-6.636926385046085e-16) (2.8484848484848486,-6.685749910455732e-16) (2.909090909090909,-9.666384036373804e-17) (2.9696969696969697,2.734954664404533e-16) (3.0303030303030303,-9.789796087117351e-17) (3.090909090909091,9.860763805712963e-16) (3.1515151515151514,-5.850647289005321e-16) (3.2121212121212124,-3.585610702627896e-17) (3.272727272727273,8.518216860046792e-17) (3.3333333333333335,-1.2838500279050285e-16) (3.393939393939394,1.7783669840585248e-16) (3.4545454545454546,8.942088536749598e-17) (3.515151515151515,-9.992007221626425e-16) (3.5757575757575757,2.331468351712824e-15) (3.6363636363636367,-2.531848417709166) (3.6969696969696972,-2.5318484177091665) (3.757575757575758,-2.531848417709166) (3.8181818181818183,-2.5318484177091665) (3.878787878787879,-2.531848417709166) (3.9393939393939394,-2.5318484177091665) (4.0,-2.531848417709166) (4.0606060606060606,-2.531848417709166) (4.121212121212121,-2.531848417709166) (4.181818181818182,-2.531848417709166) (4.242424242424242,-2.5318484177091665) (4.303030303030303,-2.531848417709167) (4.363636363636363,-2.531848417709165) (4.424242424242425,-2.531848417709165) (4.484848484848485,-2.5318484177091665) (4.545454545454546,-2.5318484177091665) (4.606060606060606,-2.5318484177091665) (4.666666666666667,-2.5318484177091665) (4.7272727272727275,-2.5318484177091656) (4.787878787878788,-2.531848417709166) (4.848484848484849,-2.531848417709166) (4.909090909090909,-2.5318484177091665) (4.96969696969697,-2.531848417709166) (5.03030303030303,-2.5318484177091665) (5.090909090909091,-2.531848417709166) (5.151515151515151,-2.5318484177091665) (5.212121212121212,-2.531848417709166) (5.2727272727272725,-2.5318484177091665) (5.333333333333334,-2.531848417709166) (5.3939393939393945,-2.5318484177091665) (5.454545454545455,-2.531848417709166) (5.515151515151516,-2.5318484177091665) (5.575757575757576,-2.531848417709166) (5.636363636363637,-2.5318484177091665) (5.696969696969697,-2.531848417709166) (5.757575757575758,-2.5318484177091665) (5.818181818181818,-2.531848417709166) (5.878787878787879,-2.5318484177091665) (5.9393939393939394,-2.5318484177091665) (6.0,-2.5318484177091665) };
\addplot[black,thin,mark=none,solid,mark size=3pt] coordinates {(0.0,2.5318484177091665) (0.06060606060606061,2.5318484177091665) (0.12121212121212122,2.5318484177091665) (0.18181818181818182,2.5318484177091665) (0.24242424242424243,2.5318484177091665) (0.30303030303030304,2.5318484177091665) (0.36363636363636365,2.5318484177091665) (0.42424242424242425,2.5318484177091665) (0.48484848484848486,2.5318484177091665) (0.5454545454545454,2.5318484177091665) (0.6060606060606061,2.5318484177091665) (0.6666666666666667,2.5318484177091665) (0.7272727272727273,2.5318484177091665) (0.7878787878787878,2.5318484177091665) (0.8484848484848485,2.5318484177091665) (0.9090909090909092,2.5318484177091665) (0.9696969696969697,2.5318484177091665) (1.0303030303030303,2.5318484177091665) (1.0909090909090908,2.5318484177091665) (1.1515151515151516,2.5318484177091665) (1.2121212121212122,2.5318484177091665) (1.2727272727272727,2.5318484177091665) (1.3333333333333335,2.5318484177091665) (1.393939393939394,2.5318484177091665) (1.4545454545454546,2.5318484177091665) (1.5151515151515151,2.5318484177091665) (1.5757575757575757,2.5318484177091665) (1.6363636363636365,2.5318484177091665) (1.696969696969697,2.5318484177091665) (1.7575757575757576,2.5318484177091665) (1.8181818181818183,2.5318484177091665) (1.878787878787879,2.5318484177091665) (1.9393939393939394,2.5318484177091665) (2.0,2.5318484177091665) (2.0606060606060606,2.5318484177091665) (2.121212121212121,2.5318484177091665) (2.1818181818181817,2.5318484177091665) (2.2424242424242427,2.5318484177091665) (2.303030303030303,2.5318484177091665) (2.3636363636363638,2.5318484177091665) (2.4242424242424243,0.0) (2.484848484848485,0.0) (2.5454545454545454,0.0) (2.606060606060606,0.0) (2.666666666666667,0.0) (2.7272727272727275,0.0) (2.787878787878788,0.0) (2.8484848484848486,0.0) (2.909090909090909,0.0) (2.9696969696969697,0.0) (3.0303030303030303,-0.0) (3.090909090909091,-0.0) (3.1515151515151514,-0.0) (3.2121212121212124,-0.0) (3.272727272727273,-0.0) (3.3333333333333335,-0.0) (3.393939393939394,-0.0) (3.4545454545454546,-0.0) (3.515151515151515,-0.0) (3.5757575757575757,-0.0) (3.6363636363636367,-2.5318484177091665) (3.6969696969696972,-2.5318484177091665) (3.757575757575758,-2.5318484177091665) (3.8181818181818183,-2.5318484177091665) (3.878787878787879,-2.5318484177091665) (3.9393939393939394,-2.5318484177091665) (4.0,-2.5318484177091665) (4.0606060606060606,-2.5318484177091665) (4.121212121212121,-2.5318484177091665) (4.181818181818182,-2.5318484177091665) (4.242424242424242,-2.5318484177091665) (4.303030303030303,-2.5318484177091665) (4.363636363636363,-2.5318484177091665) (4.424242424242425,-2.5318484177091665) (4.484848484848485,-2.5318484177091665) (4.545454545454546,-2.5318484177091665) (4.606060606060606,-2.5318484177091665) (4.666666666666667,-2.5318484177091665) (4.7272727272727275,-2.5318484177091665) (4.787878787878788,-2.5318484177091665) (4.848484848484849,-2.5318484177091665) (4.909090909090909,-2.5318484177091665) (4.96969696969697,-2.5318484177091665) (5.03030303030303,-2.5318484177091665) (5.090909090909091,-2.5318484177091665) (5.151515151515151,-2.5318484177091665) (5.212121212121212,-2.5318484177091665) (5.2727272727272725,-2.5318484177091665) (5.333333333333334,-2.5318484177091665) (5.3939393939393945,-2.5318484177091665) (5.454545454545455,-2.5318484177091665) (5.515151515151516,-2.5318484177091665) (5.575757575757576,-2.5318484177091665) (5.636363636363637,-2.5318484177091665) (5.696969696969697,-2.5318484177091665) (5.757575757575758,-2.5318484177091665) (5.818181818181818,-2.5318484177091665) (5.878787878787879,-2.5318484177091665) (5.9393939393939394,-2.5318484177091665) (6.0,-2.5318484177091665) };
    %% SECOND PLOT
    \nextgroupplot[title={(b) time $t = 4.84\times 10^{-4} $ s.},legend style={at={($(0.62,-0.35)+(0.9cm,1cm)$)},legend columns=4}]
    \addplot[Red,very thick,mark=none,dashed,mark size=3pt] coordinates {(0.0,2.5010841927729737) (0.12244897959183673,2.469190114266878) (0.24489795918367346,2.388524326251144) (0.36734693877551017,2.2273018421403252) (0.4897959183673469,1.9469489967062041) (0.6122448979591837,1.5210982531835537) (0.7346938775510203,0.9639084261801771) (0.8571428571428571,0.35350426741421853) (0.9795918367346939,-0.17346522218220087) (1.1020408163265305,-0.4738949800515782) (1.2244897959183674,-0.4781129013135259) (1.346938775510204,-0.2546047420872028) (1.4693877551020407,0.03038529149069727) (1.5918367346938775,0.17867317133486063) (1.7142857142857142,0.17318519647638472) (1.836734693877551,0.009715941997365168) (1.9591836734693877,-0.02832074608631413) (2.0816326530612246,-0.1225397720856652) (2.204081632653061,0.0701451861529502) (2.326530612244898,-0.07977872830116252) (2.4489795918367347,0.1588281886075272) (2.571428571428571,-0.1717009307352768) (2.693877551020408,0.18355506260116172) (2.816326530612245,-0.2148933313216084) (2.9387755102040813,0.22532520111024354) (3.061224489795918,-0.2253252011102409) (3.183673469387755,0.2148933313216084) (3.306122448979592,-0.18355506260115909) (3.4285714285714284,0.17170093073527581) (3.5510204081632653,-0.1588281886075268) (3.673469387755102,0.07977872830116275) (3.7959183673469385,-0.0701451861529482) (3.9183673469387754,0.12253977208566608) (4.040816326530612,0.028320746086314623) (4.163265306122449,-0.009715941997368367) (4.285714285714286,-0.17318519647638597) (4.408163265306122,-0.17867317133486066) (4.530612244897959,-0.03038529149069498) (4.653061224489796,0.2546047420872032) (4.775510204081632,0.47811290131352563) (4.8979591836734695,0.47389498005157743) (5.020408163265306,0.17346522218220176) (5.142857142857142,-0.35350426741421775) (5.26530612244898,-0.9639084261801814) (5.387755102040816,-1.5210982531835537) (5.5102040816326525,-1.9469489967062041) (5.63265306122449,-2.2273018421403257) (5.755102040816326,-2.388524326251146) (5.877551020408163,-2.469190114266878) (6.0,-2.501084192772974) };
\addplot[Duck,thin,mark=*,solid,mark size=3pt] coordinates {(0.0,2.234925373069375) (0.12244897959183673,2.1436343280442416) (0.24489795918367346,1.9670189729629377) (0.36734693877551017,1.7188027982540781) (0.4897959183673469,1.4218218306010555) (0.6122448979591837,1.1063541637291312) (0.7346938775510203,0.8046794530615926) (0.8571428571428571,0.5437755042666179) (0.9795918367346939,0.3394941446862871) (1.1020408163265305,0.1947732778067091) (1.2244897959183674,0.10214778146582487) (1.346938775510204,0.04870933140730956) (1.4693877551020407,0.021000419082633063) (1.5918367346938775,0.008135558165171995) (1.7142857142857142,0.002812166789596373) (1.836734693877551,0.0008602565870818799) (1.9591836734693877,0.00023060813391489812) (2.0816326530612246,5.3519506461213165e-05) (2.204081632653061,1.0588930108556058e-05) (2.326530612244898,1.7502820355703498e-06) (2.4489795918367347,2.3508522331180639e-07) (2.571428571428571,2.4641221959385793e-08) (2.693877551020408,1.890518510644965e-09) (2.816326530612245,9.441019734304557e-11) (2.9387755102040813,2.3023629390609346e-12) (3.061224489795918,-2.3031420543752327e-12) (3.183673469387755,-9.44112612284182e-11) (3.306122448979592,-1.8905195141505267e-09) (3.4285714285714284,-2.4641222426181802e-08) (3.5510204081632653,-2.350852239423365e-07) (3.673469387755102,-1.7502820364285166e-06) (3.7959183673469385,-1.0588930108744782e-05) (3.9183673469387754,-5.351950646084991e-05) (4.040816326530612,-0.0002306081339143132) (4.163265306122449,-0.0008602565870813735) (4.285714285714286,-0.0028121667895960427) (4.408163265306122,-0.00813555816517166) (4.530612244897959,-0.021000419082632695) (4.653061224489796,-0.04870933140730885) (4.775510204081632,-0.1021477814658242) (4.8979591836734695,-0.1947732778067091) (5.020408163265306,-0.33949414468628647) (5.142857142857142,-0.5437755042666172) (5.26530612244898,-0.8046794530615946) (5.387755102040816,-1.1063541637291325) (5.5102040816326525,-1.4218218306010573) (5.63265306122449,-1.7188027982540806) (5.755102040816326,-1.9670189729629406) (5.877551020408163,-2.1436343280442447) (6.0,-2.2349253730693786) };
\addplot[Orange,very thick,mark=none,densely dotted,mark size=3pt] coordinates {(0.0,2.512922289601415) (0.06060606060606061,2.507159803666336) (0.12121212121212122,2.4933072053261904) (0.18181818181818182,2.4713644945809774) (0.24242424242424243,2.4341811221412843) (0.30303030303030304,2.381757088007112) (0.36363636363636365,2.302584217587779) (0.42424242424242425,2.1966625108832853) (0.48484848484848486,2.051636723745046) (0.5454545454545454,1.8675068561730643) (0.6060606060606061,1.6395141655892098) (0.6666666666666667,1.367658651993486) (0.7272727272727273,1.0659200652949896) (0.7878787878787878,0.7342984054937217) (0.8484848484848485,0.4105666475914286) (0.9090909090909092,0.09472479158810959) (0.9696969696969697,-0.16372314354382467) (1.0303030303030303,-0.3647771578043735) (1.0909090909090908,-0.47704489390672733) (1.1515151515151516,-0.5005263518508837) (1.2121212121212122,-0.4498657239486042) (1.2727272727272727,-0.32506301019988637) (1.3333333333333335,-0.18348993400450103) (1.393939393939394,-0.02514649536244179) (1.4545454545454546,0.09159748054237633) (1.5151515151515151,0.16674199370995474) (1.5757575757575757,0.18691089986382955) (1.6363636363636365,0.1521041990039989) (1.696969696969697,0.0975148523025528) (1.7575757575757576,0.023142859759495475) (1.8181818181818183,-0.030403372776003856) (1.878787878787879,-0.06312384530394177) (1.9393939393939394,-0.06886040223069545) (2.0,-0.04761304355625976) (2.0606060606060606,-0.023072733446745507) (2.121212121212121,0.004760528097847733) (2.1818181818181817,0.021185203676438287) (2.2424242424242427,0.026201293289025814) (2.303030303030303,0.018938648459264136) (2.3636363636363638,-0.0006027308128485212) (2.4242424242424243,-0.0025695183595292256) (2.484848484848485,0.013038285819222022) (2.5454545454545454,-0.00841329842817132) (2.606060606060606,-0.06692427110170765) (2.666666666666667,-0.0009936080373927803) (2.7272727272727275,0.1893786907647767) (2.787878787878788,0.0028453568328217853) (2.8484848484848486,-0.5605936098332589) (2.909090909090909,0.0012272845525674915) (2.9696969696969697,1.6883080399903014) (3.0303030303030303,-1.6883080399902968) (3.090909090909091,-0.001227284552569761) (3.1515151515151514,0.5605936098332589) (3.2121212121212124,-0.002845356832822726) (3.272727272727273,-0.1893786907647747) (3.3333333333333335,0.0009936080373919875) (3.393939393939394,0.06692427110170816) (3.4545454545454546,0.008413298428171269) (3.515151515151515,-0.01303828581922206) (3.5757575757575757,0.0025695183595286896) (3.6363636363636367,0.0006027308128476435) (3.6969696969696972,-0.018938648459265153) (3.757575757575758,-0.02620129328902751) (3.8181818181818183,-0.02118520367643984) (3.878787878787879,-0.004760528097848594) (3.9393939393939394,0.023072733446745295) (4.0,0.04761304355625988) (4.0606060606060606,0.06886040223069606) (4.121212121212121,0.06312384530394438) (4.181818181818182,0.03040337277600426) (4.242424242424242,-0.023142859759493872) (4.303030303030303,-0.09751485230255147) (4.363636363636363,-0.15210419900399794) (4.424242424242425,-0.18691089986382994) (4.484848484848485,-0.16674199370995568) (4.545454545454546,-0.09159748054237618) (4.606060606060606,0.025146495362441262) (4.666666666666667,0.1834899340044987) (4.7272727272727275,0.32506301019988665) (4.787878787878788,0.44986572394860463) (4.848484848484849,0.5005263518508843) (4.909090909090909,0.4770448939067263) (4.96969696969697,0.3647771578043749) (5.03030303030303,0.1637231435438261) (5.090909090909091,-0.09472479158810942) (5.151515151515151,-0.4105666475914277) (5.212121212121212,-0.7342984054937214) (5.2727272727272725,-1.0659200652949885) (5.333333333333334,-1.3676586519934855) (5.3939393939393945,-1.6395141655892107) (5.454545454545455,-1.867506856173064) (5.515151515151516,-2.051636723745046) (5.575757575757576,-2.1966625108832845) (5.636363636363637,-2.3025842175877775) (5.696969696969697,-2.3817570880071117) (5.757575757575758,-2.4341811221412852) (5.818181818181818,-2.4713644945809774) (5.878787878787879,-2.4933072053261904) (5.9393939393939394,-2.5071598036663363) (6.0,-2.512922289601414) };
\addplot[Blue,very thick,mark=none,solid,mark size=3pt] coordinates {(0.0,2.531848417709166) (0.12244897959183673,2.531848417709167) (0.24489795918367346,2.531848417709167) (0.36734693877551017,2.531848417709165) (0.4897959183673469,2.5318484177091647) (0.6122448979591837,-6.661338147750939e-16) (0.7346938775510203,-2.242047466345328e-15) (0.8571428571428571,-1.336287918866768e-16) (0.9795918367346939,1.3990820254935266e-15) (1.1020408163265305,-1.7763568394002503e-15) (1.2244897959183674,-3.240243116178825e-15) (1.346938775510204,2.841366885177086e-15) (1.4693877551020407,-1.3538690466452047e-15) (1.5918367346938775,-3.772748139067242e-16) (1.7142857142857142,-5.993194188317551e-16) (1.836734693877551,2.3520646964787014e-15) (1.9591836734693877,9.097798367951425e-16) (2.0816326530612246,2.9513841153104565e-15) (2.204081632653061,-1.886374069533622e-15) (2.326530612244898,1.5090992556268973e-15) (2.4489795918367347,-3.772748139067235e-16) (2.571428571428571,1.1318244417201719e-15) (2.693877551020408,-1.1318244417201734e-15) (2.816326530612245,1.886374069533621e-15) (2.9387755102040813,-3.0181985112537953e-15) (3.061224489795918,1.509099255626895e-15) (3.183673469387755,-3.3954733251605203e-15) (3.306122448979592,1.131824441720175e-15) (3.4285714285714284,-7.54549627813448e-16) (3.5510204081632653,2.2204460492503017e-16) (3.673469387755102,-1.7527452776469433e-15) (3.7959183673469385,1.819559673590282e-15) (3.9183673469387754,1.3362879188667458e-16) (4.040816326530612,-1.5759136515702374e-15) (4.163265306122449,-7.997626066617703e-16) (4.285714285714286,-4.521297884831934e-17) (4.408163265306122,-3.906376930953917e-15) (4.530612244897959,3.79434955616226e-15) (4.653061224489796,-3.861163952105597e-15) (4.775510204081632,1.6643294646085906e-15) (4.8979591836734695,-4.1068201187839345e-15) (5.020408163265306,6.209208359267724e-16) (5.142857142857142,-2.7745524892337463e-15) (5.26530612244898,9.98195649833496e-16) (5.387755102040816,-6.661338147750939e-16) (5.5102040816326525,-2.531848417709167) (5.63265306122449,-2.531848417709165) (5.755102040816326,-2.531848417709168) (5.877551020408163,-2.531848417709166) (6.0,-2.531848417709165) };
\addplot[Purple,very thick,mark=|,solid,mark size=3pt] coordinates {(0.0,2.4502878879638077) (0.06060606060606061,2.4331156617141234) (0.12121212121212122,2.3912249841562767) (0.18181818181818182,2.334329048267753) (0.24242424242424243,2.2418909444576474) (0.30303030303030304,2.133656683027458) (0.36363636363636365,1.982993211922006) (0.42424242424242425,1.8174087611146408) (0.48484848484848486,1.6154775867478133) (0.5454545454545454,1.4047401021710633) (0.6060606060606061,1.1801065385155256) (0.6666666666666667,0.9574083982688202) (0.7272727272727273,0.7516285480780078) (0.7878787878787878,0.5581100772293209) (0.8484848484848485,0.40510029277008563) (0.9090909090909092,0.2688057300718126) (0.9696969696969697,0.1784269686067307) (1.0303030303030303,0.10222920629772528) (1.0909090909090908,0.06123973598946104) (1.1515151515151516,0.028491831319577256) (1.2121212121212122,0.015077503968816922) (1.2727272727272727,0.004852500743675085) (1.3333333333333335,0.002132768618835191) (1.393939393939394,9.497890449658636e-05) (1.4545454545454546,-3.613906028997789e-05) (1.5151515151515151,-0.0001771883066314252) (1.5757575757575757,-8.197840812272274e-05) (1.6363636363636365,-3.778569739159183e-05) (1.696969696969697,-1.3168311989532245e-05) (1.7575757575757576,-7.18048230077542e-07) (1.8181818181818183,1.2525161562481774e-07) (1.878787878787879,7.57143015280698e-07) (1.9393939393939394,2.448660220997095e-07) (2.0,7.135513801818636e-08) (2.0606060606060606,1.4595492827506847e-08) (2.121212121212121,-6.363153796720189e-09) (2.1818181818181817,-1.8377351508576813e-09) (2.2424242424242427,-9.025903498190151e-10) (2.303030303030303,-1.5217965502392056e-10) (2.3636363636363638,3.1015363059108134e-11) (2.4242424242424243,7.240747155190924e-12) (2.484848484848485,4.49718045151787e-12) (2.5454545454545454,4.994640475976035e-13) (2.606060606060606,-1.0901286833703581e-13) (2.666666666666667,-1.5949332804559343e-14) (2.7272727272727275,-8.856384153319333e-15) (2.787878787878788,-1.6410925375832847e-15) (2.8484848484848486,-6.505398960621357e-16) (2.909090909090909,-1.0083526676638663e-15) (2.9696969696969697,-1.001626558081926e-15) (3.0303030303030303,-8.578917651911006e-16) (3.090909090909091,-8.575534244519106e-16) (3.1515151515151514,-5.996684329169147e-16) (3.2121212121212124,-8.861306261599516e-17) (3.272727272727273,6.792476193910383e-15) (3.3333333333333335,1.3748077565965016e-14) (3.393939393939394,1.0666160267206766e-13) (3.4545454545454546,-5.017226512299146e-13) (3.515151515151515,-4.499410687924808e-12) (3.5757575757575757,-7.243247705404374e-12) (3.6363636363636367,-3.1017486754894407e-11) (3.6969696969696972,1.5217809344571135e-10) (3.757575757575758,9.025888787161362e-10) (3.8181818181818183,1.8377338911614902e-09) (3.878787878787879,6.363153398776474e-09) (3.9393939393939394,-1.4595492913112185e-08) (4.0,-7.135513834340724e-08) (4.0606060606060606,-2.4486602290663936e-07) (4.121212121212121,-7.571430163470533e-07) (4.181818181818182,-1.252516171290449e-07) (4.242424242424242,7.180482286441556e-07) (4.303030303030303,1.3168311988130426e-05) (4.363636363636363,3.7785697390737625e-05) (4.424242424242425,8.197840812161226e-05) (4.484848484848485,0.00017718830662956148) (4.545454545454546,3.613906028778548e-05) (4.606060606060606,-9.497890449961183e-05) (4.666666666666667,-0.002132768618838574) (4.7272727272727275,-0.00485250074367774) (4.787878787878788,-0.015077503968819619) (4.848484848484849,-0.02849183131958028) (4.909090909090909,-0.0612397359894635) (4.96969696969697,-0.10222920629772878) (5.03030303030303,-0.17842696860673407) (5.090909090909091,-0.2688057300718156) (5.151515151515151,-0.40510029277008763) (5.212121212121212,-0.5581100772293224) (5.2727272727272725,-0.7516285480780089) (5.333333333333334,-0.9574083982688223) (5.3939393939393945,-1.1801065385155272) (5.454545454545455,-1.4047401021710655) (5.515151515151516,-1.6154775867478157) (5.575757575757576,-1.8174087611146434) (5.636363636363637,-1.982993211922008) (5.696969696969697,-2.1336566830274606) (5.757575757575758,-2.2418909444576505) (5.818181818181818,-2.3343290482677563) (5.878787878787879,-2.39122498415628) (5.9393939393939394,-2.433115661714126) (6.0,-2.4502878879638113) };
\addplot[Green,thick,mark=x,only marks,mark size=3pt] coordinates {(0.0,2.5318484177091647) (0.06060606060606061,2.5318484177091647) (0.12121212121212122,2.5318484177091642) (0.18181818181818182,2.5318484177091642) (0.24242424242424243,2.5318484177091625) (0.30303030303030304,2.5318484177091625) (0.36363636363636365,2.531848417709162) (0.42424242424242425,2.5318484177091616) (0.48484848484848486,2.5318484177091647) (0.5454545454545454,2.5318484177091647) (0.6060606060606061,-2.5535129566378593e-15) (0.6666666666666667,-4.107825191113079e-15) (0.7272727272727273,-1.1821865067599611e-15) (0.7878787878787878,-1.0382595424903476e-15) (0.8484848484848485,-2.594083204450083e-16) (0.9090909090909092,-5.423644308750513e-16) (0.9696969696969697,-2.7327311521224373e-15) (1.0303030303030303,-2.729968157964989e-15) (1.0909090909090908,-1.849588141779186e-15) (1.1515151515151516,-2.591303956721436e-15) (1.2121212121212122,3.2358887055886763e-16) (1.2727272727272727,-4.140148282555056e-16) (1.3333333333333335,-2.964668530546082e-15) (1.393939393939394,-2.985322823581436e-15) (1.4545454545454546,-3.431128126883216e-16) (1.5151515151515151,-3.6823398093509377e-16) (1.5757575757575757,-4.976978604008548e-16) (1.6363636363636365,-7.913669349592946e-16) (1.696969696969697,9.30104610666006e-16) (1.7575757575757576,1.781082693008645e-16) (1.8181818181818183,-3.200542091552869e-16) (1.878787878787879,-2.57663792581449e-16) (1.9393939393939394,-1.066037019433047e-16) (2.0,-2.7025089943367067e-17) (2.0606060606060606,-1.393229467111157e-15) (2.121212121212121,-5.16756164175769e-16) (2.1818181818181817,-4.3607449790516604e-16) (2.2424242424242427,-3.184751299082477e-16) (2.303030303030303,4.580155256428032e-16) (2.3636363636363638,9.174549380974228e-16) (2.4242424242424243,-1.6015820734760246e-15) (2.484848484848485,-4.852351838876055e-16) (2.5454545454545454,-4.2783451322396646e-17) (2.606060606060606,2.1961507739911167e-16) (2.666666666666667,-2.13962396391271e-15) (2.7272727272727275,-2.210842176891267e-15) (2.787878787878788,5.451677388715817e-16) (2.8484848484848486,3.862135150185818e-16) (2.909090909090909,-1.0581116162267363e-15) (2.9696969696969697,-1.0287056411368938e-15) (3.0303030303030303,2.4107097344619627e-16) (3.090909090909091,-1.0744218155952165e-16) (3.1515151515151514,3.612033351810279e-16) (3.2121212121212124,5.269750845190583e-16) (3.272727272727273,-9.56438222610769e-16) (3.3333333333333335,-5.958638672061637e-16) (3.393939393939394,9.031576787357558e-16) (3.4545454545454546,8.731991606644893e-16) (3.515151515151515,-4.649512583958853e-16) (3.5757575757575757,-4.664299954942793e-16) (3.6363636363636367,-2.658188422087487e-16) (3.6969696969696972,8.898721613203251e-17) (3.757575757575758,9.018038775143556e-16) (3.8181818181818183,4.3046375203582656e-16) (3.878787878787879,-2.006864452247978e-15) (3.9393939393939394,-2.386804522746043e-15) (4.0,2.916840469072776e-16) (4.0606060606060606,5.964943727928409e-16) (4.121212121212121,4.3783343187558093e-16) (4.181818181818182,8.94434197674601e-16) (4.242424242424242,-7.57073748850323e-16) (4.303030303030303,-4.067871315316464e-17) (4.363636363636363,5.414274681889935e-16) (4.424242424242425,6.140085352842909e-16) (4.484848484848485,-1.2141910461027111e-15) (4.545454545454546,-1.4031210894910502e-15) (4.606060606060606,-1.6965880323555225e-16) (4.666666666666667,-8.95351242541286e-16) (4.7272727272727275,1.334674378172877e-15) (4.787878787878788,4.41682461227368e-16) (4.848484848484849,-1.0242381251598384e-15) (4.909090909090909,-1.4162423843572238e-15) (4.96969696969697,1.3047396462906642e-15) (5.03030303030303,9.157064029596423e-16) (5.090909090909091,-8.666079337415476e-16) (5.151515151515151,-8.193229479620654e-16) (5.212121212121212,-3.2176366267546826e-16) (5.2727272727272725,-2.991571732513092e-16) (5.333333333333334,-6.661338147750951e-16) (5.3939393939393945,2.44249065417534e-15) (5.454545454545455,-2.531848417709165) (5.515151515151516,-2.531848417709165) (5.575757575757576,-2.5318484177091656) (5.636363636363637,-2.531848417709166) (5.696969696969697,-2.531848417709165) (5.757575757575758,-2.5318484177091656) (5.818181818181818,-2.531848417709166) (5.878787878787879,-2.531848417709166) (5.9393939393939394,-2.5318484177091656) (6.0,-2.5318484177091656) };
\addplot[black,thin,mark=none,solid,mark size=3pt] coordinates {(0.0,2.5318484177091665) (0.06060606060606061,2.5318484177091665) (0.12121212121212122,2.5318484177091665) (0.18181818181818182,2.5318484177091665) (0.24242424242424243,2.5318484177091665) (0.30303030303030304,2.5318484177091665) (0.36363636363636365,2.5318484177091665) (0.42424242424242425,2.5318484177091665) (0.48484848484848486,2.5318484177091665) (0.5454545454545454,2.5318484177091665) (0.6060606060606061,0.0) (0.6666666666666667,0.0) (0.7272727272727273,0.0) (0.7878787878787878,0.0) (0.8484848484848485,0.0) (0.9090909090909092,0.0) (0.9696969696969697,0.0) (1.0303030303030303,0.0) (1.0909090909090908,0.0) (1.1515151515151516,0.0) (1.2121212121212122,0.0) (1.2727272727272727,0.0) (1.3333333333333335,0.0) (1.393939393939394,0.0) (1.4545454545454546,0.0) (1.5151515151515151,0.0) (1.5757575757575757,0.0) (1.6363636363636365,0.0) (1.696969696969697,0.0) (1.7575757575757576,0.0) (1.8181818181818183,0.0) (1.878787878787879,0.0) (1.9393939393939394,0.0) (2.0,0.0) (2.0606060606060606,0.0) (2.121212121212121,0.0) (2.1818181818181817,0.0) (2.2424242424242427,0.0) (2.303030303030303,0.0) (2.3636363636363638,0.0) (2.4242424242424243,0.0) (2.484848484848485,0.0) (2.5454545454545454,0.0) (2.606060606060606,0.0) (2.666666666666667,0.0) (2.7272727272727275,0.0) (2.787878787878788,0.0) (2.8484848484848486,0.0) (2.909090909090909,0.0) (2.9696969696969697,0.0) (3.0303030303030303,-0.0) (3.090909090909091,-0.0) (3.1515151515151514,-0.0) (3.2121212121212124,-0.0) (3.272727272727273,-0.0) (3.3333333333333335,-0.0) (3.393939393939394,-0.0) (3.4545454545454546,-0.0) (3.515151515151515,-0.0) (3.5757575757575757,-0.0) (3.6363636363636367,-0.0) (3.6969696969696972,-0.0) (3.757575757575758,-0.0) (3.8181818181818183,-0.0) (3.878787878787879,-0.0) (3.9393939393939394,-0.0) (4.0,-0.0) (4.0606060606060606,-0.0) (4.121212121212121,-0.0) (4.181818181818182,-0.0) (4.242424242424242,-0.0) (4.303030303030303,-0.0) (4.363636363636363,-0.0) (4.424242424242425,-0.0) (4.484848484848485,-0.0) (4.545454545454546,-0.0) (4.606060606060606,-0.0) (4.666666666666667,-0.0) (4.7272727272727275,-0.0) (4.787878787878788,-0.0) (4.848484848484849,-0.0) (4.909090909090909,-0.0) (4.96969696969697,-0.0) (5.03030303030303,-0.0) (5.090909090909091,-0.0) (5.151515151515151,-0.0) (5.212121212121212,-0.0) (5.2727272727272725,-0.0) (5.333333333333334,-0.0) (5.3939393939393945,-0.0) (5.454545454545455,-2.5318484177091665) (5.515151515151516,-2.5318484177091665) (5.575757575757576,-2.5318484177091665) (5.636363636363637,-2.5318484177091665) (5.696969696969697,-2.5318484177091665) (5.757575757575758,-2.5318484177091665) (5.818181818181818,-2.5318484177091665) (5.878787878787879,-2.5318484177091665) (5.9393939393939394,-2.5318484177091665) (6.0,-2.5318484177091665) };
\addlegendentry{usl 1ppc}
\addlegendentry{usl-pic 1ppc}
\addlegendentry{usl 2ppc}
\addlegendentry{dgmpm 1ppc}
\addlegendentry{dgmpm 2ppc}
\addlegendentry{dgmpm 2ppc (RK2)}
\addlegendentry{exact}
%\legend{usl 1ppc,usl-pic 1ppc,usl 2ppc,dgmpm 1ppc,dgmpm 2ppc,dgmpm 2ppc (RK2),exact}

  \end{groupplot}
\end{tikzpicture}
%%% Local Variables:
%%% mode: latex
%%% TeX-master: "../../mainManuscript"
%%% End:


}
  \caption{elastic RP velo}
  \label{fig:velo_elastic_RP}
\end{figure}
%%% Local Variables:
%%% mode: latex
%%% TeX-master: "../mainManuscript"
%%% End:
