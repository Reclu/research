% \begin{remark}
%   This problem can be solved in a two-dimensional setting by considering an infinite medium in direction $\vect{e}_3$, of dimension $l\times h$ in directions $\vect{e}_1$ and $\vect{e}_2$, which bottom and top boundaries are assumed vertically fixed.
%   %A two-dimensional setting may be used to solve this problem by considering an infinite medium in direction $\vect{e}_3$, of dimension $l\times h$ in directions $\vect{e}_1$ and $\vect{e}_2$, which bottom and top boundaries are assumed vertically fixed.
%   In that case, DGMPM results still show a good agreement with the exact solution of the problem \cite{DGMPM}.
% \end{remark}


\subsubsection*{Plane wave in a two-dimensional hyperelastic medium}
We move to multi-dimensional simulations by considering an infinite medium in directions $\vect{e}_2$ and $\vect{e}_3$, and of width $l$ in direction $\vect{e}_1$ (see figure \ref{fig:2dHEbar}), with Riemann-type data on the initial velocity ($v_1(x)=v_0$ for $x <l/2$ and $v_1(x)=-v_0$ for $x >l/2$ with $v_0=100 \: m/s$). Both ends of the domain are traction-free.
The solid is now made of a compressible hyperelastic neo-Hookean material which, unlike the SVK model, is based on a polyconvex stored energy function so that hyperbolicity is ensured whatever $\tens{F}$ (see section \ref{sec:constitutive-equations}).
While it has been seen that a compressive load on a SVK material leads to a rarefaction wave, this is not the case for the neo-Hookean model for which a compressive load involves a shock wave (see remark \ref{rq:charach_neoHook}).
\begin{figure}[h!]
  \centering
  \begin{tikzpicture}[scale=0.6]
  % \draw[thick] (0,0) --(4,0)--(4,3)--(0,3)--(0,0);
  \draw[thick] (0,0) --(6,0)--(6,1)--(0,1)--(0,0);
  \foreach \x in {0.5,1.,...,5.5} 
  \draw(\x,-0.2)circle(0.2);
  \foreach \x in {0.5,1.,...,5.5} 
  \draw(\x,1.2)circle(0.2);
  \draw(0,-0.4)--(6.,-0.4);
  \draw(0,1.4)--(6.,1.4);
  \fill [pattern=north east lines](0.0,-0.8)rectangle+(6.,0.4);
  \fill [pattern=north east lines](0.,1.4)rectangle+(6.,0.4);
  \draw[>=stealth,<->](6.2,0)--node[right]{\scriptsize $h$}(6.2,1.);
  \draw[>=stealth,<->](0,1.9)--node[above=1pt]{\scriptsize$l$}(6,1.9);
  \draw[>=stealth,->](-3.2+1.75,0.)--(-2.2+1.75,0.)node(a)[anchor=north]{\scriptsize$\vect{e}_1$};
  \draw[>=stealth,->](-3.2+1.75,0.)--(-3.2+1.75,1)node(a)[anchor=east]{\scriptsize$\vect{e}_2$};
\end{tikzpicture}

%%% Local Variables:
%%% mode: latex
%%% TeX-master: "../../presentation"
%%% End:

  \caption{Geometry and loading conditions for the plane wave problem in a two-dimensional solid.}
  \label{fig:2dHEbar}
\end{figure}

The domain is modeled by a finite medium, with zero shear stress and transverse velocity components prescribed on the top, bottom and right boundaries (figure \ref{fig:2dHEbar}).
In addition zero out-of-plane velocity and strain components are assumed so that plane waves can be simulated.
The solid is discretized with $80\times 4$ material points in a regular grid made of either $80\times 4$ or $40\times 2$ cells so that the 1ppc and 4ppc discretizations are used.
\begin{figure}[h!]
  \centering
  \begin{tikzpicture}[scale=0.8]
  \begin{groupplot}[group style={group size=2 by 4,
      ylabels at=edge left, yticklabels at=edge left,
      horizontal sep=4.ex,
      vertical sep=4ex,},
    enlargelimits=0,
    xmin=0.,xmax=10., ymin=-0.,ymax=2.
    ,axis equal image
    ,axis on top,scale only axis,xtick=\empty,ytick=\empty,width=0.45\linewidth,
    ,colorbar style={
      title style={
        font=\scriptsize,
        at={(-2.,-1.25)},
        anchor=north west
      },yticklabel style={font=\scriptsize}
      ,at={(current axis.south east)},anchor=south west
    }]
    %% FIRST ROW (time 1 = 3.5e-4s)
    %%% RANGE -2.0e7 -- 2.9e8
    \nextgroupplot[title={(a) $t= 9.94 \times 10^{-4}\:s$},ylabel={dgmpm (1ppc)},ylabel style={font=\footnotesize}]\addplot graphics[xmin=-0.,xmax=10., ymin=-0.,ymax=2.] {chapter4/pngFigures/PW_dgmpm1ppc_65.png};
    \nextgroupplot[title={(b) $t= 2.54 \times 10^{-4}\:s$},ylabel={\color{white}dgmpm (1pc)},ylabel style={font=\footnotesize}]\addplot graphics[xmin=-0.,xmax=10., ymin=-0.,ymax=2.] {chapter4/pngFigures/PW_dgmpm1ppc_132.png};

    \nextgroupplot[ylabel={dgmpm (4ppc)},ylabel style={font=\footnotesize}]\addplot graphics[xmin=0.,xmax=10., ymin=-0.,ymax=2.] {chapter4/pngFigures/PW_dgmpm4ppc_65.png};
    \nextgroupplot[ylabel={\color{white}dgmpm (4ppc)},ylabel style={font=\footnotesize}]\addplot graphics[xmin=-0.,xmax=10., ymin=-0.,ymax=2.] {chapter4/pngFigures/PW_dgmpm4ppc_132.png};
    
    \nextgroupplot[ylabel={mpm (1ppc)},ylabel style={font=\footnotesize}]\addplot graphics[xmin=0.,xmax=10., ymin=-0.,ymax=2.] {chapter4/pngFigures/PW_mpm1ppc_65.png};
    \nextgroupplot[,ylabel={\color{white}mpm (1ppc)},ylabel style={font=\footnotesize}]\addplot graphics[xmin=-0.,xmax=10., ymin=-0.,ymax=2.] {chapter4/pngFigures/PW_mpm1ppc_132.png};
    
    \nextgroupplot[ylabel={mpm (4ppc)},ylabel style={font=\footnotesize},colorbar horizontal,colorbar  style={
      title style={yshift=-1.5cm},
      title= {$\Pi_{11}\: (GPa)$},
      xtick={0.,20.},
      xticklabels={-6.2,0},
    }]
    %\addplot[scatter,scatter src=y,mark size=0.pt] coordinates {(0.,-5.7e-2) (0.,2.9e-1)};% Fake extreme values to fix scale
    \addplot[scatter,scatter src=x,mark size=0.pt] coordinates {(0.,0.) (20.,0.)};% Fake extreme values to fix scale
    \addplot graphics[xmin=0.,xmax=10., ymin=-0.,ymax=2.] {chapter4/pngFigures/PW_mpm4ppc_65.png};
    
    \nextgroupplot[colorbar horizontal,colorbar style={
      title= {$\Pi_{11}\: (GPa)$},
      title style={yshift=-1.5cm},
      xtick={0.,20.},
      xticklabels={-6.8,0},
    },ylabel={\color{white}mpm (4ppc)},ylabel style={font=\footnotesize}]
    \addplot[scatter,scatter src=x,mark size=0.pt] coordinates {(0.,0.) (20.,0.)};% Fake extreme values to fix scale
    \addplot graphics[xmin=-0.,xmax=10., ymin=-0.,ymax=2.] {chapter4/pngFigures/PW_mpm4ppc_132.png};   
    

    % \nextgroupplot[title={(c) DGMPM 4ppc}]\addplot graphics[xmin=-0.,xmax=1., ymin=-0.,ymax=1.] {chapter4/pngFigures/dgmpm4ppc_stress_115.png};
    % \nextgroupplot[title={(d) MPM 1ppc},
    % colorbar,colorbar style={
    % title= {$\sigma_{11}\: (MPa)$},
    %   ytick={-5.7e-2,2.e-1,2.9e-1},
    %   yticklabels={-57,200,290},
    % }]
    % \addplot[scatter,scatter src=y,mark size=0.pt] coordinates {(0.,-5.7e-2) (0.,2.9e-1)};% Fake extreme values to fix scale
    % \addplot graphics[xmin=-0.,xmax=1., ymin=-0.,ymax=1.] {chapter4/pngFigures/mpm_stress_115.png};

    % %% SECOND ROW (time 2 =1.e-3s)
    % %%% RANGE -4.4e7 -- 4.2e8
    % \nextgroupplot[ylabel={$t=1.0\times 10^{-3} \:s$}]\addplot graphics[xmin=0.,xmax=1., ymin=-0.,ymax=1.] {chapter4/pngFigures/fem_stress_338.png};
    % \nextgroupplot[]\addplot graphics[xmin=-0.,xmax=1., ymin=-0.,ymax=1.] {chapter4/pngFigures/dgmpm1ppc_stress_338.png};
    % \nextgroupplot[]\addplot graphics[xmin=-0.,xmax=1., ymin=-0.,ymax=1.] {chapter4/pngFigures/dgmpm4ppc_stress_338.png};
    % \nextgroupplot[colorbar,colorbar style={
    %   title= {$\sigma_{11}\: (MPa)$},
    %   ytick={-4.3e-2,2.e-1,4.8e-1},
    %   yticklabels={-43,200,480},
    % }]
    % \addplot[scatter,scatter src=y,mark size=0.pt] coordinates {(0.,-4.3e-2) (0.,4.8e-1)};% Fake extreme values to fix scale
    % \addplot graphics[xmin=-0.,xmax=1., ymin=-0.,ymax=1.] {chapter4/pngFigures/mpm_stress_338.png};
    
  \end{groupplot}
\end{tikzpicture}



%%% Local Variables:
%%% mode: latex
%%% TeX-master: "../../mainManuscript"
%%% End:

  \caption{Isovalues of the longitudinal PK1 stress $\Pi_{11}$ solution of the plane wave problem in a two-dimensional compressible neo-Hookean material with Riemann-type data on the initial velocity. Comparison between DGMPM-DCU with 1ppc (CFL=$0.57$) or 4ppc (CFL=$0.23$) and MPM solutions also using 1ppc and 4ppc (CFL=$0.7$).}
  \label{fig:2dplane_Wave}
\end{figure}
%The first space discretization in which the particles are located at elements centers is referred to as $1ppc$ while the second, consisting in material points that are symemtrically placed with respect to cells centers, is referred to as $4ppc$.  
The simulation has been performed with the DGMPM-DCU, since BCs reproduce a plane wave that does not involve Poisson's effect, and compared to MPM solutions. A similar problem is considered in \cite{DGMPM} with a traction force on the left boundary applied on a SVK material, for which a comparison with the exact solution shows a good agreement.

The isovalues of the longitudinal PK1 stress $\Pi_{11}$ can be seen in figure \ref{fig:2dplane_Wave} at two different times before the wave reaches the left and right ends of the domain.
As for one-dimensional problems, oscillations appear in MPM solutions but not in DGMPM ones so that the maximum amplitude of stress in the former is mush higher that in the latter.
In contrast to the previous simulations, the lower CFL numbers used for this two-dimensional problem lead to a less accurate resolution of the shock wave with additional numerical diffusion.
Nevertheless, the one-dimensional dependency of the solution is not disturbed within the numerical schemes.
\begin{figure}[h!]
  \centering
  \begin{tikzpicture}[scale=.8]
\begin{axis}[xlabel=$time (s)$,ylabel=$\frac{e}{e_{max}}$,ymajorgrids=true,xmajorgrids=true,legend pos=outer north east,title={},legend style={at={($(0.1,-0.35)+(0.5cm,1cm)$)},legend columns=2}]%legend pos=outer north east
\addplot[Red,very thick,mark=none,dashed,mark size=3pt] coordinates {(9.000573014794848e-06,1.0) (1.796381527756706e-05,0.9914533911626189) (2.6909286505214812e-05,0.9880756808914547) (3.585756699770927e-05,0.9854163150120561) (4.4822911312038465e-05,0.9789097295366027) (5.378423847837861e-05,0.9708660526421296) (6.273631604137162e-05,0.9656702364262686) (7.169197372887131e-05,0.9625409739506655) (8.066025051487872e-05,0.9574679354059819) (8.961671522275958e-05,0.9500647102454508) (9.857118385750588e-05,0.9437541147042068) (0.00010753083860163041,0.9398798249683428) (0.00011649057391700033,0.9356832648318265) (0.0001254456574397425,0.9291492959903835) (0.00013440464853789163,0.9222530895675854) (0.0001433667710713556,0.9174592561135397) (0.00015232285930250012,0.9135982466881382) (0.00016128049723290437,0.9080047243813228) (0.00017024437615889375,0.9010493561227031) (0.00017920147124044156,0.8953825087396217) (0.00018815854511947014,0.8913497349914383) (0.00019712199080299233,0.8865572214476816) (0.00020607998559766153,0.8800037607395182) (0.0002150368836279206,0.8736769151351442) (0.0002239987476753767,0.8691054692043627) (0.00023295751227616783,0.8648169199264896) (0.00024191442874660712,0.8589184707817774) (0.00025087522253969616,0.8522957120637725) };
\addplot[Orange,very thick,mark=none,densely dotted,mark size=3pt] coordinates {(1.8001146029589695e-05,1.0) (3.59337432830172e-05,0.987287010657628) (5.38434339576714e-05,0.9814549372480271) (7.177081672658308e-05,0.970637903879259) (8.968366215918615e-05,0.9554754935028931) (0.00010757624410062916,0.9454593249947213) (0.0001254888371660142,0.9382623428226724) (0.00014341501849920492,0.9259931945416342) (0.00016132703431225403,0.9120343329658626) (0.00017924741148391378,0.9029541004990023) (0.00019716238450120902,0.8947404029938869) (0.00021508320559939724,0.8818562426281039) (0.0002330026139290141,0.8689169578863969) (0.0002509126129702663,0.8603520482744869) };
\addplot[Purple,thick,mark=x,solid,mark size=3pt] coordinates {(7.314899467950566e-06,1.0) (1.4598146985673229e-05,0.9901528409608024) (2.186753874650882e-05,0.9809991185505665) (2.9130901458336235e-05,0.9721874325775748) (3.63916412745582e-05,0.9635884129780377) (4.365124162858778e-05,0.9551376250276616) (5.0910347345400474e-05,0.9467976667376566) (5.8169238149383194e-05,0.9385446028223614) (6.542803550311455e-05,0.9303620393969964) (7.268679212790383e-05,0.9222381699438393) (7.994553091400178e-05,0.9141641574908818) (8.72042617922543e-05,0.9061331831412421) (9.446298907826645e-05,0.8981398539433616) (0.00010172171466011121,0.890179817127578) (0.00010898043935922209,0.8822494990710554) (0.00011623916354340497,0.874345922941322) (0.00012349788737204238,0.8664665778426528) (0.00013075661094321585,0.858609322671656) (0.00013801533428144574,0.850772314137439) (0.00014527405742351267,0.8429539517751796) (0.00015253278038167682,0.8351528354294445) (0.00015979150316819838,0.8273677318145014) (0.00016705022578307734,0.8195975479492331) (0.0001743089482508341,0.8118413097554148) (0.00018156767057146868,0.8040981447385332) (0.0001888263927572412,0.7963672677259795) (0.00019608511482041192,0.7886479691085402) (0.00020334383676098083,0.7809396050484623) (0.0002106025585789479,0.7732415892416323) (0.00021786128028657335,0.7655533859586525) (0.00022512000188385718,0.7578745041586655) (0.00023237872338305955,0.7502044924212865) (0.0002396374447841805,0.7425429346279839) (0.00024689616608722,0.7348894462080099) (0.00025415488730443824,0.7272436708773569) };
\addplot[Green,very thick,mark=+,solid,mark size=3pt] coordinates {(5.8804885337294695e-06,1.0) (1.174826095073322e-05,0.9871640924586731) (1.7607631385637616e-05,0.9762308082517568) (2.3460937455554108e-05,0.9662965796826042) (2.9309741742850362e-05,0.9569353639389021) (3.515517745412317e-05,0.9479551306920194) (4.0998087388800795e-05,0.9392550356546122) (4.683910268494923e-05,0.930772911643581) (5.267869681390552e-05,0.9224666140508109) (5.851722504021463e-05,0.9143059680447483) (6.435495392470625e-05,0.9062685194043427) (7.019208344645524e-05,0.8983370369661849) (7.602876357621121e-05,0.8904979605583739) (8.186510678872026e-05,0.88274039349438) (8.77011974127584e-05,0.8750554248472826) (9.353709865908055e-05,0.8674356582122611) (9.93728579048248e-05,0.8598748748510392) (0.00010520851066206334,0.8523677870686908) (0.0001110440835480803,0.8449098543186081) (0.0001168795965148667,0.8374971431097786) (0.0001227150645257888,0.8301262193670743) (0.0001285504987783776,0.8227940637651736) (0.00013438590768043973,0.8154980046605792) (0.0001402212975401827,0.8082356645278571) (0.00014605667306898687,0.8010049165342669) (0.00015189203781513662,0.7938038486770571) (0.0001577273944299691,0.7866307348750146) (0.00016356274491429617,0.7794840103317744) (0.0001693980907564002,0.7723622512701634) (0.0001752334330798815,0.7652641576883985) (0.00018106877273236632,0.758188538890039) (0.00018690411034464512,0.7511343011030119) (0.00019273944638981014,0.7441004370098065) (0.00019857478122268016,0.7370860170751803) (0.00020441011511922514,0.7300901822562937) (0.00021024544827656633,0.7231121386431058) (0.00021608078086225672,0.7161511535868276) (0.00022191611297485688,0.7092065543688695) (0.00022775144471292734,0.7022777293268355) (0.00023358677614546044,0.6953641326839319) (0.00023942210731188037,0.6884652935895073) (0.0002452574382614674,0.6815808308188118) (0.00025109276902378965,0.6747104742805968) };
\legend{mpm 1ppc,mpm 4ppc,dgmpm 1ppc,dgmpm 4ppc}
\end{axis}
\end{tikzpicture}
%%% Local Variables:
%%% mode: latex
%%% TeX-master: "../../aRenaud"
%%% End:

  \caption{Evolution of total energy for the plane wave problem in a two-dimensional compressible neo-Hookean material.}
  \label{fig:pw_energy}
\end{figure}

The evolutions of total energy during DGMPM and MPM computations are compared in figure \ref{fig:pw_energy}.
As for the elastic case, it can be seen that the DGMPM is more dissipative than the MPM. It is however worth noticing that both methods dissipate more energy than in the linear elastic case. 

%%% Local Variables:
%%% mode: latex
%%% ispell-local-dictionary: "american"
%%% TeX-master: "../mainManuscript"
%%% End:
