The above solid is now assumed to be made of an elastic-plastic material with linear isotropic hardening. A tensile impact of amplitude $\sigma^d=2\sigma^y$ leading to plastic flow is considered. 
% is still considered on the bottom left part of the domain, as depicted in figure \ref{fig:2D_planeStrain}\subref{subfig:2D_problem}, with the traction force $\sigma^d=2\sigma^y$ so that plastic flow occurs.

Comparison between FEM (Cast3M), MPM and DGMPM using either one or four particles per element is made.
%on the equivalent plastic strain $p$ (similar observation as above being made on the longitudinal stress $\sigma_{11}$).
The computation of intercell fluxes within the DGMPM is based on an elastic Riemann solver, and plastic flow is integrated by means of a radial return algorithm.
The evolution of longitudinal stress $\sigma_{11}$ and plastic strain $\eps^p_{11}$ are depicted in figure \ref{fig:elastlines_stress}.
It can first be seen that similar remarks as above can be made about the longitudinal stress though the integration of plastic flow leads to fewer oscillations in the finite element stress.
Next, DGMPM longitudinal plastic strains before the reflection of the pressure wave (figure \ref{fig:elastlines_stress}\subref{subfig:2dplast1}) are quite close to each other.
On the other hand, the FEM plastic strain curve is above the others and the MPM one is in advance compared to the other solutions.
After the reflection (figure \ref{fig:elastlines_stress}\subref{subfig:2dplast2}), the stress profiles are rather close to each other.
However, the MPM solution is far higher than the others on the right end of the domain.
The same observations can be made on the longitudinal plastic strain which is quite similar for both DGMPM and FEM solutions, though the DGMPM maximum value computed with one ppc is higher than the others.

\begin{figure}[h!]
  {\phantomsubcaption \label{subfig:2dplast1}}
  {\phantomsubcaption \label{subfig:2dplast2}}
  \begin{tikzpicture}
  \begin{groupplot}[group style={group size=2 by 2,
ylabels at=edge left, yticklabels at=edge left,horizontal sep=3.ex,vertical sep=4.ex,
xticklabels at=edge bottom,xlabels at=edge bottom},
ymajorgrids=true,xmajorgrids=true,
axis on top,scale only axis,width=0.4\linewidth, every x tick scale label/.style={at={(xticklabel* cs:1.05,0.75cm)},anchor=near yticklabel}]
\nextgroupplot[ylabel=$\sigma_{11}$,title={(a) $t=2.5 \times 10^{-4} \: s$},ymin=-0.5e6,ymax=1.5e9]
\addplot[Red,very thick,no markers] table[x=Points:0,y=S11] {chapter4/csvFiles/2dEP_fem_115.csv};
\addplot[Blue,very thick,mark=+,only marks,mark size=3pt] table[x=Points:0,y=stress_11] {chapter4/csvFiles/2dEP_ctu1ppc_115.csv};
\addplot[Purple,very thick,mark=square*,only marks] table[x=Points:0,y=stress_11] {chapter4/csvFiles/2dEP_ctu4ppc_115.csv};
\addplot[Orange,very thick,mark=x,only marks,mark size=3pt] table[x=Points:0,y= mpm_S11] {chapter4/csvFiles/2dEP_mpm_115.csv};

\nextgroupplot[title={(b) $t=1.0 \times 10^{-3} \: s$},ymin=-0.5e6,ymax=1.5e9]
\addplot[Red,very thick,no markers] table[x=Points:0,y=S11] {chapter4/csvFiles/2dEP_fem_338.csv};
\addplot[Blue,very thick,mark=+,only marks,mark size=3pt] table[x=Points:0,y=stress_11] {chapter4/csvFiles/2dEP_ctu1ppc_338.csv};
\addplot[Purple,very thick,mark=square*,only marks] table[x=Points:0,y=stress_11] {chapter4/csvFiles/2dEP_ctu4ppc_338.csv};
\addplot[Orange,very thick,mark=x,only marks,mark size=3pt] table[x=Points:0,y= mpm_S11] {chapter4/csvFiles/2dEP_mpm_338.csv};

\nextgroupplot[ylabel=$\eps^p_{11}$,xlabel=$x (m)$,ymin=-0.1e-3,ymax=6.75e-3]
\addplot[Red,very thick,no markers] table[x=Points:0,y=EP11] {chapter4/csvFiles/2dEP_fem_115.csv};
\addplot[Blue,very thick,mark=+,only marks,mark size=3pt] table[x=Points:0,y=epsp_11] {chapter4/csvFiles/2dEP_ctu1ppc_115.csv};
\addplot[Purple,very thick,mark=square*,only marks] table[x=Points:0,y=epsp_11] {chapter4/csvFiles/2dEP_ctu4ppc_115.csv};
\addplot[Orange,very thick,mark=x,only marks,mark size=3pt] table[x=Points:0,y= mpm_epsp11] {chapter4/csvFiles/2dEP_mpm_115.csv};

\nextgroupplot[legend style={at={($(0.12,-0.35)+(0.9cm,1cm)$)},legend columns=2},xlabel=$x (m)$,ymin=-0.1e-3,ymax=6.75e-3]
\addplot[Red,very thick,no markers] table[x=Points:0,y=EP11] {chapter4/csvFiles/2dEP_fem_338.csv};
\addplot[Blue,very thick,mark=+,only marks,mark size=3pt] table[x=Points:0,y=epsp_11] {chapter4/csvFiles/2dEP_ctu1ppc_338.csv};
\addplot[Purple,very thick,mark=square*,only marks] table[x=Points:0,y=epsp_11] {chapter4/csvFiles/2dEP_ctu4ppc_338.csv};
\addplot[Orange,very thick,mark=x,only marks,mark size=3pt] table[x=Points:0,y= mpm_epsp11] {chapter4/csvFiles/2dEP_mpm_338.csv};
\addlegendentry{fem}
\addlegendentry{ctu 1ppc}
\addlegendentry{ctu 4ppc}
\addlegendentry{mpm}
   
  \end{groupplot}
\end{tikzpicture}


%%% Local Variables:
%%% mode: latex
%%% TeX-master: "../../mainManuscript"
%%% End:




































%%% Local Variables:
%%% mode: latex
%%% TeX-master: "../../mainManuscript"
%%% End:

  %\begin{tikzpicture}
  \begin{groupplot}[group style={group size=2 by 1,
ylabels at=edge left, yticklabels at=edge left,horizontal sep=3.ex,
xticklabels at=edge bottom,xlabels at=edge bottom},
ymajorgrids=true,xmajorgrids=true,ymin=-0.1e-3,ymax=6.75e-3,
axis on top,scale only axis,width=0.4\linewidth, every x tick scale label/.style={at={(xticklabel* cs:1.05,0.75cm)},anchor=near yticklabel}]
\nextgroupplot[ylabel=$\eps^p_{11}$,xlabel=$x (m)$,title={(a) $t=2.5 \times 10^{-4} \: s$}]
\addplot[Red,very thick,no markers] table[x=Points:0,y=EP11] {chapter4/csvFiles/2dEP_fem_115.csv};
\addplot[Blue,very thick,mark=+,only marks,mark size=3pt] table[x=Points:0,y=epsp_11] {chapter4/csvFiles/2dEP_ctu1ppc_115.csv};
\addplot[Purple,very thick,mark=square*,only marks] table[x=Points:0,y=epsp_11] {chapter4/csvFiles/2dEP_ctu4ppc_115.csv};
\addplot[Orange,very thick,mark=x,only marks,mark size=3pt] table[x=Points:0,y= mpm_epsp11] {chapter4/csvFiles/2dEP_mpm_115.csv};

\nextgroupplot[legend style={at={($(0.12,-0.35)+(0.9cm,1cm)$)},legend columns=2},xlabel=$x (m)$,title={(b) $t=1.0 \times 10^{-3} \: s$}]
\addplot[Red,very thick,no markers] table[x=Points:0,y=EP11] {chapter4/csvFiles/2dEP_fem_338.csv};
\addplot[Blue,very thick,mark=+,only marks,mark size=3pt] table[x=Points:0,y=epsp_11] {chapter4/csvFiles/2dEP_ctu1ppc_338.csv};
\addplot[Purple,very thick,mark=square*,only marks] table[x=Points:0,y=epsp_11] {chapter4/csvFiles/2dEP_ctu4ppc_338.csv};
\addplot[Orange,very thick,mark=x,only marks,mark size=3pt] table[x=Points:0,y= mpm_epsp11] {chapter4/csvFiles/2dEP_mpm_338.csv};
\addlegendentry{fem}
\addlegendentry{ctu 1ppc}
\addlegendentry{ctu 4ppc}
\addlegendentry{mpm}
   
  \end{groupplot}
\end{tikzpicture}


%%% Local Variables:
%%% mode: latex
%%% TeX-master: "../../mainManuscript"
%%% End:




































%%% Local Variables:
%%% mode: latex
%%% TeX-master: "../../mainManuscript"
%%% End:

  \caption{Evolution of longitudinal stress $\sigma_{11}$ and plastic strain $\eps^p_{11}$ along the bottom boundary of the elastic-plastic square. Comparison between FEM (CFL=$0.9$), DGMPM using 1ppc (CFL=$1$) and DGMPM using 4ppc (CFL=$0.23$) solutions.}
  \label{fig:elastlines_stress}
\end{figure}

Figure \ref{fig:2dEP_comparison} furthermore shows FEM and DGMPM solutions in terms of equivalent plastic strain $p$, before and after reflection of the longitudinal pressure wave on the right end of the domain.
MPM results for the same time steps are plotted separately in figure \ref{fig:2dEP_mpm} due to the large amplitude of equivalent plastic strain.
\begin{figure}[h!]
  \centering
  \begin{tikzpicture}[scale=0.8]
  \begin{groupplot}[group style={group size=3 by 2,
      ylabels at=edge left, yticklabels at=edge left,
      horizontal sep=1.ex,
      vertical sep=2.ex,},
    enlargelimits=0,
    xmin=0.,xmax=1., ymin=-0.,ymax=1.
    ,axis on top,scale only axis,xtick=\empty,ytick=\empty,width=0.27\linewidth,
    colorbar style={
      title style={
        font=\scriptsize,
        at={(1,.5)},
        anchor=north west
      },yticklabel style={font=\scriptsize}
      ,at={(current axis.south east)},anchor=south west
    }]
    %% FIRST ROW (equivalent plastic strain)
    %%% RANGE 0 -- 4.3e-3
    \nextgroupplot[ylabel={$t=3.5\times 10^{-4} \:s$},title={(a) FEM}]\addplot graphics[xmin=0.,xmax=1., ymin=-0.,ymax=1.] {chapter4/pngFigures/ep_fem_p115.png};
    \nextgroupplot[title={(b) DGMPM 1ppc}]\addplot graphics[xmin=-0.,xmax=1., ymin=-0.,ymax=1.] {chapter4/pngFigures/ep_dgmpm1ppc_p115.png};
    \nextgroupplot[title={(c) DGMPM 4ppc},colorbar,colorbar style={
      title= {$p \: (\%)$},
      ytick={0.,4.3},
      yticklabels={0,0.43},
    }]
    \addplot[scatter,scatter src=y,mark size=0.pt] coordinates {(0.,0.) (0.,4.3)};% Fake extreme values to fix scale
    \addplot graphics[xmin=-0.,xmax=1., ymin=-0.,ymax=1.] {chapter4/pngFigures/ep_dgmpm4ppc_p115.png};
    %% THIRD ROW (equivalent plastic strain)
    %%% RANGE 0 -- 2.e-2
    \nextgroupplot[ylabel={$t=1.0\times 10^{-3} \:s$}]\addplot graphics[xmin=0.,xmax=1., ymin=-0.,ymax=1.] {chapter4/pngFigures/ep_fem_p338.png};
    \nextgroupplot[]\addplot graphics[xmin=-0.,xmax=1., ymin=-0.,ymax=1.] {chapter4/pngFigures/ep_dgmpm1ppc_p338.png};
    \nextgroupplot[colorbar,colorbar style={
      title= {$p \: (\%)$},
      ytick={0.,2.},
      yticklabels={0,2},
    }]
    \addplot[scatter,scatter src=y,mark size=0.pt] coordinates {(0.,0.) (0.,2.)};% Fake extreme values to fix scale
    \addplot graphics[xmin=-0.,xmax=1., ymin=-0.,ymax=1.] {chapter4/pngFigures/ep_dgmpm4ppc_p338.png};
  \end{groupplot}
\end{tikzpicture}



%%% Local Variables:
%%% mode: latex
%%% TeX-master: "../mainManuscript"
%%% End:
  \caption{Isovalues of equivalent plastic strain $p$ in an elastic-plastic plate linear isotropic hardening material at two different times. Comparison between FEM (CFL=$0.9$), DGMPM using 1ppc (CFL=$1$) and DGMPM using 4ppc (CFL=$0.23$) solutions.}
  \label{fig:2dEP_comparison}
\end{figure}
\begin{figure}[h!]
  \centering
  \begin{tikzpicture}
  \begin{groupplot}[group style={group size=2 by 1,
      ylabels at=edge left, yticklabels at=edge left,
      horizontal sep=4.ex,
      vertical sep=2ex,},
    enlargelimits=0,
    xmin=0.,xmax=1., ymin=-0.,ymax=1.
    ,axis on top,scale only axis,xtick=\empty,ytick=\empty,width=0.27\linewidth,
    colorbar style={
      title style={
        font=\scriptsize,
        at={(1,.5)},
        anchor=north west
      },yticklabel style={font=\scriptsize}
      ,at={(current axis.south east)},anchor=south west
    }]
    %% FIRST ROW (first stress sigma11)
    
    % \nextgroupplot[title={(a) MPM results at time $t=3.5e-4 \:s$},
    % colorbar,colorbar style={
    %   title= {$\sigma_{11}\: (MPa)$},
    %   ytick={-0.27,8,9.2},
    %   yticklabels={-27,800,920},
    % }]
    % \addplot[scatter,scatter src=y,mark size=0.pt] coordinates {(0.,-0.27) (0.,9.2)};% Fake extreme values to fix scale
    % \addplot graphics[xmin=-0.,xmax=1., ymin=-0.,ymax=1.] {chapter4/pngFigures/ep_mpm_stress115.png};
    
    % \nextgroupplot[title={(b) MPM results at time $t=1.0e-3 \:s$},
    % colorbar,colorbar style={
    %   title= {$\sigma_{11}\: (MPa)$},
    %   ytick={-0.210,0.8,1.4},
    %   yticklabels={-210,800,1.4e3},
    % }]
    % \addplot[scatter,scatter src=y,mark size=0.pt] coordinates {(0.,-0.210) (0.,1.4)};% Fake extreme values to fix scale
    % \addplot graphics[xmin=-0.,xmax=1., ymin=-0.,ymax=1.] {chapter4/pngFigures/ep_mpm_stress338.png};
    
    %% SECOND ROW (first plastic strain epsp11)
    % \nextgroupplot[
    % colorbar,colorbar style={
    %   title= {$\eps^p_{11}$},
    %   ytick={-2.6,9.5},
    %   yticklabels={-2.6e-3,9.5e-3},
    % }]
    % \addplot[scatter,scatter src=y,mark size=0.pt] coordinates {(0.,-2.6) (0.,9.5)};% Fake extreme values to fix scale
    % \addplot graphics[xmin=-0.,xmax=1., ymin=-0.,ymax=1.] {chapter4/pngFigures/ep_mpm_epsp115.png};
    
    % \nextgroupplot[
    % colorbar,colorbar style={
    %   title= {$\eps^p_{11}$},
    %   ytick={-0.42,2.5},
    %   yticklabels={-4.2e-3,2.5e-2},
    % }]
    % \addplot[scatter,scatter src=y,mark size=0.pt] coordinates {(0.,-0.42) (0.,2.5)};% Fake extreme values to fix scale
    % \addplot graphics[xmin=-0.,xmax=1., ymin=-0.,ymax=1.] {chapter4/pngFigures/ep_mpm_epsp338.png};
    
    %% THIRD ROW (equivalent plastic strain)
    \nextgroupplot[title={(a) time $t=3.5e-4 \:s$}] \addplot graphics[xmin=-0.,xmax=1., ymin=-0.,ymax=1.] {chapter4/pngFigures/ep_mpm_p115.png};
    
    \nextgroupplot[title={(b) time $t=1.0e-3 \:s$},
    colorbar,colorbar style={
      title= {$p \: (\%)$},
      ytick={0,3.1},
      yticklabels={0,3.1},
    }]
    \addplot[scatter,scatter src=y,mark size=0.pt] coordinates {(0.,0) (0.,3.1)};% Fake extreme values to fix scale
    \addplot graphics[xmin=-0.,xmax=1., ymin=-0.,ymax=1.] {chapter4/pngFigures/ep_mpm_p338.png};
   
  \end{groupplot}
\end{tikzpicture}



%%% Local Variables:
%%% mode: latex
%%% TeX-master: "../mainManuscript"
%%% End:
  \caption{MPM isovalues of plastic strain $p$ in an elastic-plastic plate made of a linear isotropic hardening material at two different times (CFL=$0.7$).}
  \label{fig:2dEP_mpm}
\end{figure}
A concentration of plastic strain occurs at the interface between loaded and traction free parts of the left boundary of the domain in DGMPM solutions. 
Such concentrations in the high gradient region can also be seen in MPM solutions depicted in figure \ref{fig:2dEP_mpm} so that this phenomenon seems to be a singularity owed to boundary conditions.
Although the final DGMPM and FEM profiles of equivalent plastic strain have the same shape, the results highlight the fact that different numerical methods yield different assessments of irreversible deformations.
This point is crucial for the accurate simulation of solid mechanics applications such as forming techniques or crash problems.




%%% Local Variables:
%%% mode: latex
%%% ispell-local-dictionary: "american"
%%% TeX-master: "../mainManuscript"
%%% End: