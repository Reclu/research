The above solid is now supposed made of an elastic-plastic material with linear isotropic hardening. A tensile impact is still considered on the bottom left part of the domain, as depicted in figure \ref{fig:2D_planeStrain}\subref{subfig:2D_problem}, with the traction force $\sigma^d=2\sigma^y$ so that plastic flow occurs. A comparison between FEM (Cast3M Drexus), MPM and DGMPM using either one or four particles per elements is made on the equivalent plastic strain $p$, similar observation as above being made on the longitudinal stress $\sigma_{11}$.
The computation of intercell fluxes within the DGMPM is based on an elastic Riemann solver. Hence, the plastic flow is integrated by means of a radial return algorithm in every methods.

Figure \ref{fig:2dEP_comparison} shows FEM and DGMPM solutions before and after reflection of the longitudinal pressure wave on the right end of the domain. In addition, due to the large amplitude of equivalent plastic strain, MPM results for the same time steps are plotted separately in figure \ref{fig:2dEP_mpm}. A good agreement between FEM and DGMPM equivalent plastic strains can be seen in figure \ref{fig:2dEP_comparison}. 
\begin{figure}[h!]
  \centering
  \begin{tikzpicture}[scale=0.8]
  \begin{groupplot}[group style={group size=3 by 2,
      ylabels at=edge left, yticklabels at=edge left,
      horizontal sep=1.ex,
      vertical sep=2.ex,},
    enlargelimits=0,
    xmin=0.,xmax=1., ymin=-0.,ymax=1.
    ,axis on top,scale only axis,xtick=\empty,ytick=\empty,width=0.27\linewidth,
    colorbar style={
      title style={
        font=\scriptsize,
        at={(1,.5)},
        anchor=north west
      },yticklabel style={font=\scriptsize}
      ,at={(current axis.south east)},anchor=south west
    }]
    %% FIRST ROW (equivalent plastic strain)
    %%% RANGE 0 -- 4.3e-3
    \nextgroupplot[ylabel={$t=3.5\times 10^{-4} \:s$},title={(a) FEM}]\addplot graphics[xmin=0.,xmax=1., ymin=-0.,ymax=1.] {chapter4/pngFigures/ep_fem_p115.png};
    \nextgroupplot[title={(b) DGMPM 1ppc}]\addplot graphics[xmin=-0.,xmax=1., ymin=-0.,ymax=1.] {chapter4/pngFigures/ep_dgmpm1ppc_p115.png};
    \nextgroupplot[title={(c) DGMPM 4ppc},colorbar,colorbar style={
      title= {$p \: (\%)$},
      ytick={0.,4.3},
      yticklabels={0,0.43},
    }]
    \addplot[scatter,scatter src=y,mark size=0.pt] coordinates {(0.,0.) (0.,4.3)};% Fake extreme values to fix scale
    \addplot graphics[xmin=-0.,xmax=1., ymin=-0.,ymax=1.] {chapter4/pngFigures/ep_dgmpm4ppc_p115.png};
    %% THIRD ROW (equivalent plastic strain)
    %%% RANGE 0 -- 2.e-2
    \nextgroupplot[ylabel={$t=1.0\times 10^{-3} \:s$}]\addplot graphics[xmin=0.,xmax=1., ymin=-0.,ymax=1.] {chapter4/pngFigures/ep_fem_p338.png};
    \nextgroupplot[]\addplot graphics[xmin=-0.,xmax=1., ymin=-0.,ymax=1.] {chapter4/pngFigures/ep_dgmpm1ppc_p338.png};
    \nextgroupplot[colorbar,colorbar style={
      title= {$p \: (\%)$},
      ytick={0.,2.},
      yticklabels={0,2},
    }]
    \addplot[scatter,scatter src=y,mark size=0.pt] coordinates {(0.,0.) (0.,2.)};% Fake extreme values to fix scale
    \addplot graphics[xmin=-0.,xmax=1., ymin=-0.,ymax=1.] {chapter4/pngFigures/ep_dgmpm4ppc_p338.png};
  \end{groupplot}
\end{tikzpicture}



%%% Local Variables:
%%% mode: latex
%%% TeX-master: "../mainManuscript"
%%% End:
  \caption{Isovalues of equivalent plastic strain $p$ in an elastic-plastic plate linear isotropic hardening material at two different times. Comparison between FEM ($CFL=0.9$), DGMPM using 1ppc ($CFL=1$) and DGMP using 4ppc ($CFL=0.23$) solutions.}
  \label{fig:2dEP_comparison}
\end{figure}
\begin{figure}[h!]
  \centering
  \begin{tikzpicture}
  \begin{groupplot}[group style={group size=2 by 1,
      ylabels at=edge left, yticklabels at=edge left,
      horizontal sep=4.ex,
      vertical sep=2ex,},
    enlargelimits=0,
    xmin=0.,xmax=1., ymin=-0.,ymax=1.
    ,axis on top,scale only axis,xtick=\empty,ytick=\empty,width=0.27\linewidth,
    colorbar style={
      title style={
        font=\scriptsize,
        at={(1,.5)},
        anchor=north west
      },yticklabel style={font=\scriptsize}
      ,at={(current axis.south east)},anchor=south west
    }]
    %% FIRST ROW (first stress sigma11)
    
    % \nextgroupplot[title={(a) MPM results at time $t=3.5e-4 \:s$},
    % colorbar,colorbar style={
    %   title= {$\sigma_{11}\: (MPa)$},
    %   ytick={-0.27,8,9.2},
    %   yticklabels={-27,800,920},
    % }]
    % \addplot[scatter,scatter src=y,mark size=0.pt] coordinates {(0.,-0.27) (0.,9.2)};% Fake extreme values to fix scale
    % \addplot graphics[xmin=-0.,xmax=1., ymin=-0.,ymax=1.] {chapter4/pngFigures/ep_mpm_stress115.png};
    
    % \nextgroupplot[title={(b) MPM results at time $t=1.0e-3 \:s$},
    % colorbar,colorbar style={
    %   title= {$\sigma_{11}\: (MPa)$},
    %   ytick={-0.210,0.8,1.4},
    %   yticklabels={-210,800,1.4e3},
    % }]
    % \addplot[scatter,scatter src=y,mark size=0.pt] coordinates {(0.,-0.210) (0.,1.4)};% Fake extreme values to fix scale
    % \addplot graphics[xmin=-0.,xmax=1., ymin=-0.,ymax=1.] {chapter4/pngFigures/ep_mpm_stress338.png};
    
    %% SECOND ROW (first plastic strain epsp11)
    % \nextgroupplot[
    % colorbar,colorbar style={
    %   title= {$\eps^p_{11}$},
    %   ytick={-2.6,9.5},
    %   yticklabels={-2.6e-3,9.5e-3},
    % }]
    % \addplot[scatter,scatter src=y,mark size=0.pt] coordinates {(0.,-2.6) (0.,9.5)};% Fake extreme values to fix scale
    % \addplot graphics[xmin=-0.,xmax=1., ymin=-0.,ymax=1.] {chapter4/pngFigures/ep_mpm_epsp115.png};
    
    % \nextgroupplot[
    % colorbar,colorbar style={
    %   title= {$\eps^p_{11}$},
    %   ytick={-0.42,2.5},
    %   yticklabels={-4.2e-3,2.5e-2},
    % }]
    % \addplot[scatter,scatter src=y,mark size=0.pt] coordinates {(0.,-0.42) (0.,2.5)};% Fake extreme values to fix scale
    % \addplot graphics[xmin=-0.,xmax=1., ymin=-0.,ymax=1.] {chapter4/pngFigures/ep_mpm_epsp338.png};
    
    %% THIRD ROW (equivalent plastic strain)
    \nextgroupplot[title={(a) time $t=3.5e-4 \:s$}] \addplot graphics[xmin=-0.,xmax=1., ymin=-0.,ymax=1.] {chapter4/pngFigures/ep_mpm_p115.png};
    
    \nextgroupplot[title={(b) time $t=1.0e-3 \:s$},
    colorbar,colorbar style={
      title= {$p \: (\%)$},
      ytick={0,3.1},
      yticklabels={0,3.1},
    }]
    \addplot[scatter,scatter src=y,mark size=0.pt] coordinates {(0.,0) (0.,3.1)};% Fake extreme values to fix scale
    \addplot graphics[xmin=-0.,xmax=1., ymin=-0.,ymax=1.] {chapter4/pngFigures/ep_mpm_p338.png};
   
  \end{groupplot}
\end{tikzpicture}



%%% Local Variables:
%%% mode: latex
%%% TeX-master: "../mainManuscript"
%%% End:
  \caption{MPM isovalues of plastic strain $p$ in an elastic-plastic plate made of a linear isotropic hardening material at two different times ($CFL=0.7$).}
  \label{fig:2dEP_mpm}
\end{figure}
Nevertheless, a concentration of plastic strains occurs at the interface between loaded and traction free parts of the left boundary of the domain in DGMPM solutions. 
Such concentrations in the high gradients region can also be seen in MPM solutions depicted in figure \ref{fig:2dEP_mpm} so that this phenomenon seems to be a geometric effect arrising due to the space discretization rather than to numerical methods themselves.
%% Also plot stress
% \begin{figure}[h!]
%   \centering
%   \begin{tikzpicture}
  \begin{groupplot}[group style={group size=3 by 3,
      ylabels at=edge left, yticklabels at=edge left,
      horizontal sep=1.ex,
      vertical sep=5ex,},
    enlargelimits=0,
    xmin=0.,xmax=1., ymin=-0.,ymax=1.
    ,axis on top,scale only axis,xtick=\empty,ytick=\empty,width=0.27\linewidth,
    colorbar style={
      title style={
        font=\scriptsize,
        at={(1,.5)},
        anchor=north west
      },yticklabel style={font=\scriptsize}
      ,at={(current axis.south east)},anchor=south west
    }]
    %% FIRST ROW (first stress sigma11)
    %%% RANGE -4.8e7 -- 1.1e9
    \nextgroupplot[title={(a) FEM}]\addplot graphics[xmin=0.,xmax=1., ymin=-0.,ymax=1.] {chapter4/pngFigures/ep_fem_stress77.png};
    \nextgroupplot[title={(b) DGMPM 1ppc}]\addplot graphics[xmin=-0.,xmax=1., ymin=-0.,ymax=1.] {chapter4/pngFigures/ep_dgmpm1ppc_stress77.png};
    \nextgroupplot[title={(c) DGMPM 4ppc},
    colorbar,colorbar style={
      title= {$\sigma_{11}\: (MPa)$},
      ytick={-0.048,1.1},
      yticklabels={-48,1.1e3},
    }]
    \addplot[scatter,scatter src=y,mark size=0.pt] coordinates {(0.,-0.048) (0.,1.1)};% Fake extreme values to fix scale
    \addplot graphics[xmin=-0.,xmax=1., ymin=-0.,ymax=1.] {chapter4/pngFigures/ep_dgmpm4ppc_stress77.png};
    
    %% SECOND ROW (first plastic strain epsp11)
    %%% RANGE -4.3e-5 -- 3.9e-3
    \nextgroupplot[]\addplot graphics[xmin=0.,xmax=1., ymin=-0.,ymax=1.] {chapter4/pngFigures/ep_fem_epsp77.png};
    \nextgroupplot[]\addplot graphics[xmin=-0.,xmax=1., ymin=-0.,ymax=1.] {chapter4/pngFigures/ep_dgmpm1ppc_epsp77.png};
    \nextgroupplot[colorbar,colorbar style={
      title= {$\eps^p_{11}$},
      ytick={-0.043,3.9},
      yticklabels={-4.3e-5,3.9e-3},
    }]
    \addplot[scatter,scatter src=y,mark size=0.pt] coordinates {(0.,-0.043) (0.,3.9)};% Fake extreme values to fix scale
    \addplot graphics[xmin=-0.,xmax=1., ymin=-0.,ymax=1.] {chapter4/pngFigures/ep_dgmpm4ppc_epsp77.png};
    
    %% THIRD ROW (equivalent plastic strain)
    %%% RANGE 0 -- 4.3e-3
    \nextgroupplot[]\addplot graphics[xmin=0.,xmax=1., ymin=-0.,ymax=1.] {chapter4/pngFigures/ep_fem_p77.png};
    \nextgroupplot[]\addplot graphics[xmin=-0.,xmax=1., ymin=-0.,ymax=1.] {chapter4/pngFigures/ep_dgmpm1ppc_p77.png};
    \nextgroupplot[colorbar,colorbar style={
      title= {$p$},
      ytick={0.,4.3},
      yticklabels={0,4.3e-3},
    }]
    \addplot[scatter,scatter src=y,mark size=0.pt] coordinates {(0.,0.) (0.,4.3)};% Fake extreme values to fix scale
    \addplot graphics[xmin=-0.,xmax=1., ymin=-0.,ymax=1.] {chapter4/pngFigures/ep_dgmpm4ppc_p77.png};
  \end{groupplot}
\end{tikzpicture}



%%% Local Variables:
%%% mode: latex
%%% TeX-master: "../mainManuscript"
%%% End:
%   \caption{Isovalues of longitudinal stress $\sigma_{11}$ and equivalent plastic strain $p$ in an elastic-plastic plate linear isotropic hardening material at time $t=3.5e-4 \:s$. Comparison between FEM ($CFL=0.9$), DGMPM using 1ppc ($CFL=1$) and DGMP using 4ppc ($CFL=0.23$) solutions.}
%   \label{fig:2dEP_comparison1}
% \end{figure}
% \begin{figure}[h!]
%   \centering
%   \begin{tikzpicture}
  \begin{groupplot}[group style={group size=3 by 2,
      ylabels at=edge left, yticklabels at=edge left,
      horizontal sep=1.ex,
      vertical sep=2.ex,},
    enlargelimits=0,
    xmin=0.,xmax=1., ymin=-0.,ymax=1.
    ,axis on top,scale only axis,xtick=\empty,ytick=\empty,width=0.27\linewidth,
    colorbar style={
      title style={
        font=\scriptsize,
        at={(1,.5)},
        anchor=north west
      },yticklabel style={font=\scriptsize}
      ,at={(current axis.south east)},anchor=south west
    }]
    %% FIRST ROW (first stress sigma11)
    %%% RANGE -7.2e7 -- 1.1e9
    \nextgroupplot[title={(a) FEM}]\addplot graphics[xmin=0.,xmax=1., ymin=-0.,ymax=1.] {chapter4/pngFigures/ep_fem_stress338.png};
    \nextgroupplot[title={(b) DGMPM 1ppc}]\addplot graphics[xmin=-0.,xmax=1., ymin=-0.,ymax=1.] {chapter4/pngFigures/ep_dgmpm1ppc_stress338.png};
    \nextgroupplot[title={(c) DGMPM 4ppc},
    colorbar,colorbar style={
      title= {$\sigma_{11}\: (MPa)$},
      ytick={-0.73,8,8.9},
      yticklabels={-73,800,870},
    }]
    \addplot[scatter,scatter src=y,mark size=0.pt] coordinates {(0.,-0.73) (0.,8.9)};% Fake extreme values to fix scale
    \addplot graphics[xmin=-0.,xmax=1., ymin=-0.,ymax=1.] {chapter4/pngFigures/ep_dgmpm4ppc_stress338.png};
    
    %% SECOND ROW (first plastic strain epsp11)
    %%% RANGE -3.2e-5 -- 1.7e-2
    % \nextgroupplot[]\addplot graphics[xmin=0.,xmax=1., ymin=-0.,ymax=1.] {chapter4/pngFigures/ep_fem_epsp338.png};
    % \nextgroupplot[]\addplot graphics[xmin=-0.,xmax=1., ymin=-0.,ymax=1.] {chapter4/pngFigures/ep_dgmpm1ppc_epsp338.png};
    % \nextgroupplot[colorbar,colorbar style={
    %   title= {$\eps^p_{11}\cdot 10^{-2}$},
    %   ytick={-0.0032,1.7},
    %   yticklabels={-3.2e-5,1.7e-2},
    % }]
    % \addplot[scatter,scatter src=y,mark size=0.pt] coordinates {(0.,-0.0032) (0.,1.7)};% Fake extreme values to fix scale
    % \addplot graphics[xmin=-0.,xmax=1., ymin=-0.,ymax=1.] {chapter4/pngFigures/ep_dgmpm4ppc_epsp338.png};
      
    %% THIRD ROW (equivalent plastic strain)
    %%% RANGE 0 -- 2.e-2
    \nextgroupplot[]\addplot graphics[xmin=0.,xmax=1., ymin=-0.,ymax=1.] {chapter4/pngFigures/ep_fem_p338.png};
    \nextgroupplot[]\addplot graphics[xmin=-0.,xmax=1., ymin=-0.,ymax=1.] {chapter4/pngFigures/ep_dgmpm1ppc_p338.png};
    \nextgroupplot[colorbar,colorbar style={
      title= {$p \: (\%)$},
      ytick={0.,2.},
      yticklabels={0,2},
    }]
    \addplot[scatter,scatter src=y,mark size=0.pt] coordinates {(0.,0.) (0.,2.)};% Fake extreme values to fix scale
    \addplot graphics[xmin=-0.,xmax=1., ymin=-0.,ymax=1.] {chapter4/pngFigures/ep_dgmpm4ppc_p338.png};
  \end{groupplot}
\end{tikzpicture}



%%% Local Variables:
%%% mode: latex
%%% TeX-master: "../mainManuscript"
%%% End:
%   \caption{Isovalues of longitudinal stress $\sigma_{11}$ and equivalent plastic strain $p$ in an elastic-plastic plate linear isotropic hardening material at time $t=1.0e-3 \:s$. Comparison between FEM ($CFL=0.9$), DGMPM using 1ppc ($CFL=1$) and DGMP using 4ppc ($CFL=0.23$) solutions.}
%   \label{fig:2dEP_comparison2}
% \end{figure}
% \begin{figure}[h!]
%   \centering
%   \begin{tikzpicture}
  \begin{groupplot}[group style={group size=2 by 2,
      ylabels at=edge left, yticklabels at=edge left,
      horizontal sep=18.ex,
      vertical sep=2ex,},
    enlargelimits=0,
    xmin=0.,xmax=1., ymin=-0.,ymax=1.
    ,axis on top,scale only axis,xtick=\empty,ytick=\empty,width=0.25\linewidth,
    colorbar style={
      title style={
        font=\scriptsize,
        at={(1,.5)},
        anchor=north west
      },yticklabel style={font=\scriptsize}
      ,at={(current axis.south east)},anchor=south west
    }]
    %% FIRST ROW (first stress sigma11)
    
    \nextgroupplot[title={(a) MPM results at time $t=3.5e-4 \:s$},
    colorbar,colorbar style={
      title= {$\sigma_{11}\: (MPa)$},
      ytick={-0.27,8,9.2},
      yticklabels={-27,800,920},
    }]
    \addplot[scatter,scatter src=y,mark size=0.pt] coordinates {(0.,-0.27) (0.,9.2)};% Fake extreme values to fix scale
    \addplot graphics[xmin=-0.,xmax=1., ymin=-0.,ymax=1.] {chapter4/pngFigures/ep_mpm_stress115.png};
    
    \nextgroupplot[title={(b) MPM results at time $t=1.0e-3 \:s$},
    colorbar,colorbar style={
      title= {$\sigma_{11}\: (MPa)$},
      ytick={-0.210,0.8,1.4},
      yticklabels={-210,800,1.4e3},
    }]
    \addplot[scatter,scatter src=y,mark size=0.pt] coordinates {(0.,-0.210) (0.,1.4)};% Fake extreme values to fix scale
    \addplot graphics[xmin=-0.,xmax=1., ymin=-0.,ymax=1.] {chapter4/pngFigures/ep_mpm_stress338.png};
    
    %% SECOND ROW (first plastic strain epsp11)
    % \nextgroupplot[
    % colorbar,colorbar style={
    %   title= {$\eps^p_{11}$},
    %   ytick={-2.6,9.5},
    %   yticklabels={-2.6e-3,9.5e-3},
    % }]
    % \addplot[scatter,scatter src=y,mark size=0.pt] coordinates {(0.,-2.6) (0.,9.5)};% Fake extreme values to fix scale
    % \addplot graphics[xmin=-0.,xmax=1., ymin=-0.,ymax=1.] {chapter4/pngFigures/ep_mpm_epsp115.png};
    
    % \nextgroupplot[
    % colorbar,colorbar style={
    %   title= {$\eps^p_{11}$},
    %   ytick={-0.42,2.5},
    %   yticklabels={-4.2e-3,2.5e-2},
    % }]
    % \addplot[scatter,scatter src=y,mark size=0.pt] coordinates {(0.,-0.42) (0.,2.5)};% Fake extreme values to fix scale
    % \addplot graphics[xmin=-0.,xmax=1., ymin=-0.,ymax=1.] {chapter4/pngFigures/ep_mpm_epsp338.png};
    
    %% THIRD ROW (equivalent plastic strain)
    \nextgroupplot[
    colorbar,colorbar style={
      title= {$p \: (\%)$},
      ytick={-0.,1.1},
      yticklabels={0,1.1},
    }]
    \addplot[scatter,scatter src=y,mark size=0.pt] coordinates {(0.,0.) (0.,1.1)};% Fake extreme values to fix scale
    \addplot graphics[xmin=-0.,xmax=1., ymin=-0.,ymax=1.] {chapter4/pngFigures/ep_mpm_p115.png};
    
    \nextgroupplot[
    colorbar,colorbar style={
      title= {$p \: (\%)$},
      ytick={0,3.1},
      yticklabels={0,3.1},
    }]
    \addplot[scatter,scatter src=y,mark size=0.pt] coordinates {(0.,0) (0.,3.1)};% Fake extreme values to fix scale
    \addplot graphics[xmin=-0.,xmax=1., ymin=-0.,ymax=1.] {chapter4/pngFigures/ep_mpm_p338.png};
   
  \end{groupplot}
\end{tikzpicture}



%%% Local Variables:
%%% mode: latex
%%% TeX-master: "../mainManuscript"
%%% End:
%   \caption{Isovalues of MPM longitudinal stress $\sigma_{11}$ and plastic strain $p$ in an elastic-plastic plate made of a linear isotropic hardening material at two different times.}
%   \label{fig:2dEP_mpm}
% \end{figure}





%%% Local Variables:
%%% mode: latex
%%% TeX-master: "../mainManuscript"
%%% End: