% On ne regarde qu'une dimension spatiale en faisant des hypothèse sur les champs alors que nous on se limite à une direction particulière $\vect{n}$. En plus, on se limite à l'étude d'ondes simples alors que des chocs peuvent exister (voir Mandell car il semble y etre démontré que les shock n'arrivent que pour $\tau=0$).
% Il y a la question des vitesses charactéristiques plastiques... sont-elles collées aux vitesses élastiques ?
% Assumptions (isothermal, linear hardenings though extendable, small deformation, cartesian,concave constitutive model,rate independent)
% Un intérêt de développer ce qui est fait là est de s'autoriser à limiter les ondes plastique dans des méthodes de haut ordre.
% dependance des vitesses caractéristiques à l'angle entre la direction principale de sigma et la direction de propagation, c'est dit dans la thèse de Clifou en page 90.

\subsection{Properties of the integral curves}
%\subsubsection*{Orthogonality of loading paths}
With the left eigenvectors of the acoustic tensor defined in equation \eqref{eq:ch5_eigenvectAcc}, the functions $\psi^s_1$ and $\psi^f_1$ introduced in the previous section satisfy the following orthogonality condition: $\psi^s_1\psi^f_1=-1$. 

\begin{proof}
  We first write the product of the two functions according to equation \eqref{eq:ch5_eigenvectAcc}:
  \begin{equation*}
    \psi^s_1\psi^f_1 = \frac{l^1_2}{l^1_1}\: \frac{l_2^2}{l^2_1} = \frac{(A_{11}^{ep}-\rho c_s^2)A^{ep}_{12}}{(A_{22}^{ep}-\rho c_f^2)A^{ep}_{12}}
  \end{equation*}
  Introduction of the expressions of $\rho c_f^2 = \omega_1$ and $\rho c_s^2 = \omega_2$ given by equations \eqref{eq:ch5_eigenAcc1} and \eqref{eq:ch5_eigenAcc1} leads to:
  \begin{equation*}
    \psi^s_1\psi^f_1 = \frac{A_{11}^{ep}-A_{22}^{ep}+\sqrt{(A_{11}^{ep}-A_{22}^{ep})^2 + 4A_{12}^{ep}}}{A_{22}^{ep}-A_{11}^{ep}-\sqrt{(A_{11}^{ep}-A_{22}^{ep})^2 + 4A_{12}^{ep}}}=-1
  \end{equation*}
  This ends the proof.
\end{proof}
In other words, $\vect{l}^1 \cdot \vect{l}^2=0$ so that the eigenvectors of the acoustic tensor are orthogonal regardless of the stress state (je pense que c'est normal). This allows us to restrict the study of the functions $\psi_1$ to $\psi^s_1$.
A similar property is identified in \cite{Clifton} for the thin-walled tube problem consisting in a superimposition of a plane wave and a pure shear wave.(il y en a un autre qui en parle et qui précise que c'est valable si les courbes intégrales ont une intersection \cite[p.13]{Ting68})
\begin{remark}
  The orthogonality property of the loading pass is also valid for the vector $\vect{n}=\vect{e}_2$ since the functions involved in the intergal curves are $\psi^s_2=1/\psi^s_1$ and $\psi^f_2=1/\psi^f_1$.
\end{remark}

%\subsubsection*{Degeneracy of the hyperbolic system}

A vanishing function $\psi_1^f$ yields a longitudinal stress $\sigma_{11}$ that is constant along a loading path according to equation \eqref{eq:sigSlow_n=e1}. Conversely, the stress $\sigma_{12}$ is constant if $\psi_1^f\rightarrow \infty$. Such situations make the integration of the integral curve easier, some of them are now identified.

Looking for vanishing $\psi^f_1$ or $1/\psi^f_1$ amounts to finding roots of the components of $\vect{l}^2$. Thus:
\begin{align}
  & l_1^2 = 0 \Leftrightarrow A_{12}^{ep}=0  \\
  & l_2^2 = 0 \Leftrightarrow A_{11}^{ep}-\rho c_s^2 =0
\end{align}
Introduction of this condition in the second yields:
\begin{equation}
  A_{11}^{ep}-\rho c_s^2 = \frac{1}{2}\(A^{ep}_{11}-A^{ep}_{22} + \abs{A^{ep}_{11}-A^{ep}_{22}}\) = \left\langle A^{ep}_{11}-A^{ep}_{22}  \right\rangle
\end{equation}
where $\left\langle \bullet \right\rangle$ denotes the positive part operator.

Therefore, if $A_{12}^{ep}=0$ and $A_{11}^{ep}\neq A_{22}^{ep}$, the functions satisfy $\psi^f_1 \rightarrow \infty$ and $\psi^s_1 \rightarrow 0$ so that the stress path in the ($\sigma_{11},\sigma_{12}$) plane are (locally) horizontal (\textit{resp. vertical}) through a fast (\textit{resp. slow}) wave. 
On the other hand, if $A_{11}^{ep} = A_{22}^{ep}$, both components of the eigenvectors vanish and $\psi^f_1$ and $\psi^s_1$ are undetermined. The latter situation moreover leads to characteristic speeds that "collide" according to equations \eqref{eq:ch5_eigenAcc1} and \eqref{eq:ch5_eigenAcc2}, namely:
\begin{subequations}
  \begin{alignat}{1}
    \label{eq:eigen_acc_planeStrain1}
    & \rho c_s^2 = \frac{1}{2}\(A^{ep}_{11}+A^{ep}_{22} - \sqrt{(A^{ep}_{11}-A^{ep}_{22})^2+{4A^{ep}_{12}}^2}\) = \frac{A^{ep}_{11}+A^{ep}_{22}}{2}\\
    \label{eq:eigen_acc_planeStrain2}
    & \rho c_f^2 = \frac{1}{2}\(A^{ep}_{11}+A^{ep}_{22} + \sqrt{(A^{ep}_{11}-A^{ep}_{22})^2+{4A^{ep}_{12}}^2}\)= \frac{A^{ep}_{11}+A^{ep}_{22}}{2} 
  \end{alignat}
\end{subequations}
The system thus degenerates in that case since the solution no longer involves two but only one family of simple waves. 

The above discussion is now specified to plane strain and plane stress cases, for which loading conditions that lead to $A_{12}^{ep}=0$ and $A^{ep}_{11}-A^{ep}_{22}=0$ are identified. Without loss of generality, attention is paid to only one direction of space, say $\vect{n}=\vect{e}_1$ and right-going simple waves.

\subsection{The plane strain case}
The expressions of the tangent modulus and the acoustic tensors are recalled here for convenience:
\begin{align}
  & C^{ep}_{ijkl} = \lambda \delta_{ij}\delta_{kl} + \mu \(\delta_{il}\delta_{jk} + \delta_{ik}\delta_{jl}\) - \beta s_{ij}s_{kl} \\
  & A^{ep}_{ij} = \lambda n_i n_j + \mu \(n_k n_k \delta_{ij} +n_i n_j \) - \beta s_{ip}n_p s_{jq}n_q
\end{align}
where $s_{ij}$ are the components of the deviatoric part of Cauchy stress tensor, that is $s_{ij}=\sigma_{ij} - \frac{1}{3}\sigma_{kk}\delta_{ij}$. Hence, for $n_2=0$ the components of the (symmetric) acoustic tensor are:
\begin{align}
  & A_{11}^{ep}= \lambda + 2\mu -\beta s_{11}^2 \\
  & A_{22}^{ep}= \mu -\beta s_{12}^2 \\
  & A_{12}^{ep}=-\beta s_{11}s_{12}
\end{align}
The eigenvalues of the acoustic tensor are then:
\begin{subequations}
  \begin{alignat}{1}
    \label{eq:eigen_acc_DP1}
    & \rho c_s^2 = \frac{1}{2}\( \lambda +3\mu -\beta (s_{11}^2+ s_{12}^2) - \sqrt{(\lambda + \mu -\beta (s_{11}^2-s_{12}^2) )^2 +4(\beta s_{11}s_{12})^2} \) \\
    \label{eq:eigen_acc_DP2}
    & \rho c_f^2 = \frac{1}{2}\( \lambda +3\mu -\beta (s_{11}^2+ s_{12}^2) + \sqrt{(\lambda + \mu -\beta (s_{11}^2-s_{12}^2) )^2 +4(\beta s_{11}s_{12})^2}  \)
  \end{alignat}
\end{subequations}
We first study the sign of the functions $\psi^f$ by noticing that $\mu=\rho c_2^2$ so that $A_{22}^{ep}$ may be rewritten to yield:
\begin{equation*}
  \psi^f = -\frac{A_{12}^{ep}}{A_{22}-\rho c_f^2}= -\frac{\beta s_{11}s_{12}}{\rho c_f^2-\rho c_2^2 +\beta s_{12}^2 }
\end{equation*}
Since the denominator is positive for $c_f \geq c_2$, it comes out that $\sign (\psi^f) = - \sign(s_{12}) \sign(s_{11})$. Moreover, two roots of the loading function $\psi^f$ can be identified.

%Next, from the expressions of the acoustic tensor components, we see that particular cases leading to $\psi^f_1\rightarrow \infty$ or $\psi^f_1\rightarrow 0$ arise when $s_{11}=0$ and $s_{12}=0$. In particular, such stress values yield respectively $\rho c_s^2(s_{12}=0) = \mu/\rho = \rho c_2^2$ and $\rho c_f^2(s_{11}=0) = (\lambda + 2\mu)/\rho = \rho c_1^2$ according to equations \eqref{eq:eigen_acc_planeStrain1} and \eqref{eq:eigen_acc_planeStrain2}. (supposing $\abs{s_{11}}\leq \sqrt{\frac{\lambda+\mu}{\beta}}$)

\paragraph*{Condition $s_{12}=0$ :} 
According to equation \eqref{eq:eigen_acc_DP1}, the eigenvalue of the acoustic tensor associated to the speed of slow waves reads for a shear-free state:
\begin{align*}
  & \rho c_s^2 = \frac{1}{2}\( \lambda +3\mu -\beta s_{11}^2 - \abs{\lambda + \mu -\beta s_{11}^2 } \) \\
  & \rho c_f^2 = \frac{1}{2}\( \lambda +3\mu -\beta s_{11}^2 + \abs{\lambda + \mu -\beta s_{11}^2 } \) 
\end{align*}
Then, assuming that $\beta s_{11}^2 < \lambda + \mu$, the expression further reduces to:
\begin{align*}
  & \rho c_s^2 = \mu
  & \rho c_f^2 = \lambda +2\mu -\beta s_{11}^2 
\end{align*}
so that the characteristic speed of slow waves reduces to that of elastic shear waves $c_s=c_2=\sqrt{\mu/\rho}$. Conversely, assuming that $\beta s_{11}^2 > \lambda + \mu$, one gets from equation \eqref{eq:eigen_acc_DP2}:
\begin{align*}
  & \rho c_s^2 = \lambda +2\mu -\beta s_{11}^2  \\
  & \rho c_f^2 =  \mu \\
\end{align*}
The first equation may however lead to a loss of hyperbolicity of the problem since for $\beta s_{11}^2 > \lambda +2\mu$, the characteristic speed of slow waves becomes complex.
At last, the equality $\beta s_{11}^2 = \lambda + \mu$ leads to $A_{11}^{ep}-A_{22}^{ep}=0$. It then appears that $\abs{s_{11}} \neq \sqrt{\frac{\lambda+\mu}{\beta}}$ must be satisfied in order to ensure the hyperbolicity of the problem.
%We first look at the shear-free state for which the subtraction of the acoustic tensor diagonal entries reads: $A_{11}^{ep}-A_{22}^{ep}=\lambda + \mu -\beta s_{11}^2$. 
%Hence, the \textbf{undeterminancy} of the functions $\psi^$ arises for $s_{11} = \pm \sqrt{\frac{\lambda+\mu}{\beta}}$ (on the dowstream side), while if $s_{11} \neq \pm \sqrt{\frac{\lambda+\mu}{\beta}}$, $\psi^f_1 \rightarrow \infty$ and $\psi^s_1 \rightarrow 0$.

Recall that $\psi^f_1$ tending to infinity implies that the loading path are horizontal in $(\sigma_{11},\sigma_{12})$ plane and hence, the fast wave has no influence on the shear stress if, and only if, $\sigma_{12}=0$ downstream. Conversely, the stress paths through slow simple waves are vertical. Moreover, with regard the last row of table \ref{tab:simpleWavesEquations}, $\sigma_{22}$ is also unchanged in that case. As a consequence, if the initial state is shear-free the solution no longer contain combined waves, but longitudinal stress and shear stress simple waves.

\paragraph*{Condition $s_{11}=0$ :} The functions $\psi$ cannot be undetermined in the case $s_{11}=0$ since the equation $A_{11}^{ep}-A_{22}^{ep}=\lambda + \mu + \beta s_{12}^2$ does not admit real solutions. Considering the relation \eqref{eq:plane_strain_stress33} between stress components for plane strain, one has:
%However, one can try to give a "physical meaning" to the condition $s_{11}=0$ by considering the relation \eqref{eq:plane_strain_stress33} between stress components for plane strain:
\begin{equation*}
  s_{11}= \frac{2}{3}\sigma_{11}-\frac{1}{3}(\sigma_{22}+\nu(\sigma_{11}+\sigma_{22})-E\eps^p_{33})
\end{equation*}
so that the previous condition is equivalent to:
\begin{equation}
  \label{eq:plane_strain_s11=0}
  \sigma_{11}=\frac{1+\nu}{2-\nu}\sigma_{22}-E\eps^p_{33}
\end{equation}
In such a stress state, the characteristic speeds are of the form:
\begin{align*}
  & \rho c_s^2 = \mu -\beta s_{12}^2 \\
  & \rho c_f^2 = \lambda +2\mu 
\end{align*}
so that the celerity of fast waves identifies to that of elastic pressure wave $c_f=\sqrt{(\lambda + 2\mu)/\rho}=c_1$.


\paragraph*{Integration of loading paths under plane strain:}
It is assumed that the stress $\sigma_{22}$ in initially zero everywhere in the domain. Several stress paths followed through a fast simple wave and starting from an arbitrary point of the initial yield surface are plotted in figure \ref{fig:fast_path_plane_strains}. Figure \ref{fig:fast_path_plane_strains}\subref{subfig:fastDP_stress} shows the projections in ($\sigma_{11},\sigma_{12}$) and ($\sigma_{22},\sigma_{12}$) planes while figure \ref{fig:fast_path_plane_strains}\subref{subfig:fastDP_dev} shows the stress path in the principal deviatoric stress components space. Note that the projection in that space is orthogonal to the hydrostatic axis $s_1+s_2+s_3=0$ so that the von-Mises yield surface is a circle. Rather, the von-Mises yield surface in that space is a cylindre which axis is used to look at the stress paths in the deviator plane.

Remark, the characteristic speeds are supposed to decrease along the integral curves. It is not the case for all the stress paths depicted in the figures below. In addition, both slow and fast waves lead to loading paths restricted to the yield surface until the direction of pure shear is reached.
\begin{figure}[h!]
  \centering
  \subcaptionbox{Projections of loading paths in ($\sigma_{11},\sigma_{12}$) and ($\sigma_{22},\sigma_{12}$) planes \label{subfig:fastDP_stress}}{\begin{tikzpicture}[scale=0.9]
\begin{groupplot}[group style={group size=2 by 1,
ylabels at=edge left, yticklabels at=edge left,horizontal sep=3.ex,
xticklabels at=edge bottom,xlabels at=edge bottom},
ymajorgrids=true,xmajorgrids=true,ylabel=$\sigma_{12} \: (Pa)$,
axis on top,scale only axis,width=0.4\linewidth,ymin=0,ymax=100000000.0
, every x tick scale label/.style={at={(xticklabel* cs:1.05,0.75cm)},anchor=near yticklabel},colormap={ry}{rgb255(0cm)=(255,255,0);rgb255(1cm)=(255,0,0)}]
\nextgroupplot[xlabel=$\sigma_{11} (Pa)$]
\addplot[mesh,point meta = \thisrow{p},very thick,no markers] table[x=sigma_11,y=sigma_12] {chapter5/pgfFigures/pgf_fastWavesPlaneStrain/DPfastStressPlane_frame0_Stress0.pgf} node[above right,black] {$\textbf{1}$};
\addplot[mesh,point meta = \thisrow{p},very thick,no markers] table[x=sigma_11,y=sigma_12] {chapter5/pgfFigures/pgf_fastWavesPlaneStrain/DPfastStressPlane_frame1_Stress0.pgf} node[above right,black] {$\textbf{2}$};
\addplot[mesh,point meta = \thisrow{p},very thick,no markers] table[x=sigma_11,y=sigma_12] {chapter5/pgfFigures/pgf_fastWavesPlaneStrain/DPfastStressPlane_frame2_Stress0.pgf} node[above right,black] {$\textbf{3}$};
\addplot[mesh,point meta = \thisrow{p},very thick,no markers] table[x=sigma_11,y=sigma_12] {chapter5/pgfFigures/pgf_fastWavesPlaneStrain/DPfastStressPlane_frame3_Stress0.pgf} node[above right,black] {$\textbf{4}$};
\addplot[mesh,point meta = \thisrow{p},very thick,no markers] table[x=sigma_11,y=sigma_12] {chapter5/pgfFigures/pgf_fastWavesPlaneStrain/DPfastStressPlane_frame4_Stress0.pgf} node[above right,black] {$\textbf{5}$};
\addplot[mesh,point meta = \thisrow{p},very thick,no markers] table[x=sigma_11,y=sigma_12] {chapter5/pgfFigures/pgf_fastWavesPlaneStrain/DPfastStressPlane_frame5_Stress0.pgf} node[above right,black] {$\textbf{6}$};
\addplot[gray,dashed,thin] table[x=sigma_11,y=sigma_12] {chapter5/pgfFigures/pgf_fastWavesPlaneStrain/DPfast_yield0_s11s12_Stress0.pgf};

\nextgroupplot[colorbar,colorbar style={title= {$ c_f \: (m/s)$},every y tick scale label/.style={at={(2.,-.1125)}} },xlabel=$\sigma_{22}  (Pa)$]
\addplot[mesh,point meta = \thisrow{p},very thick,no markers] table[x=sigma_22,y=sigma_12] {chapter5/pgfFigures/pgf_fastWavesPlaneStrain/DPfastStressPlane_frame0_Stress0.pgf} node[above right,black] {$\textbf{1}$};
\addplot[mesh,point meta = \thisrow{p},very thick,no markers] table[x=sigma_22,y=sigma_12] {chapter5/pgfFigures/pgf_fastWavesPlaneStrain/DPfastStressPlane_frame1_Stress0.pgf} node[above right,black] {$\textbf{2}$};
\addplot[mesh,point meta = \thisrow{p},very thick,no markers] table[x=sigma_22,y=sigma_12] {chapter5/pgfFigures/pgf_fastWavesPlaneStrain/DPfastStressPlane_frame2_Stress0.pgf} node[above right,black] {$\textbf{3}$};
\addplot[mesh,point meta = \thisrow{p},very thick,no markers] table[x=sigma_22,y=sigma_12] {chapter5/pgfFigures/pgf_fastWavesPlaneStrain/DPfastStressPlane_frame3_Stress0.pgf} node[above right,black] {$\textbf{4}$};
\addplot[mesh,point meta = \thisrow{p},very thick,no markers] table[x=sigma_22,y=sigma_12] {chapter5/pgfFigures/pgf_fastWavesPlaneStrain/DPfastStressPlane_frame4_Stress0.pgf} node[above right,black] {$\textbf{5}$};
\addplot[mesh,point meta = \thisrow{p},very thick,no markers] table[x=sigma_22,y=sigma_12] {chapter5/pgfFigures/pgf_fastWavesPlaneStrain/DPfastStressPlane_frame5_Stress0.pgf} node[above right,black] {$\textbf{6}$};
\addplot[gray,dashed,thin] table[x=sigma_22,y=sigma_12] {chapter5/pgfFigures/pgf_fastWavesPlaneStrain/DPfast_yield0_s22s12_frame0_Stress0.pgf};

\addplot[gray,dashed,thin] table[x=sigma_22,y=sigma_12] {chapter5/pgfFigures/pgf_fastWavesPlaneStrain/DPfast_yield0_s22s12_frame1_Stress0.pgf};

\addplot[gray,dashed,thin] table[x=sigma_22,y=sigma_12] {chapter5/pgfFigures/pgf_fastWavesPlaneStrain/DPfast_yield0_s22s12_frame2_Stress0.pgf};

\addplot[gray,dashed,thin] table[x=sigma_22,y=sigma_12] {chapter5/pgfFigures/pgf_fastWavesPlaneStrain/DPfast_yield0_s22s12_frame3_Stress0.pgf};

\addplot[gray,dashed,thin] table[x=sigma_22,y=sigma_12] {chapter5/pgfFigures/pgf_fastWavesPlaneStrain/DPfast_yield0_s22s12_frame4_Stress0.pgf};

\addplot[gray,dashed,thin] table[x=sigma_22,y=sigma_12] {chapter5/pgfFigures/pgf_fastWavesPlaneStrain/DPfast_yield0_s22s12_frame5_Stress0.pgf};

\end{groupplot}
\end{tikzpicture}
%%% Local Variables:
%%% mode: latex
%%% TeX-master: "../../mainManuscript"
%%% End:
}
  \subcaptionbox{Loading path in deviatoric plane \label{subfig:fastDP_dev}}{\tikzset{cross/.style={cross out, draw=black, minimum size=2*(#1-\pgflinewidth), inner sep=0pt, outer sep=0pt},cross/.default={2.5pt}}
\begin{tikzpicture}[spy using outlines={rectangle, magnification=3, size=2.cm, connect spies},scale=0.9]
\begin{axis}[width=.75\textwidth,view={135}{35.2643},xlabel=$s_1 $,ylabel=$s_2 $,zlabel=$s_3$,xmin=-1.e8,xmax=1.e8,ymin=-1.e8,ymax=1.e8,axis equal,axis lines=center,axis on top,xtick=\empty,ytick=\empty,ztick=\empty,every axis y label/.style={at={(rel axis cs:0.,.5,-0.65)}, anchor=west}, every axis x label/.style={at={(rel axis cs:0.5,.,-0.65)}, anchor=east}, every axis z label/.style={at={(rel axis cs:0.,.0,.18)}, anchor=north},legend style={at={(.2,.68)}}]
\node[below] at (1.1e8,0.,0.) {$\sigma^y$};
\node[above] at (-1.1e8,0.,0.) {$-\sigma^y$};
\draw (1.e8,0.,0.) node[cross,rotate=10] {};
\draw (-1.e8,0.,0.) node[cross,rotate=10] {};
\node[white]  at (0,0.,1.42e8) {};
\addplot3[Red,thick,arrows along my path] file {chapter5/pgfFigures/pgf_fastWavesPlaneStrain/DPfastDevPlane_frame0_Stress0.pgf};\addlegendentry{loading path 1}
\addplot3[Blue,thick,arrows along my path] file {chapter5/pgfFigures/pgf_fastWavesPlaneStrain/DPfastDevPlane_frame1_Stress0.pgf};\addlegendentry{loading path 2}
\addplot3[Orange,thick,arrows along my path] file {chapter5/pgfFigures/pgf_fastWavesPlaneStrain/DPfastDevPlane_frame2_Stress0.pgf};\addlegendentry{loading path 3}
\addplot3[Purple,thick,arrows along my path] file {chapter5/pgfFigures/pgf_fastWavesPlaneStrain/DPfastDevPlane_frame3_Stress0.pgf};\addlegendentry{loading path 4}
\addplot3[Green,thick,arrows along my path] file {chapter5/pgfFigures/pgf_fastWavesPlaneStrain/DPfastDevPlane_frame4_Stress0.pgf};\addlegendentry{loading path 5}
\addplot3[Duck,thick,arrows along my path] file {chapter5/pgfFigures/pgf_fastWavesPlaneStrain/DPfastDevPlane_frame5_Stress0.pgf};\addlegendentry{loading path 6}
\addplot3+[gray,dashed,thin,no markers] file {chapter5/pgfFigures/pgf_fastWavesPlaneStrain/CylindreDevPlane.pgf};\addlegendentry{initial yield surface}
\newcommand\radius{1.*0.82e8}
\addplot3[dotted,thick] coordinates {(0.75*\radius,-0.75*\radius,0.) (-0.75*\radius,0.75*\radius,0.)};
\addplot3[dotted,thick] coordinates {(0.,-0.75*\radius,0.75*\radius) (0.,0.75*\radius,-0.75*\radius)};
\addplot3[dotted,thick] coordinates {(-0.75*\radius,0.,0.75*\radius) (0.75*\radius,0.,-0.75*\radius)};
\begin{scope}
\spy[black,size=1.75cm] on (6.75,3.2) in node [fill=none] at (9.5,5.5);
\end{scope}

\end{axis}
\end{tikzpicture}
%%% Local Variables:
%%% mode: latex
%%% TeX-master: "../../mainManuscript"
%%% End:
}
  \caption{Loading paths through a fast simple wave with initial condition $\sigma_{22}=0$ for different starting points on the initial yield surface.}
  \label{fig:fast_path_plane_strains}
\end{figure}


\begin{figure}[h!]
  \centering
  \subcaptionbox{Slice ($\sigma_{11},\sigma_{12}$) plane}{\begin{tikzpicture}[scale=0.9]
\begin{groupplot}[group style={group size=2 by 1,
ylabels at=edge left, yticklabels at=edge left,horizontal sep=3.ex,
xticklabels at=edge bottom,xlabels at=edge bottom},
ymajorgrids=true,xmajorgrids=true,ylabel=$\sigma_{12} \: (Pa)$,
axis on top,scale only axis,width=0.45\linewidth,ymin=0,ymax=68618075.3103
, every x tick scale label/.style={at={(xticklabel* cs:1.05,0.75cm)},anchor=near yticklabel}]
\nextgroupplot[xlabel=$\sigma_{11} (Pa)$]
\addplot[mesh,point meta = \thisrow{p},very thick,no markers] table[x=sigma_11,y=sigma_12] {chapter5/pgfFigures/pgf_slowWavesPlaneStrain/DPslowStressPlane_frame0_Stress1.pgf};
\addplot[mesh,point meta = \thisrow{p},very thick,no markers] table[x=sigma_11,y=sigma_12] {chapter5/pgfFigures/pgf_slowWavesPlaneStrain/DPslowStressPlane_frame1_Stress1.pgf};
\addplot[mesh,point meta = \thisrow{p},very thick,no markers] table[x=sigma_11,y=sigma_12] {chapter5/pgfFigures/pgf_slowWavesPlaneStrain/DPslowStressPlane_frame2_Stress1.pgf};
\addplot[mesh,point meta = \thisrow{p},very thick,no markers] table[x=sigma_11,y=sigma_12] {chapter5/pgfFigures/pgf_slowWavesPlaneStrain/DPslowStressPlane_frame3_Stress1.pgf};
\addplot[gray,thin] table[x=sigma_11,y=sigma_12] {chapter5/pgfFigures/pgf_slowWavesPlaneStrain/DPslow_yield0_s11s12_Stress1.pgf};

\nextgroupplot[colorbar,colorbar style={title= {$\rho c^2$},every y tick scale label/.style={at={(2.,-.1125)}} },xlabel=$\sigma_{22}  (Pa)$]
\addplot[mesh,point meta = \thisrow{p},very thick,no markers] table[x=sigma_22,y=sigma_12] {chapter5/pgfFigures/pgf_slowWavesPlaneStrain/DPslowStressPlane_frame0_Stress1.pgf};
\addplot[mesh,point meta = \thisrow{p},very thick,no markers] table[x=sigma_22,y=sigma_12] {chapter5/pgfFigures/pgf_slowWavesPlaneStrain/DPslowStressPlane_frame1_Stress1.pgf};
\addplot[mesh,point meta = \thisrow{p},very thick,no markers] table[x=sigma_22,y=sigma_12] {chapter5/pgfFigures/pgf_slowWavesPlaneStrain/DPslowStressPlane_frame2_Stress1.pgf};
\addplot[mesh,point meta = \thisrow{p},very thick,no markers] table[x=sigma_22,y=sigma_12] {chapter5/pgfFigures/pgf_slowWavesPlaneStrain/DPslowStressPlane_frame3_Stress1.pgf};
\end{groupplot}
\end{tikzpicture}
%%% Local Variables:
%%% mode: latex
%%% TeX-master: "../../mainManuscript"
%%% End:
}
  \subcaptionbox{Deviatoric plane}{\begin{tikzpicture}[scale=0.9]
\begin{axis}[width=.75\textwidth,view={135}{35.2643},xlabel=$s_1 $,ylabel=$s_2 $,zlabel=$s_3$,xmin=-1.e8,xmax=1.e8,ymin=-1.e8,ymax=1.e8,axis equal,axis lines=center,axis on top,ztick=\empty]
\addplot3+[Red,very thick,no markers] file {chapter5/pgfFigures/pgf_slowWavesPlaneStrain/DPslowDevPlane_frame0_Stress1.pgf};
\addplot3+[Blue,very thick,no markers] file {chapter5/pgfFigures/pgf_slowWavesPlaneStrain/DPslowDevPlane_frame1_Stress1.pgf};
\addplot3+[Orange,very thick,no markers] file {chapter5/pgfFigures/pgf_slowWavesPlaneStrain/DPslowDevPlane_frame2_Stress1.pgf};
\addplot3+[Purple,very thick,no markers] file {chapter5/pgfFigures/pgf_slowWavesPlaneStrain/DPslowDevPlane_frame3_Stress1.pgf};
\addplot3+[gray,dashed,thin,no markers] file {chapter5/pgfFigures/pgf_slowWavesPlaneStrain/CylindreDevPlane.pgf};
\end{axis}
\end{tikzpicture}
%%% Local Variables:
%%% mode: latex
%%% TeX-master: "../../mainManuscript"
%%% End:
}
  \caption{loading paths through slow simple waves. Stresses in Pa (if required)}
  \label{fig:slow_path_plane_strains}
\end{figure}

\begin{figure}[h!]
  \centering
  \subcaptionbox{Slice ($\sigma_{11},\sigma_{12}$) plane}{\begin{tikzpicture}[scale=0.9]
\begin{groupplot}[group style={group size=2 by 1,
ylabels at=edge left, yticklabels at=edge left,horizontal sep=3.ex,
xticklabels at=edge bottom,xlabels at=edge bottom},
ymajorgrids=true,xmajorgrids=true,ylabel=$\sigma_{12} \: (Pa)$,
axis on top,scale only axis,width=0.4\linewidth,ymin=0,ymax=94979909.10761759
, every x tick scale label/.style={at={(xticklabel* cs:1.05,0.75cm)},anchor=near yticklabel},colormap name=viridis]
\nextgroupplot[xlabel=$\sigma_{11} (Pa)$]
\addplot[mesh,point meta = \thisrow{p},very thick,no markers] table[x=sigma_11,y=sigma_12] {chapter5/pgfFigures/pgf_slowWavesPlaneStrain/DPslowStressPlane_frame0_Stress2.pgf} node[above,black] {$\textbf{1}$};
\addplot[mesh,point meta = \thisrow{p},very thick,no markers] table[x=sigma_11,y=sigma_12] {chapter5/pgfFigures/pgf_slowWavesPlaneStrain/DPslowStressPlane_frame1_Stress2.pgf} node[above,black] {$\textbf{2}$};
\addplot[mesh,point meta = \thisrow{p},very thick,no markers] table[x=sigma_11,y=sigma_12] {chapter5/pgfFigures/pgf_slowWavesPlaneStrain/DPslowStressPlane_frame2_Stress2.pgf} node[above,black] {$\textbf{3}$};
\addplot[mesh,point meta = \thisrow{p},very thick,no markers] table[x=sigma_11,y=sigma_12] {chapter5/pgfFigures/pgf_slowWavesPlaneStrain/DPslowStressPlane_frame3_Stress2.pgf} node[above,black] {$\textbf{4}$};
\addplot[gray,dashed,thin] table[x=sigma_11,y=sigma_12] {chapter5/pgfFigures/pgf_slowWavesPlaneStrain/DPslow_yield0_s11s12_Stress2.pgf};

\addplot[gray,dashed,thin] table[x=sigma_11,y=sigma_12] {chapter5/pgfFigures/pgf_slowWavesPlaneStrain/DPslow_yieldfin_s11s12_frame0_Stress2.pgf};

\addplot[gray,dashed,thin] table[x=sigma_11,y=sigma_12] {chapter5/pgfFigures/pgf_slowWavesPlaneStrain/DPslow_yieldfin_s11s12_frame1_Stress2.pgf};

\addplot[gray,dashed,thin] table[x=sigma_11,y=sigma_12] {chapter5/pgfFigures/pgf_slowWavesPlaneStrain/DPslow_yieldfin_s11s12_frame2_Stress2.pgf};

\addplot[gray,dashed,thin] table[x=sigma_11,y=sigma_12] {chapter5/pgfFigures/pgf_slowWavesPlaneStrain/DPslow_yieldfin_s11s12_frame3_Stress2.pgf};

\nextgroupplot[colorbar,colorbar style={title= {$ c_s \: (m/s)$},every y tick scale label/.style={at={(2.,-.1125)}} },xlabel=$\sigma_{22}  (Pa)$]
\addplot[mesh,point meta = \thisrow{p},very thick,no markers] table[x=sigma_22,y=sigma_12] {chapter5/pgfFigures/pgf_slowWavesPlaneStrain/DPslowStressPlane_frame0_Stress2.pgf} node[above,black] {$\textbf{1}$};
\addplot[mesh,point meta = \thisrow{p},very thick,no markers] table[x=sigma_22,y=sigma_12] {chapter5/pgfFigures/pgf_slowWavesPlaneStrain/DPslowStressPlane_frame1_Stress2.pgf} node[above,black] {$\textbf{2}$};
\addplot[mesh,point meta = \thisrow{p},very thick,no markers] table[x=sigma_22,y=sigma_12] {chapter5/pgfFigures/pgf_slowWavesPlaneStrain/DPslowStressPlane_frame2_Stress2.pgf} node[above,black] {$\textbf{3}$};
\addplot[mesh,point meta = \thisrow{p},very thick,no markers] table[x=sigma_22,y=sigma_12] {chapter5/pgfFigures/pgf_slowWavesPlaneStrain/DPslowStressPlane_frame3_Stress2.pgf} node[above,black] {$\textbf{4}$};
\addplot[gray,dashed,thin] table[x=sigma_22,y=sigma_12] {chapter5/pgfFigures/pgf_slowWavesPlaneStrain/DPslow_yieldfin_s22s12_frame0_Stress2.pgf};

\addplot[gray,dashed,thin] table[x=sigma_22,y=sigma_12] {chapter5/pgfFigures/pgf_slowWavesPlaneStrain/DPslow_yieldfin_s22s12_frame1_Stress2.pgf};

\addplot[gray,dashed,thin] table[x=sigma_22,y=sigma_12] {chapter5/pgfFigures/pgf_slowWavesPlaneStrain/DPslow_yieldfin_s22s12_frame2_Stress2.pgf};

\addplot[gray,dashed,thin] table[x=sigma_22,y=sigma_12] {chapter5/pgfFigures/pgf_slowWavesPlaneStrain/DPslow_yieldfin_s22s12_frame3_Stress2.pgf};

\end{groupplot}
\end{tikzpicture}
%%% Local Variables:
%%% mode: latex
%%% TeX-master: "../../mainManuscript"
%%% End:
}
  \subcaptionbox{Deviatoric plane}{\tikzset{cross/.style={cross out, draw=black, minimum size=2*(#1-\pgflinewidth), inner sep=0pt, outer sep=0pt},cross/.default={2.5pt}}
\begin{tikzpicture}[scale=0.9]
\begin{axis}[width=.75\textwidth,view={135}{35.2643},xlabel=$s_1 $,ylabel=$s_2 $,zlabel=$s_3$,xmin=-1.e8,xmax=1.e8,ymin=-1.e8,ymax=1.e8,axis equal,axis lines=center,axis on top,xtick=\empty,ytick=\empty,ztick=\empty,every axis y label/.style={at={(rel axis cs:0.,.5,-0.65)}, anchor=west}, every axis x label/.style={at={(rel axis cs:0.5,.,-0.65)}, anchor=east}, every axis z label/.style={at={(rel axis cs:0.,.0,.18)}, anchor=north},legend style={at={(.2,.68)}}]
\node[below] at (1.1e8,0.,0.) {$\sigma^y$};
\node[above] at (-1.1e8,0.,0.) {$-\sigma^y$};
\draw (1.e8,0.,0.) node[cross,rotate=10] {};
\draw (-1.e8,0.,0.) node[cross,rotate=10] {};
\node[white]  at (0,0.,1.1e8) {};
\addplot3[arrows along my path,Red,very thick] file {chapter5/pgfFigures/pgf_slowWavesPlaneStrain/DPslowDevPlane_frame0_Stress2.pgf};\addlegendentry{loading path 1}
\addplot3[arrows along my path,Blue,very thick] file {chapter5/pgfFigures/pgf_slowWavesPlaneStrain/DPslowDevPlane_frame1_Stress2.pgf};\addlegendentry{loading path 2}
\addplot3[arrows along my path,Orange,very thick] file {chapter5/pgfFigures/pgf_slowWavesPlaneStrain/DPslowDevPlane_frame2_Stress2.pgf};\addlegendentry{loading path 3}
\addplot3[arrows along my path,Purple,very thick] file {chapter5/pgfFigures/pgf_slowWavesPlaneStrain/DPslowDevPlane_frame3_Stress2.pgf};\addlegendentry{loading path 4}
\addplot3[arrows along my path,Green,very thick] file {chapter5/pgfFigures/pgf_slowWavesPlaneStrain/DPslowDevPlane_frame4_Stress2.pgf};\addlegendentry{loading path 5}
\addplot3+[gray,dashed,thin,no markers] file {chapter5/pgfFigures/pgf_slowWavesPlaneStrain/CylindreDevPlane.pgf};\addlegendentry{initial yield surface}
\newcommand\radius{1.*0.82e8}
\addplot3[dotted,thick] coordinates {(0.75*\radius,-0.75*\radius,0.) (-0.75*\radius,0.75*\radius,0.)};
\addplot3[dotted,thick] coordinates {(0.,-0.75*\radius,0.75*\radius) (0.,0.75*\radius,-0.75*\radius)};
\addplot3[dotted,thick] coordinates {(-0.75*\radius,0.,0.75*\radius) (0.75*\radius,0.,-0.75*\radius)};
\end{axis}
\end{tikzpicture}
%%% Local Variables:
%%% mode: latex
%%% TeX-master: "../../mainManuscript"
%%% End:
}
  \caption{loading paths through slow simple waves. Stresses in Pa (if required)}
  \label{fig:slow_path_plane_strains}
\end{figure}


\begin{figure}[h!]
  \centering
  \subcaptionbox{Slice ($\sigma_{11},\sigma_{12}$) plane}{\begin{tikzpicture}[scale=0.9]
\begin{groupplot}[group style={group size=2 by 1,
ylabels at=edge left, yticklabels at=edge left,horizontal sep=3.ex,
xticklabels at=edge bottom,xlabels at=edge bottom},
ymajorgrids=true,xmajorgrids=true,ylabel=$\sigma_{12} \: (Pa)$,
axis on top,scale only axis,width=0.4\linewidth,ymin=0,ymax=68618075.3102588
, every x tick scale label/.style={at={(xticklabel* cs:1.05,0.75cm)},anchor=near yticklabel},colormap={ry}{rgb255(0cm)=(255,255,0);rgb255(1cm)=(255,0,0)}]
\nextgroupplot[xlabel=$\sigma_{11} (Pa)$]
\addplot[mesh,point meta = \thisrow{p},very thick,no markers] table[x=sigma_11,y=sigma_12] {chapter5/pgfFigures/pgf_slowWavesPlaneStrain/DPslowStressPlane_frame0_Stress3.pgf} node[above right,black] {$\textbf{1}$};
\addplot[mesh,point meta = \thisrow{p},very thick,no markers] table[x=sigma_11,y=sigma_12] {chapter5/pgfFigures/pgf_slowWavesPlaneStrain/DPslowStressPlane_frame1_Stress3.pgf} node[above right,black] {$\textbf{2}$};
\addplot[mesh,point meta = \thisrow{p},very thick,no markers] table[x=sigma_11,y=sigma_12] {chapter5/pgfFigures/pgf_slowWavesPlaneStrain/DPslowStressPlane_frame2_Stress3.pgf} node[above right,black] {$\textbf{3}$};
\addplot[mesh,point meta = \thisrow{p},very thick,no markers] table[x=sigma_11,y=sigma_12] {chapter5/pgfFigures/pgf_slowWavesPlaneStrain/DPslowStressPlane_frame3_Stress3.pgf} node[above right,black] {$\textbf{4}$};
\addplot[gray,dashed,thin] table[x=sigma_11,y=sigma_12] {chapter5/pgfFigures/pgf_slowWavesPlaneStrain/DPslow_yield0_s11s12_Stress3.pgf};

\nextgroupplot[colorbar,colorbar style={title= {$ c_s \: (m/s)$},every y tick scale label/.style={at={(2.,-.1125)}} },xlabel=$\sigma_{22}  (Pa)$]
\addplot[mesh,point meta = \thisrow{p},very thick,no markers] table[x=sigma_22,y=sigma_12] {chapter5/pgfFigures/pgf_slowWavesPlaneStrain/DPslowStressPlane_frame0_Stress3.pgf} node[above right,black] {$\textbf{1}$};
\addplot[mesh,point meta = \thisrow{p},very thick,no markers] table[x=sigma_22,y=sigma_12] {chapter5/pgfFigures/pgf_slowWavesPlaneStrain/DPslowStressPlane_frame1_Stress3.pgf} node[above right,black] {$\textbf{2}$};
\addplot[mesh,point meta = \thisrow{p},very thick,no markers] table[x=sigma_22,y=sigma_12] {chapter5/pgfFigures/pgf_slowWavesPlaneStrain/DPslowStressPlane_frame2_Stress3.pgf} node[above right,black] {$\textbf{3}$};
\addplot[mesh,point meta = \thisrow{p},very thick,no markers] table[x=sigma_22,y=sigma_12] {chapter5/pgfFigures/pgf_slowWavesPlaneStrain/DPslowStressPlane_frame3_Stress3.pgf} node[above right,black] {$\textbf{4}$};
\addplot[gray,dashed,thin] table[x=sigma_22,y=sigma_12] {chapter5/pgfFigures/pgf_slowWavesPlaneStrain/DPslow_yield0_s22s12_frame0_Stress3.pgf};

\addplot[gray,dashed,thin] table[x=sigma_22,y=sigma_12] {chapter5/pgfFigures/pgf_slowWavesPlaneStrain/DPslow_yield0_s22s12_frame1_Stress3.pgf};

\addplot[gray,dashed,thin] table[x=sigma_22,y=sigma_12] {chapter5/pgfFigures/pgf_slowWavesPlaneStrain/DPslow_yield0_s22s12_frame2_Stress3.pgf};

\addplot[gray,dashed,thin] table[x=sigma_22,y=sigma_12] {chapter5/pgfFigures/pgf_slowWavesPlaneStrain/DPslow_yield0_s22s12_frame3_Stress3.pgf};

\end{groupplot}
\end{tikzpicture}
%%% Local Variables:
%%% mode: latex
%%% TeX-master: "../../mainManuscript"
%%% End:
}
  \subcaptionbox{Deviatoric plane}{\begin{tikzpicture}[scale=0.9]
\begin{axis}[width=.75\textwidth,view={135}{35.2643},xlabel=$s_1 $,ylabel=$s_2 $,zlabel=$s_3$,xmin=-1.e8,xmax=1.e8,ymin=-1.e8,ymax=1.e8,axis equal,axis lines=center,axis on top,ztick=\empty]
\addplot3+[Red,very thick,no markers] file {chapter5/pgfFigures/pgf_slowWavesPlaneStrain/DPslowDevPlane_frame0_Stress3.pgf};
\addplot3+[Blue,very thick,no markers] file {chapter5/pgfFigures/pgf_slowWavesPlaneStrain/DPslowDevPlane_frame1_Stress3.pgf};
\addplot3+[Orange,very thick,no markers] file {chapter5/pgfFigures/pgf_slowWavesPlaneStrain/DPslowDevPlane_frame2_Stress3.pgf};
\addplot3+[Purple,very thick,no markers] file {chapter5/pgfFigures/pgf_slowWavesPlaneStrain/DPslowDevPlane_frame3_Stress3.pgf};
\addplot3+[gray,dashed,thin,no markers] file {chapter5/pgfFigures/pgf_slowWavesPlaneStrain/CylindreDevPlane.pgf};
\end{axis}
\end{tikzpicture}
%%% Local Variables:
%%% mode: latex
%%% TeX-master: "../../mainManuscript"
%%% End:
}
  \caption{loading paths through slow simple waves. Stresses in Pa (if required)}
  \label{fig:slow_path_plane_strains}
\end{figure}


\subsection{The plane stress case}
Under plane stress, the elastoplastic tangent modulus can be rewritten by taking into account the vanishing stress component $\sigma_{33}$:

\begin{figure}[h!]
  \centering
  {\begin{tikzpicture}[scale=0.7]
  \begin{axis}[ymajorgrids=true,xmajorgrids=true,ylabel=$\tau \: (Pa)$,xlabel=$\sigma \: (Pa)$,xmin=-0.1e8,xmax=2.e8,ymin=0.,ymax=7.5e7]
    %%
    \addplot[very thick] table [x=sigma_11,y=sigma_12] {section5/pgfFigures/pgf_thinWalledTubeSlowWave/TWslowStressPlane_Stress0.pgf};
    %%
    \addplot[very thick] table [x=sigma_11,y=sigma_12] {section5/pgfFigures/pgf_thinWalledTubeSlowWave/TWslowStressPlane_Stress1.pgf};
    %%
    \addplot[very thick] table [x=sigma_11,y=sigma_12] {section5/pgfFigures/pgf_thinWalledTubeSlowWave/TWslowStressPlane_Stress2.pgf};
    %%
    \addplot[very thick] table [x=sigma_11,y=sigma_12] {section5/pgfFigures/pgf_thinWalledTubeSlowWave/TWslowStressPlane_Stress3.pgf};
    %%
    \addplot[very thick] table [x=sigma_11,y=sigma_12] {section5/pgfFigures/pgf_thinWalledTubeSlowWave/TWslowStressPlane_Stress4.pgf};
    %%
    \addplot[very thick] table [x=sigma_11,y=sigma_12] {section5/pgfFigures/pgf_thinWalledTubeSlowWave/TWslowStressPlane_Stress5.pgf};
    %%
    \addplot[very thick] table [x=sigma_11,y=sigma_12] {section5/pgfFigures/pgf_thinWalledTubeSlowWave/TWslowStressPlane_Stress6.pgf};
    %% Yield surface
    \addplot[black,dashed] table  [x=sigma_11,y=sigma_12] {section5/pgfFigures/pgf_thinWalledTubeSlowWave/TWslow_yield0.pgf};

    %\addplot[very thick,Orange,restrict y to domain=4.e7:6.75e7] table [x=sigma_11,y=sigma_12]{section5/pgfFigures/pgf_thinWalledTubeSlowWave/TWslowStressPlane_Stress1.pgf};


  \end{axis}
\end{tikzpicture}

%%% Local Variables:
%%% mode: latex
%%% TeX-master: "../../presentation"
%%% End:}
  \caption{comparison between Clifton and own solution}
\end{figure}
\begin{figure}[h!]
  \centering
  {\tikzset{cross/.style={cross out, draw=black, minimum size=2*(#1-\pgflinewidth), inner sep=0pt, outer sep=0pt},
%default radius will be 1pt. 
cross/.default={2.5pt}}
\begin{tikzpicture}[scale=0.9]
  \begin{axis}[width=.75\textwidth,view={135}{35.2643},xlabel=$s_1 $,
    ylabel=$s_2 $,zlabel=$s_3$,xmin=-1.e8,xmax=1.e8,ymin=-1.e8,ymax=1.e8,axis equal,axis lines=center,axis on top,xtick=\empty,ytick=\empty,ztick=\empty,
    every axis y label/.style={at={(rel axis cs:0.,.5,-0.65)}, anchor=west},
    every axis x label/.style={at={(rel axis cs:0.5,.,-0.65)}, anchor=east},
    every axis z label/.style={at={(rel axis cs:0.,.0,.18)}, anchor=north}
    ]
    \node[below] at (1.1e8,0.,0.) {$\sigma^y$};
    \node[above] at (-1.1e8,0.,0.) {$-\sigma^y$};
    \draw (1.e8,0.,0.) node[cross,rotate=10] {};
    \draw (-1.e8,0.,0.) node[cross,rotate=10] {};
    \node[white]  at (0,0.,1.42e8) {};
    %%
    \addplot3[Green,mark=x,only marks,mark repeat=20,very thick] file {chapter5/pgfFigures/pgf_thinWalledTubeSlowWave/slowDevPlane_Stress0.pgf};
    \addplot3[Green,thick] file {chapter5/pgfFigures/pgf_thinWalledTubeSlowWave/slowDevPlane_Stress0.pgf};
    %%
    \addplot3[Duck,mark=x,only marks,mark repeat=20,very thick] file {chapter5/pgfFigures/pgf_thinWalledTubeSlowWave/slowDevPlane_Stress1.pgf};
    \addplot3[Duck,thick] file {chapter5/pgfFigures/pgf_thinWalledTubeSlowWave/slowDevPlane_Stress1.pgf};
    %%
    \addplot3[Red,mark=x,only marks,mark repeat=20,very thick] file {chapter5/pgfFigures/pgf_thinWalledTubeSlowWave/slowDevPlane_Stress2.pgf};
    \addplot3[Red,thick] file {chapter5/pgfFigures/pgf_thinWalledTubeSlowWave/slowDevPlane_Stress2.pgf};
    %%
    \addplot3[Purple,mark=x,only marks,mark repeat=20,very thick] file {chapter5/pgfFigures/pgf_thinWalledTubeSlowWave/slowDevPlane_Stress3.pgf};
    \addplot3[Purple,thick] file {chapter5/pgfFigures/pgf_thinWalledTubeSlowWave/slowDevPlane_Stress3.pgf};
    %%
    \addplot3[Blue,mark=x,only marks,mark repeat=20,very thick] file {chapter5/pgfFigures/pgf_thinWalledTubeSlowWave/slowDevPlane_Stress4.pgf};
    \addplot3[Blue,thick] file {chapter5/pgfFigures/pgf_thinWalledTubeSlowWave/slowDevPlane_Stress4.pgf};
    %% 
    \addplot3[Orange,mark=x,only marks,mark repeat=20,very thick] file {chapter5/pgfFigures/pgf_thinWalledTubeSlowWave/slowDevPlane_Stress5.pgf};
    \addplot3[Orange,thick] file {chapter5/pgfFigures/pgf_thinWalledTubeSlowWave/slowDevPlane_Stress5.pgf};
    %% 
    \addplot3[Yellow,mark=x,only marks,mark repeat=5,very thick] file {chapter5/pgfFigures/pgf_thinWalledTubeSlowWave/slowDevPlane_Stress6.pgf};
    \addplot3[Yellow,thick] file {chapter5/pgfFigures/pgf_thinWalledTubeSlowWave/slowDevPlane_Stress6.pgf};
    %% Yield surface
    \addplot3[black,dashed] file {chapter5/pgfFigures/pgf_thinWalledTubeSlowWave/TWCylindreDevPlane.pgf};
    \newcommand\radius{0.82e8}
    \addplot3[dotted,thick] coordinates {(0.75*\radius,-0.75*\radius,0.) (-0.75*\radius,0.75*\radius,0.)};
    \addplot3[dotted,thick] coordinates {(0.,-0.75*\radius,0.75*\radius) (0.,0.75*\radius,-0.75*\radius)};
    \addplot3[dotted,thick] coordinates {(-0.75*\radius,0.,0.75*\radius) (0.75*\radius,0.,-0.75*\radius)};

    % \newcommand\radius{0.82e8}
    % \addplot3[dotted,very thick] coordinates {(1.05*\radius,-1.05*\radius,0.) (-1.05*\radius,1.05*\radius,0.)};

  \end{axis}
\end{tikzpicture}

%%% Local Variables:
%%% mode: latex
%%% TeX-master: "../../mainManuscript"
%%% End:}
  \caption{Projection in deviatoric plane}
\end{figure}
\begin{equation*}
  \dot{\sigma}_{33}=C^{ep}_{33ij} \dot{\eps}_{ij} =0
\end{equation*}
Then, one gets:
\begin{equation*}
  C^{ep}_{3333} \dot{\eps}_{33} = - C^{ep}_{33ij}\dot{\eps}_{ij} \qquad i,j=\{1,2\}
\end{equation*}
Hence, the constitutive equation reads:
\begin{equation}
  \label{eq:CP_constitutive}
  \dot{\sigma}_{ij}=C^{ep}_{ijkl} \dot{\eps}_{kl} - \frac{C^{ep}_{ij33}C^{ep}_{33kl}}{C^{ep}_{3333}}\dot{\eps}_{kl} \qquad i,j,k,l=\{1,2\} 
\end{equation}
Thus, the components of the acoustic tensor, after some simplifications, read:
\begin{align}
  & A_{11}^{ep}= C^{ep}_{1111} - \frac{{C^{ep}_{1133}}^2}{C^{ep}_{3333}} = \lambda + 2\mu -\beta s_{11}^2 -\frac{\(\lambda -\beta s_{11}s_{33}\)^2}{\lambda + 2\mu - \beta s_{33}^2} \\
  & A_{22}^{ep}= C^{ep}_{1212} - \frac{{C^{ep}_{1233}}^2}{C^{ep}_{3333}}= \mu - \beta s_{12}^2 -\frac{\(\beta s_{12}s_{33}\)^2}{\lambda + 2\mu - \beta s_{33}^2} \\
  & A_{12}^{ep} = C^{ep}_{1112} - \frac{C^{ep}_{1133}C^{ep}_{1233}}{C^{ep}_{3333}} =\beta s_{12} \frac{\lambda s_{33} - (\lambda + 2\mu)s_{11} }{\lambda + 2\mu - \beta s_{33}^2}  
  % & A_{12}^{ep}=C^{ep}_{1112} - \frac{C^{ep}_{1133}C^{ep}_{1233}}{C^{ep}_{3333}} = -\beta s_{11}s_{12}- \frac{\(\lambda - \beta s_{11}s_{33}\)\( -\beta s_{12}s_{33}\)}{\lambda + 2\mu - \beta s_{33}^2}
\end{align}
% \begin{equation*}
%   A_{12}^{ep} = \beta s_{12} \( s_{33}\frac{\lambda - \beta s_{11}s_{33}}{\lambda + 2\mu - \beta s_{33}^2} -s_{11}  \) =\beta s_{12} \frac{\lambda s_{33} - (\lambda + 2\mu)s_{11} }{\lambda + 2\mu - \beta s_{33}^2}  
% \end{equation*}
From the last equation, $A_{12}^{ep}$ obviously vanishes for $s_{12}=0$ and $s_{11}=\lambda s_{33}/(\lambda+2\mu)$, the latter condition leading to $\sigma_{11}=\frac{1-2\nu}{2-\nu} \sigma_{22}$. Hence, the function $\psi^f_1 \rightarrow \infty$ for $s_{12}=0$ and $\sigma_{11}=\frac{1-2\nu}{2-\nu} \sigma_{22}$. In order to ensure the hyperbolicity of the system, the component of the acoustic tensor also have to be defined, that is $C^{ep}_{3333}\neq 0$. This condition leads to:
\begin{equation*}
  \lambda + 2\mu - \beta s_{33}^2 \neq 0 \quad \Leftrightarrow \quad s_{33}\neq \frac{\lambda + 2\mu}{\beta}
\end{equation*}

\paragraph*{Case $s_{12}=0$ :}
\begin{align}
  & \rho c_s^2 =\frac{1}{2}\(A_{11}^{ep}+A_{22}^{ep} - \abs{A_{11}^{ep}-A_{22}^{ep}}\) \\
  & \rho c_f^2 =\frac{1}{2}\(A_{11}^{ep}+A_{22}^{ep} + \abs{A_{11}^{ep}-A_{22}^{ep}}\)
\end{align}

The other singularity $A^{ep}_{11}=A^{ep}_{22}$ is on the other hand, more complex to deal with.

\begin{figure}[h!]
  \centering
  \subcaptionbox{Projections of loading paths in ($\sigma_{11},\sigma_{12}$) and ($\sigma_{22},\sigma_{12}$) planes}{\begin{tikzpicture}[scale=0.9]
\begin{groupplot}[group style={group size=2 by 1,
ylabels at=edge left, yticklabels at=edge left,horizontal sep=3.ex,
xticklabels at=edge bottom,xlabels at=edge bottom},
ymajorgrids=true,xmajorgrids=true,ylabel=$\sigma_{12} \: (Pa)$,
axis on top,scale only axis,width=0.4\linewidth,ymin=0,ymax=63499406.78820015
, every x tick scale label/.style={at={(xticklabel* cs:1.05,0.75cm)},anchor=near yticklabel},colormap name=viridis]
\nextgroupplot[xlabel=$\sigma_{11} (Pa)$]
\addplot[arrows along my path,black,thick] table[x=sigma_11,y=sigma_12] {chapter5/pgfFigures/pgf_fastWavesPlaneStress/CPfastStressPlane_frame0_Stress0.pgf};\addplot[mesh,point meta = \thisrow{p},very thick,no markers] table[x=sigma_11,y=sigma_12] {chapter5/pgfFigures/pgf_fastWavesPlaneStress/CPfastStressPlane_frame0_Stress0.pgf} node[above right,black] {$\textbf{1}$};
\addplot[arrows along my path,black,thick] table[x=sigma_11,y=sigma_12] {chapter5/pgfFigures/pgf_fastWavesPlaneStress/CPfastStressPlane_frame1_Stress0.pgf};\addplot[mesh,point meta = \thisrow{p},very thick,no markers] table[x=sigma_11,y=sigma_12] {chapter5/pgfFigures/pgf_fastWavesPlaneStress/CPfastStressPlane_frame1_Stress0.pgf} node[above right,black] {$\textbf{2}$};
\addplot[gray,dashed,thin] table[x=sigma_11,y=sigma_12] {chapter5/pgfFigures/pgf_fastWavesPlaneStress/CPfast_yield0_s11s12_Stress0.pgf};

\nextgroupplot[colorbar,colorbar style={title= {$c_f \: (m/s)$},every y tick scale label/.style={at={(2.,-.1125)}} },xlabel=$\sigma_{22}  (Pa)$]
\addplot[arrows along my path,black,thick] table[x=sigma_22,y=sigma_12] {chapter5/pgfFigures/pgf_fastWavesPlaneStress/CPfastStressPlane_frame0_Stress0.pgf};\addplot[mesh,point meta = \thisrow{p},very thick,no markers] table[x=sigma_22,y=sigma_12] {chapter5/pgfFigures/pgf_fastWavesPlaneStress/CPfastStressPlane_frame0_Stress0.pgf} node[above right,black] {$\textbf{1}$};
\addplot[arrows along my path,black,thick] table[x=sigma_22,y=sigma_12] {chapter5/pgfFigures/pgf_fastWavesPlaneStress/CPfastStressPlane_frame1_Stress0.pgf};\addplot[mesh,point meta = \thisrow{p},very thick,no markers] table[x=sigma_22,y=sigma_12] {chapter5/pgfFigures/pgf_fastWavesPlaneStress/CPfastStressPlane_frame1_Stress0.pgf} node[above right,black] {$\textbf{2}$};
\end{groupplot}
\end{tikzpicture}
%%% Local Variables:
%%% mode: latex
%%% TeX-master: "../../mainManuscript"
%%% End:
}
  \subcaptionbox{Loading path in deviatoric plane}{\tikzset{cross/.style={cross out, draw=black, minimum size=2*(#1-\pgflinewidth), inner sep=0pt, outer sep=0pt},cross/.default={2.5pt}}
\begin{tikzpicture}[scale=0.9]
\begin{axis}[width=.75\textwidth,view={135}{35.2643},xlabel=$s_1 $,ylabel=$s_2 $,zlabel=$s_3$,xmin=-1.e8,xmax=1.e8,ymin=-1.e8,ymax=1.e8,axis equal,axis lines=center,axis on top,xtick=\empty,ytick=\empty,ztick=\empty,every axis y label/.style={at={(rel axis cs:0.,.5,-0.65)}, anchor=west}, every axis x label/.style={at={(rel axis cs:0.5,.,-0.65)}, anchor=east}, every axis z label/.style={at={(rel axis cs:0.,.0,.18)}, anchor=north},legend style={at={(.225,.59)}}]
\node[below] at (1.1e8,0.,0.) {$\sigma^y$};
\node[above] at (-1.1e8,0.,0.) {$-\sigma^y$};
\draw (1.e8,0.,0.) node[cross,rotate=10] {};
\draw (-1.e8,0.,0.) node[cross,rotate=10] {};
\node[white]  at (0,0.,1.1e8) {};
\addplot3[gray,dashed,thin,no markers] file {chapter5/pgfFigures/pgf_fastWavesPlaneStress/CPCylindreDevPlane.pgf};\addlegendentry{initial yield surface}
%\addplot3[Red,mark=star,mark repeat=20,mark size=3pt,very thick] file {chapter5/pgfFigures/pgf_fastWavesPlaneStress/CPfastDevPlane_frame0_Stress0.pgf};
\addplot3[arrows along my path,Red,very thick] file {chapter5/pgfFigures/pgf_fastWavesPlaneStress/CPfastDevPlane_frame0_Stress0.pgf};
\addlegendentry{loading path 1}
%\addplot3[Blue,mark=asterisk,mark repeat=20,mark size=3pt,very thick] file {chapter5/pgfFigures/pgf_fastWavesPlaneStress/CPfastDevPlane_frame1_Stress0.pgf};
\addplot3[arrows along my path,Blue,very thick] file {chapter5/pgfFigures/pgf_fastWavesPlaneStress/CPfastDevPlane_frame1_Stress0.pgf};
\addlegendentry{loading path 2}
\newcommand\radius{1.*0.82e8}
\addplot3[dotted,thick] coordinates {(0.75*\radius,-0.75*\radius,0.) (-0.75*\radius,0.75*\radius,0.)};
\addplot3[dotted,thick] coordinates {(0.,-0.75*\radius,0.75*\radius) (0.,0.75*\radius,-0.75*\radius)};
\addplot3[dotted,thick] coordinates {(-0.75*\radius,0.,0.75*\radius) (0.75*\radius,0.,-0.75*\radius)};
\end{axis}
\end{tikzpicture}
%%% Local Variables:
%%% mode: latex
%%% TeX-master: "../../mainManuscript"
%%% End:
}
  \caption{Loading paths through a fast simple wave with initial condition $\sigma_{22}=0$ for different starting points on the initial yield surface. Stresses in Pa}
  \label{fig:fast_path_plane_strains}
\end{figure}


\begin{figure}[h!]
  \centering
  \subcaptionbox{Slice ($\sigma_{11},\sigma_{12}$) plane}{\begin{tikzpicture}[scale=0.9]
\begin{groupplot}[group style={group size=2 by 1,
ylabels at=edge left, yticklabels at=edge left,horizontal sep=3.ex,
xticklabels at=edge bottom,xlabels at=edge bottom},
ymajorgrids=true,xmajorgrids=true,ylabel=$\sigma_{12} \: (Pa)$,
axis on top,scale only axis,width=0.4\linewidth,ymin=0,ymax=109528891.78848386
, every x tick scale label/.style={at={(xticklabel* cs:1.05,0.75cm)},anchor=near yticklabel},colormap name=viridis]
, every x tick scale label/.style={at={(xticklabel* cs:1.05,0.75cm)},anchor=near yticklabel},colormap name=viridis]
\nextgroupplot[xlabel=$\sigma_{11} \: (Pa)$]
\addplot[arrows along my path,black,thick] table[x=sigma_11,y=sigma_12] {chapter5/pgfFigures/pgf_slowWavesPlaneStress/CPslowStressPlane_frame0_Stress1.pgf};
\addplot[mesh,point meta = \thisrow{p},very thick,no markers] table[x=sigma_11,y=sigma_12] {chapter5/pgfFigures/pgf_slowWavesPlaneStress/CPslowStressPlane_frame0_Stress1.pgf} node[above right,black] {$\textbf{1}$};
\addplot[arrows along my path,black,thick] table[x=sigma_11,y=sigma_12] {chapter5/pgfFigures/pgf_slowWavesPlaneStress/CPslowStressPlane_frame1_Stress1.pgf};
\addplot[mesh,point meta = \thisrow{p},very thick,no markers] table[x=sigma_11,y=sigma_12] {chapter5/pgfFigures/pgf_slowWavesPlaneStress/CPslowStressPlane_frame1_Stress1.pgf} node[above right,black] {$\textbf{2}$};
\addplot[arrows along my path,black,thick] table[x=sigma_11,y=sigma_12] {chapter5/pgfFigures/pgf_slowWavesPlaneStress/CPslowStressPlane_frame2_Stress1.pgf};
\addplot[mesh,point meta = \thisrow{p},very thick,no markers] table[x=sigma_11,y=sigma_12] {chapter5/pgfFigures/pgf_slowWavesPlaneStress/CPslowStressPlane_frame2_Stress1.pgf} node[above right,black] {$\textbf{3}$};
\addplot[arrows along my path,black,thick] table[x=sigma_11,y=sigma_12] {chapter5/pgfFigures/pgf_slowWavesPlaneStress/CPslowStressPlane_frame3_Stress1.pgf};
\addplot[mesh,point meta = \thisrow{p},very thick,no markers] table[x=sigma_11,y=sigma_12] {chapter5/pgfFigures/pgf_slowWavesPlaneStress/CPslowStressPlane_frame3_Stress1.pgf} node[above right,black] {$\textbf{4}$};
\addplot[gray,dashed,thin] table[x=sigma_11,y=sigma_12] {chapter5/pgfFigures/pgf_slowWavesPlaneStress/CPslow_yield0_s11s12_Stress1.pgf};

\nextgroupplot[colorbar,colorbar style={title= {$ c_s \: (m/s)$},every y tick scale label/.style={at={(2.,-.1125)}} },xlabel=$\sigma_{22} \: (Pa)$]
\addplot[arrows along my path,black,thick] table[x=sigma_22,y=sigma_12] {chapter5/pgfFigures/pgf_slowWavesPlaneStress/CPslowStressPlane_frame0_Stress1.pgf};
\addplot[mesh,point meta = \thisrow{p},very thick,no markers] table[x=sigma_22,y=sigma_12] {chapter5/pgfFigures/pgf_slowWavesPlaneStress/CPslowStressPlane_frame0_Stress1.pgf} node[above right,black] {$\textbf{1}$};
\addplot[arrows along my path,black,thick] table[x=sigma_22,y=sigma_12] {chapter5/pgfFigures/pgf_slowWavesPlaneStress/CPslowStressPlane_frame1_Stress1.pgf};
\addplot[mesh,point meta = \thisrow{p},very thick,no markers] table[x=sigma_22,y=sigma_12] {chapter5/pgfFigures/pgf_slowWavesPlaneStress/CPslowStressPlane_frame1_Stress1.pgf} node[above right,black] {$\textbf{2}$};
\addplot[arrows along my path,black,thick] table[x=sigma_22,y=sigma_12] {chapter5/pgfFigures/pgf_slowWavesPlaneStress/CPslowStressPlane_frame2_Stress1.pgf};
\addplot[mesh,point meta = \thisrow{p},very thick,no markers] table[x=sigma_22,y=sigma_12] {chapter5/pgfFigures/pgf_slowWavesPlaneStress/CPslowStressPlane_frame2_Stress1.pgf} node[above right,black] {$\textbf{3}$};
\addplot[arrows along my path,black,thick] table[x=sigma_22,y=sigma_12] {chapter5/pgfFigures/pgf_slowWavesPlaneStress/CPslowStressPlane_frame3_Stress1.pgf};
\addplot[mesh,point meta = \thisrow{p},very thick,no markers] table[x=sigma_22,y=sigma_12] {chapter5/pgfFigures/pgf_slowWavesPlaneStress/CPslowStressPlane_frame3_Stress1.pgf} node[above right,black] {$\textbf{4}$};
\end{groupplot}
\end{tikzpicture}
%%% Local Variables:
%%% mode: latex
%%% TeX-master: "../../mainManuscript"
%%% End:
}
  \subcaptionbox{Deviatoric plane}{\begin{tikzpicture}[scale=0.9]
\begin{axis}[width=.75\textwidth,view={135}{35.2643},xlabel=$s_1 $,ylabel=$s_2 $,zlabel=$s_3$,xmin=-1.e8,xmax=1.e8,ymin=-1.e8,ymax=1.e8,axis equal,axis lines=center,axis on top,ztick=\empty,legend style={at={(.225,.59)}}]
\addplot3+[Red,mark=star,mark repeat=20,mark size=3pt,very thick] file {chapter5/pgfFigures/pgf_slowWavesPlaneStress/CPslowDevPlane_frame0_Stress1.pgf};
\addlegendentry{loading path 1}
\addplot3+[Blue,mark=asterisk,mark repeat=20,mark size=3pt,very thick] file {chapter5/pgfFigures/pgf_slowWavesPlaneStress/CPslowDevPlane_frame1_Stress1.pgf};
\addlegendentry{loading path 2}
\addplot3+[Orange,mark=+,mark repeat=20,mark size=3pt,very thick] file {chapter5/pgfFigures/pgf_slowWavesPlaneStress/CPslowDevPlane_frame2_Stress1.pgf};
\addlegendentry{loading path 3}
\addplot3+[Purple,mark=x,mark repeat=20,mark size=3pt,very thick] file {chapter5/pgfFigures/pgf_slowWavesPlaneStress/CPslowDevPlane_frame3_Stress1.pgf};
\addlegendentry{loading path 4}
\addplot3+[gray,dashed,thin,no markers] file {chapter5/pgfFigures/pgf_slowWavesPlaneStress/CPCylindreDevPlane.pgf};\addlegendentry{initial yield surface}
\end{axis}
\end{tikzpicture}
%%% Local Variables:
%%% mode: latex
%%% TeX-master: "../../mainManuscript"
%%% End:
}
  \caption{loading paths through slow simple waves. Stresses in Pa (if required)}
  \label{fig:slow_path_plane_strains}
\end{figure}

\begin{figure}[h!]
  \centering
  \subcaptionbox{Slice ($\sigma_{11},\sigma_{12}$) plane}{\begin{tikzpicture}[scale=0.9]
\begin{groupplot}[group style={group size=2 by 1,
ylabels at=edge left, yticklabels at=edge left,horizontal sep=3.ex,
xticklabels at=edge bottom,xlabels at=edge bottom},
ymajorgrids=true,xmajorgrids=true,ylabel=$\sigma_{12} \: (Pa)$,
axis on top,scale only axis,width=0.4\linewidth,ymin=0,ymax=126473070.316
, every x tick scale label/.style={at={(xticklabel* cs:1.05,0.75cm)},anchor=near yticklabel}
,colormap name =viridis]
\nextgroupplot[xlabel=$\sigma_{11} \: (Pa)$]
%\addplot[arrows along my path,black,thick] table[x=sigma_11,y=sigma_12] {chapter5/pgfFigures/pgf_slowWavesPlaneStress/CPslowStressPlane_frame0_Stress2.pgf};
\addplot[mesh,point meta = \thisrow{p},very thick,no markers] table[x=sigma_11,y=sigma_12] {chapter5/pgfFigures/pgf_slowWavesPlaneStress/CPslowStressPlane_frame0_Stress2.pgf} node[above,black] {$\textbf{1}$};
%\addplot[arrows along my path,black,thick] table[x=sigma_11,y=sigma_12] {chapter5/pgfFigures/pgf_slowWavesPlaneStress/CPslowStressPlane_frame1_Stress2.pgf};
\addplot[mesh,point meta = \thisrow{p},very thick,no markers] table[x=sigma_11,y=sigma_12] {chapter5/pgfFigures/pgf_slowWavesPlaneStress/CPslowStressPlane_frame1_Stress2.pgf} node[above,black] {$\textbf{2}$};
%\addplot[arrows along my path,black,thick] table[x=sigma_11,y=sigma_12] {chapter5/pgfFigures/pgf_slowWavesPlaneStress/CPslowStressPlane_frame2_Stress2.pgf};
\addplot[mesh,point meta = \thisrow{p},very thick,no markers] table[x=sigma_11,y=sigma_12] {chapter5/pgfFigures/pgf_slowWavesPlaneStress/CPslowStressPlane_frame2_Stress2.pgf} node[above,black] {$\textbf{3}$};
%\addplot[arrows along my path,black,thick] table[x=sigma_11,y=sigma_12] {chapter5/pgfFigures/pgf_slowWavesPlaneStress/CPslowStressPlane_frame3_Stress2.pgf};
\addplot[mesh,point meta = \thisrow{p},very thick,no markers] table[x=sigma_11,y=sigma_12] {chapter5/pgfFigures/pgf_slowWavesPlaneStress/CPslowStressPlane_frame3_Stress2.pgf} node[above,black] {$\textbf{4}$};
\addplot[gray,dashed,thin] table[x=sigma_11,y=sigma_12] {chapter5/pgfFigures/pgf_slowWavesPlaneStress/CPslow_yield0_s11s12_Stress2.pgf};

\nextgroupplot[colorbar,colorbar style={title= {$ c_s \: (m/s)$},every y tick scale label/.style={at={(2.,-.1125)}} },xlabel=$\sigma_{22} \: (Pa)$]
\addplot[arrows along my path,black!70,thick] table[x=sigma_22,y=sigma_12] {chapter5/pgfFigures/pgf_slowWavesPlaneStress/CPslowStressPlane_frame0_Stress2.pgf};\addplot[mesh,point meta = \thisrow{p},very thick,no markers] table[x=sigma_22,y=sigma_12] {chapter5/pgfFigures/pgf_slowWavesPlaneStress/CPslowStressPlane_frame0_Stress2.pgf};
\addplot[arrows along my path,black!70,thick] table[x=sigma_22,y=sigma_12] {chapter5/pgfFigures/pgf_slowWavesPlaneStress/CPslowStressPlane_frame1_Stress2.pgf};\addplot[mesh,point meta = \thisrow{p},very thick,no markers] table[x=sigma_22,y=sigma_12] {chapter5/pgfFigures/pgf_slowWavesPlaneStress/CPslowStressPlane_frame1_Stress2.pgf} ;
\addplot[arrows along my path,black!70,thick] table[x=sigma_22,y=sigma_12] {chapter5/pgfFigures/pgf_slowWavesPlaneStress/CPslowStressPlane_frame2_Stress2.pgf};\addplot[mesh,point meta = \thisrow{p},very thick,no markers] table[x=sigma_22,y=sigma_12] {chapter5/pgfFigures/pgf_slowWavesPlaneStress/CPslowStressPlane_frame2_Stress2.pgf} ;
\addplot[arrows along my path,black!70,thick] table[x=sigma_22,y=sigma_12] {chapter5/pgfFigures/pgf_slowWavesPlaneStress/CPslowStressPlane_frame3_Stress2.pgf};\addplot[mesh,point meta = \thisrow{p},very thick,no markers] table[x=sigma_22,y=sigma_12] {chapter5/pgfFigures/pgf_slowWavesPlaneStress/CPslowStressPlane_frame3_Stress2.pgf} ;
\end{groupplot}
\end{tikzpicture}
%%% Local Variables:
%%% mode: latex
%%% TeX-master: "../../mainManuscript"
%%% End:
}
  \subcaptionbox{Deviatoric plane}{\tikzset{cross/.style={cross out, draw=black, minimum size=2*(#1-\pgflinewidth), inner sep=0pt, outer sep=0pt},cross/.default={2.5pt}}
\begin{tikzpicture}[scale=0.9]
\begin{axis}[width=.75\textwidth,view={135}{35.2643},xlabel=$s_1 $,ylabel=$s_2 $,zlabel=$s_3$,xmin=-1.e8,xmax=1.e8,ymin=-1.e8,ymax=1.e8,axis equal,axis lines=center,axis on top,xtick=\empty,ytick=\empty,ztick=\empty,every axis y label/.style={at={(rel axis cs:0.,.5,-0.65)}, anchor=west}, every axis x label/.style={at={(rel axis cs:0.5,.,-0.65)}, anchor=east}, every axis z label/.style={at={(rel axis cs:0.,.0,.18)}, anchor=north},legend style={at={(.225,.59)}}]
\node[below] at (1.1e8,0.,0.) {$\sigma^y$};
\node[above] at (-1.1e8,0.,0.) {$-\sigma^y$};
\draw (1.e8,0.,0.) node[cross,rotate=10] {};
\draw (-1.e8,0.,0.) node[cross,rotate=10] {};
\node[white]  at (0,0.,1.42e8) {};
\addplot3+[Red,mark=star,mark repeat=20,mark size=3pt,very thick] file {chapter5/pgfFigures/pgf_slowWavesPlaneStress/CPslowDevPlane_frame0_Stress2.pgf};
\addlegendentry{loading path 1}
\addplot3+[Blue,mark=asterisk,mark repeat=20,mark size=3pt,very thick] file {chapter5/pgfFigures/pgf_slowWavesPlaneStress/CPslowDevPlane_frame1_Stress2.pgf};
\addlegendentry{loading path 2}
\addplot3+[Orange,mark=+,mark repeat=20,mark size=3pt,very thick] file {chapter5/pgfFigures/pgf_slowWavesPlaneStress/CPslowDevPlane_frame2_Stress2.pgf};
\addlegendentry{loading path 3}
\addplot3+[Purple,mark=x,mark repeat=20,mark size=3pt,very thick] file {chapter5/pgfFigures/pgf_slowWavesPlaneStress/CPslowDevPlane_frame3_Stress2.pgf};
\addlegendentry{loading path 4}
\addplot3+[gray,dashed,thin,no markers] file {chapter5/pgfFigures/pgf_slowWavesPlaneStress/CPCylindreDevPlane.pgf};\addlegendentry{initial yield surface}
\newcommand\radius{0.82e8}
\addplot3[dotted,thick] coordinates {(0.75*\radius,-0.75*\radius,0.) (-0.75*\radius,0.75*\radius,0.)};
\addplot3[dotted,thick] coordinates {(0.,-0.75*\radius,0.75*\radius) (0.,0.75*\radius,-0.75*\radius)};
\addplot3[dotted,thick] coordinates {(-0.75*\radius,0.,0.75*\radius) (0.75*\radius,0.,-0.75*\radius)};
\end{axis}
\end{tikzpicture}
%%% Local Variables:
%%% mode: latex
%%% TeX-master: "../../mainManuscript"
%%% End:
}
  \caption{loading paths through slow simple waves. Stresses in Pa (if required)}
  \label{fig:slow_path_plane_strains}
\end{figure}


\begin{figure}[h!]
  \centering
  \subcaptionbox{Slice ($\sigma_{11},\sigma_{12}$) plane}{\begin{tikzpicture}[scale=0.9]
\begin{groupplot}[group style={group size=2 by 1,
ylabels at=edge left, yticklabels at=edge left,horizontal sep=3.ex,
xticklabels at=edge bottom,xlabels at=edge bottom},
ymajorgrids=true,xmajorgrids=true,ylabel=$\sigma_{12} \: (Pa)$,
axis on top,scale only axis,width=0.4\linewidth,ymin=0,ymax=109528891.788
, every x tick scale label/.style={at={(xticklabel* cs:1.05,0.75cm)},anchor=near yticklabel}
,colormap name =viridis]
\nextgroupplot[xlabel=$\sigma_{11} \: (Pa)$]
%\addplot[arrows along my path,black,thick] table[x=sigma_11,y=sigma_12] {chapter5/pgfFigures/pgf_slowWavesPlaneStress/CPslowStressPlane_frame0_Stress3.pgf};
\addplot[mesh,point meta = \thisrow{p},very thick,no markers] table[x=sigma_11,y=sigma_12] {chapter5/pgfFigures/pgf_slowWavesPlaneStress/CPslowStressPlane_frame0_Stress3.pgf} node[above,black] {$\textbf{1}$};
%\addplot[arrows along my path,black,thick] table[x=sigma_11,y=sigma_12] {chapter5/pgfFigures/pgf_slowWavesPlaneStress/CPslowStressPlane_frame1_Stress3.pgf};
\addplot[mesh,point meta = \thisrow{p},very thick,no markers] table[x=sigma_11,y=sigma_12] {chapter5/pgfFigures/pgf_slowWavesPlaneStress/CPslowStressPlane_frame1_Stress3.pgf} node[above,black] {$\textbf{2}$};
%\addplot[arrows along my path,black,thick] table[x=sigma_11,y=sigma_12] {chapter5/pgfFigures/pgf_slowWavesPlaneStress/CPslowStressPlane_frame2_Stress3.pgf};
\addplot[mesh,point meta = \thisrow{p},very thick,no markers] table[x=sigma_11,y=sigma_12] {chapter5/pgfFigures/pgf_slowWavesPlaneStress/CPslowStressPlane_frame2_Stress3.pgf} node[above,black] {$\textbf{3}$};
%\addplot[arrows along my path,black,thick] table[x=sigma_11,y=sigma_12] {chapter5/pgfFigures/pgf_slowWavesPlaneStress/CPslowStressPlane_frame3_Stress3.pgf};
\addplot[mesh,point meta = \thisrow{p},very thick,no markers] table[x=sigma_11,y=sigma_12] {chapter5/pgfFigures/pgf_slowWavesPlaneStress/CPslowStressPlane_frame3_Stress3.pgf} node[above,black] {$\textbf{4}$};
\addplot[gray,dashed,thin] table[x=sigma_11,y=sigma_12] {chapter5/pgfFigures/pgf_slowWavesPlaneStress/CPslow_yield0_s11s12_Stress3.pgf};

\nextgroupplot[colorbar,colorbar style={title= {$ c_s \: (m/s)$},every y tick scale label/.style={at={(2.,-.1125)}} },xlabel=$\sigma_{22} \: (Pa)$]
\addplot[arrows along my path,black!70,thick] table[x=sigma_22,y=sigma_12] {chapter5/pgfFigures/pgf_slowWavesPlaneStress/CPslowStressPlane_frame0_Stress3.pgf};\addplot[mesh,point meta = \thisrow{p},very thick,no markers] table[x=sigma_22,y=sigma_12] {chapter5/pgfFigures/pgf_slowWavesPlaneStress/CPslowStressPlane_frame0_Stress3.pgf} node[above,black] {$\textbf{1}$};
\addplot[arrows along my path,black!70,thick] table[x=sigma_22,y=sigma_12] {chapter5/pgfFigures/pgf_slowWavesPlaneStress/CPslowStressPlane_frame1_Stress3.pgf};\addplot[mesh,point meta = \thisrow{p},very thick,no markers] table[x=sigma_22,y=sigma_12] {chapter5/pgfFigures/pgf_slowWavesPlaneStress/CPslowStressPlane_frame1_Stress3.pgf} node[above,black] {$\textbf{2}$};
\addplot[arrows along my path,black!70,thick] table[x=sigma_22,y=sigma_12] {chapter5/pgfFigures/pgf_slowWavesPlaneStress/CPslowStressPlane_frame2_Stress3.pgf};\addplot[mesh,point meta = \thisrow{p},very thick,no markers] table[x=sigma_22,y=sigma_12] {chapter5/pgfFigures/pgf_slowWavesPlaneStress/CPslowStressPlane_frame2_Stress3.pgf} node[above,black] {$\textbf{3}$};
\addplot[arrows along my path,black!70,thick] table[x=sigma_22,y=sigma_12] {chapter5/pgfFigures/pgf_slowWavesPlaneStress/CPslowStressPlane_frame3_Stress3.pgf};\addplot[mesh,point meta = \thisrow{p},very thick,no markers] table[x=sigma_22,y=sigma_12] {chapter5/pgfFigures/pgf_slowWavesPlaneStress/CPslowStressPlane_frame3_Stress3.pgf} node[above,black] {$\textbf{4}$};
\end{groupplot}
\end{tikzpicture}
%%% Local Variables:
%%% mode: latex
%%% TeX-master: "../../mainManuscript"
%%% End:
}
  \subcaptionbox{Deviatoric plane}{\tikzset{cross/.style={cross out, draw=black, minimum size=2*(#1-\pgflinewidth), inner sep=0pt, outer sep=0pt},cross/.default={2.5pt}}
\begin{tikzpicture}[scale=0.9]
\begin{axis}[width=.75\textwidth,view={135}{35.2643},xlabel=$s_1 $,ylabel=$s_2 $,zlabel=$s_3$,xmin=-1.e8,xmax=1.e8,ymin=-1.e8,ymax=1.e8,axis equal,axis lines=center,axis on top,xtick=\empty,ytick=\empty,ztick=\empty,every axis y label/.style={at={(rel axis cs:0.,.5,-0.65)}, anchor=west}, every axis x label/.style={at={(rel axis cs:0.5,.,-0.65)}, anchor=east}, every axis z label/.style={at={(rel axis cs:0.,.0,.18)}, anchor=north},legend style={at={(.225,.59)}}]
\node[below] at (1.1e8,0.,0.) {$\sigma^y$};
\node[above] at (-1.1e8,0.,0.) {$-\sigma^y$};
\draw (1.e8,0.,0.) node[cross,rotate=10] {};
\draw (-1.e8,0.,0.) node[cross,rotate=10] {};
\node[white]  at (0,0.,1.42e8) {};
\addplot3+[Red,mark=star,mark repeat=20,mark size=3pt,very thick] file {chapter5/pgfFigures/pgf_slowWavesPlaneStress/CPslowDevPlane_frame0_Stress3.pgf};
\addlegendentry{loading path 1}
\addplot3+[Blue,mark=asterisk,mark repeat=20,mark size=3pt,very thick] file {chapter5/pgfFigures/pgf_slowWavesPlaneStress/CPslowDevPlane_frame1_Stress3.pgf};
\addlegendentry{loading path 2}
\addplot3+[Orange,mark=+,mark repeat=20,mark size=3pt,very thick] file {chapter5/pgfFigures/pgf_slowWavesPlaneStress/CPslowDevPlane_frame2_Stress3.pgf};
\addlegendentry{loading path 3}
\addplot3+[Purple,mark=x,mark repeat=20,mark size=3pt,very thick] file {chapter5/pgfFigures/pgf_slowWavesPlaneStress/CPslowDevPlane_frame3_Stress3.pgf};
\addlegendentry{loading path 4}
\addplot3+[gray,dashed,thin,no markers] file {chapter5/pgfFigures/pgf_slowWavesPlaneStress/CPCylindreDevPlane.pgf};\addlegendentry{initial yield surface}
\newcommand\radius{1.*0.82e8}
\addplot3[dotted,thick] coordinates {(0.75*\radius,-0.75*\radius,0.) (-0.75*\radius,0.75*\radius,0.)};
\addplot3[dotted,thick] coordinates {(0.,-0.75*\radius,0.75*\radius) (0.,0.75*\radius,-0.75*\radius)};
\addplot3[dotted,thick] coordinates {(-0.75*\radius,0.,0.75*\radius) (0.75*\radius,0.,-0.75*\radius)};
\end{axis}
\end{tikzpicture}
%%% Local Variables:
%%% mode: latex
%%% TeX-master: "../../mainManuscript"
%%% End:
}
  \caption{loading paths through slow simple waves. Stresses in Pa (if required)}
  \label{fig:slow_path_plane_strains}
\end{figure}




%%%% REMARQUES A LA VOLEE
% It is shown in \cite{Ting73} that the plastic celerities only depends on $\tens{\sigma}/\norm{\tens{\sigma}}$ so that they are constant along in ray of the stress space $(\sigma_{11}, \sigma_{22}, \sigma_{12})$. Thus, look at the loading path along integral curves and see the evolution of celerities.

% For now, it is assumed that the characteristic speeds satisfy: $c_1 \geq c_f \geq c_2 \geq c_s \geq 0$ and that the plastic celerities are monotonically decreasing functions of the stress. The latter assumption is in particular satisfied in the quarter-space $(\sigma_{11}\geq 0, \sigma_{22}\geq 0, \sigma_{12}\geq 0)$ for in that case, every elements of the acoustic tensor $\tens{A}^{ep}$ decrease with increasing stress (pas assez général. Vrai en écrouissage isotrope. Dépend de la normale. Vrai pour un état de contrainte donnée mais dépend du trajet de chargement. Peut-être qu'il faut ommettre ça pour le moment).

%This is in particular true if we restrict our attention to the quarter-space $(\sigma_{11} \geq 0, \sigma_{22} \geq 0 , \sigma_{12}\geq 0)$ in which every components of the tensor  
%Assuming that no shock occurs, the integration of ODEs \eqref{eq:ch5_ODEs} yields simple wave solutions of the problem.
%This assumption seems to be valid with the convex flux function used in equation \eqref{eq:ch5_conservative} that leads to monotonically decreasing wave speeds with respect to the stress tensor. Furthermore, the medium is homogeneous 
%% Ne pas regarder genuinely non-linear car ça n'apporte rien. Ca donne juste une indication sur la variation des vitesses le long des courbes intégrales mais pas en fonction de la contrainte.





% It has been shown in \cite{Ting69} (homogeneous function so that c are constant along radial lines in the stress space)

%This is in paticular true when looking at the normal vectors $\vect{n} = \vect{e}_1$ and $\vect{n} = \vect{e}_2$ that yield an acoustic tensor $A_{ij}^{ep}=A_{ij}^{elas} - \beta m_{pi}m_{jq}n_p n_q\deta_{pq}$.



%%% Local Variables:
%%% mode: latex
%%% TeX-master: "../mainManuscript"
%%% End:
