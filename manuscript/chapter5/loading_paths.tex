% On ne regarde qu'une dimension spatiale en faisant des hypothèse sur les champs alors que nous on se limite à une direction particulière $\vect{n}$.
% En plus, on se limite à l'étude d'ondes simples alors que des chocs peuvent exister (voir Mandell car il semble y etre démontré que les shock n'arrivent que pour $\tau=0$).
% Il y a la question des vitesses charactéristiques plastiques... sont-elles collées aux vitesses élastiques ?
% dependance des vitesses caractéristiques à l'angle entre la direction principale de sigma et la direction de propagation, c'est dit dans la thèse de Clifou en page 90.

\subsection{Properties of the loading paths}
The stress paths followed within slow and fast simple waves are governed by the mathematical properties of the loading functions \eqref{eq:loading_func}.
Before specializing the discussion to plane stress and plane strain cases, some general properties holding regardless of the loading conditions are highlighted.
%The analysis is here carried out for the special case $\vect{n}=\vect{e}_1$, similar results being obtained for the other situation $\vect{n}=\vect{e}_2$.

First, the functions satisfy the orthogonality properties: $\psi^s_1\psi^f_1=-1$ and $\psi^s_2\psi^f_2=-1$.
Indeed, considering the left eigenvectors of the acoustic tensor given in equation \eqref{eq:ch5_eigenvectAcc}, the product $\psi^s_1\psi^f_1$ reads:
\begin{equation*}
  \psi^s_1\psi^f_1 = \frac{l^1_2}{l^1_1}\: \frac{l_2^2}{l^2_1}  
\end{equation*}
Since the eigenvectors of symmetric second-order tensors all satisfy $\vect{l}^1 \cdot \vect{l}^2=0$, it comes out that the above product is equal to $-1$.
Whereas this orthogonality has already been noticed for particular plane strain and plane stress cases \cite{Clifton,Ting68}, \textbf{the generic formulation proposed here allows to show that it is valid for all problems in two space dimensions}. 
% but now obviously appears as true for all problems in two space dimensions.
%However, the eigenvectors of symmetric second-order tensors all satisfy this property in such a way that it is valid for all problems in two space dimensions.
As a result, the study can be restricted to one function in each direction, say $\psi_1^s$ and $\psi_2^s$.

% Indeed, considering the left eigenvectors of the acoustic tensor given in equation \eqref{eq:ch5_eigenvectAcc}, the product $\psi^s_1\psi^f_1$ reads:
% \begin{equation*}
%   \psi^s_1\psi^f_1 = \frac{l^1_2}{l^1_1}\: \frac{l_2^2}{l^2_1} = \frac{(A_{11}-\omega_2)A_{12}}{(A_{22}-\omega_1)A_{12}}
% \end{equation*}
% Introduction of the expressions of eigenvalues $\omega_i$ from equations \eqref{eq:ch5_eigenAcc1} and \eqref{eq:ch5_eigenAcc1} further leads to:
% \begin{equation*}
%   \psi^s_1\psi^f_1 = \frac{A_{11} -A_{22} +\sqrt{(A_{11} -A_{22} )^2 + 4A_{12}^2 }}{A_{22} -A_{11} -\sqrt{(A_{11} -A_{22} )^2 + 4A_{12}^2 }}=-1
% \end{equation*}
% or equivalently, $\vect{l}^1 \cdot \vect{l}^2=0$.
% %as expected by the symmetry of $\tens{A}$.
% This orthogonality has already been noticed for particular plane strain and plane stress cases \cite{Clifton,Ting68}. % but now obviously appears as true for all problems in two space dimensions.
% However, the eigenvectors of symmetric second-order tensors all satisfy this property in such a way that it is valid for all problems in two space dimensions.
% As a result, the study can be restricted to one function in each direction, say $\psi_1^s$ and $\psi_2^s$.

Second, if the function $\psi_1^s$ vanishes at some point of the stress space, the projection in the $(\sigma_{11},\sigma_{12})$ plane of the loading path followed inside a slow wave is vertical according to the ODE \eqref{eq:sigSlow_n=e1} (\textit{i.e} $d\sigma_{11}=0$).
Conversely, if $\psi_1^s\rightarrow \infty$, the loading path is horizontal in the $(\sigma_{11},\sigma_{12})$ plane (\textit{i.e} $d\sigma_{12}=0$).
%Looking for vanishing $\psi^f_1$ or $1/\psi^f_1$ amounts to finding roots of the components of $\vect{l}^2$:
These situations respectively correspond to:
\begin{subequations}
  \begin{alignat}{1}
    \label{eq:first_root}
    \psi_1^s = 0   & \Leftrightarrow A_{12} =0  \\
    \label{eq:second_root}
    \psi_1^s \rightarrow \infty & \Leftrightarrow A_{22} -\omega_1 =0
  \end{alignat}
\end{subequations}
In particular, if $A_{12}=0$ equation \eqref{eq:second_root} reads:
\begin{equation}
  A_{22} -\omega_1 = \frac{1}{2}\(A_{22} -A_{11} -\sqrt{(A_{11} -A_{22} )^2 + 4A_{12}^2 }\) = -\left\langle A _{11}-A _{22}  \right\rangle
\end{equation}
where $\left\langle \bullet \right\rangle$ denotes the positive part operator.
Hence, if $A_{12} =0$ and $A_{11} \neq A_{22} $, one has $\psi^s_1 =0$ and hence $\psi^f_1 \rightarrow -\infty $.
If moreover $A_{11}  = A_{22} $, both components of the eigenvectors vanish and the functions $\psi^s_1$ and $\psi^f_1$ are undetermined.
At last, it follows from equation \eqref{eq:diff_celerities} that the simultaneous satisfaction of conditions \eqref{eq:first_root} and \eqref{eq:second_root} leads to characteristic speeds of simple waves that are identical. Hence, the situation $c_f=c_s$ corresponds to a loss of hyperbolicity of the system.
% Hence, if $A_{12} =0$ and $A_{11} \neq A_{22} $, one has $\psi^s_1 =0$ and $\psi^f_1 \rightarrow \infty $, so that the stress path in the ($\sigma_{11},\sigma_{12}$) plane are vertical (\textit{resp. horizontal}) through a slow (\textit{resp. fast}) wave. 

Analogously, the function $\psi_2^s$ is such that:
\begin{subequations}
  \begin{alignat}{1}
    \label{eq:first_root_psi2cp}
    \psi_2^s \rightarrow \infty  & \Leftrightarrow A_{12} =0  \\
    \label{eq:second_root_psi2cp}
    \psi_2^s =0 & \Leftrightarrow A_{22} -\omega_1 =0
  \end{alignat}
\end{subequations}
Therefore, if both conditions \eqref{eq:first_root_psi2cp} and \eqref{eq:second_root_psi2cp} are satisfied on $A_{12}$ and $A_{22} -\omega_1$, the system is no longer hyperbolic with characteristic speeds of fast and slow waves that are identical.

According to the ODEs of table \ref{tab:simpleWavesEquations}, the particular values of the loading functions $\psi_i^{s,f}$ through the simple waves propagating in direction $\vect{e}_i$ for $i=\{1,2\}$, provide information about the loading paths in stress space.
First, $\psi^{s,f}_i =0$ leads to $d\sigma_{ii}=0$ (no sum on $i$) so that the longitudinal stress is constant within the simple wave.
Conversely, with loading functions tending to infinity, the stress $\sigma_{12}$ does not vary.
Notice that the coefficients $\alpha_{ij}$ of the left eigenvector of the Jacobian matrix associated to the zero eigenvalue \eqref{eq:ch5_null_left_eigen} also have to be regarded.
Nevertheless, those terms resulting from products of the components of the elastoplastic tangent modulus have complex expressions and are assumed to have non-zero values in the remainder of the manuscript.

The above discussions are now specified to plane strain and plane stress, for which loading conditions leading to $A_{12} =0$ and $A _{11}-A _{22}=0$ are identified.



\subsection{The plane strain case}
The case of plane strain is first considered by using the elastoplastic tangent modulus so that the components of the acoustic tensor for $\vect{n}=\vect{e}_1$ read:
%The elastoplastic tangent modulus under consideration is now that given in equation \eqref{eq:elastoplastic_tangent}, so that the components of the acoustic tensor for $\vect{n}=\vect{e}_1$ read: 
\begin{subequations}
  \begin{alignat}{1}
    \label{eq:DP_A11}
    & A_{11}^{ep}= C_{1111}^{ep} = \lambda + 2\mu -\beta s_{11}^2 \\
    \label{eq:DP_A22}
    & A_{22}^{ep}= C_{2121}^{ep}= \mu -\beta s_{12}^2 \\
    \label{eq:DP_A12}
    & A_{12}^{ep}= C_{1121}^{ep}=-\beta s_{11}s_{12}
  \end{alignat}
\end{subequations}
The associated eigenvalues are then:
\begin{subequations}
  \label{eq:eigen_acc_DP}
  \begin{alignat}{1}
    \label{eq:eigen_acc_DP1}
    & \rho c_s^2 = \frac{1}{2}\( \lambda +3\mu -\beta (s_{11}^2+ s_{12}^2) - \sqrt{(\lambda + \mu -\beta (s_{11}^2-s_{12}^2) )^2 +4(\beta s_{11}s_{12})^2} \) \\
    \label{eq:eigen_acc_DP2}
    & \rho c_f^2 = \frac{1}{2}\( \lambda +3\mu -\beta (s_{11}^2+ s_{12}^2) + \sqrt{(\lambda + \mu -\beta (s_{11}^2-s_{12}^2) )^2 +4(\beta s_{11}s_{12})^2}  \)
  \end{alignat}
\end{subequations}
Subtracting equations \eqref{eq:DP_A11} and \eqref{eq:DP_A22}, one gets: $A_{11}^{ep}-A_{22}^{ep}= \lambda + \mu -\beta \(s_{11}^2-s_{12}^2\)$.
Hence, the equation $A_{11}^{ep}-A_{22}^{ep}=0$ admits a set of solutions in the deviatoric stress space.
On the other hand, we see from equation \eqref{eq:DP_A12} that $A_{12}^{ep}$ vanishes for $s_{12}=0$ or $s_{11}=0$.
% Recall that $A^{ep}_{12}=0$ leads to vertical and horizontal loading paths across slow and fast waves respectively. 
Each solution is studied in more details below.

%% Sign of one of the functions psi... but not used afterwards
% We first study the sign of the functions $\psi^f$ by noticing that $\mu=\rho c_2^2$ so that $A_{22}^{ep}$ may be rewritten to yield:
% \begin{equation*}
%   \psi^f = -\frac{A_{12}^{ep}}{A_{22}-\rho c_f^2}= -\frac{\beta s_{11}s_{12}}{\rho c_f^2-\rho c_2^2 +\beta s_{12}^2 }
% \end{equation*}
% Since the denominator is positive for $c_f \geq c_2$, it comes out that $\sign (\psi^f) = - \sign(s_{12}) \sign(s_{11})$. Moreover, two roots of the loading function $\psi^f$ can be identified.

\paragraph*{Condition $s_{12}=0$:} 
According to equations \eqref{eq:eigen_acc_DP1} and \eqref{eq:eigen_acc_DP2}, the eigenvalues of the acoustic tensor become:
\begin{align*}
  & \rho c_s^2 = \frac{1}{2}\( \lambda +3\mu -\beta s_{11}^2 - \abs{\lambda + \mu -\beta s_{11}^2 } \) \\
  & \rho c_f^2 = \frac{1}{2}\( \lambda +3\mu -\beta s_{11}^2 + \abs{\lambda + \mu -\beta s_{11}^2 } \)
\end{align*}
Two cases are to be considered:
\begin{itemize}
\item[(i)] if $\beta s_{11}^2 < \lambda + \mu$, the expression further reduces to:
  \begin{align*}
    & \rho c_s^2 = \mu \\
    & \rho c_f^2 = \lambda +2\mu -\beta s_{11}^2 
  \end{align*}
  The characteristic speed of slow waves therefore identifies with this of elastic shear waves for plane strain $c_s=c_2=\sqrt{\mu/\rho}$. 
\item[(ii)] if $ \lambda + \mu - \beta s_{11}^2 <0$, the characteristic speeds read: 
  \begin{align*}
    & \rho c_s^2 = \lambda +2\mu -\beta s_{11}^2  \\
    & \rho c_f^2 =  \mu 
  \end{align*}
  Therefore, the celerity of fast waves reduces to that of elastic shear waves.
  Note, however, that the characteristic speed of slow waves remains real if and only if $\lambda +2\mu >\beta s_{11}^2$.
  One then gets the following bounds: $\lambda +2\mu > \beta s_{11}^2 > \lambda +\mu$
\end{itemize}
At last, the equality $\beta s_{11}^2 = \lambda + \mu$ leads to $A_{11}^{ep}-A_{22}^{ep}=0$ and hence, to undetermined loading functions. 
%% Set of admissible values for s11 (depends on s itself)
% It then appears that the values of $s_{11}$ ensuring hyperbolicity of the system are:
% \begin{equation}
%   s_{11} \in ]-\infty,-\sqrt{\frac{\lambda + \mu}{\beta}}[\: \cup\: ]-\sqrt{\frac{\lambda + \mu}{\beta}},\sqrt{\frac{\lambda + \mu}{\beta}}[\: \cup \:]\sqrt{\frac{\lambda + \mu}{\beta}} ,\infty[
% \end{equation}

%% Discussion about the loading path direction
% Recall that $\psi^f_1$ tending to infinity implies that the loading path are horizontal in $(\sigma_{11},\sigma_{12})$ plane and hence, the fast wave has no influence on the shear stress if, and only if, $\sigma_{12}=0$ downstream.
% Conversely, the stress paths through slow simple waves are vertical.
% Moreover, with regard the last row of table \ref{tab:simpleWavesEquations}, $\sigma_{22}$ is also unchanged in that case.
% As a consequence, if the initial state is shear-free the solution no longer contain combined waves, but longitudinal stress and shear stress simple waves.

\paragraph*{Condition $s_{11}=0$:}
Considering the relation \eqref{eq:plane_strain_stress33} between stress components for plane strain, one writes:
\begin{equation*}
  s_{11}= \frac{2}{3}\sigma_{11}-\frac{1}{3}(\sigma_{22}+\nu(\sigma_{11}+\sigma_{22})-E\eps^p_{33})
\end{equation*}
so that $s_{11}=0$ is equivalent to:
\begin{equation}
  \label{eq:plane_strain_s11=0}
  \sigma_{11}=\frac{1+\nu}{2-\nu}\sigma_{22}-E\eps^p_{33}
\end{equation}
In contrast to what has been seen previously, the functions $\psi$ cannot be undetermined in the case $s_{11}=0$ since the equation $A_{11}^{ep}-A_{22}^{ep}=\lambda + \mu + \beta s_{12}^2=0$ does not admit real solutions.
However, the stress state \eqref{eq:plane_strain_s11=0} yields the following characteristic speeds:
\begin{align*}
  & \rho c_s^2 = \mu -\beta s_{12}^2 \\
  & \rho c_f^2 = \lambda +2\mu 
\end{align*}
so that the celerity of fast waves identifies with that of elastic pressure waves under plane strains $c_f=\sqrt{(\lambda + 2\mu)/\rho}=c_1$.

%%%% n=e2
$\newline$
The same analysis can be carried out in the direction $\vect{n}=\vect{e}_2$ by considering the following acoustic tensor components:
\begin{subequations}
  \begin{alignat}{1}
    \label{eq:DP_A11_n2}
    & A_{11}^{ep}= C_{1212}^{ep} = \mu -\beta s_{12}^2 \\
    \label{eq:DP_A22_n2}
    & A_{22}^{ep}= C_{2222}^{ep}= \lambda + 2\mu -\beta s_{22}^2 \\
    \label{eq:DP_A12_n2}
    & A_{12}^{ep}= C_{1222}^{ep}=-\beta s_{22}s_{12}
  \end{alignat}
\end{subequations}
The characteristic speeds are then:
\begin{subequations}
  \label{eq:eigen_acc_DP_n2}
  \begin{alignat}{1}
    \label{eq:eigen_acc_DP1_n2}
    & \rho c_s^2 = \frac{1}{2}\( \lambda +3\mu -\beta (s_{22}^2+ s_{12}^2) - \sqrt{(\lambda +\mu -\beta (s_{22}^2-s_{12}^2) )^2 +4(\beta s_{22}s_{12})^2} \) \\
    \label{eq:eigen_acc_DP2_n2}
    & \rho c_f^2 = \frac{1}{2}\( \lambda +3\mu -\beta (s_{22}^2+ s_{12}^2) + \sqrt{(\lambda + \mu -\beta (s_{22}^2-s_{12}^2) )^2 +4(\beta s_{22}s_{12})^2}  \)
  \end{alignat}
\end{subequations}
With the above expressions, the same remarks than for $\vect{n}=\vect{e}_1$ can obviously be made by replacing $s_{11}$ with $s_{22}$.

Among the above results, the most significant arises from the condition $s_{12}=0$.
Indeed, it has been seen that $A_{12}^{ep}=0$ leads to $\psi_1^s=0$ and $\psi^s_2\rightarrow \infty$ in such a way that the corresponding loading paths in the $(\sigma_{11},\sigma_{12})$ plane are respectively vertical and horizontal.
In virtue of the orthogonality property of the loading functions, the stress path followed in a fast wave propagating in the direction $\vect{e}_1$ is horizontal in the same plane.
Hence, if the path through a fast wave intersects the plane $\sigma_{12}=0$, the shear stress component remains constant afterwards.
The same result holds for the slow wave propagating in the direction $\vect{e}_2$.
The above conclusion are summarized in table \ref{tab:stress_paths_properties}.
\begin{table}[h!]
  \centering
    \begin{tabular}{c|ccN}
    \hline
    & \multicolumn{2}{c}{Stress path in ($\sigma_{11},\sigma_{12}$) plane for $\sigma_{12}=0$} &\\ [8pt]
     & $\vect{n}=\vect{e}_1$ ($i=1$)& $\vect{n}=\vect{e}_2$ ($i=2$) &  \\
    \hline
    \hline
    Slow wave: $\ddroit{\sigma_{ii}}{\sigma_{12}}=\psi^s_i \: (i=\{1,2\} \text{ no sum on i})$ & $\psi_1^s=0 \Rightarrow$ vertical path & $\psi_2^s \rightarrow \infty \Rightarrow$ horizontal path &  \\ [15pt]
    Fast wave: $\ddroit{\sigma_{ii}}{\sigma_{12}}=\psi^f_i \: (i=\{1,2\} \text{ no sum on i})$ & $\psi_1^f\rightarrow \infty \Rightarrow$ horizontal path & $\psi_2^f=0 \Rightarrow$ vertical path & \\ [15pt]
    \hline
  \end{tabular}

%%% Local Variables:
%%% mode: latex
%%% TeX-master: "../../mainManuscript"
%%% End:

  \caption{Loading paths in projection into ($\sigma_{11},\sigma_{12}$) plane followed across slow and fast simple waves, under the condition $\sigma_{12}=0$ assuming that $A_{11}^{ep}-A_{22}^{ep}\neq 0$.}
  \label{tab:stress_paths_properties}
\end{table}
\subsection{The plane stress case}
The elastoplastic tangent modulus under consideration is now that given in equation \eqref{eq:CP_constitutive}.
Let's first consider $\psi_1^s$ related to the vector $\vect{n}=\vect{e}_1$.
Thus:
\begin{subequations}
  \label{eq:CP_Acoustic}
  \begin{alignat}{1}
    \label{eq:CP_A11}
    & \widetilde{A}_{11}^{ep}= C^{ep}_{1111} - \frac{(C^{ep}_{1133})^2}{C^{ep}_{3333}} = \lambda + 2\mu -\beta s_{11}^2 -\frac{\(\lambda -\beta s_{11}s_{33}\)^2}{\lambda + 2\mu - \beta s_{33}^2} \\
    \label{eq:CP_A22}
    & \widetilde{A}_{22}^{ep}= C^{ep}_{2121} - \frac{(C^{ep}_{2133})^2}{C^{ep}_{3333}}= \mu - \beta s_{12}^2 -\frac{\(\beta s_{12}s_{33}\)^2}{\lambda + 2\mu - \beta s_{33}^2} \\
    \label{eq:CP_A12}
    & \widetilde{A}_{12}^{ep} = C^{ep}_{1121} - \frac{C^{ep}_{1133}C^{ep}_{1233}}{C^{ep}_{3333}} =\beta s_{12} \frac{\lambda s_{33} - (\lambda + 2\mu)s_{11} }{\lambda + 2\mu - \beta s_{33}^2} 
  \end{alignat}
\end{subequations}
In order to ensure the hyperbolicity of the system, the component of the acoustic tensor also have to be defined, that is $C^{ep}_{3333}> 0$. This condition leads to:
\begin{equation*}
  \lambda + 2\mu - \beta s_{33}^2 > 0 \quad \Leftrightarrow \quad s_{33}^2 < \frac{\lambda + 2\mu}{\beta}
\end{equation*}
Second, from equation \eqref{eq:CP_A12}, $\widetilde{A}_{12}^{ep}$ admits two roots in terms of the components of the deviatoric stress tensor, namely: 
\begin{equation}
  s_{12}=0 \quad ; \quad s_{11}= \frac{\lambda}{\lambda+2\mu}s_{33}
\end{equation}
In terms of the components of Cauchy stress tensor, those conditions read:
% \begin{align}
%   & \frac{2}{3}\sigma_{11}-\frac{1}{3}\sigma_{22} = -\frac{\lambda}{3\lambda+6\mu}(\sigma_{11}+\sigma_{22}) \\
%   & 2\sigma_{11}-\sigma_{22} = -\frac{\lambda}{\lambda+2\mu}(\sigma_{11}+\sigma_{22}) \\
%   & \sigma_{11}(2 +\frac{\lambda}{\lambda+2\mu})=\sigma_{22}(1-\frac{\lambda}{\lambda+2\mu})\\
%   & \sigma_{11}\frac{3\lambda+4\mu}{\lambda+2\mu}=\sigma_{22}\frac{2\mu}{\lambda+2\mu}\\
%   & \sigma_{11}=\sigma_{22}\frac{2\mu}{3\lambda+4\mu}
% \end{align}
\begin{equation}
  \label{eq:CP_roots}
  \sigma_{12}=0 \quad ; \quad \sigma_{11}=\frac{2\mu}{3\lambda+4\mu}\sigma_{22}
  % \sigma_{12}=0 \quad ; \quad \sigma_{11}=\frac{1-2\nu}{2-\nu} \sigma_{22}
\end{equation}
% Hence, the loading path through a slow simple wave is vertical, that is $\psi^s_1 = 0$, for stress values satisfying \eqref{eq:CP_roots}, providing that the $\widetilde{A}_{11}^{ep}$ and $\widetilde{A}_{22}^{ep}$ are not equal.
% Conversely, such stress states yield horizontal path through a fast wave.

%% Attempt to show that if s12=0, cs=c2 but depends on the sign of A11-A22
% \begin{align}
%   &\omega_2=\frac{1}{2}\(A_{11}+A_{22} - \abs{A_{11}-A_{22}}\)\\
%   &\omega_2=\frac{1}{2}\(A_{11}+A_{22} - A_{11}+A_{22}\) \quad ;\quad \omega_2=\frac{1}{2}\(A_{11}+A_{22} + A_{11}-A_{22}\) \\
%   &\omega_2=A_{22} \quad ;\quad \omega_2=A_{11}
% \end{align}


% \paragraph*{Case $s_{12}=0$ :}
% \begin{align}
%   & \rho c_s^2 =\frac{1}{2}\(\widetilde{A}_{11}^{ep}+\widetilde{A}_{22}^{ep} - \abs{\widetilde{A}_{11}^{ep}-\widetilde{A}_{22}^{ep}}\) \\
%   & \rho c_f^2 =\frac{1}{2}\(\widetilde{A}_{11}^{ep}+\widetilde{A}_{22}^{ep} + \abs{\widetilde{A}_{11}^{ep}-\widetilde{A}_{22}^{ep}}\)
% \end{align}

If on the other hand the vector $\vect{n}=\vect{e}_2$ is considered, the acoustic tensors components read:
\begin{subequations}
  \begin{alignat}{1}
    \label{eq:CP_A11_n=e2}
    & \widetilde{A}_{11}^{ep}= C^{ep}_{1212} - \frac{(C^{ep}_{1233})^2}{C^{ep}_{3333}} = \mu -\beta s_{12}^2 -\frac{\(\lambda -\beta s_{12}s_{33}\)^2}{\lambda + 2\mu - \beta s_{33}^2} \\
    \label{eq:CP_A22_n=e2}
    & \widetilde{A}_{22}^{ep}= C^{ep}_{2222} - \frac{(C^{ep}_{2233})^2}{C^{ep}_{3333}}= \lambda +2\mu - \beta s_{22}^2 -\frac{\(\beta s_{22}s_{33}\)^2}{\lambda + 2\mu - \beta s_{33}^2} \\
    \label{eq:CP_A12_n=e2}
    % \widetilde{A}_{12}^{ep}    = -\beta s_{12}s_{22} - \frac{(-\beta s_{12}s_{33})(\lambda - \beta s_{22}s_{33})}{\lambda + 2\mu - \beta s_{33}^2}
    % \widetilde{A}_{12}^{ep}    = \beta s_{12}\( s_{33}\frac{\lambda - \beta s_{22}s_{33}}{\lambda + 2\mu - \beta s_{33}^2}-s_{22}\)
    % \widetilde{A}_{12}^{ep}    = \beta s_{12}\( \frac{\lambda s_{33}  -s_{22}(\lambda + 2\mu ) }{\lambda + 2\mu - \beta s_{33}^2}\)
    & \widetilde{A}_{12}^{ep} = C^{ep}_{1222} - \frac{C^{ep}_{1233}C^{ep}_{2233}}{C^{ep}_{3333}} =\beta s_{12} \frac{\lambda s_{33} - (\lambda + 2\mu)s_{22} }{\lambda + 2\mu - \beta s_{33}^2}
  \end{alignat}
\end{subequations}
Which are similar to these obtained for $\vect{n}=\vect{e}_1$ \eqref{eq:CP_Acoustic} with $s_{22}$ instead of $s_{11}$.
It comes out that $\widetilde{A}_{12}^{ep}$ admits to roots:
\begin{equation}
  \label{eq:CP_roots_n=e2}
  \sigma_{12}=0 \quad ; \quad \sigma_{22}=\frac{2\mu}{3\lambda+4\mu}\sigma_{11}
\end{equation}

The complexity introduced by the plane stress tangent modulus prevents the finding of other singular configurations for the hyperbolic system. 
In particular, it is difficult to deal with the equation $\widetilde{A}^{ep}_{11}=\widetilde{A}^{ep}_{22}$ due to the expressions given in equations \eqref{eq:CP_A11} and \eqref{eq:CP_A22}.
Nevertheless, since the stress state $s_{12}=0$ also constitutes a singular point for plane stresses, the same remarks than these made for plane strains on the loading paths hold.
Namely, $\sigma_{12}$ becomes constant if it falls to zero along the loading path followed inside a fast (\textit{resp. slow}) wave propagating in direction $\vect{e}_1$ (\textit{resp. $\vect{e}_2$}) as summarized in table \ref{tab:stress_paths_properties}.
%Namely, if $\sigma_{12}$ falls to zero along the loading path followed inside a fast (\textit{resp. slow}) wave propagating in direction $\vect{e}_1$ (\textit{resp. $\vect{e}_2$}), is restricted to that value.
%As we shall see below, more singular behaviors can be identified for plane strain.





%%% Local Variables:
%%% mode: latex
%%% TeX-master: "../mainManuscript"
%%% End:
