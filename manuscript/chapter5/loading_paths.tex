\subsection{Properties of the integral curves}
%\subsubsection*{Orthogonality of loading paths}
With the left eigenvectors of the acoustic tensor defined in equation \eqref{eq:ch5_eigenvectAcc}, the functions $\psi^s_1$ and $\psi^f_1$ introduced in the previous section satisfy the following orthogonality condition: $\psi^s_1\psi^f_1=-1$. 

\begin{proof}
  We first write the product of the two functions according to equation \eqref{eq:ch5_eigenvectAcc}:
  \begin{equation*}
    \psi^s_1\psi^f_1 = \frac{l^1_2}{l^1_1}\: \frac{l_2^2}{l^2_1} = \frac{(A_{11}^{ep}-\rho c_s^2)A^{ep}_{12}}{(A_{22}^{ep}-\rho c_f^2)A^{ep}_{12}}
  \end{equation*}
  Introduction of the expressions of $\rho c_f^2 = \omega_1$ and $\rho c_s^2 = \omega_2$ given by equations \eqref{eq:ch5_eigenAcc1} and \eqref{eq:ch5_eigenAcc1} leads to:
  \begin{equation*}
    \psi^s_1\psi^f_1 = \frac{A_{11}^{ep}-A_{22}^{ep}+\sqrt{(A_{11}^{ep}-A_{22}^{ep})^2 + 4A_{12}^{ep}}}{A_{22}^{ep}-A_{11}^{ep}-\sqrt{(A_{11}^{ep}-A_{22}^{ep})^2 + 4A_{12}^{ep}}}=-1
  \end{equation*}
  This ends the proof.
\end{proof}
In other words, $\vect{l}^1 \cdot \vect{l}^2=0$ so that the eigenvectors of the acoustic tensor are orthogonal regardless of the stress state (je pense que c'est normal). This allows us to restrict the study of the functions $\psi_1$ to $\psi^s_1$.
A similar property is identified in \cite{Clifton} for the thin-walled tube problem consisting in a superimposition of a plane wave and a pure shear wave.(il y en a un autre qui en parle et qui précise que c'est valable si les courbes intégrales ont une intersection \cite[p.13]{Ting68})
\begin{remark}
  The orthogonality property of the loading pass is also valid for the vector $\vect{n}=\vect{e}_2$ since the functions involved in the intergal curves are $\psi^s_2=1/\psi^s_1$ and $\psi^s_2=1/\psi^s_1$.
\end{remark}
%\subsubsection*{Degeneracy of the hyperbolic system}

A vanishing function $\psi_1^f$ yields a longitudinal stress $\sigma_{11}$ that is constant along a loading path according to equation \eqref{eq:sigSlow_n=e1}. Conversely, the stress $\sigma_{12}$ is constant if $\psi_1^f\rightarrow \infty$. Such situations make the integration of the integral curve easier, some of them are now identified.

Looking for vanishing $\psi^f_1$ or $1/\psi^f_1$ amounts to find roots of the components of $\vect{l}^2$. Thus:
\begin{align}
  & l_1^2 = 0 \Leftrightarrow A_{12}^{ep}=0  \\
  & l_2^2 = 0 \Leftrightarrow A_{11}^{ep}-\rho c_s^2 =0
\end{align}
Introduction of this condition in the second yields:
\begin{equation}
  A_{11}^{ep}-\rho c_s^2 = \frac{1}{2}\(A^{ep}_{11}-A^{ep}_{22} + \abs{A^{ep}_{11}-A^{ep}_{22}}\) = \left\langle A^{ep}_{11}-A^{ep}_{22} \right\rangle
\end{equation}
where $\left\langle \bullet \right\rangle$ denotes the positive part operator.

Therefore, if $A_{12}^{ep}=0$ and $A_{11}^{ep}\neq A_{22}^{ep}$, the functions satisfy $\psi^f_1 \rightarrow \infty$ and $\psi^s_1 \rightarrow 0$ so that the stress path in the ($\sigma_{11},\sigma_{12}$) plane are (locally) horizontal (\textit{resp. vertical}) through a fast (\textit{resp. slow}) wave. 
On the other hand, if $A_{11}^{ep} = A_{22}^{ep}$, both component of the eigenvectors vanishes and $\psi^f_1$ and $\psi^s_1$ are undetermined. The latter situation moreover leads to characteristic speeds that "collide" according to equations \eqref{eq:ch5_eigenAcc1} and \eqref{eq:ch5_eigenAcc2}, namely:
\begin{align}
  & \rho c_s^2 = \frac{1}{2}\(A^{ep}_{11}+A^{ep}_{22} - \sqrt{(A^{ep}_{11}-A^{ep}_{22})^2+{4A^{ep}_{12}}^2}\) = \frac{A^{ep}_{11}+A^{ep}_{22}}{2}\\
  & \rho c_f^2 = \frac{1}{2}\(A^{ep}_{11}+A^{ep}_{22} + \sqrt{(A^{ep}_{11}-A^{ep}_{22})^2+{4A^{ep}_{12}}^2}\)= \frac{A^{ep}_{11}+A^{ep}_{22}}{2} 
\end{align}
The system thus degenerates in that case since the solution no longer involves two but only one family of simple waves. 

The above discussion is now specified to plane strain and plane stress cases, for which loading conditions that lead to $A_{12}^{ep}=0$ and $A^{ep}_{11}-A^{ep}_{22}=0$ are identified. Without loss of generality, attention is paid to only one direction of space, say $\vect{n}=\vect{e}_1$ and right-going simple waves.

\subsection{The plane strain case}
The expressions of the tangent modulus and the acoustic tensors are recalled here for convenience:
\begin{align}
  & C^{ep}_{ijkl} = \lambda \delta_{ij}\delta_{kl} + \mu \(\delta_{il}\delta_{jk} + \delta_{ik}\delta_{jl}\) - \beta s_{ij}s_{kl} \\
  & A^{ep}_{ij} = \lambda n_i n_j + \mu \(n_k n_k \delta_{ij} +n_i n_j \) - \beta s_{ip}n_p s_{jq}n_q
\end{align}
where $s_{ij}$ are the components of the deviatoric part of Cauchy stress tensor, that is $s_{ij}=\sigma_{ij} - \frac{1}{3}\sigma_{kk}\delta_{ij}$. Hence, for $n_2=0$ the components of the (symmetric) acoustic tensor are:
\begin{align}
  & A_{11}^{ep}= \lambda + 2\mu -\beta s_{11}^2 \\
  & A_{22}^{ep}= \mu -\beta s_{12}^2 \\
  & A_{12}^{ep}=-\beta s_{11}s_{12}
\end{align}
We first study the sign of the functions $\psi^s$ by $\mu=\rho c_2^2$ so that:
\begin{equation*}
  \psi^f = -\frac{A_{12}^{ep}}{A_{22}-\rho c_f^2}=\frac{\beta s_{11}s_{12}}{\rho c_2^2 -\rho c_f^2 -\beta s_{12}^2 } = -\frac{\beta s_{11}s_{12}}{\rho c_f^2-\rho c_2^2 +\beta s_{12}^2 }
\end{equation*}
Since the denominator is positive for $c_f \geq c_2$, it comes out that $\sign (\psi^f) = - \sign(s_{12}) \sign(s_{11})$.

Next, from the expressions of the acoustic tensor components, we see that particular cases leading to $\psi^f_1\rightarrow \infty$ or $\psi^f_1\rightarrow 0$ arise when $s_{11}=0$ and $s_{12}=0$. 
\paragraph*{Condition $s_{12}=0$ :} 
We first look at the shear-free state for which the subtraction of the acoustic tensor diagonal entries reads: $A_{11}^{ep}-A_{22}^{ep}=\lambda + \mu -\beta s_{11}^2$. Hence, the \textbf{undeterminancy} of the functions $\psi^$ arises for $s_{11} = \pm \sqrt{\frac{\lambda+\mu}{\beta}}$ (on the dowstream side), while if $s_{11} \neq \pm \sqrt{\frac{\lambda+\mu}{\beta}}$, $\psi^f_1 \rightarrow \infty$ and $\psi^s_1 \rightarrow 0$.

Recall that $\psi^f_1$ tending to infinity implies that tha loading path are horizontal in $(\sigma_{11},\sigma_{12})$ plane and hence, tha fast wave has no influence on the shear stress if, and only if, $\sigma_{12}=0$ downstream. Conversely, the stress paths through slow simple waves are vertical. Moreover, with regard the last row of table \ref{tab:simpleWavesEquations}, $\sigma_{22}$ is also unchanged in that case. As a consequence, if the initial state is shear-free the solution no longer contain combined waves, but longitudinal stress and shear stress simple waves.

\paragraph*{Condition $s_{11}=0$ :} The functions $psi$ cannot be undetermined in the case $s_{11}=0$ since the equation $A_{11}^{ep}-A_{22}^{ep}=\lambda + \mu + \beta s_{12}^2$ does not admit real solutions.
However, one can try to give a "physical meaning" to the condition $s_{11}=0$ by considering the relation \eqref{eq:plane_strain_stress33} between stress components for plane strain:
\begin{equation*}
  s_{11}= \frac{2}{3}\sigma_{11}-\frac{1}{3}(\sigma_{22}+\nu(\sigma_{11}+\sigma_{22})-E\eps^p_{33})
\end{equation*}
so that $s_{11}=0$ occurs for:
\begin{equation}
  \label{eq:plane_strain_s11=0}
  \sigma_{11}=\frac{1+\nu}{2-\nu}\sigma_{22}-E\eps^p_{33}
\end{equation}


\subsection{The plane stress case}
% Orthogonalité des loading paths \cite{Clifton,Ting68}




%%%% REMARQUES A LA VOLEE
% It is shown in \cite{Ting73} that the plastic celerities only depends on $\tens{\sigma}/\norm{\tens{\sigma}}$ so that they are constant along in ray of the stress space $(\sigma_{11}, \sigma_{22}, \sigma_{12})$. Thus, look at the loading path along integral curves and see the evolution of celerities.

% For now, it is assumed that the characteristic speeds satisfy: $c_1 \geq c_f \geq c_2 \geq c_s \geq 0$ and that the plastic celerities are monotonically decreasing functions of the stress. The latter assumption is in particular satisfied in the quarter-space $(\sigma_{11}\geq 0, \sigma_{22}\geq 0, \sigma_{12}\geq 0)$ for in that case, every elements of the acoustic tensor $\tens{A}^{ep}$ decrease with increasing stress (pas assez général. Vrai en écrouissage isotrope. Dépend de la normale. Vrai pour un état de contrainte donnée mais dépend du trajet de chargement. Peut-être qu'il faut ommettre ça pour le moment).

%This is in particular true if we restrict our attention to the quarter-space $(\sigma_{11} \geq 0, \sigma_{22} \geq 0 , \sigma_{12}\geq 0)$ in which every components of the tensor  
%Assuming that no shock occurs, the integration of ODEs \eqref{eq:ch5_ODEs} yields simple wave solutions of the problem.
%This assumption seems to be valid with the convex flux function used in equation \eqref{eq:ch5_conservative} that leads to monotonically decreasing wave speeds with respect to the stress tensor. Furthermore, the medium is homogeneous 
%% Ne pas regarder genuinely non-linear car ça n'apporte rien. Ca donne juste une indication sur la variation des vitesses le long des courbes intégrales mais pas en fonction de la contrainte.





% It has been shown in \cite{Ting69} (homogeneous function so that c are constant along radial lines in the stress space)

%This is in paticular true when looking at the normal vectors $\vect{n} = \vect{e}_1$ and $\vect{n} = \vect{e}_2$ that yield an acoustic tensor $A_{ij}^{ep}=A_{ij}^{elas} - \beta m_{pi}m_{jq}n_p n_q\deta_{pq}$.



%%% Local Variables:
%%% mode: latex
%%% TeX-master: "../mainManuscript"
%%% End:
