Although some properties of the simple waves have been emphasized in section \ref{sec:stress_paths}, the complexity of the equations prevents the complete characterization of the loading paths followed.
In order to get additional information on the evolution of the stress states, the systems of ODEs gathered in table \ref{tab:simpleWavesEquations} are here numerically integrated for plane stress and plane strain loadings. %, based on the material parameters used in chapter \ref{chap:chap3}.
In particular, the thin-walled tube problem considered by \textsc{Clifton} \cite{Clifton} is first looked at so that the above developments can be validated.
Next, the plane stress and plane strain cases are treated.
%The identified behaviors should provide some simplification assumptions for the building of a procedure that lead to approximate solutions of the problems.
The values of the elastic properties considered here are those used in the previous chapter (see table \ref{tab:material}).
On the other hand, the tensile yield stress $\sigma^y=1 \times 10^{8} \: Pa$ and the hardening modulus $C=1\times10^8 \: Pa$ are set here arbitrarily.
Finally, we restrict here to positive shear stress $\sigma_{12}\geq 0$.
\subsection{Thin-walled tube problem}
\label{sec:num_thin-walled}
%% Hypothèses du problème
Consider the semi-infinite domain in Cartesian coordinate system: $x_1 \times x_2 \times x_3 \in [0,\infty[ \times [-h,h] \times [-e,e]$, being acted upon by a traction vector $\vect{T}^d$ at $x_1=0 $ and free surfaces $x_2=\pm h$ and $x_3=\pm e$.
Only the first two components of $\vect{T}^d$ are non-null so that the stress and strain tensors within the medium are of the form:
\begin{equation}
  \tens{\sigma} = \matrice{\sigma_{11} & \sigma_{12} & 0\\  & 0 & 0\\sym & & 0} \quad ;\quad\tens{\eps} = \matrice{\eps_{11} & \eps_{12} &0 \\  & \eps_{22}&0 \\sym & & \eps_{33}}
\end{equation}
By using the following mapping of coordinates: $(1,2,3) \mapsto (z,\theta,r)$, such a state corresponds also to that 
%of the thin-walled cylinder studied by \textsc{Clifton} \cite{Clifton}.
holding in a hollow cylinder with radius and length much bigger that its thickness, submitted to combined longitudinal and torsional loads.
Hence the name of thin-walled tube problem. 
As a particular plane stress case, the set of ODEs along characteristic derived in section \ref{sec:stress_paths} applies by nevertheless taking into account the vanishing stress component $\sigma_{22}$:
\begin{align*}
  & \dot{\sigma}_{22}=\widetilde{C}^{ep}_{22ij} \dot{\eps}_{ij} =0 \quad i,j=\{1,2\} \\
  \Rightarrow  \quad  &\widetilde{C}^{ep}_{2222} \dot{\eps}_{22} = - \widetilde{C}^{ep}_{22ij}\dot{\eps}_{ij} \quad ij=\{11,12,21\}
\end{align*}
where $\widetilde{\Cbb}^{ep}$ is the plane stress tangent modulus \eqref{eq:CP_constitutive}.
Thus, inverting the above equation and introducing it in the constitutive equation, we are left with the following law:
\begin{equation}
  \label{eq:ch5_TW_tangent}
  \dot{\sigma}_{ij}=\widetilde{C}^{ep}_{ijkl} \dot{\eps}_{kl} - \frac{\widetilde{C}^{ep}_{ij22}\widetilde{C}^{ep}_{22kl}}{\widetilde{C}^{ep}_{2222}}\dot{\eps}_{kl}= \widehat{C}^{ep}_{ijkl} \dot{\eps}_{kl}\qquad ij,kl=\{11,12,21\} 
\end{equation}
%with $\widetilde{\Cbb}^{ep}$ is referred to as the thin-walled tube tangent modulus.
The characteristic analysis of the hyperbolic system based on this tangent modulus also leads to loading paths followed across slow and fast waves, involving however two components of stress rather than three. For the sake of simplicity, the stress components are denoted by $\sigma_{11}=\sigma$ and $\sigma_{12}=\tau$ whereas the velocity components reads $v_1=u$ and $v_2=v$.

Thus, the ODEs governing the evolution of stress components inside the waves of combined-stress read: 
\begin{equation}
  \label{eq:tw_paths}
  d\sigma = \psi^{s,f} d\tau
\end{equation}
where the loading functions $\psi^{s,f}$ depend on the component of the acoustic tensor that corresponds to the tangent modulus \eqref{eq:ch5_TW_tangent}.
Equations \eqref{eq:tw_paths} as well as these of \textsc{Clifton} \cite{Clifton} have been numerically integrated, starting from several arbitrary points lying on the initial yield surface.
Since the loading functions are odd functions of $\sigma$ and $\tau$ \cite{Clifton}, $\tau(\sigma)$ and $\sigma(\tau)$ are even functions and hence, the loading paths exhibit symmetries with respect to $\tau$ and $\sigma$ axes.
Therefore, the study is restricted to the quarter-plane ($\sigma>0,\tau>0$).
\begin{figure}[h!]
  \centering
  \subcaptionbox{Stress path in $(\sigma,\tau)$ plane \label{subfig:tw_fast_stress}}{\begin{tikzpicture}[scale=0.9]
  \begin{axis}[ymajorgrids=true,xmajorgrids=true,ylabel=$\tau \: (Pa)$,xlabel=$\sigma \: (Pa)$,legend style={legend pos=south west}]
    %%
    \addplot[Blue,mark=x,only marks,mark repeat=10,very thick,mark size=3pt] table [x=sigma_11,y=sigma_12] {chapter5/pgfFigures/pgf_thinWalledTubeFastWave/fastStressPlane_Stress.pgf};
    \addlegendentry{Present work}
    \addplot[arrows along my path,Red,thick] table [x=sigma_11,y=sigma_12] {chapter5/pgfFigures/pgf_thinWalledTubeFastWave/TWfastStressPlane_Stress.pgf};
    \addlegendentry{Clifton}
    %% Yield surface
    \addplot[black,dashed] table  [x=sigma_11,y=sigma_12] {chapter5/pgfFigures/pgf_thinWalledTubeSlowWave/TWslow_yield0.pgf};
    \addlegendentry{initial yield surface}
  \end{axis}
\end{tikzpicture}

%%% Local Variables:
%%% mode: latex
%%% TeX-master: "../../mainManuscript"
%%% End:} \qquad
  \subcaptionbox{Stress path in deviatoric plane\label{subfig:tw_fast_dev}}{\tikzset{cross/.style={cross out, draw=black, minimum size=2*(#1-\pgflinewidth), inner sep=0pt, outer sep=0pt},
%default radius will be 1pt. 
cross/.default={2.5pt}}
\begin{tikzpicture}[scale=0.8]
  \begin{axis}[width=.75\textwidth,view={135}{35.2643},xlabel=$s_1 $,
    ylabel=$s_2 $,zlabel=$s_3$,xmin=-1.e8,xmax=1.e8,ymin=-1.e8,ymax=1.e8,axis equal,axis lines=center,axis on top,xtick=\empty,ytick=\empty,ztick=\empty,
    every axis y label/.style={at={(rel axis cs:0.,.5,-0.65)}, anchor=west},
    every axis x label/.style={at={(rel axis cs:0.5,.,-0.65)}, anchor=east},
    every axis z label/.style={at={(rel axis cs:0.,.0,.18)}, anchor=north},
    legend style={at={(1.125,.59)}}
    ]
    \node[below] at (axis cs:1.1e8,0.,0.) {$\sigma^y$};
    \node[above] at (axis cs:-1.1e8,0.,0.) {$-\sigma^y$};
    \draw (axis cs:1.e8,0.,0.) node[cross,rotate=10] {};
    \draw (axis cs:-1.e8,0.,0.) node[cross,rotate=10] {};
    \node[white]  at (axis cs:0,0.,1.42e8) {};
    %%
    \addplot3[black,mark=x,only marks,mark repeat=20,thick,mark size=3pt] file {section7/pgfFigures/pgf_thinWalledTubeFastWave/TWfastDevPlane_Stress.pgf};
    \addplot3[black,arrows along my path,thick] file {section7/pgfFigures/pgf_thinWalledTubeFastWave/fastDevPlane_Stress.pgf};
    \addlegendentry{Clifton}
    \addlegendentry{This work}
    %% Yield surface
    \addplot3[black,dashed] file {section7/pgfFigures/pgf_thinWalledTubeSlowWave/TWCylindreDevPlane.pgf};
    \addlegendentry{Initial yield surface}
    \newcommand\radius{0.82e8}
    \addplot3[dotted,thick] coordinates {(0.75*\radius,-0.75*\radius,0.) (-0.75*\radius,0.75*\radius,0.)};
    \addplot3[dotted,thick] coordinates {(0.,-0.75*\radius,0.75*\radius) (0.,0.75*\radius,-0.75*\radius)};
    \addplot3[dotted,thick] coordinates {(-0.75*\radius,0.,0.75*\radius) (0.75*\radius,0.,-0.75*\radius)};

  \end{axis}
  \begin{scope}[shift={(8.5,0.)}]
    \begin{axis}[width=.75\textwidth,view={135}{35.2643},xlabel=$s_1 $,
    ylabel=$s_2 $,zlabel=$s_3$,xmin=-1.e8,xmax=1.e8,ymin=-1.e8,ymax=1.e8,axis equal,axis lines=center,axis on top,xtick=\empty,ytick=\empty,ztick=\empty,
    every axis y label/.style={at={(rel axis cs:0.,.5,-0.65)}, anchor=west},
    every axis x label/.style={at={(rel axis cs:0.5,.,-0.65)}, anchor=east},
    every axis z label/.style={at={(rel axis cs:0.,.0,.18)}, anchor=north}
    ]
    \node[below] at (axis cs:1.1e8,0.,0.) {$\sigma^y$};
    \node[above] at (axis cs:-1.1e8,0.,0.) {$-\sigma^y$};
    \draw (axis cs:1.e8,0.,0.) node[cross,rotate=10] {};
    \draw (axis cs:-1.e8,0.,0.) node[cross,rotate=10] {};
    \node[white]  at (axis cs:0,0.,1.42e8) {};
    %%
    \addplot3[black,mark=x,only marks,mark repeat=30,thick] file {section7/pgfFigures/pgf_thinWalledTubeSlowWave/TWslowDevPlane_Stress0.pgf};
    \addplot3[black,arrows along my path,thick] file {section7/pgfFigures/pgf_thinWalledTubeSlowWave/slowDevPlane_Stress0.pgf};
    %%
    \addplot3[black,mark=x,only marks,mark repeat=30,thick] file {section7/pgfFigures/pgf_thinWalledTubeSlowWave/TWslowDevPlane_Stress1.pgf};
    \addplot3[black,arrows along my path,thick] file {section7/pgfFigures/pgf_thinWalledTubeSlowWave/slowDevPlane_Stress1.pgf};
    %%
    \addplot3[black,mark=x,only marks,mark repeat=30,thick] file {section7/pgfFigures/pgf_thinWalledTubeSlowWave/TWslowDevPlane_Stress2.pgf};
    \addplot3[black,arrows along my path,thick] file {section7/pgfFigures/pgf_thinWalledTubeSlowWave/slowDevPlane_Stress2.pgf};
    %%
    \addplot3[black,mark=x,only marks,mark repeat=30,thick] file {section7/pgfFigures/pgf_thinWalledTubeSlowWave/TWslowDevPlane_Stress3.pgf};
    \addplot3[black,arrows along my path,thick] file {section7/pgfFigures/pgf_thinWalledTubeSlowWave/slowDevPlane_Stress3.pgf};
    %%
    \addplot3[black,mark=x,only marks,mark repeat=30,thick] file {section7/pgfFigures/pgf_thinWalledTubeSlowWave/TWslowDevPlane_Stress4.pgf};
    \addplot3[black,arrows along my path,thick] file {section7/pgfFigures/pgf_thinWalledTubeSlowWave/slowDevPlane_Stress4.pgf};
    %% 
    \addplot3[black,mark=x,only marks,mark repeat=30,thick] file {section7/pgfFigures/pgf_thinWalledTubeSlowWave/TWslowDevPlane_Stress5.pgf};
    \addplot3[black,arrows along my path,thick] file {section7/pgfFigures/pgf_thinWalledTubeSlowWave/slowDevPlane_Stress5.pgf};
    %% 
    \addplot3[black,mark=x,only marks,mark repeat=5,thick] file {section7/pgfFigures/pgf_thinWalledTubeSlowWave/TWslowDevPlane_Stress6.pgf};
    \addplot3[black,arrows along my path,thick] file {section7/pgfFigures/pgf_thinWalledTubeSlowWave/slowDevPlane_Stress6.pgf};
    %% Yield surface
    \addplot3[black,dashed] file {section7/pgfFigures/pgf_thinWalledTubeSlowWave/TWCylindreDevPlane.pgf};
    \newcommand\radius{0.82e8}
    \addplot3[dotted,thick] coordinates {(0.75*\radius,-0.75*\radius,0.) (-0.75*\radius,0.75*\radius,0.)};
    \addplot3[dotted,thick] coordinates {(0.,-0.75*\radius,0.75*\radius) (0.,0.75*\radius,-0.75*\radius)};
    \addplot3[dotted,thick] coordinates {(-0.75*\radius,0.,0.75*\radius) (0.75*\radius,0.,-0.75*\radius)};

    % \newcommand\radius{0.82e8}
    % \addplot3[dotted,very thick] coordinates {(1.05*\radius,-1.05*\radius,0.) (-1.05*\radius,1.05*\radius,0.)};

  \end{axis}
\end{scope}
\node at (4.95,6.75) {\text{Fast wave}};
\node at (14.35,6.75) {\text{Slow waves}};
\end{tikzpicture}

%%% Local Variables:
%%% mode: latex
%%% TeX-master: "../../presentation"
%%% End:}
  \caption{Stress path followed in a fast simple wave for the thin-walled tube problem. Comparison between the results obtained from equations \eqref{eq:tw_paths} and these of \cite{Clifton}.}
  \label{fig:fast_clifton}
\end{figure}
Figure \ref{fig:fast_clifton} shows one stress path resulting from the integration of the ODE related to right-going fast waves with $\sigma$ used as a driving parameter.
The initial stress state lies on the yield surface at $\sigma=0$ and the ODE is discretized by means of the backward Euler method, the integration being performed until the stress reaches the value $\sigma=\sigma^y $.
The path is respectively depicted in the stress space and in the deviatoric plane in figures \ref{fig:fast_clifton}\subref{subfig:tw_fast_stress} and \ref{fig:fast_clifton}\subref{subfig:tw_fast_dev}.
The deviatoric plane projection is obtained by drawing the paths in the eigenstresses space and projecting them onto the plane perpendicular to the hydrostatic axis $\sigma_1=\sigma_2=\sigma_3$.
In this plane, the von-Mises yield surface is a circle drawn with dashed lines.
%Furthermore, the direction of the path is given by the arrows in figure \ref{sec:stress_paths}.
As observed by \textsc{Clifton}, the path inside fast waves first follows the initial yield surface up to the intersection with the $\sigma$-axis.
Then, the loading path is such that $d\tau=0$ while $\sigma$ increases as far as hyperbolicity holds, that is for $c_f < c_2 = \sqrt{\mu/\rho} $ \cite{Clifton}.
Notice that these conclusions are similar to those made in the previous section.
The ODEs derived in section \ref{sec:stress_paths} for plane stresses, once adapted to the thin walled-tube problem, then allow to retrieve the solution originally proposed by \textsc{Clifton}.

Adopting the same approach with $\tau$ as driving parameter, some stress paths through slow waves have been reported in figure \ref{fig:tw_slow}.
Since fast waves lead to loading paths following the initial yield surface, the orthogonality property of the loading functions implies that these of slow waves move away from it.
It is seen in figure \ref{fig:tw_slow}\subref{subfig:tw_slow_stress}.
Nonetheless, this property holds in $(\sigma,\tau)$ plane but not in the deviatoric plane, as can be seen in figure \ref{fig:tw_slow}\subref{subfig:tw_slow_dev}, since the quasi-linear form \eqref{eq:ch5_quasilinear_normal} and hence, the ODEs, are not written in terms of $s_1,s_2,s_3$.
%This is however in general not the case in the deviatoric plane (\ref{fig:tw_slow}\subref{subfig:tw_slow_dev}) except for the initial condition $\sigma=0$ leads to a radial loading path following the direction of pure shear (horizontal dotted line in figure \ref{fig:tw_slow}\subref{subfig:tw_slow_dev}).
\begin{figure}[h!]
  \centering
  \subcaptionbox{Stress path in $(\sigma,\tau)$ plane \label{subfig:tw_slow_stress}}{\begin{tikzpicture}[scale=0.7]
  \begin{axis}[ymajorgrids=true,xmajorgrids=true,ylabel=$\tau \: (Pa)$,xlabel=$\sigma \: (Pa)$,xmin=-0.1e8,xmax=2.e8,ymin=0.,ymax=7.5e7]
    %%
    \addplot[very thick] table [x=sigma_11,y=sigma_12] {section5/pgfFigures/pgf_thinWalledTubeSlowWave/TWslowStressPlane_Stress0.pgf};
    %%
    \addplot[very thick] table [x=sigma_11,y=sigma_12] {section5/pgfFigures/pgf_thinWalledTubeSlowWave/TWslowStressPlane_Stress1.pgf};
    %%
    \addplot[very thick] table [x=sigma_11,y=sigma_12] {section5/pgfFigures/pgf_thinWalledTubeSlowWave/TWslowStressPlane_Stress2.pgf};
    %%
    \addplot[very thick] table [x=sigma_11,y=sigma_12] {section5/pgfFigures/pgf_thinWalledTubeSlowWave/TWslowStressPlane_Stress3.pgf};
    %%
    \addplot[very thick] table [x=sigma_11,y=sigma_12] {section5/pgfFigures/pgf_thinWalledTubeSlowWave/TWslowStressPlane_Stress4.pgf};
    %%
    \addplot[very thick] table [x=sigma_11,y=sigma_12] {section5/pgfFigures/pgf_thinWalledTubeSlowWave/TWslowStressPlane_Stress5.pgf};
    %%
    \addplot[very thick] table [x=sigma_11,y=sigma_12] {section5/pgfFigures/pgf_thinWalledTubeSlowWave/TWslowStressPlane_Stress6.pgf};
    %% Yield surface
    \addplot[black,dashed] table  [x=sigma_11,y=sigma_12] {section5/pgfFigures/pgf_thinWalledTubeSlowWave/TWslow_yield0.pgf};

    %\addplot[very thick,Orange,restrict y to domain=4.e7:6.75e7] table [x=sigma_11,y=sigma_12]{section5/pgfFigures/pgf_thinWalledTubeSlowWave/TWslowStressPlane_Stress1.pgf};


  \end{axis}
\end{tikzpicture}

%%% Local Variables:
%%% mode: latex
%%% TeX-master: "../../presentation"
%%% End:} \qquad
  \subcaptionbox{Stress path in deviatoric plane \label{subfig:tw_slow_dev}}{\tikzset{cross/.style={cross out, draw=black, minimum size=2*(#1-\pgflinewidth), inner sep=0pt, outer sep=0pt},
%default radius will be 1pt. 
cross/.default={2.5pt}}
\begin{tikzpicture}[scale=0.9]
  \begin{axis}[width=.75\textwidth,view={135}{35.2643},xlabel=$s_1 $,
    ylabel=$s_2 $,zlabel=$s_3$,xmin=-1.e8,xmax=1.e8,ymin=-1.e8,ymax=1.e8,axis equal,axis lines=center,axis on top,xtick=\empty,ytick=\empty,ztick=\empty,
    every axis y label/.style={at={(rel axis cs:0.,.5,-0.65)}, anchor=west},
    every axis x label/.style={at={(rel axis cs:0.5,.,-0.65)}, anchor=east},
    every axis z label/.style={at={(rel axis cs:0.,.0,.18)}, anchor=north}
    ]
    \node[below] at (1.1e8,0.,0.) {$\sigma^y$};
    \node[above] at (-1.1e8,0.,0.) {$-\sigma^y$};
    \draw (1.e8,0.,0.) node[cross,rotate=10] {};
    \draw (-1.e8,0.,0.) node[cross,rotate=10] {};
    \node[white]  at (0,0.,1.42e8) {};
    %%
    \addplot3[Green,mark=x,only marks,mark repeat=20,very thick] file {chapter5/pgfFigures/pgf_thinWalledTubeSlowWave/slowDevPlane_Stress0.pgf};
    \addplot3[Green,thick] file {chapter5/pgfFigures/pgf_thinWalledTubeSlowWave/slowDevPlane_Stress0.pgf};
    %%
    \addplot3[Duck,mark=x,only marks,mark repeat=20,very thick] file {chapter5/pgfFigures/pgf_thinWalledTubeSlowWave/slowDevPlane_Stress1.pgf};
    \addplot3[Duck,thick] file {chapter5/pgfFigures/pgf_thinWalledTubeSlowWave/slowDevPlane_Stress1.pgf};
    %%
    \addplot3[Red,mark=x,only marks,mark repeat=20,very thick] file {chapter5/pgfFigures/pgf_thinWalledTubeSlowWave/slowDevPlane_Stress2.pgf};
    \addplot3[Red,thick] file {chapter5/pgfFigures/pgf_thinWalledTubeSlowWave/slowDevPlane_Stress2.pgf};
    %%
    \addplot3[Purple,mark=x,only marks,mark repeat=20,very thick] file {chapter5/pgfFigures/pgf_thinWalledTubeSlowWave/slowDevPlane_Stress3.pgf};
    \addplot3[Purple,thick] file {chapter5/pgfFigures/pgf_thinWalledTubeSlowWave/slowDevPlane_Stress3.pgf};
    %%
    \addplot3[Blue,mark=x,only marks,mark repeat=20,very thick] file {chapter5/pgfFigures/pgf_thinWalledTubeSlowWave/slowDevPlane_Stress4.pgf};
    \addplot3[Blue,thick] file {chapter5/pgfFigures/pgf_thinWalledTubeSlowWave/slowDevPlane_Stress4.pgf};
    %% 
    \addplot3[Orange,mark=x,only marks,mark repeat=20,very thick] file {chapter5/pgfFigures/pgf_thinWalledTubeSlowWave/slowDevPlane_Stress5.pgf};
    \addplot3[Orange,thick] file {chapter5/pgfFigures/pgf_thinWalledTubeSlowWave/slowDevPlane_Stress5.pgf};
    %% 
    \addplot3[Yellow,mark=x,only marks,mark repeat=5,very thick] file {chapter5/pgfFigures/pgf_thinWalledTubeSlowWave/slowDevPlane_Stress6.pgf};
    \addplot3[Yellow,thick] file {chapter5/pgfFigures/pgf_thinWalledTubeSlowWave/slowDevPlane_Stress6.pgf};
    %% Yield surface
    \addplot3[black,dashed] file {chapter5/pgfFigures/pgf_thinWalledTubeSlowWave/TWCylindreDevPlane.pgf};
    \newcommand\radius{0.82e8}
    \addplot3[dotted,thick] coordinates {(0.75*\radius,-0.75*\radius,0.) (-0.75*\radius,0.75*\radius,0.)};
    \addplot3[dotted,thick] coordinates {(0.,-0.75*\radius,0.75*\radius) (0.,0.75*\radius,-0.75*\radius)};
    \addplot3[dotted,thick] coordinates {(-0.75*\radius,0.,0.75*\radius) (0.75*\radius,0.,-0.75*\radius)};

    % \newcommand\radius{0.82e8}
    % \addplot3[dotted,very thick] coordinates {(1.05*\radius,-1.05*\radius,0.) (-1.05*\radius,1.05*\radius,0.)};

  \end{axis}
\end{tikzpicture}

%%% Local Variables:
%%% mode: latex
%%% TeX-master: "../../mainManuscript"
%%% End:}
  \caption{Stress paths followed in a slow simple wave for the thin-walled tube problem. Comparison between the results obtained from equations \eqref{eq:tw_paths} (cross markers) and these of \cite{Clifton} (solid lines).}
  \label{fig:tw_slow}
\end{figure}



The behaviors highlighted above allow the solution of the Picard problem in a thin-walled cylinder, that is:
\begin{itemize}
\item initial conditions $\tens{\sigma}(\vect{x},t=0)=\tens{0}$, $\vect{v}(\vect{x},t=0)=\vect{0}$
\item step-loading boundary conditions $\sigma(x_1=0,t)=\sigma^d$ and $\tau(x_1=0,t)=\tau^d$
\end{itemize}
Indeed, with given ($\sigma^d,\tau^d$) outside of the initial yield surface, one can first integrate backward the loading path through the slowest wave. Two situations are then distinguished:
\begin{itemize}
\item[(i)] if the integration leads to some point of the initial yield surface, which can be reached by elastic discontinuities, the solution is complete.
\item[(ii)] if the slow wave connects ($\sigma^d,\tau^d$) to the $\sigma$-axis at some point lying outside of the initial yield surface, then a fast wave must be integrated backward to the initial elastic convex.
  Indeed, analogously to the results of table \ref{tab:stress_paths_properties}, it is shown in \cite{Clifton} that the paths followed through slow waves (\textit{resp. fast waves}) are perpendicular (\textit{resp. parallel}) to the $\sigma$-axis. 
  As a result and in virtue of the symmetries with respect to $\sigma$ and $\tau$ axes of the loading paths, the initial yield surface can reached by considering a fast wave.  
\end{itemize}
At last, the cases $\tau^d=0$ and $\sigma^d=0$ respectively lead to one single fast wave and one single slow wave.
Once the characteristic structure of the problem has been determined (\textit{i.e. one fast wave, one slow wave, or both}), the complete set of ODEs can be integrated in order to compute the solution.
It is worth emphasizing the complexity introduced by waves of combined-stress since the characteristic structure of the solution of a Picard problem now depends on the boundary conditions.
Hence, for developing a Riemann solver that would provide the stationary solution, additional computational effort must be made.

The iterative procedure proposed by \textsc{Lin} and \textsc{Ballman} \cite{Lin_et_Ballman} to solve Riemann problems is based on the above considerations.
%\textsc{Lin} and \textsc{Ballman} \cite{Lin_et_Ballman} proposed an iterative procedure to solve Riemann problems with the stress states considered above.
The left and right initial conditions of that problem satisfy equations similar to \eqref{eq:integral_example}:
\begin{subequations}
  \label{eq:lin_ballman}
  \begin{alignat}{1}
    \label{eq:lin_ballman_left}
    & u^* = u^L + \int_{\tens{\sigma}^L}^{\tens{\sigma}^*} \frac{d\sigma}{\rho c} \quad ; \quad v^* = v^L + \int_{\tens{\sigma}^L}^{\tens{\sigma}^*} \frac{d\tau}{\rho c} \\
    \label{eq:lin_ballman_right}
    & u^* = u^R - \int_{\tens{\sigma}^R}^{\tens{\sigma}^*} \frac{d\sigma}{\rho c} \quad ; \quad v^* = v^R - \int_{\tens{\sigma}^R}^{\tens{\sigma}^*} \frac{d\tau}{\rho c}
  \end{alignat}
\end{subequations}
where the asterisk denotes the stationary state of the Riemann problem.
First, a stress state ($\bar{\sigma},\bar{\tau}$) is assumed to be connected to $\tens{\sigma}^L$ and $\tens{\sigma}^R$ (see figure \ref{fig:lin_et_ballman} for the illustration of the method).
\begin{figure}[h!]
  \centering
  % \begin{tikzpicture}[scale=1.5]
  %   \draw[->] (0,0) --(3,0);
  %   \draw[->] (0,0) --(0,3);
  %   \node[below] at (1,0) {$v^L$};
  % \end{tikzpicture}
  \begin{tikzpicture}[scale=1.5]
  %% (u,sigma) plane
  \draw[->,thick] (0,0)-- (3,0) node [right] {$u$};
  \draw[->,thick] (0,0)-- (0,3) node [above] {$\sigma$};
  \fill[black] (0.25,0.3) circle (0.05) ;
  \fill[black] (2.5,0.5) circle (0.05) ;
  \fill[black] (1.,2.8) circle (0.05) ;
  \fill[black] (1.75,2.8) circle (0.05) ;
  %% Left states
  \draw[dotted] (0.25,0.3) -- (0.25,0.) node [below] {$u^L$};
  \draw[dotted] (1.75,2.8) -- (1.75,0) node [below] {$u^1$};
  %% Right states
  \draw[dotted] (2.5,0.5) -- (2.5,0.) node [below] {$u^R$};
  \draw[dotted] (1,2.8) -- (1.,0) node [below] {$u^2$};
  \draw[dotted] (0,2.8) node [left] {$\bar{\sigma}_{11}$} -- (1.75,2.8) ;
  \draw[dashed] (0.25,0.3) .. controls (0.3,0.33) and (1.5,2.8) .. (1.75,2.8);
  \draw[dashed] (2.5,0.5) .. controls (2.,0.5) and (1,2.5) .. (1,2.8);
  %% Intersection of integral curves
  \draw[dotted] (0.,2.125) node [left] {$\widehat{\sigma}$}-- (2.5,2.125);

  %% (v,tau) plane
  \newcommand\shift{6}
  \draw[->,thick] (0+\shift,0)-- (3+\shift,0) node [right] {$v$};
  \draw[->,thick] (0+\shift,0)-- (0+\shift,3) node [above] {$\tau$};
  \fill[black] (0.25+\shift,2.3) circle (0.05) ;
  \fill[black] (1.75+\shift,1.5) circle (0.05) ;
  \fill[black] (2.8+\shift,0.8) circle (0.05) ;
  \fill[black] (1.+\shift,0.8) circle (0.05) ;
  %% Left states
  \draw[dotted] (0.25+\shift,2.3) -- (0.25+\shift,0.) node [below] {$v^L$};
  \draw[dotted] (2.8+\shift,0.8) -- (2.8+\shift,0) node [below] {$v^1$};
  %% Right states
  \draw[dotted] (1.75+\shift,1.5) -- (1.75+\shift,0.) node [below] {$v^R$};
  \draw[dotted] (1.+\shift,0.8) -- (1.+\shift,0.0) node [below] {$v^2$};
  \draw[dotted] (0+\shift,0.8) node [left] {$\bar{\sigma}_{12}$} -- (3.+\shift,0.8) ;
  %% integral curves
  \draw[dashed] (0.25+\shift,2.3) .. controls (0.75+\shift,1.3) and (2.5+\shift,0.8) .. (2.8+\shift,0.8);
  \draw[dashed] (1.75+\shift,1.5) .. controls (1.5+\shift,1.5) and (1+\shift,1.5) .. (1.+\shift,0.8);
  %% Intersection of integral curves
  \draw[dotted] (0.+\shift,1.38) node [left] {$\widehat{\tau}$}-- (1.55+\shift,1.38);
\end{tikzpicture}



%%% Local Variables:
%%% mode: latex
%%% TeX-master: "../../mainManuscript"
%%% End:
 
  \caption{Schematic representation of the iterative Riemann solver proposed in \cite{Lin_et_Ballman}.}
  \label{fig:lin_et_ballman}
\end{figure}
By taking into account the conclusions of \textsc{Clifton} the loading paths followed can be identified so that equations \eqref{eq:lin_ballman_left} and \eqref{eq:lin_ballman_right} can be integrated.
Thus, virtual integral curves are built in ($u,\sigma$) and ($v,\tau$) planes as depicted with dashed lines in figure \ref{fig:lin_et_ballman}.
Second, the intersection of the curves joining respectively $\vect{v}^L$ to $\vect{v}^1$ and $\vect{v}^R$ to $\vect{v}^2$ gives a stress state ($\widehat{\sigma},\widehat{\tau}$) that is used to apply the procedure again until some criterion $\norm{\vect{v}^1-\vect{v}^2}\leq \epsilon $ is achieved.
At last, the state obtained $(\widehat{\vect{v}},\widehat{\tens{\sigma}})$ corresponds to the stationary state of the Riemann problem and can be used to compute numerical fluxes at cell interfaces.
Notice that in this procedure, the intersection of integral curves is found by means of the tangent lines approximation at $\bar{\tens{\sigma}}$ so that this solver does not fully account for the exact solution.
Therefore, it appears that the loading paths identified in \cite{Clifton} are crucial in order to determine the wave pattern corresponding to a guessed stationary state.
This is exactly what is lacking for the more general plane strain and plane stress problems.

\subsection{Plane stress}
\label{sec:num_plane_stress}
We now move on to a more general plane stress case for which the stress $\sigma_{22} $ is not zero.
Although the equations of section \ref{sec:stress_paths} have been derived for two directions of propagation, attention is paid here to $\vect{n}=\vect{e}_1$ only. 
Thus, the system of ODEs considered reads (see table \ref{tab:simpleWavesEquations}):
\begin{align}
  \label{eq:plane_stress_paths}
  & d\sigma_{11} = \psi_1^{s,f} d\sigma_{12} \\
  & d\sigma_{22} = -\frac{\psi^{s,f}_{1}\alpha_{11}+\alpha_{12}}{\alpha_{22}}d\sigma_{12}
\end{align}
%Indeed, it has been seen that similar properties of the loading paths inside the simple waves hold for both directions.

One path through a fast simple wave is first looked at by assuming an initially free-stress state, brought to the yield surface at the point $ \sigma_{11}=\sigma_{22}=0 $.
Equations \eqref{eq:plane_stress_paths} are thus integrated implicitly with $\sigma_{12}$ as driving parameter until the shear component $\sigma_{12}$ vanishes.
%The ODEs of table \ref{tab:simpleWavesEquations} are thus integrated implicitly with $\sigma_{12}$ as driving parameter until the shear component $\sigma_{12}$ vanishes.
Two situations are considered for which the stress $\sigma_{11}$ increases or decreases.
\begin{figure}[h!]
  \centering
  \subcaptionbox{Projections of loading paths in ($\sigma_{11},\sigma_{12}$) and ($\sigma_{22},\sigma_{12}$) planes \label{subfig:CP_fast_stress}}{\begin{tikzpicture}[scale=0.9]
\begin{groupplot}[group style={group size=2 by 1,
ylabels at=edge left, yticklabels at=edge left,horizontal sep=3.ex,
xticklabels at=edge bottom,xlabels at=edge bottom},
ymajorgrids=true,xmajorgrids=true,ylabel=$\sigma_{12} \: (Pa)$,
axis on top,scale only axis,width=0.4\linewidth,ymin=0,ymax=63499406.78820015
, every x tick scale label/.style={at={(xticklabel* cs:1.05,0.75cm)},anchor=near yticklabel},colormap name=viridis]
\nextgroupplot[xlabel=$\sigma_{11} (Pa)$]
\addplot[arrows along my path,black,thick] table[x=sigma_11,y=sigma_12] {chapter5/pgfFigures/pgf_fastWavesPlaneStress/CPfastStressPlane_frame0_Stress0.pgf};\addplot[mesh,point meta = \thisrow{p},very thick,no markers] table[x=sigma_11,y=sigma_12] {chapter5/pgfFigures/pgf_fastWavesPlaneStress/CPfastStressPlane_frame0_Stress0.pgf} node[above right,black] {$\textbf{1}$};
\addplot[arrows along my path,black,thick] table[x=sigma_11,y=sigma_12] {chapter5/pgfFigures/pgf_fastWavesPlaneStress/CPfastStressPlane_frame1_Stress0.pgf};\addplot[mesh,point meta = \thisrow{p},very thick,no markers] table[x=sigma_11,y=sigma_12] {chapter5/pgfFigures/pgf_fastWavesPlaneStress/CPfastStressPlane_frame1_Stress0.pgf} node[above right,black] {$\textbf{2}$};
\addplot[gray,dashed,thin] table[x=sigma_11,y=sigma_12] {chapter5/pgfFigures/pgf_fastWavesPlaneStress/CPfast_yield0_s11s12_Stress0.pgf};

\nextgroupplot[colorbar,colorbar style={title= {$c_f \: (m/s)$},every y tick scale label/.style={at={(2.,-.1125)}} },xlabel=$\sigma_{22}  (Pa)$]
\addplot[arrows along my path,black,thick] table[x=sigma_22,y=sigma_12] {chapter5/pgfFigures/pgf_fastWavesPlaneStress/CPfastStressPlane_frame0_Stress0.pgf};\addplot[mesh,point meta = \thisrow{p},very thick,no markers] table[x=sigma_22,y=sigma_12] {chapter5/pgfFigures/pgf_fastWavesPlaneStress/CPfastStressPlane_frame0_Stress0.pgf} node[above right,black] {$\textbf{1}$};
\addplot[arrows along my path,black,thick] table[x=sigma_22,y=sigma_12] {chapter5/pgfFigures/pgf_fastWavesPlaneStress/CPfastStressPlane_frame1_Stress0.pgf};\addplot[mesh,point meta = \thisrow{p},very thick,no markers] table[x=sigma_22,y=sigma_12] {chapter5/pgfFigures/pgf_fastWavesPlaneStress/CPfastStressPlane_frame1_Stress0.pgf} node[above right,black] {$\textbf{2}$};
\end{groupplot}
\end{tikzpicture}
%%% Local Variables:
%%% mode: latex
%%% TeX-master: "../../mainManuscript"
%%% End:
}
  \subcaptionbox{Loading paths in deviatoric plane \label{subfig:CP_fast_dev}}{\tikzset{cross/.style={cross out, draw=black, minimum size=2*(#1-\pgflinewidth), inner sep=0pt, outer sep=0pt},cross/.default={2.5pt}}
\begin{tikzpicture}[scale=0.9]
\begin{axis}[width=.75\textwidth,view={135}{35.2643},xlabel=$s_1 $,ylabel=$s_2 $,zlabel=$s_3$,xmin=-1.e8,xmax=1.e8,ymin=-1.e8,ymax=1.e8,axis equal,axis lines=center,axis on top,xtick=\empty,ytick=\empty,ztick=\empty,every axis y label/.style={at={(rel axis cs:0.,.5,-0.65)}, anchor=west}, every axis x label/.style={at={(rel axis cs:0.5,.,-0.65)}, anchor=east}, every axis z label/.style={at={(rel axis cs:0.,.0,.18)}, anchor=north},legend style={at={(.225,.59)}}]
\node[below] at (1.1e8,0.,0.) {$\sigma^y$};
\node[above] at (-1.1e8,0.,0.) {$-\sigma^y$};
\draw (1.e8,0.,0.) node[cross,rotate=10] {};
\draw (-1.e8,0.,0.) node[cross,rotate=10] {};
\node[white]  at (0,0.,1.1e8) {};
\addplot3[gray,dashed,thin,no markers] file {chapter5/pgfFigures/pgf_fastWavesPlaneStress/CPCylindreDevPlane.pgf};\addlegendentry{initial yield surface}
%\addplot3[Red,mark=star,mark repeat=20,mark size=3pt,very thick] file {chapter5/pgfFigures/pgf_fastWavesPlaneStress/CPfastDevPlane_frame0_Stress0.pgf};
\addplot3[arrows along my path,Red,very thick] file {chapter5/pgfFigures/pgf_fastWavesPlaneStress/CPfastDevPlane_frame0_Stress0.pgf};
\addlegendentry{loading path 1}
%\addplot3[Blue,mark=asterisk,mark repeat=20,mark size=3pt,very thick] file {chapter5/pgfFigures/pgf_fastWavesPlaneStress/CPfastDevPlane_frame1_Stress0.pgf};
\addplot3[arrows along my path,Blue,very thick] file {chapter5/pgfFigures/pgf_fastWavesPlaneStress/CPfastDevPlane_frame1_Stress0.pgf};
\addlegendentry{loading path 2}
\newcommand\radius{1.*0.82e8}
\addplot3[dotted,thick] coordinates {(0.75*\radius,-0.75*\radius,0.) (-0.75*\radius,0.75*\radius,0.)};
\addplot3[dotted,thick] coordinates {(0.,-0.75*\radius,0.75*\radius) (0.,0.75*\radius,-0.75*\radius)};
\addplot3[dotted,thick] coordinates {(-0.75*\radius,0.,0.75*\radius) (0.75*\radius,0.,-0.75*\radius)};
\end{axis}
\end{tikzpicture}
%%% Local Variables:
%%% mode: latex
%%% TeX-master: "../../mainManuscript"
%%% End:
}
  \caption{Loading paths through a fast simple wave starting from the initial yield surface with initial condition $\sigma_{11}=\sigma_{22}=0 $ in directions of decreasing and increasing $\sigma_{11} $.}
  \label{fig:fast_path_plane_stress}
\end{figure}
The resulting loading paths are depicted in figure \ref{fig:fast_path_plane_stress}\subref{subfig:CP_fast_stress} in $(\sigma_{11},\sigma_{12})$ and $(\sigma_{22},\sigma_{12})$ planes, while the projection in the deviatoric plane can be seen in figure \ref{fig:fast_path_plane_stress}\subref{subfig:CP_fast_dev}.
In addition, figure \ref{fig:fast_path_plane_stress}\subref{subfig:CP_fast_stress} shows the evolution of the characteristic speed associated to the fast wave along the path by means of a color gradient.
Thus, it can be seen that for the loadings under consideration, the wave celerity is a decreasing function of the stress so that the simple wave solutions are valid.
Next, it appears that $\sigma_{12}$ is an even function of $\sigma_{11}$ and $\sigma_{22}$. %the paths have axial symmetry, although the property has not been shown yet.
At last, analogously to the thin-walled cylinder solution, the stress components follow the initial yield surface, which is obvious in the deviatoric plane (figure \ref{fig:fast_path_plane_stress}\subref{subfig:CP_fast_dev}).
Furthermore, according to the property \eqref{eq:CP_roots} the stress path must be horizontal in the $(\sigma_{11},\sigma_{12})$ and $(\sigma_{22},\sigma_{12})$ planes, once the $\sigma_{11}$-axis is reached.
As depicted in figure \ref{fig:fast_path_plane_stress}\subref{subfig:CP_fast_dev}, this point corresponds to a direction of pure shear in the deviatoric plane.
%As a result, the plastic flow becomes significant once a direction of pure shear is reached.
Nevertheless, the numerical integration of ODEs once the shear stress $\sigma_{12}$ vanishes is not possible owing to an indeterminacy of the loading function $\psi_1^f$ that has not been identified so far.
%condition \eqref{eq:CP_roots} yields $\psi^f_1 \rightarrow \infty$ and associated numerical issues so that the loading path cannot be integrated further.
%The integration performed here nevertheless stops once $\sigma_{12}=0$ due to a singularity that has not been identified and leads to an undefined loading function $\psi_1^f$.

% Note that while the path is restricted to the initial yield surface, there exists a relation bewteen the principle stresses $\sigma_1 $ and $\sigma_2 $.
% Indeed, since the the out-of-plane stress $\sigma_{33}=0 $ is itself an eigenvalue of Cauchy stress tensor, the intersection of the initial yield surface with the plane $\sigma_3=0$ in the principle stress space is an ellipse as shown in figure \ref{fig:ellipse}.
% Therefore, the following relation holds:
% \begin{equation}
%   \label{eq:ellipse_equation}
%   \(\frac{\sigma_1}{a}\)^2 + \(\frac{\sigma_2}{b}\)^2 = 1
% \end{equation}
% where $a$ and $b$ the semi-major and semi-minor axis of the ellipse.
% Find $a$ and $b $!.
% \begin{figure}[h!]
%   \centering
%   \begin{tikzpicture}[scale=.8]
  \fill[pattern=north west lines] (0,0) ellipse (1.95 and 0.25);
  \draw (0,0) ellipse (1.95 and 0.25);
  \draw[thick,->] (0,0,0) -- (0,3,0) node[right] {$\sigma_3$};
  \draw[thick,->] (0,0,0) -- (3,0,0) node[right] {$\sigma_1$};
  \draw[thick,->] (0,0,0) -- (0,0,-3) node[right] {$\sigma_2$};
  \draw[dashed,->] (0,0,0) -- (2,1,-2) node [right] {\text{hydrostatic axis}};%{$\tr \tens{\sigma}$};
  \begin{scope}[rotate=30]
    \draw (-3.,-1) -- (3.,-1);
    \draw (-3.,1) -- (3,1);
    \draw (3.,0) ellipse (0.35 and 1);
    \draw (-3.,0) ellipse (0.35 and 1);
  \end{scope}
  
\end{tikzpicture}

%%% Local Variables:
%%% mode: latex
%%% TeX-master: "../../mainManuscript"
%%% End:

%   \caption{Intersection of the initial yield surface with the plane $\sigma_3=0 $ in eigenstresses space.}
%   \label{fig:ellipse}
% \end{figure}

%% integration jusqu'à \tau=2\sigma^y pour \sigma_{22}=\{\pm 57735026.919,0\}
We now focus on the stress evolution inside slow waves.
The same procedure is followed for several starting points on the initial yield surface.
In addition, various initial values are considered for $\sigma_{22}$ since, even for a solid in a free stress state at $t=0$, a fast wave may lead to $\sigma_{22}\neq 0$.
\begin{figure}[h!]
  \centering
  \subcaptionbox{Projections of loading paths in ($\sigma_{11},\sigma_{12}$) and ($\sigma_{22},\sigma_{12}$) planes \label{subfig:CP_slow_stress1}}{\begin{tikzpicture}[scale=0.9]
\begin{groupplot}[group style={group size=2 by 1,
ylabels at=edge left, yticklabels at=edge left,horizontal sep=3.ex,
xticklabels at=edge bottom,xlabels at=edge bottom},
ymajorgrids=true,xmajorgrids=true,ylabel=$\sigma_{12} \: (Pa)$,
axis on top,scale only axis,width=0.4\linewidth,ymin=0,ymax=109528891.78848386
, every x tick scale label/.style={at={(xticklabel* cs:1.05,0.75cm)},anchor=near yticklabel},colormap name=viridis]
, every x tick scale label/.style={at={(xticklabel* cs:1.05,0.75cm)},anchor=near yticklabel},colormap name=viridis]
\nextgroupplot[xlabel=$\sigma_{11} \: (Pa)$]
\addplot[arrows along my path,black,thick] table[x=sigma_11,y=sigma_12] {chapter5/pgfFigures/pgf_slowWavesPlaneStress/CPslowStressPlane_frame0_Stress1.pgf};
\addplot[mesh,point meta = \thisrow{p},very thick,no markers] table[x=sigma_11,y=sigma_12] {chapter5/pgfFigures/pgf_slowWavesPlaneStress/CPslowStressPlane_frame0_Stress1.pgf} node[above right,black] {$\textbf{1}$};
\addplot[arrows along my path,black,thick] table[x=sigma_11,y=sigma_12] {chapter5/pgfFigures/pgf_slowWavesPlaneStress/CPslowStressPlane_frame1_Stress1.pgf};
\addplot[mesh,point meta = \thisrow{p},very thick,no markers] table[x=sigma_11,y=sigma_12] {chapter5/pgfFigures/pgf_slowWavesPlaneStress/CPslowStressPlane_frame1_Stress1.pgf} node[above right,black] {$\textbf{2}$};
\addplot[arrows along my path,black,thick] table[x=sigma_11,y=sigma_12] {chapter5/pgfFigures/pgf_slowWavesPlaneStress/CPslowStressPlane_frame2_Stress1.pgf};
\addplot[mesh,point meta = \thisrow{p},very thick,no markers] table[x=sigma_11,y=sigma_12] {chapter5/pgfFigures/pgf_slowWavesPlaneStress/CPslowStressPlane_frame2_Stress1.pgf} node[above right,black] {$\textbf{3}$};
\addplot[arrows along my path,black,thick] table[x=sigma_11,y=sigma_12] {chapter5/pgfFigures/pgf_slowWavesPlaneStress/CPslowStressPlane_frame3_Stress1.pgf};
\addplot[mesh,point meta = \thisrow{p},very thick,no markers] table[x=sigma_11,y=sigma_12] {chapter5/pgfFigures/pgf_slowWavesPlaneStress/CPslowStressPlane_frame3_Stress1.pgf} node[above right,black] {$\textbf{4}$};
\addplot[gray,dashed,thin] table[x=sigma_11,y=sigma_12] {chapter5/pgfFigures/pgf_slowWavesPlaneStress/CPslow_yield0_s11s12_Stress1.pgf};

\nextgroupplot[colorbar,colorbar style={title= {$ c_s \: (m/s)$},every y tick scale label/.style={at={(2.,-.1125)}} },xlabel=$\sigma_{22} \: (Pa)$]
\addplot[arrows along my path,black,thick] table[x=sigma_22,y=sigma_12] {chapter5/pgfFigures/pgf_slowWavesPlaneStress/CPslowStressPlane_frame0_Stress1.pgf};
\addplot[mesh,point meta = \thisrow{p},very thick,no markers] table[x=sigma_22,y=sigma_12] {chapter5/pgfFigures/pgf_slowWavesPlaneStress/CPslowStressPlane_frame0_Stress1.pgf} node[above right,black] {$\textbf{1}$};
\addplot[arrows along my path,black,thick] table[x=sigma_22,y=sigma_12] {chapter5/pgfFigures/pgf_slowWavesPlaneStress/CPslowStressPlane_frame1_Stress1.pgf};
\addplot[mesh,point meta = \thisrow{p},very thick,no markers] table[x=sigma_22,y=sigma_12] {chapter5/pgfFigures/pgf_slowWavesPlaneStress/CPslowStressPlane_frame1_Stress1.pgf} node[above right,black] {$\textbf{2}$};
\addplot[arrows along my path,black,thick] table[x=sigma_22,y=sigma_12] {chapter5/pgfFigures/pgf_slowWavesPlaneStress/CPslowStressPlane_frame2_Stress1.pgf};
\addplot[mesh,point meta = \thisrow{p},very thick,no markers] table[x=sigma_22,y=sigma_12] {chapter5/pgfFigures/pgf_slowWavesPlaneStress/CPslowStressPlane_frame2_Stress1.pgf} node[above right,black] {$\textbf{3}$};
\addplot[arrows along my path,black,thick] table[x=sigma_22,y=sigma_12] {chapter5/pgfFigures/pgf_slowWavesPlaneStress/CPslowStressPlane_frame3_Stress1.pgf};
\addplot[mesh,point meta = \thisrow{p},very thick,no markers] table[x=sigma_22,y=sigma_12] {chapter5/pgfFigures/pgf_slowWavesPlaneStress/CPslowStressPlane_frame3_Stress1.pgf} node[above right,black] {$\textbf{4}$};
\end{groupplot}
\end{tikzpicture}
%%% Local Variables:
%%% mode: latex
%%% TeX-master: "../../mainManuscript"
%%% End:
}
  \subcaptionbox{Loading paths in deviatoric plane \label{subfig:CP_slow_dev1}}{\begin{tikzpicture}[scale=0.9]
\begin{axis}[width=.75\textwidth,view={135}{35.2643},xlabel=$s_1 $,ylabel=$s_2 $,zlabel=$s_3$,xmin=-1.e8,xmax=1.e8,ymin=-1.e8,ymax=1.e8,axis equal,axis lines=center,axis on top,ztick=\empty,legend style={at={(.225,.59)}}]
\addplot3+[Red,mark=star,mark repeat=20,mark size=3pt,very thick] file {chapter5/pgfFigures/pgf_slowWavesPlaneStress/CPslowDevPlane_frame0_Stress1.pgf};
\addlegendentry{loading path 1}
\addplot3+[Blue,mark=asterisk,mark repeat=20,mark size=3pt,very thick] file {chapter5/pgfFigures/pgf_slowWavesPlaneStress/CPslowDevPlane_frame1_Stress1.pgf};
\addlegendentry{loading path 2}
\addplot3+[Orange,mark=+,mark repeat=20,mark size=3pt,very thick] file {chapter5/pgfFigures/pgf_slowWavesPlaneStress/CPslowDevPlane_frame2_Stress1.pgf};
\addlegendentry{loading path 3}
\addplot3+[Purple,mark=x,mark repeat=20,mark size=3pt,very thick] file {chapter5/pgfFigures/pgf_slowWavesPlaneStress/CPslowDevPlane_frame3_Stress1.pgf};
\addlegendentry{loading path 4}
\addplot3+[gray,dashed,thin,no markers] file {chapter5/pgfFigures/pgf_slowWavesPlaneStress/CPCylindreDevPlane.pgf};\addlegendentry{initial yield surface}
\end{axis}
\end{tikzpicture}
%%% Local Variables:
%%% mode: latex
%%% TeX-master: "../../mainManuscript"
%%% End:
}
  \caption{Stress paths in a slow simple wave for various starting point lying on the initial yield surface for $\sigma_{22}=-5.8\times 10^7 \: Pa$. Projections in the stress space (figure \subref{subfig:CP_slow_stress2})  and deviatoric plane (figure \subref{subfig:CP_slow_dev1}).}
  \label{fig:slow_path_plane_stress1}
\end{figure}
The loading paths thus obtained for the (arbitrary) initial values $\sigma_{22}=-5.8\times 10^7 \: Pa$, $\sigma_{22}=0$ and $\sigma_{22}=5.8\times 10^7 \: Pa$, are respectively depicted in figures \ref{fig:slow_path_plane_stress1}, \ref{fig:slow_path_plane_stress2} and \ref{fig:slow_path_plane_stress3}.
The projections in the stress space and the deviatoric plane are shown.
The evolution of the characteristic speed associated to the slow wave can also be seen by means of a color gradient.
Once again, the simple wave solution appears to be valid with the considered loading conditions.
Furthermore, one can see that the stress paths are now more complex since, for instance, no symmetry appears in the $(\sigma_{11},\sigma_{12})$ plane.
%For instance, for a negative initial value of the stress $\sigma_{22}$ (figure \ref{fig:slow_path_plane_stress1}), the projection in the $(\sigma_{11},\sigma_{12})$ plane does not allow to identify some symmetry.
Moreover, the behavior is even more complex in the $(\sigma_{22},\sigma_{12})$ plane in which the variations first mainly concern $\sigma_{22}$ and next, the slopes of curves roughly change so that the paths are almost vertical.
This sharp change in slopes is also notable in the deviatoric plane in figure \ref{fig:slow_path_plane_stress1}\subref{subfig:CP_slow_dev1}.
%However, this figure does not help in finding some interpretations. 

On the other hand, similar observations can be made for the other initial values of $\sigma_{22}$ as can be seen in figures \ref{fig:slow_path_plane_stress2} and \ref{fig:slow_path_plane_stress3}.
\begin{figure}[h!]
  \centering
  \subcaptionbox{Projections of loading paths in ($\sigma_{11},\sigma_{12}$) and ($\sigma_{22},\sigma_{12}$) planes \label{subfig:CP_slow_stress2}}{\begin{tikzpicture}[scale=0.9]
\begin{groupplot}[group style={group size=2 by 1,
ylabels at=edge left, yticklabels at=edge left,horizontal sep=3.ex,
xticklabels at=edge bottom,xlabels at=edge bottom},
ymajorgrids=true,xmajorgrids=true,ylabel=$\sigma_{12} \: (Pa)$,
axis on top,scale only axis,width=0.4\linewidth,ymin=0,ymax=126473070.316
, every x tick scale label/.style={at={(xticklabel* cs:1.05,0.75cm)},anchor=near yticklabel}
,colormap name =viridis]
\nextgroupplot[xlabel=$\sigma_{11} \: (Pa)$]
%\addplot[arrows along my path,black,thick] table[x=sigma_11,y=sigma_12] {chapter5/pgfFigures/pgf_slowWavesPlaneStress/CPslowStressPlane_frame0_Stress2.pgf};
\addplot[mesh,point meta = \thisrow{p},very thick,no markers] table[x=sigma_11,y=sigma_12] {chapter5/pgfFigures/pgf_slowWavesPlaneStress/CPslowStressPlane_frame0_Stress2.pgf} node[above,black] {$\textbf{1}$};
%\addplot[arrows along my path,black,thick] table[x=sigma_11,y=sigma_12] {chapter5/pgfFigures/pgf_slowWavesPlaneStress/CPslowStressPlane_frame1_Stress2.pgf};
\addplot[mesh,point meta = \thisrow{p},very thick,no markers] table[x=sigma_11,y=sigma_12] {chapter5/pgfFigures/pgf_slowWavesPlaneStress/CPslowStressPlane_frame1_Stress2.pgf} node[above,black] {$\textbf{2}$};
%\addplot[arrows along my path,black,thick] table[x=sigma_11,y=sigma_12] {chapter5/pgfFigures/pgf_slowWavesPlaneStress/CPslowStressPlane_frame2_Stress2.pgf};
\addplot[mesh,point meta = \thisrow{p},very thick,no markers] table[x=sigma_11,y=sigma_12] {chapter5/pgfFigures/pgf_slowWavesPlaneStress/CPslowStressPlane_frame2_Stress2.pgf} node[above,black] {$\textbf{3}$};
%\addplot[arrows along my path,black,thick] table[x=sigma_11,y=sigma_12] {chapter5/pgfFigures/pgf_slowWavesPlaneStress/CPslowStressPlane_frame3_Stress2.pgf};
\addplot[mesh,point meta = \thisrow{p},very thick,no markers] table[x=sigma_11,y=sigma_12] {chapter5/pgfFigures/pgf_slowWavesPlaneStress/CPslowStressPlane_frame3_Stress2.pgf} node[above,black] {$\textbf{4}$};
\addplot[gray,dashed,thin] table[x=sigma_11,y=sigma_12] {chapter5/pgfFigures/pgf_slowWavesPlaneStress/CPslow_yield0_s11s12_Stress2.pgf};

\nextgroupplot[colorbar,colorbar style={title= {$ c_s \: (m/s)$},every y tick scale label/.style={at={(2.,-.1125)}} },xlabel=$\sigma_{22} \: (Pa)$]
\addplot[arrows along my path,black!70,thick] table[x=sigma_22,y=sigma_12] {chapter5/pgfFigures/pgf_slowWavesPlaneStress/CPslowStressPlane_frame0_Stress2.pgf};\addplot[mesh,point meta = \thisrow{p},very thick,no markers] table[x=sigma_22,y=sigma_12] {chapter5/pgfFigures/pgf_slowWavesPlaneStress/CPslowStressPlane_frame0_Stress2.pgf};
\addplot[arrows along my path,black!70,thick] table[x=sigma_22,y=sigma_12] {chapter5/pgfFigures/pgf_slowWavesPlaneStress/CPslowStressPlane_frame1_Stress2.pgf};\addplot[mesh,point meta = \thisrow{p},very thick,no markers] table[x=sigma_22,y=sigma_12] {chapter5/pgfFigures/pgf_slowWavesPlaneStress/CPslowStressPlane_frame1_Stress2.pgf} ;
\addplot[arrows along my path,black!70,thick] table[x=sigma_22,y=sigma_12] {chapter5/pgfFigures/pgf_slowWavesPlaneStress/CPslowStressPlane_frame2_Stress2.pgf};\addplot[mesh,point meta = \thisrow{p},very thick,no markers] table[x=sigma_22,y=sigma_12] {chapter5/pgfFigures/pgf_slowWavesPlaneStress/CPslowStressPlane_frame2_Stress2.pgf} ;
\addplot[arrows along my path,black!70,thick] table[x=sigma_22,y=sigma_12] {chapter5/pgfFigures/pgf_slowWavesPlaneStress/CPslowStressPlane_frame3_Stress2.pgf};\addplot[mesh,point meta = \thisrow{p},very thick,no markers] table[x=sigma_22,y=sigma_12] {chapter5/pgfFigures/pgf_slowWavesPlaneStress/CPslowStressPlane_frame3_Stress2.pgf} ;
\end{groupplot}
\end{tikzpicture}
%%% Local Variables:
%%% mode: latex
%%% TeX-master: "../../mainManuscript"
%%% End:
}
  \subcaptionbox{Loading paths in deviatoric plane  \label{subfig:CP_slow_dev2}}{\tikzset{cross/.style={cross out, draw=black, minimum size=2*(#1-\pgflinewidth), inner sep=0pt, outer sep=0pt},cross/.default={2.5pt}}
\begin{tikzpicture}[scale=0.9]
\begin{axis}[width=.75\textwidth,view={135}{35.2643},xlabel=$s_1 $,ylabel=$s_2 $,zlabel=$s_3$,xmin=-1.e8,xmax=1.e8,ymin=-1.e8,ymax=1.e8,axis equal,axis lines=center,axis on top,xtick=\empty,ytick=\empty,ztick=\empty,every axis y label/.style={at={(rel axis cs:0.,.5,-0.65)}, anchor=west}, every axis x label/.style={at={(rel axis cs:0.5,.,-0.65)}, anchor=east}, every axis z label/.style={at={(rel axis cs:0.,.0,.18)}, anchor=north},legend style={at={(.225,.59)}}]
\node[below] at (1.1e8,0.,0.) {$\sigma^y$};
\node[above] at (-1.1e8,0.,0.) {$-\sigma^y$};
\draw (1.e8,0.,0.) node[cross,rotate=10] {};
\draw (-1.e8,0.,0.) node[cross,rotate=10] {};
\node[white]  at (0,0.,1.42e8) {};
\addplot3+[Red,mark=star,mark repeat=20,mark size=3pt,very thick] file {chapter5/pgfFigures/pgf_slowWavesPlaneStress/CPslowDevPlane_frame0_Stress2.pgf};
\addlegendentry{loading path 1}
\addplot3+[Blue,mark=asterisk,mark repeat=20,mark size=3pt,very thick] file {chapter5/pgfFigures/pgf_slowWavesPlaneStress/CPslowDevPlane_frame1_Stress2.pgf};
\addlegendentry{loading path 2}
\addplot3+[Orange,mark=+,mark repeat=20,mark size=3pt,very thick] file {chapter5/pgfFigures/pgf_slowWavesPlaneStress/CPslowDevPlane_frame2_Stress2.pgf};
\addlegendentry{loading path 3}
\addplot3+[Purple,mark=x,mark repeat=20,mark size=3pt,very thick] file {chapter5/pgfFigures/pgf_slowWavesPlaneStress/CPslowDevPlane_frame3_Stress2.pgf};
\addlegendentry{loading path 4}
\addplot3+[gray,dashed,thin,no markers] file {chapter5/pgfFigures/pgf_slowWavesPlaneStress/CPCylindreDevPlane.pgf};\addlegendentry{initial yield surface}
\newcommand\radius{0.82e8}
\addplot3[dotted,thick] coordinates {(0.75*\radius,-0.75*\radius,0.) (-0.75*\radius,0.75*\radius,0.)};
\addplot3[dotted,thick] coordinates {(0.,-0.75*\radius,0.75*\radius) (0.,0.75*\radius,-0.75*\radius)};
\addplot3[dotted,thick] coordinates {(-0.75*\radius,0.,0.75*\radius) (0.75*\radius,0.,-0.75*\radius)};
\end{axis}
\end{tikzpicture}
%%% Local Variables:
%%% mode: latex
%%% TeX-master: "../../mainManuscript"
%%% End:
}
  \caption{Stress paths in a slow simple wave for various starting point lying on the initial yield surface for $\sigma_{22}=0$. Projections in the stress space (figure \subref{subfig:CP_slow_stress2}) and deviatoric plane (figure \subref{subfig:CP_slow_dev2}).}
  \label{fig:slow_path_plane_stress2}
\end{figure}
%The last set of results obtained with the initial data $\sigma_{22}<0$, which can be see in figure \ref{fig:slow_path_plane_stress3}, shows a similar behavior to that of figure \ref{fig:slow_path_plane_stress1}.
However, the paths depicted in the ($\sigma_{22},\sigma_{12}$) plane in figure \ref{fig:slow_path_plane_stress3}\subref{subfig:CP_slow_stress3} follow a direction opposite to those corresponding to the initial condition $\sigma_{22}=-5.8\times 10^7 \: Pa$.
%case in the $(\sigma_{22},\sigma_{12})$ plane, since $\sigma_{22}$ decreases rather than increases before the slopes of paths break (see figure \ref{fig:slow_path_plane_stress3}\subref{subfig:CP_slow_stress3}).
The previous remark is also valid in the deviatoric plane in figure \ref{fig:slow_path_plane_stress3}\subref{subfig:CP_slow_dev3}.
Indeed, the integral curves first describe clock-wise curved lines until the break in slopes occurs, after which a behavior close to straight lines is seen.
\begin{figure}[h!]
  \centering
  \subcaptionbox{Projections of loading paths in ($\sigma_{11},\sigma_{12}$) and ($\sigma_{22},\sigma_{12}$) planes \label{subfig:CP_slow_stress3}}{\begin{tikzpicture}[scale=0.9]
\begin{groupplot}[group style={group size=2 by 1,
ylabels at=edge left, yticklabels at=edge left,horizontal sep=3.ex,
xticklabels at=edge bottom,xlabels at=edge bottom},
ymajorgrids=true,xmajorgrids=true,ylabel=$\sigma_{12} \: (Pa)$,
axis on top,scale only axis,width=0.4\linewidth,ymin=0,ymax=109528891.788
, every x tick scale label/.style={at={(xticklabel* cs:1.05,0.75cm)},anchor=near yticklabel}
,colormap name =viridis]
\nextgroupplot[xlabel=$\sigma_{11} \: (Pa)$]
%\addplot[arrows along my path,black,thick] table[x=sigma_11,y=sigma_12] {chapter5/pgfFigures/pgf_slowWavesPlaneStress/CPslowStressPlane_frame0_Stress3.pgf};
\addplot[mesh,point meta = \thisrow{p},very thick,no markers] table[x=sigma_11,y=sigma_12] {chapter5/pgfFigures/pgf_slowWavesPlaneStress/CPslowStressPlane_frame0_Stress3.pgf} node[above,black] {$\textbf{1}$};
%\addplot[arrows along my path,black,thick] table[x=sigma_11,y=sigma_12] {chapter5/pgfFigures/pgf_slowWavesPlaneStress/CPslowStressPlane_frame1_Stress3.pgf};
\addplot[mesh,point meta = \thisrow{p},very thick,no markers] table[x=sigma_11,y=sigma_12] {chapter5/pgfFigures/pgf_slowWavesPlaneStress/CPslowStressPlane_frame1_Stress3.pgf} node[above,black] {$\textbf{2}$};
%\addplot[arrows along my path,black,thick] table[x=sigma_11,y=sigma_12] {chapter5/pgfFigures/pgf_slowWavesPlaneStress/CPslowStressPlane_frame2_Stress3.pgf};
\addplot[mesh,point meta = \thisrow{p},very thick,no markers] table[x=sigma_11,y=sigma_12] {chapter5/pgfFigures/pgf_slowWavesPlaneStress/CPslowStressPlane_frame2_Stress3.pgf} node[above,black] {$\textbf{3}$};
%\addplot[arrows along my path,black,thick] table[x=sigma_11,y=sigma_12] {chapter5/pgfFigures/pgf_slowWavesPlaneStress/CPslowStressPlane_frame3_Stress3.pgf};
\addplot[mesh,point meta = \thisrow{p},very thick,no markers] table[x=sigma_11,y=sigma_12] {chapter5/pgfFigures/pgf_slowWavesPlaneStress/CPslowStressPlane_frame3_Stress3.pgf} node[above,black] {$\textbf{4}$};
\addplot[gray,dashed,thin] table[x=sigma_11,y=sigma_12] {chapter5/pgfFigures/pgf_slowWavesPlaneStress/CPslow_yield0_s11s12_Stress3.pgf};

\nextgroupplot[colorbar,colorbar style={title= {$ c_s \: (m/s)$},every y tick scale label/.style={at={(2.,-.1125)}} },xlabel=$\sigma_{22} \: (Pa)$]
\addplot[arrows along my path,black!70,thick] table[x=sigma_22,y=sigma_12] {chapter5/pgfFigures/pgf_slowWavesPlaneStress/CPslowStressPlane_frame0_Stress3.pgf};\addplot[mesh,point meta = \thisrow{p},very thick,no markers] table[x=sigma_22,y=sigma_12] {chapter5/pgfFigures/pgf_slowWavesPlaneStress/CPslowStressPlane_frame0_Stress3.pgf} node[above,black] {$\textbf{1}$};
\addplot[arrows along my path,black!70,thick] table[x=sigma_22,y=sigma_12] {chapter5/pgfFigures/pgf_slowWavesPlaneStress/CPslowStressPlane_frame1_Stress3.pgf};\addplot[mesh,point meta = \thisrow{p},very thick,no markers] table[x=sigma_22,y=sigma_12] {chapter5/pgfFigures/pgf_slowWavesPlaneStress/CPslowStressPlane_frame1_Stress3.pgf} node[above,black] {$\textbf{2}$};
\addplot[arrows along my path,black!70,thick] table[x=sigma_22,y=sigma_12] {chapter5/pgfFigures/pgf_slowWavesPlaneStress/CPslowStressPlane_frame2_Stress3.pgf};\addplot[mesh,point meta = \thisrow{p},very thick,no markers] table[x=sigma_22,y=sigma_12] {chapter5/pgfFigures/pgf_slowWavesPlaneStress/CPslowStressPlane_frame2_Stress3.pgf} node[above,black] {$\textbf{3}$};
\addplot[arrows along my path,black!70,thick] table[x=sigma_22,y=sigma_12] {chapter5/pgfFigures/pgf_slowWavesPlaneStress/CPslowStressPlane_frame3_Stress3.pgf};\addplot[mesh,point meta = \thisrow{p},very thick,no markers] table[x=sigma_22,y=sigma_12] {chapter5/pgfFigures/pgf_slowWavesPlaneStress/CPslowStressPlane_frame3_Stress3.pgf} node[above,black] {$\textbf{4}$};
\end{groupplot}
\end{tikzpicture}
%%% Local Variables:
%%% mode: latex
%%% TeX-master: "../../mainManuscript"
%%% End:
}
  \subcaptionbox{Loading paths in deviatoric plane  \label{subfig:CP_slow_dev3}}{\tikzset{cross/.style={cross out, draw=black, minimum size=2*(#1-\pgflinewidth), inner sep=0pt, outer sep=0pt},cross/.default={2.5pt}}
\begin{tikzpicture}[scale=0.9]
\begin{axis}[width=.75\textwidth,view={135}{35.2643},xlabel=$s_1 $,ylabel=$s_2 $,zlabel=$s_3$,xmin=-1.e8,xmax=1.e8,ymin=-1.e8,ymax=1.e8,axis equal,axis lines=center,axis on top,xtick=\empty,ytick=\empty,ztick=\empty,every axis y label/.style={at={(rel axis cs:0.,.5,-0.65)}, anchor=west}, every axis x label/.style={at={(rel axis cs:0.5,.,-0.65)}, anchor=east}, every axis z label/.style={at={(rel axis cs:0.,.0,.18)}, anchor=north},legend style={at={(.225,.59)}}]
\node[below] at (1.1e8,0.,0.) {$\sigma^y$};
\node[above] at (-1.1e8,0.,0.) {$-\sigma^y$};
\draw (1.e8,0.,0.) node[cross,rotate=10] {};
\draw (-1.e8,0.,0.) node[cross,rotate=10] {};
\node[white]  at (0,0.,1.42e8) {};
\addplot3+[Red,mark=star,mark repeat=20,mark size=3pt,very thick] file {chapter5/pgfFigures/pgf_slowWavesPlaneStress/CPslowDevPlane_frame0_Stress3.pgf};
\addlegendentry{loading path 1}
\addplot3+[Blue,mark=asterisk,mark repeat=20,mark size=3pt,very thick] file {chapter5/pgfFigures/pgf_slowWavesPlaneStress/CPslowDevPlane_frame1_Stress3.pgf};
\addlegendentry{loading path 2}
\addplot3+[Orange,mark=+,mark repeat=20,mark size=3pt,very thick] file {chapter5/pgfFigures/pgf_slowWavesPlaneStress/CPslowDevPlane_frame2_Stress3.pgf};
\addlegendentry{loading path 3}
\addplot3+[Purple,mark=x,mark repeat=20,mark size=3pt,very thick] file {chapter5/pgfFigures/pgf_slowWavesPlaneStress/CPslowDevPlane_frame3_Stress3.pgf};
\addlegendentry{loading path 4}
\addplot3+[gray,dashed,thin,no markers] file {chapter5/pgfFigures/pgf_slowWavesPlaneStress/CPCylindreDevPlane.pgf};\addlegendentry{initial yield surface}
\newcommand\radius{1.*0.82e8}
\addplot3[dotted,thick] coordinates {(0.75*\radius,-0.75*\radius,0.) (-0.75*\radius,0.75*\radius,0.)};
\addplot3[dotted,thick] coordinates {(0.,-0.75*\radius,0.75*\radius) (0.,0.75*\radius,-0.75*\radius)};
\addplot3[dotted,thick] coordinates {(-0.75*\radius,0.,0.75*\radius) (0.75*\radius,0.,-0.75*\radius)};
\end{axis}
\end{tikzpicture}
%%% Local Variables:
%%% mode: latex
%%% TeX-master: "../../mainManuscript"
%%% End:
}
  \caption{Stress paths in a slow simple wave for various starting point lying on the initial yield surface for $\sigma_{22}=5.8\times 10^7 \: Pa$. Projections in the stress space (figure \subref{subfig:CP_slow_stress3}) and deviatoric plane (figure \subref{subfig:CP_slow_dev3}).}
  \label{fig:slow_path_plane_stress3}
\end{figure}
It is worth noticing that the driving parameter used for slow waves has not been chosen arbitrarily.
As a matter of fact, the numerical integration does not go well when driven with $\sigma_{11}$.
Furthermore, numerical issues occur if the starting point is such that $\sigma_{12}=0$.

More generally, the loading paths resulting from the integration of ODEs governing the behavior inside simple waves in plane stress can be summarized as follows.
Whereas the integral curves inside a fast wave first exhibits a phase in which the stress is restricted to the initial yield surface, the passage of a slow wave makes the stress leave the elastic convex quasi-instantaneously.
It is moreover noteworthy that the shear stress component $\sigma_{12}$ undergoes the biggest variation through a slow wave, in spite of the visible combined-stress nature of the corresponding paths.
In addition, roughs change in the slopes of the integral curves associated to slow waves occur.
However, such phenomena might also be observed for fast waves once the shear waves vanishes but have not been highlighted in figure \ref{fig:fast_path_plane_stress} due to numerical issues in integrating an undetermined function.
%Indeed, the theory (equation \eqref{eq:CP_roots}) predicts that this situation would lead to loading paths (locally) perpendicular to the yield surface rather than parallel.
Furthermore, after the slopes of slow waves integral curves broke, the paths are straight the deviatoric plane.

\subsection{Plane strain}
\label{sec:num_plane_strain}
Assuming that a solid initially at rest undergoes external loads leading to a plane strain case, the previous approach is now repeated.
However, the derivation of the hyperbolic system in a two-dimensional setting lies on the writing of the out-of-plane stress component as a function of plastic strain.
Hence, the integral curves associated to simple waves are integrated implicitly, along with the plastic flow.
To do so, the consistency condition (see section \ref{sec:constitutive-equations}) of the von-Mises yield surface \eqref{eq:ch5_von-Mises_yield} is written:
\begin{equation}
  \label{eq:consistance}
  \dot{f}=0 \quad \Leftrightarrow  \quad  \sqrt{\frac{3}{2}}\:\frac{\tens{s}:\tens{\dot{\sigma}}}{\norm{\tens{s}}}=C\dot{p}
\end{equation}
In addition, combination of the plastic flow rule \eqref{eq:plastic_strain_rate}: $\tens{\dot{\eps}}^p = \dot{p}\:\sqrt{\frac{3}{2}}\frac{\tens{s}}{\norm{\tens{s}}}$ ,
and the above consistency condition yields:
\begin{equation}
  \label{eq:flow_rule}
  \tens{\dot{\eps}}^p = \frac{3}{2C}\frac{\tens{s}\otimes\tens{s}}{\norm{\tens{s}}^2} :\tens{\dot{\sigma}}
\end{equation}
% so that the evolution of the plastic strain tensor is governed by: 
% \begin{equation}
%   \label{eq:flow_rule2}
%   \tens{\dot{\eps}}^p = \frac{3}{2C}\frac{\tens{s}\otimes\tens{s}}{\norm{\tens{s}}^2} :\tens{\dot{\sigma}}
% \end{equation}
Thus, the system of ODE consists of the equations of table \ref{tab:simpleWavesEquations}:
\begin{align}
  \label{eq:plane_strain_paths}
  & d\sigma_{11} = \psi_1^{s,f} d\sigma_{12} \\
  & d\sigma_{22} = -\frac{\psi^{s,f}_{1}\alpha_{11}+\alpha_{12}}{\alpha_{22}}d\sigma_{12}
\end{align}
along with the ODE related to the out-of-plane component:
\begin{equation}
  \label{eq:sig33_plane_strain}
  d\sigma_{33}= \nu\(d\sigma_{11}+d\sigma_{22}\) - E d\eps^p_{33}
\end{equation}

Since a fast wave propagates faster than a slow one, a material particle is first acted upon by the effects of the former. 
\begin{figure}[h!]
  \centering
  \subcaptionbox{Projections of loading paths in ($\sigma_{11},\sigma_{12}$) and ($\sigma_{22},\sigma_{12}$) planes \label{subfig:fastDP_stress}}{\begin{tikzpicture}[scale=0.9]
\begin{groupplot}[group style={group size=2 by 1,
ylabels at=edge left, yticklabels at=edge left,horizontal sep=3.ex,
xticklabels at=edge bottom,xlabels at=edge bottom},
ymajorgrids=true,xmajorgrids=true,ylabel=$\sigma_{12} \: (Pa)$,
axis on top,scale only axis,width=0.4\linewidth,ymin=0,ymax=100000000.0
, every x tick scale label/.style={at={(xticklabel* cs:1.05,0.75cm)},anchor=near yticklabel},colormap={ry}{rgb255(0cm)=(255,255,0);rgb255(1cm)=(255,0,0)}]
\nextgroupplot[xlabel=$\sigma_{11} (Pa)$]
\addplot[mesh,point meta = \thisrow{p},very thick,no markers] table[x=sigma_11,y=sigma_12] {chapter5/pgfFigures/pgf_fastWavesPlaneStrain/DPfastStressPlane_frame0_Stress0.pgf} node[above right,black] {$\textbf{1}$};
\addplot[mesh,point meta = \thisrow{p},very thick,no markers] table[x=sigma_11,y=sigma_12] {chapter5/pgfFigures/pgf_fastWavesPlaneStrain/DPfastStressPlane_frame1_Stress0.pgf} node[above right,black] {$\textbf{2}$};
\addplot[mesh,point meta = \thisrow{p},very thick,no markers] table[x=sigma_11,y=sigma_12] {chapter5/pgfFigures/pgf_fastWavesPlaneStrain/DPfastStressPlane_frame2_Stress0.pgf} node[above right,black] {$\textbf{3}$};
\addplot[mesh,point meta = \thisrow{p},very thick,no markers] table[x=sigma_11,y=sigma_12] {chapter5/pgfFigures/pgf_fastWavesPlaneStrain/DPfastStressPlane_frame3_Stress0.pgf} node[above right,black] {$\textbf{4}$};
\addplot[mesh,point meta = \thisrow{p},very thick,no markers] table[x=sigma_11,y=sigma_12] {chapter5/pgfFigures/pgf_fastWavesPlaneStrain/DPfastStressPlane_frame4_Stress0.pgf} node[above right,black] {$\textbf{5}$};
\addplot[mesh,point meta = \thisrow{p},very thick,no markers] table[x=sigma_11,y=sigma_12] {chapter5/pgfFigures/pgf_fastWavesPlaneStrain/DPfastStressPlane_frame5_Stress0.pgf} node[above right,black] {$\textbf{6}$};
\addplot[gray,dashed,thin] table[x=sigma_11,y=sigma_12] {chapter5/pgfFigures/pgf_fastWavesPlaneStrain/DPfast_yield0_s11s12_Stress0.pgf};

\nextgroupplot[colorbar,colorbar style={title= {$ c_f \: (m/s)$},every y tick scale label/.style={at={(2.,-.1125)}} },xlabel=$\sigma_{22}  (Pa)$]
\addplot[mesh,point meta = \thisrow{p},very thick,no markers] table[x=sigma_22,y=sigma_12] {chapter5/pgfFigures/pgf_fastWavesPlaneStrain/DPfastStressPlane_frame0_Stress0.pgf} node[above right,black] {$\textbf{1}$};
\addplot[mesh,point meta = \thisrow{p},very thick,no markers] table[x=sigma_22,y=sigma_12] {chapter5/pgfFigures/pgf_fastWavesPlaneStrain/DPfastStressPlane_frame1_Stress0.pgf} node[above right,black] {$\textbf{2}$};
\addplot[mesh,point meta = \thisrow{p},very thick,no markers] table[x=sigma_22,y=sigma_12] {chapter5/pgfFigures/pgf_fastWavesPlaneStrain/DPfastStressPlane_frame2_Stress0.pgf} node[above right,black] {$\textbf{3}$};
\addplot[mesh,point meta = \thisrow{p},very thick,no markers] table[x=sigma_22,y=sigma_12] {chapter5/pgfFigures/pgf_fastWavesPlaneStrain/DPfastStressPlane_frame3_Stress0.pgf} node[above right,black] {$\textbf{4}$};
\addplot[mesh,point meta = \thisrow{p},very thick,no markers] table[x=sigma_22,y=sigma_12] {chapter5/pgfFigures/pgf_fastWavesPlaneStrain/DPfastStressPlane_frame4_Stress0.pgf} node[above right,black] {$\textbf{5}$};
\addplot[mesh,point meta = \thisrow{p},very thick,no markers] table[x=sigma_22,y=sigma_12] {chapter5/pgfFigures/pgf_fastWavesPlaneStrain/DPfastStressPlane_frame5_Stress0.pgf} node[above right,black] {$\textbf{6}$};
\addplot[gray,dashed,thin] table[x=sigma_22,y=sigma_12] {chapter5/pgfFigures/pgf_fastWavesPlaneStrain/DPfast_yield0_s22s12_frame0_Stress0.pgf};

\addplot[gray,dashed,thin] table[x=sigma_22,y=sigma_12] {chapter5/pgfFigures/pgf_fastWavesPlaneStrain/DPfast_yield0_s22s12_frame1_Stress0.pgf};

\addplot[gray,dashed,thin] table[x=sigma_22,y=sigma_12] {chapter5/pgfFigures/pgf_fastWavesPlaneStrain/DPfast_yield0_s22s12_frame2_Stress0.pgf};

\addplot[gray,dashed,thin] table[x=sigma_22,y=sigma_12] {chapter5/pgfFigures/pgf_fastWavesPlaneStrain/DPfast_yield0_s22s12_frame3_Stress0.pgf};

\addplot[gray,dashed,thin] table[x=sigma_22,y=sigma_12] {chapter5/pgfFigures/pgf_fastWavesPlaneStrain/DPfast_yield0_s22s12_frame4_Stress0.pgf};

\addplot[gray,dashed,thin] table[x=sigma_22,y=sigma_12] {chapter5/pgfFigures/pgf_fastWavesPlaneStrain/DPfast_yield0_s22s12_frame5_Stress0.pgf};

\end{groupplot}
\end{tikzpicture}
%%% Local Variables:
%%% mode: latex
%%% TeX-master: "../../mainManuscript"
%%% End:
}
  \subcaptionbox{Loading path in deviatoric plane \label{subfig:fastDP_dev}}{\tikzset{cross/.style={cross out, draw=black, minimum size=2*(#1-\pgflinewidth), inner sep=0pt, outer sep=0pt},cross/.default={2.5pt}}
\begin{tikzpicture}[spy using outlines={rectangle, magnification=3, size=2.cm, connect spies},scale=0.9]
\begin{axis}[width=.75\textwidth,view={135}{35.2643},xlabel=$s_1 $,ylabel=$s_2 $,zlabel=$s_3$,xmin=-1.e8,xmax=1.e8,ymin=-1.e8,ymax=1.e8,axis equal,axis lines=center,axis on top,xtick=\empty,ytick=\empty,ztick=\empty,every axis y label/.style={at={(rel axis cs:0.,.5,-0.65)}, anchor=west}, every axis x label/.style={at={(rel axis cs:0.5,.,-0.65)}, anchor=east}, every axis z label/.style={at={(rel axis cs:0.,.0,.18)}, anchor=north},legend style={at={(.2,.68)}}]
\node[below] at (1.1e8,0.,0.) {$\sigma^y$};
\node[above] at (-1.1e8,0.,0.) {$-\sigma^y$};
\draw (1.e8,0.,0.) node[cross,rotate=10] {};
\draw (-1.e8,0.,0.) node[cross,rotate=10] {};
\node[white]  at (0,0.,1.42e8) {};
\addplot3[Red,thick,arrows along my path] file {chapter5/pgfFigures/pgf_fastWavesPlaneStrain/DPfastDevPlane_frame0_Stress0.pgf};\addlegendentry{loading path 1}
\addplot3[Blue,thick,arrows along my path] file {chapter5/pgfFigures/pgf_fastWavesPlaneStrain/DPfastDevPlane_frame1_Stress0.pgf};\addlegendentry{loading path 2}
\addplot3[Orange,thick,arrows along my path] file {chapter5/pgfFigures/pgf_fastWavesPlaneStrain/DPfastDevPlane_frame2_Stress0.pgf};\addlegendentry{loading path 3}
\addplot3[Purple,thick,arrows along my path] file {chapter5/pgfFigures/pgf_fastWavesPlaneStrain/DPfastDevPlane_frame3_Stress0.pgf};\addlegendentry{loading path 4}
\addplot3[Green,thick,arrows along my path] file {chapter5/pgfFigures/pgf_fastWavesPlaneStrain/DPfastDevPlane_frame4_Stress0.pgf};\addlegendentry{loading path 5}
\addplot3[Duck,thick,arrows along my path] file {chapter5/pgfFigures/pgf_fastWavesPlaneStrain/DPfastDevPlane_frame5_Stress0.pgf};\addlegendentry{loading path 6}
\addplot3+[gray,dashed,thin,no markers] file {chapter5/pgfFigures/pgf_fastWavesPlaneStrain/CylindreDevPlane.pgf};\addlegendentry{initial yield surface}
\newcommand\radius{1.*0.82e8}
\addplot3[dotted,thick] coordinates {(0.75*\radius,-0.75*\radius,0.) (-0.75*\radius,0.75*\radius,0.)};
\addplot3[dotted,thick] coordinates {(0.,-0.75*\radius,0.75*\radius) (0.,0.75*\radius,-0.75*\radius)};
\addplot3[dotted,thick] coordinates {(-0.75*\radius,0.,0.75*\radius) (0.75*\radius,0.,-0.75*\radius)};
\begin{scope}
\spy[black,size=1.75cm] on (6.75,3.2) in node [fill=none] at (9.5,5.5);
\end{scope}

\end{axis}
\end{tikzpicture}
%%% Local Variables:
%%% mode: latex
%%% TeX-master: "../../mainManuscript"
%%% End:
}
  \caption{Loading paths through a fast simple wave with initial condition $\sigma_{22}=0$ for different starting points on the initial yield surface.}
  \label{fig:fast_path_plane_strains}
\end{figure}
Thus, figure \ref{fig:fast_path_plane_strains} shows the evolution of stresses resulting from numerical integration using $\sigma_{11}$ as driving parameter by rearranging equations \eqref{eq:plane_strain_paths}.
Several starting points on the initial yield surface are considered.
The evolution of the celerity of fast waves along the integral curve confirms the validity of the simple wave solution.
Furthermore, the starting points are chosen in such a way that a symmetry of the loading path with respect to $\sigma_{11}=0$ and $\sigma_{22}=0$ planes is notable.
Although the stress paths depicted in figure \ref{fig:fast_path_plane_strains}\subref{subfig:fastDP_stress} are rather different to these resulting from a fast wave in plane stress, the behaviors in the deviatoric plane are similar as can be seen in figure \ref{fig:fast_path_plane_strains}\subref{subfig:fastDP_dev}. 
Indeed, the fact that the computed loading paths through a fast wave is parallel to the initial yield surface is obvious when looking at the deviatoric plane.
More specifically, the von-Mises circle is traced by the integral curve even when the shear component $\sigma_{12}$ is zero (see paths $1$ and $6$ in figure \ref{fig:fast_path_plane_strains}).
Notice that the integration of loading path in the plane $\sigma_{12}=0$ is here possible contrary to plane stresses.
The integral curves of figure \ref{fig:fast_path_plane_strains} however exhibit a cusp which is not explained (see the two external arrows pointing toward axes of pure tensile/compression).
%Nevertheless, the results of figure \ref{fig:fast_path_plane_strains} must be taken carefully since the loading paths exhibits a cusp that is not explained so far (see the two external arrows pointing toward axes of pure tensile/compression).


%% For slow waves, integrated by driving with a decreasing \tau yields an increasing celerity (it goes for fast waves driven with tau in plane stress)
On the other hand, some loading paths resulting from the integration of slow waves ODEs are depicted in figures \ref{fig:slow_path_plane_strains1}, \ref{fig:slow_path_plane_strains2} and \ref{fig:slow_path_plane_strains3}.
Analogously to plane stress cases, three initial values $\sigma_{22}=-1.3 \times 10^{8} \: Pa$, $\sigma_{22}=0$ and $\sigma_{22}=1.3 \times 10^8 \: Pa$ are considered since a fast wave may modify the initial free-stress state.
The integration of loading paths through slow waves in plane strain is performed by using $\sigma_{12}$ as driving parameter.
However, numerical difficulties arise owing to the characteristic speed associated to slow waves which starts increasing rather than decreasing at some point along the path.
In order to circumvent this issue, the last stress state leading to a decreasing celerity is used as an initial condition for a second integration driven by means of $\sigma_{11}$.
The final value is set so that the variation of $\sigma_{11}$ (\textit{i.e increasing or decreasing}) undergone up to that singularity is continued.
This strategy enables to carry on the integration further.
Nevertheless, the same problem of increasing characteristic speed again occurs and the computation must be aborted.
\begin{figure}[h!]
  \centering
  \subcaptionbox{Projections of loading paths in ($\sigma_{11},\sigma_{12}$) and ($\sigma_{22},\sigma_{12}$) planes \label{subfig:slowDP_stress1}}{\begin{tikzpicture}[scale=0.9]
\begin{groupplot}[group style={group size=2 by 1,
ylabels at=edge left, yticklabels at=edge left,horizontal sep=3.ex,
xticklabels at=edge bottom,xlabels at=edge bottom},
ymajorgrids=true,xmajorgrids=true,ylabel=$\sigma_{12} \: (Pa)$,
axis on top,scale only axis,width=0.45\linewidth,ymin=0,ymax=68618075.3103
, every x tick scale label/.style={at={(xticklabel* cs:1.05,0.75cm)},anchor=near yticklabel}]
\nextgroupplot[xlabel=$\sigma_{11} (Pa)$]
\addplot[mesh,point meta = \thisrow{p},very thick,no markers] table[x=sigma_11,y=sigma_12] {chapter5/pgfFigures/pgf_slowWavesPlaneStrain/DPslowStressPlane_frame0_Stress1.pgf};
\addplot[mesh,point meta = \thisrow{p},very thick,no markers] table[x=sigma_11,y=sigma_12] {chapter5/pgfFigures/pgf_slowWavesPlaneStrain/DPslowStressPlane_frame1_Stress1.pgf};
\addplot[mesh,point meta = \thisrow{p},very thick,no markers] table[x=sigma_11,y=sigma_12] {chapter5/pgfFigures/pgf_slowWavesPlaneStrain/DPslowStressPlane_frame2_Stress1.pgf};
\addplot[mesh,point meta = \thisrow{p},very thick,no markers] table[x=sigma_11,y=sigma_12] {chapter5/pgfFigures/pgf_slowWavesPlaneStrain/DPslowStressPlane_frame3_Stress1.pgf};
\addplot[gray,thin] table[x=sigma_11,y=sigma_12] {chapter5/pgfFigures/pgf_slowWavesPlaneStrain/DPslow_yield0_s11s12_Stress1.pgf};

\nextgroupplot[colorbar,colorbar style={title= {$\rho c^2$},every y tick scale label/.style={at={(2.,-.1125)}} },xlabel=$\sigma_{22}  (Pa)$]
\addplot[mesh,point meta = \thisrow{p},very thick,no markers] table[x=sigma_22,y=sigma_12] {chapter5/pgfFigures/pgf_slowWavesPlaneStrain/DPslowStressPlane_frame0_Stress1.pgf};
\addplot[mesh,point meta = \thisrow{p},very thick,no markers] table[x=sigma_22,y=sigma_12] {chapter5/pgfFigures/pgf_slowWavesPlaneStrain/DPslowStressPlane_frame1_Stress1.pgf};
\addplot[mesh,point meta = \thisrow{p},very thick,no markers] table[x=sigma_22,y=sigma_12] {chapter5/pgfFigures/pgf_slowWavesPlaneStrain/DPslowStressPlane_frame2_Stress1.pgf};
\addplot[mesh,point meta = \thisrow{p},very thick,no markers] table[x=sigma_22,y=sigma_12] {chapter5/pgfFigures/pgf_slowWavesPlaneStrain/DPslowStressPlane_frame3_Stress1.pgf};
\end{groupplot}
\end{tikzpicture}
%%% Local Variables:
%%% mode: latex
%%% TeX-master: "../../mainManuscript"
%%% End:
}
  \subcaptionbox{Loading path in deviatoric plane \label{subfig:slowDP_dev1}}{\begin{tikzpicture}[scale=0.9]
\begin{axis}[width=.75\textwidth,view={135}{35.2643},xlabel=$s_1 $,ylabel=$s_2 $,zlabel=$s_3$,xmin=-1.e8,xmax=1.e8,ymin=-1.e8,ymax=1.e8,axis equal,axis lines=center,axis on top,ztick=\empty]
\addplot3+[Red,very thick,no markers] file {chapter5/pgfFigures/pgf_slowWavesPlaneStrain/DPslowDevPlane_frame0_Stress1.pgf};
\addplot3+[Blue,very thick,no markers] file {chapter5/pgfFigures/pgf_slowWavesPlaneStrain/DPslowDevPlane_frame1_Stress1.pgf};
\addplot3+[Orange,very thick,no markers] file {chapter5/pgfFigures/pgf_slowWavesPlaneStrain/DPslowDevPlane_frame2_Stress1.pgf};
\addplot3+[Purple,very thick,no markers] file {chapter5/pgfFigures/pgf_slowWavesPlaneStrain/DPslowDevPlane_frame3_Stress1.pgf};
\addplot3+[gray,dashed,thin,no markers] file {chapter5/pgfFigures/pgf_slowWavesPlaneStrain/CylindreDevPlane.pgf};
\end{axis}
\end{tikzpicture}
%%% Local Variables:
%%% mode: latex
%%% TeX-master: "../../mainManuscript"
%%% End:
}
  \caption{Loading paths through slow simple waves for different starting points on the initial yield surface for the initial condition $\sigma_{22}=-1.3 \times 10^{8} \: Pa$.}
  \label{fig:slow_path_plane_strains1}
\end{figure}
The integral curves depicted in figure \ref{fig:slow_path_plane_strains1} results from the negative initial value $\sigma_{22}=-1.3 \times 10^{8} \: Pa$ for several starting points on the initial yield surface.

Whereas $\sigma_{11}$ varies little as shown by the projection of the path in the ($\sigma_{11},\sigma_{12}$) plane, it is not the case for $\sigma_{22}$ (see figure \ref{fig:slow_path_plane_strains1}\subref{subfig:slowDP_stress1}).
Indeed, the projections of the integral curves in ($\sigma_{22},\sigma_{12}$) plane exhibit complex paths similar to these obtained for plane stress.
In contrast, it can be seen in figure \ref{fig:slow_path_plane_strains1}\subref{subfig:slowDP_dev1} that the paths first follow the initial yield surface and next a direction of pure shear in the deviatoric plane, which differs from plane stress solutions.
Even though the paths for plane stress and plane strain have similar shapes in the stress space, one cannot expect the same observations in the deviatoric plane due to the out-of-plane stress component.
%The fact that the curves have the same shape in stress space and not in the deviatoric plane can be explained by the out-of-plane stress that is zero are not.
Namely, the loading paths are restricted to the $(\sigma_1,\sigma_2)$ plane under plane stress, while they can take values in the whole space ($\sigma_1,\sigma_2,\sigma_3$) under plane strain.

The same behavior is observed in figures \ref{fig:slow_path_plane_strains2} and \ref{fig:slow_path_plane_strains3} which respectively show the loading paths resulting form the zero and the positive initial values of $\sigma_{22}$.
\begin{figure}[h!]
  \centering
  % \subcaptionbox{Projections of loading paths in ($\sigma_{11},\sigma_{12}$) and ($\sigma_{22},\sigma_{12}$) planes \label{subfig:slowDP_stress2}}{\begin{tikzpicture}[scale=0.9]
\begin{groupplot}[group style={group size=2 by 1,
ylabels at=edge left, yticklabels at=edge left,horizontal sep=3.ex,
xticklabels at=edge bottom,xlabels at=edge bottom},
ymajorgrids=true,xmajorgrids=true,ylabel=$\sigma_{12} \: (Pa)$,
axis on top,scale only axis,width=0.4\linewidth,ymin=0,ymax=94979909.10761759
, every x tick scale label/.style={at={(xticklabel* cs:1.05,0.75cm)},anchor=near yticklabel},colormap name=viridis]
\nextgroupplot[xlabel=$\sigma_{11} (Pa)$]
\addplot[mesh,point meta = \thisrow{p},very thick,no markers] table[x=sigma_11,y=sigma_12] {chapter5/pgfFigures/pgf_slowWavesPlaneStrain/DPslowStressPlane_frame0_Stress2.pgf} node[above,black] {$\textbf{1}$};
\addplot[mesh,point meta = \thisrow{p},very thick,no markers] table[x=sigma_11,y=sigma_12] {chapter5/pgfFigures/pgf_slowWavesPlaneStrain/DPslowStressPlane_frame1_Stress2.pgf} node[above,black] {$\textbf{2}$};
\addplot[mesh,point meta = \thisrow{p},very thick,no markers] table[x=sigma_11,y=sigma_12] {chapter5/pgfFigures/pgf_slowWavesPlaneStrain/DPslowStressPlane_frame2_Stress2.pgf} node[above,black] {$\textbf{3}$};
\addplot[mesh,point meta = \thisrow{p},very thick,no markers] table[x=sigma_11,y=sigma_12] {chapter5/pgfFigures/pgf_slowWavesPlaneStrain/DPslowStressPlane_frame3_Stress2.pgf} node[above,black] {$\textbf{4}$};
\addplot[gray,dashed,thin] table[x=sigma_11,y=sigma_12] {chapter5/pgfFigures/pgf_slowWavesPlaneStrain/DPslow_yield0_s11s12_Stress2.pgf};

\addplot[gray,dashed,thin] table[x=sigma_11,y=sigma_12] {chapter5/pgfFigures/pgf_slowWavesPlaneStrain/DPslow_yieldfin_s11s12_frame0_Stress2.pgf};

\addplot[gray,dashed,thin] table[x=sigma_11,y=sigma_12] {chapter5/pgfFigures/pgf_slowWavesPlaneStrain/DPslow_yieldfin_s11s12_frame1_Stress2.pgf};

\addplot[gray,dashed,thin] table[x=sigma_11,y=sigma_12] {chapter5/pgfFigures/pgf_slowWavesPlaneStrain/DPslow_yieldfin_s11s12_frame2_Stress2.pgf};

\addplot[gray,dashed,thin] table[x=sigma_11,y=sigma_12] {chapter5/pgfFigures/pgf_slowWavesPlaneStrain/DPslow_yieldfin_s11s12_frame3_Stress2.pgf};

\nextgroupplot[colorbar,colorbar style={title= {$ c_s \: (m/s)$},every y tick scale label/.style={at={(2.,-.1125)}} },xlabel=$\sigma_{22}  (Pa)$]
\addplot[mesh,point meta = \thisrow{p},very thick,no markers] table[x=sigma_22,y=sigma_12] {chapter5/pgfFigures/pgf_slowWavesPlaneStrain/DPslowStressPlane_frame0_Stress2.pgf} node[above,black] {$\textbf{1}$};
\addplot[mesh,point meta = \thisrow{p},very thick,no markers] table[x=sigma_22,y=sigma_12] {chapter5/pgfFigures/pgf_slowWavesPlaneStrain/DPslowStressPlane_frame1_Stress2.pgf} node[above,black] {$\textbf{2}$};
\addplot[mesh,point meta = \thisrow{p},very thick,no markers] table[x=sigma_22,y=sigma_12] {chapter5/pgfFigures/pgf_slowWavesPlaneStrain/DPslowStressPlane_frame2_Stress2.pgf} node[above,black] {$\textbf{3}$};
\addplot[mesh,point meta = \thisrow{p},very thick,no markers] table[x=sigma_22,y=sigma_12] {chapter5/pgfFigures/pgf_slowWavesPlaneStrain/DPslowStressPlane_frame3_Stress2.pgf} node[above,black] {$\textbf{4}$};
\addplot[gray,dashed,thin] table[x=sigma_22,y=sigma_12] {chapter5/pgfFigures/pgf_slowWavesPlaneStrain/DPslow_yieldfin_s22s12_frame0_Stress2.pgf};

\addplot[gray,dashed,thin] table[x=sigma_22,y=sigma_12] {chapter5/pgfFigures/pgf_slowWavesPlaneStrain/DPslow_yieldfin_s22s12_frame1_Stress2.pgf};

\addplot[gray,dashed,thin] table[x=sigma_22,y=sigma_12] {chapter5/pgfFigures/pgf_slowWavesPlaneStrain/DPslow_yieldfin_s22s12_frame2_Stress2.pgf};

\addplot[gray,dashed,thin] table[x=sigma_22,y=sigma_12] {chapter5/pgfFigures/pgf_slowWavesPlaneStrain/DPslow_yieldfin_s22s12_frame3_Stress2.pgf};

\end{groupplot}
\end{tikzpicture}
%%% Local Variables:
%%% mode: latex
%%% TeX-master: "../../mainManuscript"
%%% End:
}
  % \subcaptionbox{Loading path in deviatoric plane \label{subfig:slowDP_dev2}}{\tikzset{cross/.style={cross out, draw=black, minimum size=2*(#1-\pgflinewidth), inner sep=0pt, outer sep=0pt},cross/.default={2.5pt}}
\begin{tikzpicture}[scale=0.9]
\begin{axis}[width=.75\textwidth,view={135}{35.2643},xlabel=$s_1 $,ylabel=$s_2 $,zlabel=$s_3$,xmin=-1.e8,xmax=1.e8,ymin=-1.e8,ymax=1.e8,axis equal,axis lines=center,axis on top,xtick=\empty,ytick=\empty,ztick=\empty,every axis y label/.style={at={(rel axis cs:0.,.5,-0.65)}, anchor=west}, every axis x label/.style={at={(rel axis cs:0.5,.,-0.65)}, anchor=east}, every axis z label/.style={at={(rel axis cs:0.,.0,.18)}, anchor=north},legend style={at={(.2,.68)}}]
\node[below] at (1.1e8,0.,0.) {$\sigma^y$};
\node[above] at (-1.1e8,0.,0.) {$-\sigma^y$};
\draw (1.e8,0.,0.) node[cross,rotate=10] {};
\draw (-1.e8,0.,0.) node[cross,rotate=10] {};
\node[white]  at (0,0.,1.1e8) {};
\addplot3[arrows along my path,Red,very thick] file {chapter5/pgfFigures/pgf_slowWavesPlaneStrain/DPslowDevPlane_frame0_Stress2.pgf};\addlegendentry{loading path 1}
\addplot3[arrows along my path,Blue,very thick] file {chapter5/pgfFigures/pgf_slowWavesPlaneStrain/DPslowDevPlane_frame1_Stress2.pgf};\addlegendentry{loading path 2}
\addplot3[arrows along my path,Orange,very thick] file {chapter5/pgfFigures/pgf_slowWavesPlaneStrain/DPslowDevPlane_frame2_Stress2.pgf};\addlegendentry{loading path 3}
\addplot3[arrows along my path,Purple,very thick] file {chapter5/pgfFigures/pgf_slowWavesPlaneStrain/DPslowDevPlane_frame3_Stress2.pgf};\addlegendentry{loading path 4}
\addplot3[arrows along my path,Green,very thick] file {chapter5/pgfFigures/pgf_slowWavesPlaneStrain/DPslowDevPlane_frame4_Stress2.pgf};\addlegendentry{loading path 5}
\addplot3+[gray,dashed,thin,no markers] file {chapter5/pgfFigures/pgf_slowWavesPlaneStrain/CylindreDevPlane.pgf};\addlegendentry{initial yield surface}
\newcommand\radius{1.*0.82e8}
\addplot3[dotted,thick] coordinates {(0.75*\radius,-0.75*\radius,0.) (-0.75*\radius,0.75*\radius,0.)};
\addplot3[dotted,thick] coordinates {(0.,-0.75*\radius,0.75*\radius) (0.,0.75*\radius,-0.75*\radius)};
\addplot3[dotted,thick] coordinates {(-0.75*\radius,0.,0.75*\radius) (0.75*\radius,0.,-0.75*\radius)};
\end{axis}
\end{tikzpicture}
%%% Local Variables:
%%% mode: latex
%%% TeX-master: "../../mainManuscript"
%%% End:
}
  {\begin{tikzpicture}[scale=0.9]
\begin{groupplot}[group style={group size=2 by 1,
ylabels at=edge left, yticklabels at=edge left,horizontal sep=3.ex,
xticklabels at=edge bottom,xlabels at=edge bottom},
ymajorgrids=true,xmajorgrids=true,ylabel=$\sigma_{12} \: (Pa)$,
axis on top,scale only axis,width=0.4\linewidth,ymin=0,ymax=94979909.10761759
, every x tick scale label/.style={at={(xticklabel* cs:1.05,0.75cm)},anchor=near yticklabel},colormap name=viridis]
\nextgroupplot[xlabel=$\sigma_{11} (Pa)$]
\addplot[mesh,point meta = \thisrow{p},very thick,no markers] table[x=sigma_11,y=sigma_12] {chapter5/pgfFigures/pgf_slowWavesPlaneStrain/DPslowStressPlane_frame0_Stress2.pgf} node[above,black] {$\textbf{1}$};
\addplot[mesh,point meta = \thisrow{p},very thick,no markers] table[x=sigma_11,y=sigma_12] {chapter5/pgfFigures/pgf_slowWavesPlaneStrain/DPslowStressPlane_frame1_Stress2.pgf} node[above,black] {$\textbf{2}$};
\addplot[mesh,point meta = \thisrow{p},very thick,no markers] table[x=sigma_11,y=sigma_12] {chapter5/pgfFigures/pgf_slowWavesPlaneStrain/DPslowStressPlane_frame2_Stress2.pgf} node[above,black] {$\textbf{3}$};
\addplot[mesh,point meta = \thisrow{p},very thick,no markers] table[x=sigma_11,y=sigma_12] {chapter5/pgfFigures/pgf_slowWavesPlaneStrain/DPslowStressPlane_frame3_Stress2.pgf} node[above,black] {$\textbf{4}$};
\addplot[gray,dashed,thin] table[x=sigma_11,y=sigma_12] {chapter5/pgfFigures/pgf_slowWavesPlaneStrain/DPslow_yield0_s11s12_Stress2.pgf};

\addplot[gray,dashed,thin] table[x=sigma_11,y=sigma_12] {chapter5/pgfFigures/pgf_slowWavesPlaneStrain/DPslow_yieldfin_s11s12_frame0_Stress2.pgf};

\addplot[gray,dashed,thin] table[x=sigma_11,y=sigma_12] {chapter5/pgfFigures/pgf_slowWavesPlaneStrain/DPslow_yieldfin_s11s12_frame1_Stress2.pgf};

\addplot[gray,dashed,thin] table[x=sigma_11,y=sigma_12] {chapter5/pgfFigures/pgf_slowWavesPlaneStrain/DPslow_yieldfin_s11s12_frame2_Stress2.pgf};

\addplot[gray,dashed,thin] table[x=sigma_11,y=sigma_12] {chapter5/pgfFigures/pgf_slowWavesPlaneStrain/DPslow_yieldfin_s11s12_frame3_Stress2.pgf};

\nextgroupplot[colorbar,colorbar style={title= {$ c_s \: (m/s)$},every y tick scale label/.style={at={(2.,-.1125)}} },xlabel=$\sigma_{22}  (Pa)$]
\addplot[mesh,point meta = \thisrow{p},very thick,no markers] table[x=sigma_22,y=sigma_12] {chapter5/pgfFigures/pgf_slowWavesPlaneStrain/DPslowStressPlane_frame0_Stress2.pgf} node[above,black] {$\textbf{1}$};
\addplot[mesh,point meta = \thisrow{p},very thick,no markers] table[x=sigma_22,y=sigma_12] {chapter5/pgfFigures/pgf_slowWavesPlaneStrain/DPslowStressPlane_frame1_Stress2.pgf} node[above,black] {$\textbf{2}$};
\addplot[mesh,point meta = \thisrow{p},very thick,no markers] table[x=sigma_22,y=sigma_12] {chapter5/pgfFigures/pgf_slowWavesPlaneStrain/DPslowStressPlane_frame2_Stress2.pgf} node[above,black] {$\textbf{3}$};
\addplot[mesh,point meta = \thisrow{p},very thick,no markers] table[x=sigma_22,y=sigma_12] {chapter5/pgfFigures/pgf_slowWavesPlaneStrain/DPslowStressPlane_frame3_Stress2.pgf} node[above,black] {$\textbf{4}$};
\addplot[gray,dashed,thin] table[x=sigma_22,y=sigma_12] {chapter5/pgfFigures/pgf_slowWavesPlaneStrain/DPslow_yieldfin_s22s12_frame0_Stress2.pgf};

\addplot[gray,dashed,thin] table[x=sigma_22,y=sigma_12] {chapter5/pgfFigures/pgf_slowWavesPlaneStrain/DPslow_yieldfin_s22s12_frame1_Stress2.pgf};

\addplot[gray,dashed,thin] table[x=sigma_22,y=sigma_12] {chapter5/pgfFigures/pgf_slowWavesPlaneStrain/DPslow_yieldfin_s22s12_frame2_Stress2.pgf};

\addplot[gray,dashed,thin] table[x=sigma_22,y=sigma_12] {chapter5/pgfFigures/pgf_slowWavesPlaneStrain/DPslow_yieldfin_s22s12_frame3_Stress2.pgf};

\end{groupplot}
\end{tikzpicture}
%%% Local Variables:
%%% mode: latex
%%% TeX-master: "../../mainManuscript"
%%% End:
}
  {\tikzset{cross/.style={cross out, draw=black, minimum size=2*(#1-\pgflinewidth), inner sep=0pt, outer sep=0pt},cross/.default={2.5pt}}
\begin{tikzpicture}[scale=0.9]
\begin{axis}[width=.75\textwidth,view={135}{35.2643},xlabel=$s_1 $,ylabel=$s_2 $,zlabel=$s_3$,xmin=-1.e8,xmax=1.e8,ymin=-1.e8,ymax=1.e8,axis equal,axis lines=center,axis on top,xtick=\empty,ytick=\empty,ztick=\empty,every axis y label/.style={at={(rel axis cs:0.,.5,-0.65)}, anchor=west}, every axis x label/.style={at={(rel axis cs:0.5,.,-0.65)}, anchor=east}, every axis z label/.style={at={(rel axis cs:0.,.0,.18)}, anchor=north},legend style={at={(.2,.68)}}]
\node[below] at (1.1e8,0.,0.) {$\sigma^y$};
\node[above] at (-1.1e8,0.,0.) {$-\sigma^y$};
\draw (1.e8,0.,0.) node[cross,rotate=10] {};
\draw (-1.e8,0.,0.) node[cross,rotate=10] {};
\node[white]  at (0,0.,1.1e8) {};
\addplot3[arrows along my path,Red,very thick] file {chapter5/pgfFigures/pgf_slowWavesPlaneStrain/DPslowDevPlane_frame0_Stress2.pgf};\addlegendentry{loading path 1}
\addplot3[arrows along my path,Blue,very thick] file {chapter5/pgfFigures/pgf_slowWavesPlaneStrain/DPslowDevPlane_frame1_Stress2.pgf};\addlegendentry{loading path 2}
\addplot3[arrows along my path,Orange,very thick] file {chapter5/pgfFigures/pgf_slowWavesPlaneStrain/DPslowDevPlane_frame2_Stress2.pgf};\addlegendentry{loading path 3}
\addplot3[arrows along my path,Purple,very thick] file {chapter5/pgfFigures/pgf_slowWavesPlaneStrain/DPslowDevPlane_frame3_Stress2.pgf};\addlegendentry{loading path 4}
\addplot3[arrows along my path,Green,very thick] file {chapter5/pgfFigures/pgf_slowWavesPlaneStrain/DPslowDevPlane_frame4_Stress2.pgf};\addlegendentry{loading path 5}
\addplot3+[gray,dashed,thin,no markers] file {chapter5/pgfFigures/pgf_slowWavesPlaneStrain/CylindreDevPlane.pgf};\addlegendentry{initial yield surface}
\newcommand\radius{1.*0.82e8}
\addplot3[dotted,thick] coordinates {(0.75*\radius,-0.75*\radius,0.) (-0.75*\radius,0.75*\radius,0.)};
\addplot3[dotted,thick] coordinates {(0.,-0.75*\radius,0.75*\radius) (0.,0.75*\radius,-0.75*\radius)};
\addplot3[dotted,thick] coordinates {(-0.75*\radius,0.,0.75*\radius) (0.75*\radius,0.,-0.75*\radius)};
\end{axis}
\end{tikzpicture}
%%% Local Variables:
%%% mode: latex
%%% TeX-master: "../../mainManuscript"
%%% End:
}
  \caption{Loading paths through slow simple waves for different starting points on the initial yield surface for the initial condition $\sigma_{22}=0$.}
  \label{fig:slow_path_plane_strains2}
\end{figure}
\begin{figure}[h!]
  \centering
  % \subcaptionbox{Projections of loading paths in ($\sigma_{11},\sigma_{12}$) and ($\sigma_{22},\sigma_{12}$) planes \label{subfig:slowDP_stress3}}{\begin{tikzpicture}[scale=0.9]
\begin{groupplot}[group style={group size=2 by 1,
ylabels at=edge left, yticklabels at=edge left,horizontal sep=3.ex,
xticklabels at=edge bottom,xlabels at=edge bottom},
ymajorgrids=true,xmajorgrids=true,ylabel=$\sigma_{12} \: (Pa)$,
axis on top,scale only axis,width=0.4\linewidth,ymin=0,ymax=68618075.3102588
, every x tick scale label/.style={at={(xticklabel* cs:1.05,0.75cm)},anchor=near yticklabel},colormap={ry}{rgb255(0cm)=(255,255,0);rgb255(1cm)=(255,0,0)}]
\nextgroupplot[xlabel=$\sigma_{11} (Pa)$]
\addplot[mesh,point meta = \thisrow{p},very thick,no markers] table[x=sigma_11,y=sigma_12] {chapter5/pgfFigures/pgf_slowWavesPlaneStrain/DPslowStressPlane_frame0_Stress3.pgf} node[above right,black] {$\textbf{1}$};
\addplot[mesh,point meta = \thisrow{p},very thick,no markers] table[x=sigma_11,y=sigma_12] {chapter5/pgfFigures/pgf_slowWavesPlaneStrain/DPslowStressPlane_frame1_Stress3.pgf} node[above right,black] {$\textbf{2}$};
\addplot[mesh,point meta = \thisrow{p},very thick,no markers] table[x=sigma_11,y=sigma_12] {chapter5/pgfFigures/pgf_slowWavesPlaneStrain/DPslowStressPlane_frame2_Stress3.pgf} node[above right,black] {$\textbf{3}$};
\addplot[mesh,point meta = \thisrow{p},very thick,no markers] table[x=sigma_11,y=sigma_12] {chapter5/pgfFigures/pgf_slowWavesPlaneStrain/DPslowStressPlane_frame3_Stress3.pgf} node[above right,black] {$\textbf{4}$};
\addplot[gray,dashed,thin] table[x=sigma_11,y=sigma_12] {chapter5/pgfFigures/pgf_slowWavesPlaneStrain/DPslow_yield0_s11s12_Stress3.pgf};

\nextgroupplot[colorbar,colorbar style={title= {$ c_s \: (m/s)$},every y tick scale label/.style={at={(2.,-.1125)}} },xlabel=$\sigma_{22}  (Pa)$]
\addplot[mesh,point meta = \thisrow{p},very thick,no markers] table[x=sigma_22,y=sigma_12] {chapter5/pgfFigures/pgf_slowWavesPlaneStrain/DPslowStressPlane_frame0_Stress3.pgf} node[above right,black] {$\textbf{1}$};
\addplot[mesh,point meta = \thisrow{p},very thick,no markers] table[x=sigma_22,y=sigma_12] {chapter5/pgfFigures/pgf_slowWavesPlaneStrain/DPslowStressPlane_frame1_Stress3.pgf} node[above right,black] {$\textbf{2}$};
\addplot[mesh,point meta = \thisrow{p},very thick,no markers] table[x=sigma_22,y=sigma_12] {chapter5/pgfFigures/pgf_slowWavesPlaneStrain/DPslowStressPlane_frame2_Stress3.pgf} node[above right,black] {$\textbf{3}$};
\addplot[mesh,point meta = \thisrow{p},very thick,no markers] table[x=sigma_22,y=sigma_12] {chapter5/pgfFigures/pgf_slowWavesPlaneStrain/DPslowStressPlane_frame3_Stress3.pgf} node[above right,black] {$\textbf{4}$};
\addplot[gray,dashed,thin] table[x=sigma_22,y=sigma_12] {chapter5/pgfFigures/pgf_slowWavesPlaneStrain/DPslow_yield0_s22s12_frame0_Stress3.pgf};

\addplot[gray,dashed,thin] table[x=sigma_22,y=sigma_12] {chapter5/pgfFigures/pgf_slowWavesPlaneStrain/DPslow_yield0_s22s12_frame1_Stress3.pgf};

\addplot[gray,dashed,thin] table[x=sigma_22,y=sigma_12] {chapter5/pgfFigures/pgf_slowWavesPlaneStrain/DPslow_yield0_s22s12_frame2_Stress3.pgf};

\addplot[gray,dashed,thin] table[x=sigma_22,y=sigma_12] {chapter5/pgfFigures/pgf_slowWavesPlaneStrain/DPslow_yield0_s22s12_frame3_Stress3.pgf};

\end{groupplot}
\end{tikzpicture}
%%% Local Variables:
%%% mode: latex
%%% TeX-master: "../../mainManuscript"
%%% End:
}
  % \subcaptionbox{Loading path in deviatoric plane \label{subfig:slowDP_dev3}}{\begin{tikzpicture}[scale=0.9]
\begin{axis}[width=.75\textwidth,view={135}{35.2643},xlabel=$s_1 $,ylabel=$s_2 $,zlabel=$s_3$,xmin=-1.e8,xmax=1.e8,ymin=-1.e8,ymax=1.e8,axis equal,axis lines=center,axis on top,ztick=\empty]
\addplot3+[Red,very thick,no markers] file {chapter5/pgfFigures/pgf_slowWavesPlaneStrain/DPslowDevPlane_frame0_Stress3.pgf};
\addplot3+[Blue,very thick,no markers] file {chapter5/pgfFigures/pgf_slowWavesPlaneStrain/DPslowDevPlane_frame1_Stress3.pgf};
\addplot3+[Orange,very thick,no markers] file {chapter5/pgfFigures/pgf_slowWavesPlaneStrain/DPslowDevPlane_frame2_Stress3.pgf};
\addplot3+[Purple,very thick,no markers] file {chapter5/pgfFigures/pgf_slowWavesPlaneStrain/DPslowDevPlane_frame3_Stress3.pgf};
\addplot3+[gray,dashed,thin,no markers] file {chapter5/pgfFigures/pgf_slowWavesPlaneStrain/CylindreDevPlane.pgf};
\end{axis}
\end{tikzpicture}
%%% Local Variables:
%%% mode: latex
%%% TeX-master: "../../mainManuscript"
%%% End:
}
  {\begin{tikzpicture}[scale=0.9]
\begin{groupplot}[group style={group size=2 by 1,
ylabels at=edge left, yticklabels at=edge left,horizontal sep=3.ex,
xticklabels at=edge bottom,xlabels at=edge bottom},
ymajorgrids=true,xmajorgrids=true,ylabel=$\sigma_{12} \: (Pa)$,
axis on top,scale only axis,width=0.4\linewidth,ymin=0,ymax=68618075.3102588
, every x tick scale label/.style={at={(xticklabel* cs:1.05,0.75cm)},anchor=near yticklabel},colormap={ry}{rgb255(0cm)=(255,255,0);rgb255(1cm)=(255,0,0)}]
\nextgroupplot[xlabel=$\sigma_{11} (Pa)$]
\addplot[mesh,point meta = \thisrow{p},very thick,no markers] table[x=sigma_11,y=sigma_12] {chapter5/pgfFigures/pgf_slowWavesPlaneStrain/DPslowStressPlane_frame0_Stress3.pgf} node[above right,black] {$\textbf{1}$};
\addplot[mesh,point meta = \thisrow{p},very thick,no markers] table[x=sigma_11,y=sigma_12] {chapter5/pgfFigures/pgf_slowWavesPlaneStrain/DPslowStressPlane_frame1_Stress3.pgf} node[above right,black] {$\textbf{2}$};
\addplot[mesh,point meta = \thisrow{p},very thick,no markers] table[x=sigma_11,y=sigma_12] {chapter5/pgfFigures/pgf_slowWavesPlaneStrain/DPslowStressPlane_frame2_Stress3.pgf} node[above right,black] {$\textbf{3}$};
\addplot[mesh,point meta = \thisrow{p},very thick,no markers] table[x=sigma_11,y=sigma_12] {chapter5/pgfFigures/pgf_slowWavesPlaneStrain/DPslowStressPlane_frame3_Stress3.pgf} node[above right,black] {$\textbf{4}$};
\addplot[gray,dashed,thin] table[x=sigma_11,y=sigma_12] {chapter5/pgfFigures/pgf_slowWavesPlaneStrain/DPslow_yield0_s11s12_Stress3.pgf};

\nextgroupplot[colorbar,colorbar style={title= {$ c_s \: (m/s)$},every y tick scale label/.style={at={(2.,-.1125)}} },xlabel=$\sigma_{22}  (Pa)$]
\addplot[mesh,point meta = \thisrow{p},very thick,no markers] table[x=sigma_22,y=sigma_12] {chapter5/pgfFigures/pgf_slowWavesPlaneStrain/DPslowStressPlane_frame0_Stress3.pgf} node[above right,black] {$\textbf{1}$};
\addplot[mesh,point meta = \thisrow{p},very thick,no markers] table[x=sigma_22,y=sigma_12] {chapter5/pgfFigures/pgf_slowWavesPlaneStrain/DPslowStressPlane_frame1_Stress3.pgf} node[above right,black] {$\textbf{2}$};
\addplot[mesh,point meta = \thisrow{p},very thick,no markers] table[x=sigma_22,y=sigma_12] {chapter5/pgfFigures/pgf_slowWavesPlaneStrain/DPslowStressPlane_frame2_Stress3.pgf} node[above right,black] {$\textbf{3}$};
\addplot[mesh,point meta = \thisrow{p},very thick,no markers] table[x=sigma_22,y=sigma_12] {chapter5/pgfFigures/pgf_slowWavesPlaneStrain/DPslowStressPlane_frame3_Stress3.pgf} node[above right,black] {$\textbf{4}$};
\addplot[gray,dashed,thin] table[x=sigma_22,y=sigma_12] {chapter5/pgfFigures/pgf_slowWavesPlaneStrain/DPslow_yield0_s22s12_frame0_Stress3.pgf};

\addplot[gray,dashed,thin] table[x=sigma_22,y=sigma_12] {chapter5/pgfFigures/pgf_slowWavesPlaneStrain/DPslow_yield0_s22s12_frame1_Stress3.pgf};

\addplot[gray,dashed,thin] table[x=sigma_22,y=sigma_12] {chapter5/pgfFigures/pgf_slowWavesPlaneStrain/DPslow_yield0_s22s12_frame2_Stress3.pgf};

\addplot[gray,dashed,thin] table[x=sigma_22,y=sigma_12] {chapter5/pgfFigures/pgf_slowWavesPlaneStrain/DPslow_yield0_s22s12_frame3_Stress3.pgf};

\end{groupplot}
\end{tikzpicture}
%%% Local Variables:
%%% mode: latex
%%% TeX-master: "../../mainManuscript"
%%% End:
}
  {\begin{tikzpicture}[scale=0.9]
\begin{axis}[width=.75\textwidth,view={135}{35.2643},xlabel=$s_1 $,ylabel=$s_2 $,zlabel=$s_3$,xmin=-1.e8,xmax=1.e8,ymin=-1.e8,ymax=1.e8,axis equal,axis lines=center,axis on top,ztick=\empty]
\addplot3+[Red,very thick,no markers] file {chapter5/pgfFigures/pgf_slowWavesPlaneStrain/DPslowDevPlane_frame0_Stress3.pgf};
\addplot3+[Blue,very thick,no markers] file {chapter5/pgfFigures/pgf_slowWavesPlaneStrain/DPslowDevPlane_frame1_Stress3.pgf};
\addplot3+[Orange,very thick,no markers] file {chapter5/pgfFigures/pgf_slowWavesPlaneStrain/DPslowDevPlane_frame2_Stress3.pgf};
\addplot3+[Purple,very thick,no markers] file {chapter5/pgfFigures/pgf_slowWavesPlaneStrain/DPslowDevPlane_frame3_Stress3.pgf};
\addplot3+[gray,dashed,thin,no markers] file {chapter5/pgfFigures/pgf_slowWavesPlaneStrain/CylindreDevPlane.pgf};
\end{axis}
\end{tikzpicture}
%%% Local Variables:
%%% mode: latex
%%% TeX-master: "../../mainManuscript"
%%% End:
}
  \caption{Loading paths through slow simple waves for different starting points on the initial yield surface for the initial condition $\sigma_{22}=1.3 \times 10^{8} \: Pa$.}
  \label{fig:slow_path_plane_strains3}
\end{figure}
Nevertheless, the integral curves in figure \ref{fig:slow_path_plane_strains2} reveal that numerical issues occur "faster" than in the previous case.
Indeed, the characteristic speeds quickly start increasing so that the stress paths depicted are short.
Furthermore, the projection in ($\sigma_{22},\sigma_{12}$) plane of the integral curves $2$ and $3$ show that the slow wave mainly influences $\sigma_{22}$.
The projections in deviatoric plane in figure \ref{fig:slow_path_plane_strains2} however show that the stress paths first follow the initial yield surface until the direction of pure shear is reached, and next follow the radial direction.
In addition, considering figures \ref{fig:slow_path_plane_strains1}, \ref{fig:slow_path_plane_strains2} and \ref{fig:slow_path_plane_strains3}, it can be seen that if the initial condition on $\sigma_{11}$ is greater than the value corresponding to the maximum shear stress on the initial yield surface, $\sigma_{22}$ increases along the loading path.
Conversely, $\sigma_{22}$ decreases along the loading path if $\sigma_{11}$ is initially lower than the value that corresponds to the maximum shear stress on the initial yield surface.
%In addition, considering figures \ref{fig:slow_path_plane_strains1}, \ref{fig:slow_path_plane_strains2} and \ref{fig:slow_path_plane_strains3}, it can be seen that the loading paths in $(\sigma_{22},\sigma_{12})$ plane lead to an increasing $\sigma_{22}$ if the initial condition is such that $\sigma_{11}$ is greater than this corresponding to the maximum shear on the initial yield surface in projection in the $(\sigma_{11},\sigma_{12})$ plane.
%Conversely, an initial condition $\sigma_{11}$ lower than that associated to the maximum shear stress on the initial yield surface in the plane $(\sigma_{11},\sigma_{12})$ results in a decreasing $\sigma_{22}$ along the integral curves.
Note that the same goes for $\sigma_{11}$.
Thus, it seems that the property $\sign(d\sigma_{22})=\sign(d\sigma_{11})$ holds along the loading paths followed inside a slow simple wave in plane strain, though this has not been highlighted mathematically.

Generally speaking, it appears that for the ranges of stress considered here, the hardening of the material is mainly due to slow simple waves for plane strain cases.
Indeed, the latter may lead to radial loading paths that greatly increase the radius of the von-Mises cylinder in principal stresses space, whereas the integral curves corresponding to fast waves are restricted to the initial yield surface. Notice, however, that the above results have been obtained by using a rather low hardening modulus.



%If on the other hand, the hardening parameter is raised to $C=1\times10^{10} \: Pa$, the loading paths slightly differ.
%To illustrate this, the same plane strain problems as before are considered by using the same driving parameters and initial conditions for both fast and slow waves.
%The resulting integral curves are here depicted in deviatoric plane only in order to highlight particular behaviors.
The curves resulting from the use of the hardening parameter $C=1\times10^{10} \: Pa$ exhibit slight differences.
First, the integral curves of fast waves are depicted in figure \ref{fig:fast_H}.
As before, the integral curves follow the initial yield surface but then branch off to reach a direction of pure tensile/compression loading.
\begin{figure}[h!]
  \centering
  \tikzset{cross/.style={cross out, draw=black, minimum size=2*(#1-\pgflinewidth), inner sep=0pt, outer sep=0pt},cross/.default={2.5pt}}
\begin{tikzpicture}[spy using outlines={rectangle, magnification=3, size=2.cm, connect spies},scale=0.9]
\begin{axis}[width=.75\textwidth,view={135}{35.2643},xlabel=$s_1 $,ylabel=$s_2 $,zlabel=$s_3$,xmin=-1.e8,xmax=1.e8,ymin=-1.e8,ymax=1.e8,axis equal,axis lines=center,axis on top,xtick=\empty,ytick=\empty,ztick=\empty,every axis y label/.style={at={(rel axis cs:0.,.5,-0.65)}, anchor=west}, every axis x label/.style={at={(rel axis cs:0.5,.,-0.65)}, anchor=east}, every axis z label/.style={at={(rel axis cs:0.,.0,.18)}, anchor=north},legend style={at={(.2,.68)}}]
\node[below] at (1.1e8,0.,0.) {$\sigma^y$};
\node[above] at (-1.1e8,0.,0.) {$-\sigma^y$};
\draw (1.e8,0.,0.) node[cross,rotate=10] {};
\draw (-1.e8,0.,0.) node[cross,rotate=10] {};
\node[white]  at (0,0.,1.1e8) {};
\addplot3[Red,thick,arrows along my path] file {chapter5/pgfFigures/pgf_HfastWavesPlaneStrai/DPfastDevPlane_frame0_Stress0.pgf};\addlegendentry{loading path 1}
\addplot3[Blue,thick,arrows along my path] file {chapter5/pgfFigures/pgf_HfastWavesPlaneStrai/DPfastDevPlane_frame1_Stress0.pgf};\addlegendentry{loading path 2}
\addplot3[Orange,thick,arrows along my path] file {chapter5/pgfFigures/pgf_HfastWavesPlaneStrai/DPfastDevPlane_frame2_Stress0.pgf};\addlegendentry{loading path 3}
\addplot3[Purple,thick,arrows along my path] file {chapter5/pgfFigures/pgf_HfastWavesPlaneStrai/DPfastDevPlane_frame3_Stress0.pgf};\addlegendentry{loading path 4}
\addplot3[Green,thick,arrows along my path] file {chapter5/pgfFigures/pgf_HfastWavesPlaneStrai/DPfastDevPlane_frame4_Stress0.pgf};\addlegendentry{loading path 5}
\addplot3[Duck,thick,arrows along my path] file {chapter5/pgfFigures/pgf_HfastWavesPlaneStrai/DPfastDevPlane_frame5_Stress0.pgf};\addlegendentry{loading path 6}
\addplot3+[gray,dashed,thin,no markers] file {chapter5/pgfFigures/pgf_HfastWavesPlaneStrai/CylindreDevPlane.pgf};\addlegendentry{initial yield surface}
\newcommand\radius{1.*0.82e8}
\addplot3[dotted,thick] coordinates {(0.75*\radius,-0.75*\radius,0.) (-0.75*\radius,0.75*\radius,0.)};
\addplot3[dotted,thick] coordinates {(0.,-0.75*\radius,0.75*\radius) (0.,0.75*\radius,-0.75*\radius)};
\addplot3[dotted,thick] coordinates {(-0.75*\radius,0.,0.75*\radius) (0.75*\radius,0.,-0.75*\radius)};
\begin{scope}
\spy[black,size=1.75cm] on (6.9,3.3) in node [fill=none] at (9.5,5.5);
\end{scope}
\end{axis}
\end{tikzpicture}
%%% Local Variables:
%%% mode: latex
%%% TeX-master: "../../mainManuscript"
%%% End:

  \caption{Fast simple wave solutions of the plane strain problems $C=1\times10^{10} \: Pa$.}
  \label{fig:fast_H}
\end{figure}
Moreover, the visible cusps in the loading paths $2$, $3$, $4$ and $5$, which already arise in the solutions based on a lower hardening modulus, indicate that the path followed inside fast waves converges to a direction of pure tensile/compression under plane strain. 

The increase in hardening parameter also allows to eliminate the integration issues that occur for a lower one.
As a result, the stress paths followed inside slow simple waves depicted in figure \ref{fig:slow_H} lead to stress states lying further outside of the initial elastic convex.
\begin{figure}[h!]
  \centering
  {\phantomsubcaption \label{subfig:slow_H1}}
  {\phantomsubcaption \label{subfig:slow_H2}}
  {\phantomsubcaption \label{subfig:slow_H3}}
  \tikzset{cross/.style={cross out, draw=black, minimum size=2*(#1-\pgflinewidth), inner sep=0pt, outer sep=0pt},cross/.default={2.5pt}}
\begin{tikzpicture}
  \newcommand\radius{1.*0.82e8}
  \begin{groupplot}[group style={group size=3 by 1,
      % ylabels at=edge left, yticklabels at=edge left,
      horizontal sep=-90pt,
      % xticklabels at=edge bottom,xlabels at=edge bottom
    },
    ymajorgrids=true,xmajorgrids=true,%enlargelimits=0,
    axis on top,scale only axis,%width=0.4\linewidth,
    view={135}{35.2643},xlabel=$s_1 $,ylabel=$s_2 $,zlabel=$s_3$,
    xmin=-1.e8,xmax=1.e8,ymin=-1.e8,ymax=1.e8,axis equal,axis lines=center,
    xtick=\empty,ytick=\empty,ztick=\empty,
    every axis y label/.style={at={(rel axis cs:0.,.5,-0.65)}, anchor=west},
    every axis x label/.style={at={(rel axis cs:0.5,.,-0.65)}, anchor=east},
    every axis z label/.style={at={(rel axis cs:0.,.0,.18)}, anchor=north}]
    %%%
    \nextgroupplot[width=.5\textwidth,
    % title={(a) $\sigma_{22}=-1.3\times 10^{8} \: Pa$.}
    ]
    \node[below] at (axis cs:1.1e8,0.,0.) {$\sigma^y$};
    \node[above] at (axis cs:-1.1e8,0.,0.) {$-\sigma^y$};
    \draw (axis cs:1.e8,0.,0.) node[cross,rotate=10] {};
    \draw (axis cs:-1.e8,0.,0.) node[cross,rotate=10] {};
    %\node[white]  at (0,0.,.7e8) {};
    \addplot3[Red,thick,arrows along my path] file {section7/pgfFigures/pgf_HslowWavesPlaneStrai/DPslowDevPlane_frame0_Stress1.pgf};
    \addplot3[Blue,thick,arrows along my path] file {section7/pgfFigures/pgf_HslowWavesPlaneStrai/DPslowDevPlane_frame1_Stress1.pgf};
    \addplot3[Orange,thick,arrows along my path] file {section7/pgfFigures/pgf_HslowWavesPlaneStrai/DPslowDevPlane_frame2_Stress1.pgf};
    \addplot3[Purple,thick,arrows along my path] file {section7/pgfFigures/pgf_HslowWavesPlaneStrai/DPslowDevPlane_frame3_Stress1.pgf};
    \addplot3[Duck,thick,arrows along my path] file {section7/pgfFigures/pgf_HslowWavesPlaneStrai/DPslowDevPlane_frame4_Stress1.pgf};
    \addplot3+[gray,dashed,thin,no markers] file {section7/pgfFigures/pgf_HslowWavesPlaneStrai/CylindreDevPlane.pgf};
    \addplot3[dotted,thick] coordinates {(0.75*\radius,-0.75*\radius,0.) (-0.75*\radius,0.75*\radius,0.)};
    \addplot3[dotted,thick] coordinates {(0.,-0.75*\radius,0.75*\radius) (0.,0.75*\radius,-0.75*\radius)};
\addplot3[dotted,thick] coordinates {(-0.75*\radius,0.,0.75*\radius) (0.75*\radius,0.,-0.75*\radius)};
    %%%
\nextgroupplot[width=.5\textwidth,
%title={(b) $\sigma_{22}=0$.},
legend style={at={(1.14,.15)},legend columns=3}]
    \node[below] at (axis cs:1.1e8,0.,0.) {$\sigma^y$};
    \node[above] at (axis cs:-1.1e8,0.,0.) {$-\sigma^y$};
    \draw (axis cs:1.e8,0.,0.) node[cross,rotate=10] {};
    \draw (axis cs:-1.e8,0.,0.) node[cross,rotate=10] {};
    %\node[white]  at (0,0.,1.1e8) {};
    \addplot3[Red,thick,arrows along my path] file {section7/pgfFigures/pgf_HslowWavesPlaneStrai/DPslowDevPlane_frame0_Stress2.pgf};%\âddlegendentry{loading path 1}
    \addplot3[Blue,thick,arrows along my path] file {section7/pgfFigures/pgf_HslowWavesPlaneStrai/DPslowDevPlane_frame1_Stress2.pgf};%\âddlegendentry{loading path 2}
    \addplot3[Orange,thick,arrows along my path] file {section7/pgfFigures/pgf_HslowWavesPlaneStrai/DPslowDevPlane_frame2_Stress2.pgf};%\âddlegendentry{loading path 3}
    \addplot3[Purple,thick,arrows along my path] file {section7/pgfFigures/pgf_HslowWavesPlaneStrai/DPslowDevPlane_frame3_Stress2.pgf};%\âddlegendentry{loading path 4}
    \addplot3[Duck,thick,arrows along my path] file {section7/pgfFigures/pgf_HslowWavesPlaneStrai/DPslowDevPlane_frame4_Stress2.pgf};%\âddlegendentry{loading path 5}
    \addplot3+[gray,dashed,thin,no markers] file {section7/pgfFigures/pgf_HslowWavesPlaneStrai/CylindreDevPlane.pgf};%\âddlegendentry{initial yield surface}
    \addplot3[dotted,thick] coordinates {(0.75*\radius,-0.75*\radius,0.) (-0.75*\radius,0.75*\radius,0.)};
    \addplot3[dotted,thick] coordinates {(0.,-0.75*\radius,0.75*\radius) (0.,0.75*\radius,-0.75*\radius)};
    \addplot3[dotted,thick] coordinates {(-0.75*\radius,0.,0.75*\radius) (0.75*\radius,0.,-0.75*\radius)};
    %%%
    \nextgroupplot[width=.5\textwidth,
    % title={(c) $\sigma_{22}=1.3\times 10^{8} \: Pa$.}
    ]
    \node[below] at (axis cs:1.1e8,0.,0.) {$\sigma^y$};
    \node[above] at (axis cs:-1.1e8,0.,0.) {$-\sigma^y$};
    \draw (axis cs:1.e8,0.,0.) node[cross,rotate=10] {};
    \draw (axis cs:-1.e8,0.,0.) node[cross,rotate=10] {};
    %\node[white]  at (0,0.,1.1e8) {};
    \addplot3[Red,thick,arrows along my path] file {section7/pgfFigures/pgf_HslowWavesPlaneStrai/DPslowDevPlane_frame0_Stress3.pgf};
    \addplot3[Blue,thick,arrows along my path] file {section7/pgfFigures/pgf_HslowWavesPlaneStrai/DPslowDevPlane_frame1_Stress3.pgf};
    \addplot3[Orange,thick,arrows along my path] file {section7/pgfFigures/pgf_HslowWavesPlaneStrai/DPslowDevPlane_frame2_Stress3.pgf};
    \addplot3[Purple,thick,arrows along my path] file {section7/pgfFigures/pgf_HslowWavesPlaneStrai/DPslowDevPlane_frame3_Stress3.pgf};
    \addplot3[Duck,thick,arrows along my path] file {section7/pgfFigures/pgf_HslowWavesPlaneStrai/DPslowDevPlane_frame4_Stress3.pgf};
    \addplot3+[gray,dashed,thin,no markers] file {section7/pgfFigures/pgf_HslowWavesPlaneStrai/CylindreDevPlane.pgf};
    \addplot3[dotted,thick] coordinates {(0.75*\radius,-0.75*\radius,0.) (-0.75*\radius,0.75*\radius,0.)};
    \addplot3[dotted,thick] coordinates {(0.,-0.75*\radius,0.75*\radius) (0.,0.75*\radius,-0.75*\radius)};
    \addplot3[dotted,thick] coordinates {(-0.75*\radius,0.,0.75*\radius) (0.75*\radius,0.,-0.75*\radius)};
  \end{groupplot}
\end{tikzpicture}






























%%% Local Variables:
%%% mode: latex
%%% TeX-master: "../../mainManuscript"
%%% End:
  \caption{Slow simple wave solutions of the previous problems with $C=1\times10^{10} \: Pa$.}
  \label{fig:slow_H}
\end{figure}
Moreover, for every initial values of $\sigma_{22}$ considered in figures \ref{fig:slow_H}\subref{subfig:slow_H1}, \ref{fig:slow_H}\subref{subfig:slow_H2} and \ref{fig:slow_H}\subref{subfig:slow_H3}, the slopes of the integral curves no longer break but smoothly vary to reach the direction of pure shear in the deviatoric plane.

\begin{remark}
  The use of a higher hardening parameter for slow waves under plane stress also leads to smoother paths in the deviatoric plane.
  On the other hand, the integral curves associated to fast waves under plane stress (slightly) move away from the yield surface up to a direction of pure shear.
  Nevertheless, at that point numerical difficulties occur due to the indeterminacy of the loading functions already mentioned.
\end{remark}
%%%% REMARQUES A LA VOLEE
% It is shown in \cite{Ting73} that the plastic celerities only depends on $\tens{\sigma}/\norm{\tens{\sigma}}$ so that they are constant along rays of the stress space $(\sigma_{11}, \sigma_{22}, \sigma_{12})$. Thus, look at the loading path along integral curves and see the evolution of celerities.


%%% Local Variables:
%%% mode: latex
%%% TeX-master: "../mainManuscript"
%%% End:
