Commencer par un résumé du ce qui a été identifié au dessus.

Eventuellement ne parler que d'un cas pour lequel c'est plus simple comme les defs planes (encore que...).

On pourrait faire comme Lin et Ballman \cite{Lin_et_Ballman} en se donnant 3 composantes de contraintes et en intégrant. 
C'est ce qui paraît le plus faisable dans l'état. 
Sauf que sigma22 fout la merde à cause car on a 3 inconnues en contraintes alors que pour le thin-walled tube on en n'avait que 2 et on pouvait itérer comme ça.
En effet, les interscetions dans les plans u,sigma et v,tau des courbes intégrales nous donnait des nouveaux états de contraintes pour l'étape d'après. 
Qu'en est-il avec sigma22 ?


Pour un solveur approximé:
Identifier le trajet ?? A priori, on ne peut pas sans intégrer.
Sauf si on fait une projection radiale de l'état trial sur le critère et qu'on cherche quelle onde part de là.
Il semblerait tout de même que la plasticité est surtout répendue par les ondes slow.
Les ondes fast font tourner autour du critère. 
Clifton suppose qu'elles n'existent que si leurs point de départ est à tau=0. On peut faire la même chose ?
On peut approximer les trajets par la tangente sur le critère pour les ondes slow.
%% Tracer des structures charactéristiques (revoir dans la biblio où on met une discontinuité et où on n'en met pas)








%%% Local Variables:
%%% mode: latex
%%% TeX-master: "../mainManuscript"
%%% End:
