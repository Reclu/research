%%%%%%%%%%%%%%%%%%%%%%%%%%%%%%%%%%%%%%%%%%%
% Revenir sur ce qui a été fait dans le chapitre du point de vue analytique: ok
% Parler de Lin et Ballman (dire à la fin que c'est la seule tentative à notre connaissance pour prendre en compte la structure dans un schéma volume finis)
% Parler de l'importance de connaitre la solution analytique

% Parler des résultats numériques qui donnent quelques indices sur la structure de la solution

% Parler de pistes qui peuvent être suivies en vue de prendre en compte ces informations
%% Utiliser une forme quasi-linéaire en valeur propres de cauchy
%% Lin et Ballman
%%%%%%%%%%%%%%%%%%%%%%%%%%%%%%%%%%%%%%%%%%%

\subsection*{Summary of the chapter}

In this chapter, the characteristic structure of the solution of hyperbolic problems in elastic-plastic solids in two space dimensions has been highlighted.
It is known since the 50s that plastic flow in two-dimensional solids yields two families of waves which speeds depend on the stress state, the slow and fast waves.
As a result, shock and simple waves may occur in an elastoplastic medium even for linear hardening material.
In addition, these plastic waves may have an impact on several stresses in contrast to elastic discontinuities across which one stress component varies, hence the name of combined stress waves.
During the 60s, attention has been paid to simple waves in particular two-dimensional problems thus providing, among other, solutions of Picard problems in elastic-plastic medium undergoing step loadings \cite{Clifton,Ting68,Ting69,Ting73}. % Idem pour ting ? c'est dit dans l'intro ? voir ce qui est fait dans le 73
The singular nature of such problems lies in the fact that the characteristic structure of the solution depends on the external loading undergone.
Indeed, it has been shown \cite{Clifton} that boundary conditions can lead to plastic flow involvingon fast, one slow, or both simple waves.
Therefore, it is crucial to be able to identify typical stress paths followed in each simple waves in order to link the initial data to a given stress state, and subsequently to determine the occurring wave pattern.

%% Lin et Ballman
Based on those works, an iterative Riemann solver \cite{Lin_et_Ballman}, which procedure has been recalled in section \ref{sec:stress_paths_num}, has been developed for the numerical solution of the thin-walled tube problem. % Voir s'ils sont contents de ce que ça apporte parce que ça a pas l'air top quand même
% parler de Toro le 4-rarefactions RS
%% http://www.global-sci.com/issue/v22/n5/pdf/1224.pdf 4-rarefaction waves elastic-plastic RS
% L'idée ici c'est de généraliser cette approche pour tous les problèmes 2D
Following this approach, identifying characteristic wave patterns for general elastoplastic problems in two space dimensions should allow to enrich the numerical solution with the knowledge one has of physical phenomena.
%% Donc analyse math en plane stress et plane strain qui montrent certains comportement typiques.

%% Ensuite, puisque les équations sont quand même chaudes, on est passé au numérique et on a montré deux trois trucs intéressants.

% | prendre en compte ce qu'on vient de trouver pour construire un solveur à la Lin et Ballman
% v si on cherche la solution d'un problème de Picard. ou pas d'ailleurs à cause de sigma_22 que l'on ne connait pas dans Picard
\subsection*{Towards a two-dimensional elastoplastic Riemann solver}
% Eventuellement ne parler que d'un cas pour lequel c'est plus simple comme les defs planes (encore que...).

% On pourrait faire comme Lin et Ballman \cite{Lin_et_Ballman} en se donnant 3 composantes de contraintes et en intégrant. 
% C'est ce qui paraît le plus faisable dans l'état. 
% Sauf que sigma22 fout la merde à cause car on a 3 inconnues en contraintes alors que pour le thin-walled tube on en n'avait que 2 et on pouvait itérer comme ça.
% En effet, les interscetions dans les plans u,sigma et v,tau des courbes intégrales nous donnait des nouveaux états de contraintes pour l'étape d'après. 
% Qu'en est-il avec sigma22 ?


% Pour un solveur approximé:
% Identifier le trajet ?? A priori, on ne peut pas sans intégrer.
% Sauf si on fait une projection radiale de l'état trial sur le critère et qu'on cherche quelle onde part de là.
% Il semblerait tout de même que la plasticité est surtout répendue par les ondes slow.
% Les ondes fast font tourner autour du critère. 
% Clifton suppose qu'elles n'existent que si leurs point de départ est à tau=0. On peut faire la même chose ?
% On peut approximer les trajets par la tangente sur le critère pour les ondes slow.
%% Tracer des structures charactéristiques (revoir dans la biblio où on met une discontinuité et où on n'en met pas)








%%% Local Variables:
%%% mode: latex
%%% TeX-master: "../mainManuscript"
%%% End:
