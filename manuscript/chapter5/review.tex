Researches done on the elastic-plastic bahavior of material at high strain rates for characterization purposes. To this end, uni-axial stress or strain, pure bending or pure torsion problems have been investigated until the late 50's (vérifier ça). Rakhmatulin et Cristescu ont ouvert la voie à des problèmes plus complexes impliquant des chargements combinés.
A l'époque, de telles études permettait de développer des essais afin de mieux charactériser le comportmeent éléastioplastic des matériaux. De nos jours, les informations sur la structures exact de de la solution peuvent conduitre à une meilleur évaluation des contraintes et des déformations résiduelles, de comprendre la physique etc.

Parler de comment on gère la plasticité usuellement citer swansea, marseile etc.

Ce remonte à von Karman en 42 mais on va dater la Biblio à partir de Rakhmatulin (ou Clifton ?!). On ne regarde qu'une dimension spatiale en faisant des hypothèse sur les champs alors que nous on se limite à une direction particulière $\vect{n}$.
\begin{itemize}
%% plane stress
\item \cite{Rakhmatulin} ; \cite{CRISTESCU19591605} : Strong discontinuity; Distinction between plastic waves ($c_1,c_2$) of strong discontinuities, and combined waves ($c_f,c_s$)
%% superimposition of plane wave and shear wave
\item \cite{Bleich}: $\tens{\sigma}=\matrice{\sigma & \tau & \\ \tau & \sigma_{22} & \\ & & \sigma_{22}}$;$\tens{\eps}=\matrice{\eps & \gamma/2 & \\ \gamma/2 &  & \\ & & }$ ideal elastic-plastic continuum ; shock front at p.15 and subsection B + exact solutions ; identification of particular loading cases that lead to different characteristic structures.
%% combined shear and longitudinal waves
\item \cite{Clifton}: $\tens{\sigma}=\matrice{\sigma & \tau & \\ \tau & & \\ & & }$;$\tens{\eps}=\matrice{\eps & \gamma/2 & \\ \gamma/2 &  & \\ & & }$ Not a strong discontinuity + hardening + Emphasis unexpected stress structure (prestressed tubes)
%% superimposition of plane wave and shear wave
\item \cite{Ting68}: $\tens{\sigma}=\matrice{\sigma & \tau & \\ \tau & \sigma_{22} & \\ & & \sigma_{22}}$;$\tens{\eps}=\matrice{\eps & \gamma/2 & \\ \gamma/2 &  & \\ & & }$ with linear hardening materials; identification of particular loading cases that lead to different characteristic structures (p.8). Generalization of \cite{Bleich} tp hardening materials and of \cite{Clifton} to more complex loadings ; Emphasis unexpected stress structure (prestressed tubes)
%% superimposition of plane wave and shear waveS
\item \cite{Ting69}:  $\tens{\sigma}=\matrice{\sigma & \tau_y & \tau_z\\ \tau_y & \sigma_{22} & \\\tau_z & & \sigma_{22}}$;$\tens{\eps}=\matrice{\eps & \gamma_y/2 & \gamma_z/2\\ \gamma_y/2 &  & \\ \gamma_z/2& & }$ ; superimposition of plane wave and shear waveS ; Emphasis unexpected stress structure (prestressed tubes)
\item \cite{Ting73}: c'est pas un peu une review des deux papiers d'avant ?
%% superimposition of plane wave and shear wave
\item \cite{Li_planeStress_EP}:$\tens{\sigma}=\matrice{\sigma & \tau & \\ \tau & \sigma_{22} & \\ & & \sigma_{22}}$;$\tens{\eps}=\matrice{\eps & \gamma/2 & \\ \gamma/2 &  & \\ & & }$
\end{itemize}

Il y a la question des vitesses charactéristiques plastiques... sont-elles collées aux vitesses élastiques ?
% \subsection{The complexity of elastoplasticity}
% % Bibliography about:
% % - what has been done so far (thin-walled ; plane wave + shear wave)
% % - what is missing (solution of more general problems [i.e plane strain or stress] ; development of approximate Riemann solvers)
% % - what it will allow (comparison with experimental data in order to track plastic shocks) internal structure of plastic shock
% % - what is currently done in fluid or solid mechanics numerically
% Biblio, 
% %\section{State of the art}
% \subsection{The thin-walled tube problem}
% \cite{Clifton} + thesis
% \subsection{Superimposition of plane wave and shear waves}
% Citer Ting
% \section{Plane strain and plane stress problem}
% \subsection{General multi-dimensional formulation}
% % Based on the tangent modulus -> independent on the hardening model
% \subsection{Plane strain problems}
% \subsection{Plane stress problems}
% \section{Characteristic analysis}
% dependance des vitesses caractéristiques à l'angle entre la direction principale de sigma et la direction de propagation, c'est dit dans la thèse de Clifou en page 90.
% \subsection{Structure of the solution}
% \subsection{Integral curves and loading paths}
% \subsection{The plane strain case}
% \subsection{The plane stress case}



%%% Local Variables:
%%% mode: latex
%%% TeX-master: "../mainManuscript"
%%% End:
