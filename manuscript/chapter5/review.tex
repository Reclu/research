Until the 50s, researches on dynamic problems in elastic-plastic solids were focused on uni-axial stress or strain, pure bending or pure torsion loading conditions \cite{Taylor,vonKarman}, and were carried out for materials characterization purposes.
The first references that brought some understanding about the response of linearly hardening solids to combined shear and pressure loads are those of Rakhmatulin \cite{Rakhmatulin} and Cristescu \cite{CRISTESCU19591605}.
These early analytical investigations on plane stress impacts in the plastic regime led to the conclusion that elastic waves, as well as plastic combined-stress simple waves, can propagate in two-dimensional solids. 
While the former were well-known, the latter were shown to fall into the two \textit{fast waves} and \textit{slow waves} families.
% It has been moreover shown that the fast waves propagate faster than the plastic pressure discontinuity in uni-axial problems would, for a given compression load amplitude.
% Likewise, slow waves propa
% The maximal value of fast waves (\textit{resp. slow waves}) is higher than that of pressure (\textit{resp. shear}) plastic discontinuity occuring in one-dimensional problems, for a given compression (\textit{resp. shear}) load amplitude.

Later, Bleich and Nelson \cite{Bleich} considered superimposed plane and shear waves in an ideally elastic-plastic materials submitted to step loads.
It has then been highlighted that different loading cases yield different characteristic structures of the solution of a Picard problem, thus revealing the complexity of plastic flows in more than one dimension.
% Distinguer un peu plus ces deux contributions.
%\thomas{see \cite[p.56 pdf]{Nowacki},\cite{Goel}}. 
The same conclusions have been drawn by Clifton \cite{Clifton} for hardening materials under tension-torsion, who furthermore studied the influence of plastic pre-loading on the solution.
This contribution established the existence of loading paths through the simple waves arising from the characteristic analysis of the hyperbolic system.
Indeed, the combined-stress wave nature lies in ODEs which govern the evolution of stress components within the simple waves.
The integration of these equations of the form $d\sigma_{11}=\psi d\sigma_{12}$ allows the building of curves that connect the applied stress state of the Picard problem $(\sigma^d_{11},\sigma^d_{12})$ to the initial state of the medium.
% Indeed, the study mathematical properties of relations between stress components of the form $d\sigma_{11}=\psi d\sigma_{12}$, satisifed inside fast and slow simple waves, allows to connect the applied stress state of the Picard problem $(\sigma^d_{11},\sigma^d_{12})$ to the initial state of the medium.
It has been for instance shown that if a solid is acted upon by a traction force such that $\sigma^d_{11}=0$ and $\sigma^d_{12}$ lies outside the elastic convex, only an elastic shear discontinuity, followed by a slow simple wave, propagates.
Conversely, other loading conditions may lead to the combination of an elastic pressure discontinuity and a fast wave, possibly followed by a slow wave.
Another notable conclusion is that the combined loading paths followed inside simple waves may lead to plastic unloading, whereas only elastic unloading occurs in the one-dimensional theory.
%In addition, it is possible to meet unloading plastic simple waves with contrast to the one-dimensional theory in which the unloading waves propagate at elastic speeds (c'est pas vraiment ça attention).
%Such loading paths are supplemented by ODEs satisfied by the velocity components so that a closed form of the solution of the problem can be derived.

Experimental data collected on a thin-walled tube submitted to a dynamic tensile load \cite{Clifton_exp,Clifton_exp2} confirmed the existence of two distinct families of  simple waves, both involving combined stress paths.
Those works nevertheless exhibited some discrepancies with the theory which have been attributed to the assumption made on the von-Mises yield surface.
As a matter of fact, a constant strain region lying between the fast and slow waves that is predicted by the theory \cite{Clifton} could not be seen in experimental results.
However, by following the endochronic theory of plasticity \cite{Valanis} which does not require the introduction of a yield surface, Wu and Lin \cite{Wu_experimental} obtained numerical results that better fitted the experimental data provided by Lipkin and Clifton \cite{Clifton_exp2}.
The good agreement showed between numerical and experimental results \cite{Wu_experimental} thus confirmed the theory.

Ting and Nan \cite{Ting68} then generalized the work of Bleich and Nelson to hardening materials and Ting \cite{Ting69} widened this of Clifton to more complex loadings, that is a superimposition of one plane wave and two shear waves states.
Once again, the mathematical study of the ODE system governing the stresses evolution inside fast and slow simple waves led to the construction of loading paths in the stress space that depend on the external loads. A review of governing equations for all the cases depending on one space dimension considered above can be found in \cite{Nowacki}.

The information on characteristic structures thus provided has then be used by Lin and Ballman \cite{Lin_et_Ballman} for the development of an iterative Riemann solver.
This procedure is based on successive guesses on the stress state lying in the stationary region so that the loading paths predicted by the theory of Clifton \cite{Clifton} can be integrated numerically until convergence.
The implementation of this solver within a second-order Godunov scheme provided results that were in good agreement with the exact solutions.
Nevertheless, the theoretical investigations mentioned above restrict the development of such numerical tools to problems that depend on one space dimension.
%%
Clifton tackled the solution of plane strain problems in elastic-plastic solids by looking for bi-characteristics \cite{Clifton_thesis} in order to build finite difference schemes that account for plastic waves.
The point of view adopted here is that one can benefit from the simplifications introduced by the writing of Riemann problems in an arbitrary direction of space.
Indeed, the method of characteristics rather than the more complex method of bi-characteristics can be employed with the quasi-linear forms presented in chapter \ref{chap:chap2}.

$\newline$
On the other hand, the existence of plastic shocks in solids under the plane wave assumptions has been investigated by several authors \cite{Mandel1,Germain_shock,Mandel2}.
Those references
% Structure interne + trajet radial de Mandel
% Hypothèse hydrostatic -> non-monotonic stress-strain curve
?????????????????
PLASTIC SHOCKS
?????????????????


%%% Local Variables:
%%% mode: latex
%%% TeX-master: "../mainManuscript"
%%% End:
