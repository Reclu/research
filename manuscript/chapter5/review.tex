Until the 50s, research on dynamic problems in elastic-plastic solids were focused on uni-axial stress or strain, pure bending or pure torsion loading conditions \cite{Taylor,vonKarman}, and were carried out for materials characterization purposes.
The first references that brought some understanding about the response of linearly hardening solids to combined shear and pressure loads are those of \textsc{Rakhmatulin} \cite{Rakhmatulin} and \textsc{Cristescu} \cite{CRISTESCU19591605}.
These early analytical investigations on plane stress impacts in the plastic regime led to the conclusion that elastic waves, as well as plastic combined-stress simple waves, can propagate in two-dimensional solids. 
While the former were well-known, the latter were shown to fall into two families: the \textit{fast waves} and the \textit{slow waves}.
% It has been moreover shown that the fast waves propagate faster than the plastic pressure discontinuity in uni-axial problems would, for a given compression load amplitude.
% Likewise, slow waves propa
% The maximal value of fast waves (\textit{resp. slow waves}) is higher than that of pressure (\textit{resp. shear}) plastic discontinuity occuring in one-dimensional problems, for a given compression (\textit{resp. shear}) load amplitude.

Later, \textsc{Bleich} and \textsc{Nelson} \cite{Bleich} considered superimposed plane and shear waves in an ideally elastic-plastic material submitted to step loads.
It has then been highlighted that different loading cases yield different characteristic structures of the solution of a Picard problem, thus revealing the complexity of plastic flows in more than one dimension.
% Distinguer un peu plus ces deux contributions.
%\thomas{see \cite[p.56 pdf]{Nowacki},\cite{Goel}}. 
The same conclusions have been drawn by \textsc{Clifton} \cite{Clifton} for hardening materials under tension-torsion, who furthermore studied the influence of plastic pre-loading on the solution.
This contribution established the existence of loading paths through the simple waves arising from the characteristic analysis of the hyperbolic system.
Indeed, the combined-stress wave nature lies in ODEs which govern the evolution of stress components within the simple waves.
The integration of these equations of the form $d\sigma_{11}=\psi d\sigma_{12}$ allows the building of curves that connect the applied stress state of the Picard problem $(\sigma^d_{11},\sigma^d_{12})$ to the initial state of the medium.
% Indeed, the study mathematical properties of relations between stress components of the form $d\sigma_{11}=\psi d\sigma_{12}$, satisifed inside fast and slow simple waves, allows to connect the applied stress state of the Picard problem $(\sigma^d_{11},\sigma^d_{12})$ to the initial state of the medium.
It has been for instance shown that if a solid is acted upon by a traction force such that $\sigma^d_{11}=0$ and $\sigma^d_{12}$ lies outside the elastic domain, only an elastic shear discontinuity, followed by a slow simple wave, propagates.
Conversely, other loading conditions may lead to the combination of an elastic pressure discontinuity and a fast wave, possibly followed by a slow wave.
Another notable conclusion is that the combined loading paths followed inside simple waves may lead to plastic unloading, whereas only elastic unloading occurs in the one-dimensional theory.
%In addition, it is possible to meet unloading plastic simple waves with contrast to the one-dimensional theory in which the unloading waves propagate at elastic speeds (c'est pas vraiment ça attention).
%Such loading paths are supplemented by ODEs satisfied by the velocity components so that a closed form of the solution of the problem can be derived.

Experimental data collected on a thin-walled tube submitted to a dynamic tensile load \cite{Clifton_exp,Clifton_exp2} confirmed the existence of two distinct families of  simple waves, both involving combined stress paths.
These works nevertheless exhibited some discrepancies with the theory which have been attributed to the assumption made on the von-Mises yield surface.
As a matter of fact, a constant strain region lying between the fast and slow waves that is predicted by the theory \cite{Clifton} could not be seen in experimental results.
However, by following the endochronic theory of plasticity \cite{Valanis} which does not require the introduction of a yield surface, \textsc{Wu} and \textsc{Lin} \cite{Wu_experimental} obtained numerical results that better fit the experimental data provided by \textsc{Lipkin} and \textsc{Clifton} \cite{Clifton_exp2}.
The good agreement showed between numerical and experimental results \cite{Wu_experimental} thus confirmed the theory.

\textsc{Ting} and \textsc{Nan} \cite{Ting68} then generalized the work of \textsc{Bleich} and \textsc{Nelson} to hardening materials and \textsc{Ting} \cite{Ting69} widened that of \textsc{Clifton} to more complex loadings, that is a superimposition of one plane wave and two shear waves states.
Once again, the mathematical study of the ODE system governing the stress evolution inside fast and slow simple waves led to the construction of loading paths in stress space that depend on the external loads. A review of governing equations for all the cases depending on one space dimension considered above can be found in \cite{Nowacki}.

The information on characteristic structures thus provided has then been used by \textsc{Lin} and \textsc{Ballman} \cite{Lin_et_Ballman} for the development of an iterative Riemann solver.
This procedure is based on successive guesses of the stress state lying in the stationary region so that the loading paths predicted by the theory of \textsc{Clifton} \cite{Clifton} can be integrated numerically until convergence.
The implementation of this solver within a second-order Godunov scheme provided results that were in good agreement with the exact solutions.
Nevertheless, the theoretical investigations mentioned above restrict the development of such numerical tools to problems that depend on one space dimension.
%%
\textsc{Clifton} tackled the solution of plane strain problems in elastic-plastic solids by looking for bi-characteristics \cite{Clifton_thesis} in order to build finite difference schemes that account for plastic waves.
The point of view adopted here is that one can benefit from the simplifications introduced by the writing of Riemann problems in an arbitrary direction.
Indeed, the method of characteristics rather than the more complex method of bi-characteristics can be employed with the quasi-linear forms presented in chapter \ref{chap:chap2}.

%$\newline$
On the other hand, the existence of plastic shocks in solids under plane wave assumptions has been investigated by several authors. % \cite{Mandel1,Germain_shock,Mandel2,Claude}.
First, \textsc{Mandel} \cite{Mandel1} showed the existence of stable plastic shocks in three-dimensional elastoplastic media.
In this work, Hugoniot curves are built by assuming that the internal variables followed a radial loading path through a plastic shock.
\textsc{Lee} and \textsc{Germain} \cite{Germain_shock} considered that Hugoniot curves in elastic-plastic solids cannot be constructed without studying the internal structure of the shock.
Thus, an elastic-viscoplastic continuum problem is solved by magnifying the narrow region in the vicinity of the shock in which the fields vary sharply.
The shock solution was then taken as the limit when viscosity tends to zero.
A study of the internal structure of the shock has also been made by \textsc{Stolz} \cite{Claude}.
In the latter approach, the Hugoniot conditions across a shock moving at constant speed are derived by doing an asymptotic analysis. %a steady-state problem is solved through the shock.
The author thus provided existence and uniqueness conditions for a shock in compression provided that elastic stiffening dominates the (concave) hardening saturation.
Nevertheless, according to \textsc{Mandel} \cite{Mandel2}, such an analysis of the internal structure of the shock is not required to build Hugoniot curves, provided one chooses $\eps^p_1$ as internal variable and not the specific work $w$.
However, the propagation of plastic shocks is still an open scientific issue and subject to debate.

In what follows, simple waves are considered in elastic-plastic solids with natural initial conditions by assuming a concave hardening law with no stiffening so that plastic shocks do not arise.

% In those references high pressure impacts were considered so that the problem may be approximated as falling into the hydrodynamics theory.
% Hence, the state law describing the evolution of the hydrostatic pressure is responsible for the creation of shock waves due to colliding characteristics.
% This work gave rise to discussions about the need of studying the internal structure of the shock and more precisely to the nature of the load paths followed in the infinitely thin layer in the vicinity of the shock.

% While \textsc{Mandel} considered in \cite{Mandel1} that the radial plastic flow assumption holds across the shock, \textsc{Germain} and \textsc{Lee} \cite{Germain_shock}
%% Mandel suppose un trajet radial dans le choc et s'affranchit donc de l'analyse de la structure interne.
%% Germain utilise le travail plastic comme variable d'écrouissage. + regarde la structure interne du choc pour compléter les courbes d'hugoniot ?
%% Désaccord sur la variable d'écrouissage
%% Et puis Claude qui dit que la structure interne ne peut pas être déterminée si on ne connait que les états amont et aval...



%%% Local Variables:
%%% mode: latex
%%% TeX-master: "../mainManuscript"
%%% End:
