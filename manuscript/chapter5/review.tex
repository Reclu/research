% Researches done on the elastic-plastic bahavior of material at high strain rates for characterization purposes. 
% To this end, uni-axial stress or strain, pure bending or pure torsion problems have been investigated until the late 50's (vérifier ça). 
% Rakmathulin et Cristescu ont ouvert la voie à des problèmes plus complexes impliquant des chargements combinés.

% Elastic solver used previously

% \cite{Clifton_thesis} development of a method of characteristics in 3 independent variables that is then numerically approximated (Notion of bicharacteristics). Nous on n'en a pas besoin puisqu'on a déjà notre schéma numérique. En ravanche, on cherche à résoudre le problème dans une direction donnée pour lequel la méthode des caractéristiques s'applique.

% This is for instance the case for the simulation of forming technics that cannot, in general, be modeled in a one-dimensional setting.

Until the 50s, researches on dynamic problems in plastic solids were focused on uni-axial stress or strain, pure bending or pure torsion loading conditions \cite{Taylor,vonKarman}, (Voir \cite{Clifton_thesis}) and were led for materials characterization purposes.
The first references that brought some understanding about the response of linearly hardening solids to combined shear and pressure loads are those of Rakhmatulin \cite{Rakhmatulin} and Cristescu \cite{CRISTESCU19591605}.
Those early analytical investigations, made on plane stress impacts in the plastic regime, led to the conclusion that both elastic and combined-stress simple waves can occur in two-dimensional solids. 
While the former were well-known, the latter were shown to fall into the two \textit{fast waves} and \textit{slow waves} families.
Moreover, the maximal value of fast waves (\textit{resp. slow waves}) is higher than that of pressure (\textit{resp. shear}) plastic discontinuity in one-dimensional problem, at a given compression (\textit{resp. shear}) load amplitude.

Later, Bleich \cite{Bleich} considered sumperimposed plane and shear waves in an ideally elastic-plastic materials.
It has been shown in this work that different loading cases yield different characteristic structures of the solution of a Picard initial boundary value problem, thus revealing the complexity of plastic flows in more that one dimension. 
\thomas{see \cite[p.56 pdf]{Nowacki},\cite{Goel}}. 
% Clifton a ensuite pris un autre problème et est arrivé aux mêmes conclusions. Il a part ailleur ajouté le préchargement pour modifier la structure de la solution. Travaux qui ont été confirmés expérimentalement.
The same conclusions have been made by Clifton \cite{Clifton} for possibly nonlinear hardening materials.

, who furthermore studied the influence of plastic pre-loading on the solution.
These works have been supplementend and confirmed with experimental researches on a thin-walled \cite{Clifton_exp}.
Those two works were based on a strain formulation of the plastc flow theory. (Briefly recall equations and show some results ?)

On ne regarde qu'une dimension spatiale en faisant des hypothèse sur les champs alors que nous on se limite à une direction particulière $\vect{n}$.
\begin{itemize}
%% plane stress
%\item \cite{Rakhmatulin} ; \cite{CRISTESCU19591605} : $\tens{\sigma}=\matrice{\sigma & \tau & \\ \tau &\sigma_{22} & \\ & & }$;$\tens{\eps}=\matrice{\eps & \gamma/2 & \\ \gamma/2 &  & \\ & & }$ Strong discontinuity; Distinction between plastic waves ($c_1,c_2$) of strong discontinuities, and combined waves ($c_f,c_s$)
%% superimposition of plane wave and shear wave
%\item \cite{Bleich}: $\tens{\sigma}=\matrice{\sigma & \tau & \\ \tau & \sigma_{22} & \\ & & \sigma_{22}}$;$\tens{\eps}=\matrice{\eps & \gamma/2 & \\ \gamma/2 &  & \\ & & }$ ideal elastic-plastic continuum ; shock front at p.15 and subsection B + exact solutions ; identification of particular loading cases that lead to different characteristic structures.
%% combined shear and longitudinal waves
\item \cite{Clifton}: $\tens{\sigma}=\matrice{\sigma & \tau & \\ \tau & & \\ & & }$;$\tens{\eps}=\matrice{\eps & \gamma/2 & \\ \gamma/2 &  & \\ & & }$ Not a strong discontinuity + hardening + Emphasis unexpected stress structure (prestressed tubes)
%% superimposition of plane wave and shear wave
\item \cite{Ting68}: $\tens{\sigma}=\matrice{\sigma & \tau & \\ \tau & \sigma_{22} & \\ & & \sigma_{22}}$;$\tens{\eps}=\matrice{\eps & \gamma/2 & \\ \gamma/2 &  & \\ & & }$ with linear hardening materials; identification of particular loading cases that lead to different characteristic structures (p.8). Generalization of \cite{Bleich} to hardening materials and of \cite{Clifton} to more complex loadings ; Emphasis unexpected stress structure (prestressed tubes)
%% superimposition of plane wave and shear waveS
\item \cite{Ting69}:  $\tens{\sigma}=\matrice{\sigma & \tau_y & \tau_z\\ \tau_y & \sigma_{22} & \\\tau_z & & \sigma_{22}}$;$\tens{\eps}=\matrice{\eps & \gamma_y/2 & \gamma_z/2\\ \gamma_y/2 &  & \\ \gamma_z/2& & }$ ; superimposition of plane wave and shear waveS ; Emphasis unexpected stress structure (prestressed tubes)
\item \cite{Ting73}: c'est pas un peu une review des deux papiers d'avant ?
%% superimposition of plane wave and shear wave
\item \cite{Li_planeStress_EP}:$\tens{\sigma}=\matrice{\sigma & \tau & \\ \tau & \sigma_{22} & \\ & & \sigma_{22}}$;$\tens{\eps}=\matrice{\eps & \gamma/2 & \\ \gamma/2 &  & \\ & & }$
\end{itemize}

Il y a la question des vitesses charactéristiques plastiques... sont-elles collées aux vitesses élastiques ?


% Assumptions (isothermal, linear hardenings though extendable, small deformation, cartesian,concave constitutive model,rate independent)

% Un intérêt de développer ce qui est fait là est de s'autoriser à limiter les ondes plastique dans des méthodes de haut ordre.


% \subsection{The complexity of elastoplasticity}
% % Bibliography about:
% % - what has been done so far (thin-walled ; plane wave + shear wave)
% % - what is missing (solution of more general problems [i.e plane strain or stress] ; development of approximate Riemann solvers)
% % - what it will allow (comparison with experimental data in order to track plastic shocks) internal structure of plastic shock
% % - what is currently done in fluid or solid mechanics numerically
% Biblio, 
% %\section{State of the art}
% \subsection{The thin-walled tube problem}
% \cite{Clifton} + thesis
% \subsection{Superimposition of plane wave and shear waves}
% Citer Ting
% \section{Plane strain and plane stress problem}
% \subsection{General multi-dimensional formulation}
% % Based on the tangent modulus -> independent on the hardening model
% \subsection{Plane strain problems}
% \subsection{Plane stress problems}
% \section{Characteristic analysis}
% dependance des vitesses caractéristiques à l'angle entre la direction principale de sigma et la direction de propagation, c'est dit dans la thèse de Clifou en page 90.
% \subsection{Structure of the solution}
% \subsection{Integral curves and loading paths}
% \subsection{The plane strain case}
% \subsection{The plane stress case}



%%% Local Variables:
%%% mode: latex
%%% TeX-master: "../mainManuscript"
%%% End:
