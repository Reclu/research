We assume the infinitesimal strain tensor can be additively decomposed into an elastic part and a plastic part according to:
\begin{equation}
  \label{eq:ch5_partition}
  \tens{\eps}=\tens{\eps}^e+\tens{\eps}^p
\end{equation}
The elastic part of the infinitesimal strain tensor is:
\begin{equation}
  \label{eq:ch5_elastic_inverse}
  \tens{\eps}^e = \frac{1+\nu}{E} \tens{\sigma} - \frac{\nu}{E} \tr \tens{\sigma} \tens{I}
\end{equation}
Linear hardening
\begin{equation}
  \label{eq:ch5_von-Mises_yield}
  f\(\tens{\sigma},\Acb \)= \sqrt{\frac{3}{2}}\norm{\tens{s}-\tens{Y}} - \(Cp+\sigma^y\) \equiv 0
\end{equation}
$\tens{Y}=\frac{2}{3}C\tens{\eps}^p$ $C$ is the \textit{hardening modulus}.
At last, equations \eqref{eq:plastic_strain_rate} and \eqref{eq:p_evolution} can be successively introduced in the elastic law so that one gets \cite[eq (2.2.22)]{Simo}:
\begin{equation}
  \label{eq:elastoplastic_tangent}
  \tens{\dot{\sigma}}=\(\Cbb - \frac{6\mu^2}{3\mu +(C+R')}\tens{m}\otimes\tens{m} \):\tens{\dot{\eps}} = \Cbb^{ep}:\tens{\dot{\eps}}
\end{equation}
with $\Cbb^{ep}$ the \textit{elastoplastic tangent modulus}.
The \textit{elastoplastic acoustic tensor} is defined as:
\begin{equation}
  \label{eq:EP_acoustic}
  A_{ij}^{ep}= n_k C^{ep}_{kijl}n_l = A_{ij}^{elast} -  \frac{6\mu^2}{3\mu +(C+R')} (n_k m_{ki})(m_{jl}n_l)
\end{equation}

\subsubsection*{The general case}
Reparler de la surface de charge de von-Mises, et tout et tout.
\begin{equation}
  \label{eq:ch5_conservative}
  \Ucb_t + \sum_{i=1}^D \drond{\Fcb\cdot \vect{e}_i}{x_i} = \Scb
\end{equation}
 where the conserved quantities, fluxes and source term vectors are:
\begin{equation}
  \label{eq:ch5_vectorU}
  \Ucb =\matrice{\vect{v} \\ \tens{\eps}} \quad ; \quad \Fcb\cdot\vect{e}_i = \matrice{-\frac{1}{\rho}\tens{\sigma}\cdot\vect{e}_i\\-\frac{\vect{v}\otimes\vect{e}_i +\vect{e}_i \otimes\vect{v} }{2} } \quad ; \quad \Scb = \matrice{ \vect{b} \\\tens{0}} 
\end{equation}
The governing equations of dynamics in elastic-plastic solids are written in the following quasi-linear form:
\begin{equation}
  \Qcb_t + \Absf^i \drond{\Qcb}{x_i} = \Scb \qquad \text{with: }\Absf^i = -\matrice{\tens{0}^2 & \frac{1}{\rho}\tens{I}\otimes\vect{e}_i\\ \Cbb^{ep}\cdot \vect{e}_i & \tens{0}^4}  \label{eq:ch5_quasilinear}
\end{equation}
where $\Qcb=\matrice{\vect{v}\\ \tens{\sigma}}$ and $\Cbb^{ep}=\ddroit{\tens{\sigma}}{\tens{\eps}}=\Cbb - \beta\tens{m}\otimes\tens{m}$ is the tangent modulus with $\tens{m}=\frac{\tens{s}-\tens{Y}}{\norm{\tens{s}-\tens{Y}}}$ is the flow direction and $\beta=\frac{6\mu^2}{3\mu +(C+R')}$ depends on the shear modulus $\mu$ and kinematic or isotropic hardening modulus $C$ and $R'$ (see section \ref{sec:constitutive-equations}). In particular in the arbitrary direction $\vect{n}$:
\begin{equation}
  \Qcb_t + \Jbsf \drond{\Qcb}{x_n} = \Scb  \label{eq:ch5_quasilinear_normal}
\end{equation}
where $x_n=\vect{x}\cdot\vect{n}$ and the Jacobian matrix $\Jbsf=\Absf^in_i$ arises. The left characteristic fields $\{c_K;\Lcb^K\}$ satisfy the following equation:
\begin{equation}
  \label{eq:ch5_eigen_system}
  \vect{\Lc}^K \(\Jbsf - c_K \Ibsf\) = \vect{0}
\end{equation}
As seen in section \ref{sec:characteristic_analysis}, $6$ couples of characteristic speeds $c_K$ and left eigenvectors $\Lcb^K= \[ \vect{v}^K \: , \: \tens{S}^K \]$ are determined based on those of the acoustic tensor $\tens{A}^{ep}=\vect{n}\cdot\Cbb^{ep}\cdot \vect{n}=\tens{A}^{elast} -  \beta (\vect{n}\cdot\tens{m})\otimes(\tens{m}\cdot \vect{n})$, that are $\{\omega^p;\vect{l}^p\}$ for $p=1,2,3$:
\begin{equation}
  \label{eq:ch5_left_eigenfields}
  \left\lbrace \pm \sqrt{\frac{\omega_p}{\rho}} ; \quad \[\: \pm \rho\sqrt{\frac{\omega_p}{\rho}} \vect{l}^p , -\vect{l}^p\otimes \vect{n} \:\]  \right\rbrace ,\quad p=1,2,3
\end{equation}
In addition, three independent left eigenvectors associated to the zero eigenvalue of system \eqref{eq:ch5_quasilinear_normal}, which is of multiplicity $3$, are found by solving:
\begin{equation}
  \label{eq:ch5_null_eigen}
  \tens{\sigma}^K:\(\Cbb^{ep}\cdot  \vect{n}\) =\vect{0},\quad K=1,2,3
\end{equation}

\subsubsection*{Problems in two space dimensions}
We now focus on the solid domain bounded by $x_1 \times x_2 \times x_3 \in [0,\infty[ \times ]-\infty,\infty[ \times [-e,e]$ in a Cartesian coordinates system, where $e$ is an arbitrary length.
The solid is subject on the plane $x_1=0$ to a traction force $\vect{T}$ restricted to the $(\vect{e}_1,\vect{e}_2)$ plane, that is $T_3=0$. It is moreover assumed that all quantities except the velocity component $v_3$ depend solely on $x_1$ and $x_2$. Given the geometry of the problem, the vector $\vect{n}$ may be reduced to $\vect{e}_1$ or $\vect{e}_2$.

First, the solid is under plane strain, that is $\tens{\eps}\cdot\vect{e}_3=\vect{0}$, if for instance the velocity $v_3$ vanishes on both ends $x_3=\pm h$. In such a case, combination of equations \eqref{eq:ch5_partition} and \eqref{eq:ch5_elastic_inverse}, along with the kinematic condition $\eps_{33}=0$, allows to write a relation between $\sigma_{33}$ and other stress components:
\begin{equation}
  \label{eq:plane_strain_stress33}
  \sigma_{33}=\nu\(\sigma_{11}+\sigma_{22}\) - E\eps^p_{33}
\end{equation}

Conversely, if the planes $x_3=\pm e$ are traction free, a plane stress state reading $\tens{\sigma}\cdot\vect{e}_3=\vect{0}$ holds in the solid. For both plane strain and plane stress problems, the stress component $\sigma_{33}$ can then be removed from system \eqref{eq:ch5_quasilinear_normal}, leading to the unknown vector $\Qcb=[v_1,v_2,\sigma_{11},\sigma_{22},\sigma_{12}]$. The problem can then be solved in a two-dimensional setting, for which the acoustic tensor admits two distinct real eigenvalues:
\begin{subequations}
  \begin{alignat}{1}
    \label{eq:ch5_eigenAcc1}
    &\omega_1 = \frac{1}{2}\(A^{ep}_{11}+A^{ep}_{22} + \sqrt{(A^{ep}_{11}-A^{ep}_{22})^2+{4A^{ep}_{12}}^2}\) \\
    \label{eq:ch5_eigenAcc2}
    &\omega_2 = \frac{1}{2}\(A^{ep}_{11}+A^{ep}_{22} - \sqrt{(A^{ep}_{11}-A^{ep}_{22})^2+{4A^{ep}_{12}}^2}\)     
  \end{alignat}
\end{subequations}
with associated left eigenvector:
\begin{equation}
  \label{eq:ch5_eigenvectAcc}
  \vect{l}^p=\matrice{-A^{ep}_{12} \\ A^{ep}_{11}-\omega_p} = \matrice{ A^{ep}_{22}-\omega_p \\ -A^{ep}_{12}}
\end{equation}
One gets from equation \eqref{eq:ch5_left_eigenfields} that two families of waves with celerities $c_f=\pm \sqrt{\omega_1/\rho}$ and $c_s = \pm \sqrt{\omega_2/\rho}$ which are respectively referred to as \textit{fast} and \textit{slow} waves, may travel in the domain.
Four left eigenfields of the Jacobian matrix then read:
\begin{subequations}
  \begin{alignat}{1}
    \label{eq:ch5_Jac_eigenfield_fast}
    &\left\lbrace \pm c_f ; \quad \Lcb^{c_f^\pm}=\[\: \pm \rho c_f \vect{l}^1 , -\vect{l}^1\otimes \vect{n} \:\]  \right\rbrace \\
  \label{eq:ch5_Jac_eigenfield_slow}
    &\left\lbrace \pm c_s ; \quad \Lcb^{c_s^\pm}=\[\: \pm \rho c_s \vect{l}^2 , -\vect{l}^2\otimes \vect{n} \:\]  \right\rbrace
  \end{alignat}
\end{subequations}
where $\Lcb^{c_f^+}$ and $\Lcb^{c_f^-}$ are associated to the right-going and left-goind fast waves respectively. The same goes for $\Lcb^{c_s^-}$ and $\Lcb^{c_s^-}$. In addition, one stationary wave associated to the zero eigenvalue of the Jacobian matrix has to be added. The left eigenvector of the corresponding characteristic field satisfies equation \eqref{eq:ch5_null_eigen} and can be taken as:
\begin{equation}
  \label{eq:ch5_null_left_eigen}
  {\Lcb^0}^T = \matrice{v_1^0 \\[5.pt] v_2^0 \\[5.pt] \sigma_{11}^0 \\[5.pt] \sigma^0_{22} \\[5.pt] \sigma^0_{12} }= \matrice{0 \\[5.pt] 0 \\[5.pt] \(C^{ep}_{121i}C^{ep}_{222j}-C^{ep}_{221i}C^{ep}_{122j}\)n_in_j \\[5.pt] \(C^{ep}_{111i}C^{ep}_{122j}-C^{ep}_{112i}C^{ep}_{121j}\)n_in_j \\[5.pt] \(C^{ep}_{112i}C^{ep}_{221j}-C^{ep}_{111i}C^{ep}_{222j}\)\frac{n_in_j}{2}} = \matrice{0 \\ 0 \\ \alpha_{11} \\ \alpha_{22} \\ \alpha_{12} }
\end{equation}

Thus, the characteristic equations $\Lcb^K \cdot d\Qcb = 0$ yield:
\begin{subequations}
  \label{eq:ch5_ODEs}
  \begin{alignat}{3}
    \label{eq:charac_fr}
    & \rho c_f \vect{l}^1 \cdot d\vect{v} - l^1_i n_j d\sigma_{ij} =0 \qquad && \text{along }\: dx/dt = c_f\\
    \label{eq:charac_fl}
    -& \rho c_f \vect{l}^1 \cdot d\vect{v} - l^1_i n_j d\sigma_{ij} =0 \qquad && \text{along }\: dx/dt = - c_f \\
    \label{eq:charac_sr}
    & \rho c_s \vect{l}^2 \cdot d\vect{v} - l^2_i n_j d\sigma_{ij} =0 \qquad  && \text{along }\: dx/dt =  c_s \\
    \label{eq:charac_sl}
    -& \rho c_s \vect{l}^2 \cdot d\vect{v} - l^2_i n_j d\sigma_{ij} =0 \qquad  && \text{along }\: dx/dt = - c_s \\
    \label{eq:charac_contact}
    &\alpha_{11}d\sigma_{11} + \alpha_{12}d\sigma_{12} + \alpha_{22}d\sigma_{22}=0 \qquad && \text{along }\: dx/dt =0 
  \end{alignat}
\end{subequations}
Is is shown in \cite{Ting73} that the plastic celerities only depends on $\tens{\sigma}/\norm{\tens{\sigma}}$ so that they are constant along in ray of the stress space $(\sigma_{11}, \sigma_{22}, \sigma_{12})$.
For now, it is assumed that the characteristic speeds satisfy: $c_1 \geq c_f \geq c_2 \geq c_s \geq 0$ and that the plastic celerities are monotonically decreasing functions of the stress.

Using the following expressions for the eigenvectors of the acoustic tensor:
\begin{equation*}
  \vect{l}^1=\matrice{ A^{ep}_{22}- \rho c_f^2 \\ -A^{ep}_{12}} \quad ;\quad  \vect{l}^2=\matrice{ -A^{ep}_{12} \\A^{ep}_{11}-\rho c_s^2  }
\end{equation*}
\begin{figure}[h!]
  \centering
  {\input{chapter5/pgfFigures/simpleWaves}}
  \caption{Characteristics of slow, fast and stationary waves in the $(x_n,t)$ plane.}
  \label{fig:ch5_charac_method}
\end{figure}
%This is in particular true if we restrict our attention to the quarter-space $(\sigma_{11} \geq 0, \sigma_{22} \geq 0 , \sigma_{12}\geq 0)$ in which every components of the tensor  
%Assuming that no shock occurs, the integration of ODEs \eqref{eq:ch5_ODEs} yields simple wave solutions of the problem.
%This assumption seems to be valid with the convex flux function used in equation \eqref{eq:ch5_conservative} that leads to monotonically decreasing wave speeds with respect to the stress tensor. Furthermore, the medium is homogeneous 
%% Ne pas regarder genuinely non-linear car ça n'apporte rien. Ca donne juste une indication sur la variation des vitesses le long des courbes intégrales mais pas en fonction de la contrainte.





% It has been shown in \cite{Ting69} (homogeneous function so that c are constant along radial lines in the stress space)

%This is in paticular true when looking at the normal vectors $\vect{n} = \vect{e}_1$ and $\vect{n} = \vect{e}_2$ that yield an acoustic tensor $A_{ij}^{ep}=A_{ij}^{elas} - \beta m_{pi}m_{jq}n_p n_q\deta_{pq}$.

Clifton écrit l'équation caractéristique et ensuite cherche des solutions d'ondes simples. Donc on peut faire de même et regarder après l'équation caractéristiques comment évoluent les vitesses.








% Orthogonalité des loading paths \cite{Clifton,Ting68}

%%% Local Variables:
%%% mode: latex
%%% TeX-master: "../mainManuscript"
%%% End:
