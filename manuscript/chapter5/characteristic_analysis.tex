We assume the infinitesimal strain tensor can be additively decomposed into an elastic part and a plastic part according to:
\begin{equation}
  \label{eq:ch5_partition}
  \tens{\eps}=\tens{\eps}^e+\tens{\eps}^p
\end{equation}
The elastic part of the infinitesimal strain tensor is:
\begin{equation}
  \label{eq:ch5_elastic_inverse}
  \tens{\eps}^e = \frac{1+\nu}{E} \tens{\sigma} - \frac{\nu}{E} \tr \tens{\sigma} \tens{I}
\end{equation}

\subsubsection*{The general case}
The governing equations of dynamics in elastic-plastic solids are written in the following quasi-linear form:
\begin{equation}
  \Qcb_t + \Absf^i \drond{\Qcb}{x_i} = \Scb \qquad \text{with: }\Absf^i = -\matrice{\tens{0}^2 & \frac{1}{\rho}\tens{I}\otimes\vect{e}_i\\ \Cbb^{ep}\cdot \vect{e}_i & \tens{0}^4}  \label{eq:ch5_quasilinear}
\end{equation}
where $\Qcb=\matrice{\vect{v}\\ \tens{\sigma}}$ and $\Cbb^{ep}=\ddroit{\tens{\sigma}}{\tens{\eps}}=\Cbb - \beta\tens{m}\otimes\tens{m}$ is the tangent modulus with $\tens{m}=\frac{\tens{s}-\tens{Y}}{\norm{\tens{s}-\tens{Y}}}$ is the flow direction and $\beta=\frac{6\mu^2}{3\mu +(C+R')}$ depends on the shear modulus $\mu$ and kinematic or isotropic hardening modulus $C$ and $R'$ (see section \ref{sec:constitutive-equations}). In particular in the arbitrary direction $\vect{n}$:
\begin{equation}
  \Qcb_t + \Jbsf \drond{\Qcb}{x_n} = \Scb  \label{eq:ch5_quasilinear_normal}
\end{equation}
where $x_n=\vect{x}\cdot\vect{n}$ and the Jacobian matrix $\Jbsf=\Absf^in_i$ arises. The left characteristic fields $\{c_K;\Lcb^K\}$ satisfy the following equation:
\begin{equation}
  \label{eq:ch5_eigen_system}
  \vect{\Lc}^K \(\Jbsf - c_K \Ibsf\) = \vect{0}
\end{equation}
As seen in section \ref{sec:characteristic_analysis}, $6$ couples of characteristic speeds $c_K$ and left eigenvectors $\Lcb^K= \[ \vect{v}^K \: , \: \tens{S}^K \]$ are determined based on those of the acoustic tensor $\tens{A}=\vect{n}\cdot\Cbb^{ep}\cdot \vect{n}$, that are $\{\omega^p;\vect{l}^p\}$ for $p=1,2,3$:
\begin{equation}
  \label{eq:ch5_left_eigenfields}
  \left\lbrace \pm \sqrt{\frac{\omega_p}{\rho_0}} ; \quad \[\: \pm \rho_0\sqrt{\frac{\omega_p}{\rho_0}} \vect{l}^p , -\vect{l}^p\otimes \vect{N} \:\]  \right\rbrace ,\quad p=1,2,3
\end{equation}
In addition, three independent left eigenvectors associated to the zero eigenvalue of system \eqref{eq:ch5_quasilinear_normal}, which is of multiplicity $3$, are found by solving:
\begin{equation}
  \label{eq:ch5_null_eigen}
  \tens{\sigma}^K:\(\Cbb^{ep}\cdot  \vect{n}\) =\vect{0},\quad K=1,2,3
\end{equation}

\subsubsection*{Problems in two space dimensions}
We now focus on the solid domain bounded by $x_1 \times x_2 \times x_3 \in [0,\infty[ \times ]-infty,infty[ \times [-e,e]$ in a Cartesian coordinates system, where $e$ is an arbitrary length.
The solid is subject on the plane $x_1=0$ to a traction force $\vect{T}$ restricted to the $(\vect{e}_1,\vect{e}_2)$ plane, that is $T_3=0$. It is moreover assumed that all quantities except the velocity component $v_3$ depend solely on $x_1$ and $x_2$.

First, the solid is under plane strain, that is $\tens{\eps}\cdot\vect{e}_3=\vect{0}$, if the velocity $v_3$ vanishes on both ends $x_3=\pm h$. Thus, combination of equations \eqref{eq:ch5_partition} and \eqref{eq:ch5_elastic_inverse}, along with the kinematic condition $\eps_{33}=0$, allows to write a relation between $\sigma_{33}$ and other stress components:
\begin{equation}
  \label{eq:plane_strain_stress33}
  \sigma_{33}=\nu\(\sigma_{11}+\sigma_{22}\) - E\eps^p_{33}
\end{equation}

Conversely, if the planes $x_3=\pm e$ are traction free, a plane stress state reading $\tens{\sigma}\cdot\vect{e}_3=\vect{0}$ holds in the solid. For both plane strain and plane stress problems, the stress component $\sigma_{33}$ can then be removed from system \eqref{eq:ch5_quasilinear_normal}, leading to the unknown vector $\Qcb=[v_1,v_2,\sigma_{11},\sigma_{22},\sigma_{12}]$. The problem can then be solved in a two-dimensional setting, for which the acoustic tensor admits two distinct real eigenvalues:
\begin{subequations}
  \begin{alignat}{1}
    \label{eq:ch5_eigenAcc1}
    &\omega_1 = \frac{1}{2}\(A_{11}+A_{22} - \sqrt{(A_{11}-A_{22})^2+4A_{12}}\) \\
    \label{eq:ch5_eigenAcc2}
    &\omega_2 = \frac{1}{2}\(A_{11}+A_{22} + \sqrt{(A_{11}-A_{22})^2+4A_{12}}\) 
  \end{alignat}
\end{subequations}
with associated left eigenvector:
\begin{equation}
  \label{eq:ch5_eigenvectAcc}
  \vect{l}^p=\matrice{-A_{12} \\ A_{11}-\omega_p} = \matrice{ A_{22}-\omega_p \\ -A_{12}}
\end{equation}
From equation \eqref{eq:ch5_left_eigenfields}, one gets that the problem involves two families of waves travelling at speeds $c_1 = \pm \sqrt{\omega_1/\rho}$ and $c_2 = \pm \sqrt{\omega_2/\rho}$. For elastic evolutions, th

that are referred to as \textit{slow} and \textit{fast} waves, travelling respectively at speeds $c_s = \pm \sqrt{\omega_1/\rho}$ and $c_f = \pm \sqrt{\omega_2/\rho}$.
Considering equation \eqref{eq:ch5_left_eigenfields}, the problems involve 









% Orthogonalité des loading paths \cite{Clifton,Ting68}

%%% Local Variables:
%%% mode: latex
%%% TeX-master: "../mainManuscript"
%%% End:
