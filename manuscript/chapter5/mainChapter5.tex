\chapter{Contribution to the solution of elastic-plastic problems in two space dimensions}
%% Faire un historique des formulations faites.
%% Formuler le problème à notre sauce et identifier les cas évoqués en intro en particularisant les modules tangents etc. a ce moment, parler des trajets de chargement et des types d'ondes

\section*{Introduction}
It has been shown throughout this manuscript that hyperbolic problems of solid mechanics are solved in a different manner depending on the numerical explicit method employed. 
In particular, irreversible deformations which are usually computed numerically based on well-known constitutive integrators, may greatly differ from one scheme to another, even for one-dimensional problems.
However, the accurate assessment of residual stresses and strains are of major importance for many industrial applications such as, among others, high-speed metal forming, crash-proof design or the study of earthquakes impact on structures.
The simulations performed in chapter \ref{chap:chap4} emphasized the improvements enabled by the knowledge of the characteristic structure of the solution of conservation laws systems, especially for elastoplastic solids.
Nevertheless, the introduction of the exact solution by means of approximate Riemann solvers is so far only carried out for one-dimensional and specific multi-dimensional loading cases for elastic-plastic solids.

The purpose of this chapter is to provide solutions and clues for future works for more general two-dimensional elastoplasticity problems under small strains. 
More information on the structure of solutions to these problems allow a better understanding of physical phenomena occuring in media on the one hand, and the ability of accurately deal with them numerically on the other hand.
The chapter is organized as follows.
A brief historical review of the solution of plastic waves in two-dimension space is made in section \ref{sec:review}.
Then, the equations of plastic flow, which characteristic analysis is carried out, are recalled in section \ref{sec:charac_plast}.
Attention is next paid in section \ref{sec:stress_paths} to the evolution of stress components through simple waves possibly arising in the solution. 
Some thus identified loading paths are finally discussed in section \ref{sec:ep_Riemman_solver} in the context of the development of dedicated approximate Riemann solvers. 

\section{Historical review}
\label{sec:review}
% Researches done on the elastic-plastic bahavior of material at high strain rates for characterization purposes. 
% To this end, uni-axial stress or strain, pure bending or pure torsion problems have been investigated until the late 50's (vérifier ça). 
% Rakmathulin et Cristescu ont ouvert la voie à des problèmes plus complexes impliquant des chargements combinés.

% Elastic solver used previously

% \cite{Clifton_thesis} development of a method of characteristics in 3 independent variables that is then numerically approximated (Notion of bicharacteristics). Nous on n'en a pas besoin puisqu'on a déjà notre schéma numérique. En ravanche, on cherche à résoudre le problème dans une direction donnée pour lequel la méthode des caractéristiques s'applique.

% This is for instance the case for the simulation of forming technics that cannot, in general, be modeled in a one-dimensional setting.

Until the 50s, researches on dynamic problems in plastic solids were focused on uni-axial stress or strain, pure bending or pure torsion loading conditions \cite{Taylor,vonKarman}, and were carried out for materials characterization purposes.
The first references that brought some understanding about the response of linearly hardening solids to combined shear and pressure loads are those of Rakhmatulin \cite{Rakhmatulin} and Cristescu \cite{CRISTESCU19591605}.
These early analytical investigations on plane stress impacts in the plastic regime led to the conclusion that elastic waves, as well as plastic combined-stress simple waves, can propagate in two-dimensional solids. 
While the former were well-known, the latter were shown to fall into the two \textit{fast waves} and \textit{slow waves} families.
The maximal value of fast waves (\textit{resp. slow waves}) is higher than that of pressure (\textit{resp. shear}) plastic discontinuity occuring in one-dimensional problems, for a given compression (\textit{resp. shear}) load amplitude.

Later, Bleich and Nelson \cite{Bleich} considered sumperimposed plane and shear waves in an ideally elastic-plastic materials submitted to step loads.
It has thus been highlighted that different loading cases yield different characteristic structures of the solution of a Picard problem, thus revealing the complexity of plastic flows in more than one dimension.
% Distinguer un peu plus ces deux contributions.
%\thomas{see \cite[p.56 pdf]{Nowacki},\cite{Goel}}. 
The same conclusions have been drawn by Clifton \cite{Clifton} for hardening materials under tension-torsion, who furthermore studied the influence of plastic pre-loading on the solution.
This contribution established the existence of loading paths through the simple waves arising from the characteristic analysis of the hyperbolic system.
Indeed, the combined-stress wave nature lies in ODEs which govern the evolution of stress components within the simple waves.
The integration of these equations of the form $d\sigma_{11}=\psi d\sigma_{12}$ allows the building of curves which connect the applied stress state of the Picard problem $(\sigma^d_{11},\sigma^d_{12})$ to the initial state of the medium.
% Indeed, the study mathematical properties of relations between stress components of the form $d\sigma_{11}=\psi d\sigma_{12}$, satisifed inside fast and slow simple waves, allows to connect the applied stress state of the Picard problem $(\sigma^d_{11},\sigma^d_{12})$ to the initial state of the medium.
It has been for instance shown that if a solid is acted upon by a traction force such that $\sigma^d_{11}=0$ and $\sigma^d_{12}$ lies outside the elastic convex, only an elastic shear discontinuity, followed by a slow simple wave, propagates.
Conversely, other loading conditions may lead to the combination of elastic pressure discontinuity and a fast wave, possibly followed by a slow wave.
Another notable conclusion is that the combined loading paths followed inside simple waves may lead to plastic unloading, while only elastic unloading occurs in the one-dimensional theory.
%In addition, it is possible to meet unloading plastic simple waves with contrast to the one-dimensional theory in which the unloading waves propagate at elastic speeds (c'est pas vraiment ça attention).
%Such loading paths are supplemented by ODEs satisfied by the velocity components so that a closed form of the solution of the problem can be derived.

Experimental data collected on a thin-walled tube submitted to a dynamic tensile load \cite{Clifton_exp,Clifton_exp2} confirmed the existence of two distinct families of  simple waves, both involving combined stress paths.
Those works nevertheless exhibited some discrepancies with the theory which have been attributed to the assumption made on the von-Mises yield surface.
As a matter of fact, a constant strain region lying between the fast and slow waves that is predicted by the theory \cite{Clifton} could not be seen in experimental results.
However, by following the endochronic theory of plasticity \cite{Valanis} which does not require the introduction of a yield surface, Wu and Lin \cite{Wu_experimental} obtained numerical results that better fitted the experimental data provided by Lipkin and Clifton \cite{Clifton_exp2}.
The good agreement showed between numerical and experimental results \cite{Wu_experimental} thus confirmed the theory.

Ting and Nan \cite{Ting68} then generalized the work of Bleich and Nelson to hardening materials and Ting \cite{Ting69} widened this of Clifton to more complex loadings, that is a superimposition of one plane wave and two shear waves states.
Once again, the mathematical study of the ODE system governing the stresses evolution inside fast and slow simple waves led to the construction of loading paths in the stress space that depend on the external loads. A review of governing equations for all the cases depending on one space dimension considered above can be found in \cite{Nowacki}.

The information on characteristic structures thus provided has then be used by Lin and Ballman \cite{Lin_et_Ballman} for the development of an iterative Riemann solver.
This procedure is based on successive guesses on the stress state lying in the stationary region so that the loading paths preticted by the theory of Clifton \cite{Clifton} can integrated numerically until convergence.
The implementation of this solver within a second-order Godunov scheme provided results that were in good agreement the exact solutions.
Nevertheless, the theoretical investigations mentioned above restrict the development of such numerical tools to problems that depend on one space dimension.
%%
Clifton tackled the solution of plane strain problems in elastic-plastic solids by looking for bi-characteristics \cite{Clifton_thesis} in order to build finite difference schemes that account for plastic waves.
The point of view adopted here is that one can benefit from the simplifications introduced by the writing of Riemann problems in an arbitray direction of space.
Indeed, the method of characteristics rather than the more complex method of bi-characteristics can be employed with the quasilinear forms presented in chapter \ref{chap:chap2}.




%%% Local Variables:
%%% mode: latex
%%% TeX-master: "../mainManuscript"
%%% End:


\section{Elastic-plastic wave structure in two space dimensions}
\label{sec:charac_plast}
We assume the infinitesimal strain tensor can be additively decomposed into an elastic part and a plastic part according to:
\begin{equation}
  \label{eq:ch5_partition}
  \tens{\eps}=\tens{\eps}^e+\tens{\eps}^p
\end{equation}
The elastic part of the infinitesimal strain tensor is:
\begin{equation}
  \label{eq:ch5_elastic_inverse}
  \tens{\eps}^e = \frac{1+\nu}{E} \tens{\sigma} - \frac{\nu}{E} \tr \tens{\sigma} \tens{I}
\end{equation}

\subsubsection*{The general case}
The governing equations of dynamics in elastic-plastic solids are written in the following quasi-linear form:
\begin{equation}
  \Qcb_t + \Absf^i \drond{\Qcb}{x_i} = \Scb \qquad \text{with: }\Absf^i = -\matrice{\tens{0}^2 & \frac{1}{\rho}\tens{I}\otimes\vect{e}_i\\ \Cbb^{ep}\cdot \vect{e}_i & \tens{0}^4}  \label{eq:ch5_quasilinear}
\end{equation}
where $\Qcb=\matrice{\vect{v}\\ \tens{\sigma}}$ and $\Cbb^{ep}=\ddroit{\tens{\sigma}}{\tens{\eps}}=\Cbb - \beta\tens{m}\otimes\tens{m}$ is the tangent modulus with $\tens{m}=\frac{\tens{s}-\tens{Y}}{\norm{\tens{s}-\tens{Y}}}$ is the flow direction and $\beta=\frac{6\mu^2}{3\mu +(C+R')}$ depends on the shear modulus $\mu$ and kinematic or isotropic hardening modulus $C$ and $R'$ (see section \ref{sec:constitutive-equations}). In particular in the arbitrary direction $\vect{n}$:
\begin{equation}
  \Qcb_t + \Jbsf \drond{\Qcb}{x_n} = \Scb  \label{eq:ch5_quasilinear_normal}
\end{equation}
where $x_n=\vect{x}\cdot\vect{n}$ and the Jacobian matrix $\Jbsf=\Absf^in_i$ arises. The left characteristic fields $\{c_K;\Lcb^K\}$ satisfy the following equation:
\begin{equation}
  \label{eq:ch5_eigen_system}
  \vect{\Lc}^K \(\Jbsf - c_K \Ibsf\) = \vect{0}
\end{equation}
As seen in section \ref{sec:characteristic_analysis}, $6$ couples of characteristic speeds $c_K$ and left eigenvectors $\Lcb^K= \[ \vect{v}^K \: , \: \tens{S}^K \]$ are determined based on those of the acoustic tensor $\tens{A}=\vect{n}\cdot\Cbb^{ep}\cdot \vect{n}$, that are $\{\omega^p;\vect{l}^p\}$ for $p=1,2,3$:
\begin{equation}
  \label{eq:ch5_left_eigenfields}
  \left\lbrace \pm \sqrt{\frac{\omega_p}{\rho_0}} ; \quad \[\: \pm \rho_0\sqrt{\frac{\omega_p}{\rho_0}} \vect{l}^p , -\vect{l}^p\otimes \vect{N} \:\]  \right\rbrace ,\quad p=1,2,3
\end{equation}
In addition, three independent left eigenvectors associated to the zero eigenvalue of system \eqref{eq:ch5_quasilinear_normal}, which is of multiplicity $3$, are found by solving:
\begin{equation}
  \label{eq:ch5_null_eigen}
  \tens{\sigma}^K:\(\Cbb^{ep}\cdot  \vect{n}\) =\vect{0},\quad K=1,2,3
\end{equation}

\subsubsection*{Problems in two space dimensions}
We now focus on the solid domain bounded by $x_1 \times x_2 \times x_3 \in [0,\infty[ \times ]-infty,infty[ \times [-e,e]$ in a Cartesian coordinates system, where $e$ is an arbitrary length.
The solid is subject on the plane $x_1=0$ to a traction force $\vect{T}$ restricted to the $(\vect{e}_1,\vect{e}_2)$ plane, that is $T_3=0$. It is moreover assumed that all quantities except the velocity component $v_3$ depend solely on $x_1$ and $x_2$.

First, the solid is under plane strain, that is $\tens{\eps}\cdot\vect{e}_3=\vect{0}$, if the velocity $v_3$ vanishes on both ends $x_3=\pm h$. Thus, combination of equations \eqref{eq:ch5_partition} and \eqref{eq:ch5_elastic_inverse}, along with the kinematic condition $\eps_{33}=0$, allows to write a relation between $\sigma_{33}$ and other stress components:
\begin{equation}
  \label{eq:plane_strain_stress33}
  \sigma_{33}=\nu\(\sigma_{11}+\sigma_{22}\) - E\eps^p_{33}
\end{equation}

Conversely, if the planes $x_3=\pm e$ are traction free, a plane stress state reading $\tens{\sigma}\cdot\vect{e}_3=\vect{0}$ holds in the solid. For both plane strain and plane stress problems, the stress component $\sigma_{33}$ can then be removed from system \eqref{eq:ch5_quasilinear_normal}, leading to the unknown vector $\Qcb=[v_1,v_2,\sigma_{11},\sigma_{22},\sigma_{12}]$. The problem can then be solved in a two-dimensional setting, for which the acoustic tensor admits two distinct real eigenvalues:
\begin{subequations}
  \begin{alignat}{1}
    \label{eq:ch5_eigenAcc1}
    &\omega_1 = \frac{1}{2}\(A_{11}+A_{22} - \sqrt{(A_{11}-A_{22})^2+4A_{12}}\) \\
    \label{eq:ch5_eigenAcc2}
    &\omega_2 = \frac{1}{2}\(A_{11}+A_{22} + \sqrt{(A_{11}-A_{22})^2+4A_{12}}\) 
  \end{alignat}
\end{subequations}
with associated left eigenvector:
\begin{equation}
  \label{eq:ch5_eigenvectAcc}
  \vect{l}^p=\matrice{-A_{12} \\ A_{11}-\omega_p} = \matrice{ A_{22}-\omega_p \\ -A_{12}}
\end{equation}
From equation \eqref{eq:ch5_left_eigenfields}, one gets that the problem involves two families of waves travelling at speeds $c_1 = \pm \sqrt{\omega_1/\rho}$ and $c_2 = \pm \sqrt{\omega_2/\rho}$. For elastic evolutions, th

that are referred to as \textit{slow} and \textit{fast} waves, travelling respectively at speeds $c_s = \pm \sqrt{\omega_1/\rho}$ and $c_f = \pm \sqrt{\omega_2/\rho}$.
Considering equation \eqref{eq:ch5_left_eigenfields}, the problems involve 









% Orthogonalité des loading paths \cite{Clifton,Ting68}

%%% Local Variables:
%%% mode: latex
%%% TeX-master: "../mainManuscript"
%%% End:


\section{Loading paths through simple waves}
\label{sec:stress_paths}
% On ne regarde qu'une dimension spatiale en faisant des hypothèse sur les champs alors que nous on se limite à une direction particulière $\vect{n}$.
% En plus, on se limite à l'étude d'ondes simples alors que des chocs peuvent exister (voir Mandell car il semble y etre démontré que les shock n'arrivent que pour $\tau=0$).
% Il y a la question des vitesses charactéristiques plastiques... sont-elles collées aux vitesses élastiques ?
% dependance des vitesses caractéristiques à l'angle entre la direction principale de sigma et la direction de propagation, c'est dit dans la thèse de Clifou en page 90.

\subsection{Properties of the loading paths}
The stress paths followed within slow and fast simple waves is governed by the mathematical properties of functions $\psi^s_1$ and $\psi^f_1$ involved in the relations of table \ref{tab:simpleWavesEquations}.
Then, before specializing the discussion to plane stress and plane strain cases, some general properties holding regardless of the loading conditions are highlighted.
The analysis is here carried out for the special case $\vect{n}=\vect{e}_1$, similar results being obtained for the other situation $\vect{n}=\vect{e}_2$.

First, the functions are orthogonal in the stress space, that is $\psi^s_1\psi^f_1=-1$.
Indeed, considering the left eigenvectors of the acoustic tensor in equation \eqref{eq:ch5_eigenvectAcc}, the product $\psi^s_1\psi^f_1$ reads:
\begin{equation*}
  \psi^s_1\psi^f_1 = \frac{l^1_2}{l^1_1}\: \frac{l_2^2}{l^2_1} = \frac{(A_{11}-\omega_2)A_{12}}{(A_{22}-\omega_1)A_{12}}
\end{equation*}
Introduction of the expressions of eigenvalues $\omega_i$ from equations \eqref{eq:ch5_eigenAcc1} and \eqref{eq:ch5_eigenAcc1} further yields:
\begin{equation*}
  \psi^s_1\psi^f_1 = \frac{A_{11} -A_{22} +\sqrt{(A_{11} -A_{22} )^2 + 4A_{12}^2 }}{A_{22} -A_{11} -\sqrt{(A_{11} -A_{22} )^2 + 4A_{12}^2 }}=-1
\end{equation*}
In other words, $\vect{l}^1 \cdot \vect{l}^2=0$ as expected by the symmetry of $\tens{A}$.
This orthogonality has already been noticed for particular plane strain and plane stress cases \cite{Clifton,Ting68} but now obviously appears as true for all problems in two space dimensions.
%Moreover, since $\psi^s_2=1/\psi^s_1$ and $\psi^f_2=1/\psi^f_1$, the same holds for the functions $\psi^s_2$ and $\psi^f_2$.
Therefore, it allows us to restrict the study to one function only, say $\psi_1^f$.

Second, if the function $\psi_1^f$ vanishes at some point of the stress space, the projection of the stress path in the $(\sigma_{11},\sigma_{12})$ plane is vertical according to the ODE \eqref{eq:sigSlow_n=e1}.
Conversely, if the inverse of $\psi_1^f\rightarrow \infty$, the loading path is horizontal in the $(\sigma_{11},\sigma_{12})$ plane.
%Looking for vanishing $\psi^f_1$ or $1/\psi^f_1$ amounts to finding roots of the components of $\vect{l}^2$:
It then comes out:
\begin{subequations}
  \begin{alignat}{1}
    \label{eq:first_root}
    \psi_1^f = 0  & \Leftrightarrow A_{12} =0  \\
    \label{eq:second_root}
    \psi_1^f\rightarrow \infty & \Leftrightarrow A_{11} -\omega_2 =0
  \end{alignat}
\end{subequations}
In particular, if $A_{12}=0$ the second equation reads:
\begin{equation}
  A_{11} -\omega_2 = \frac{1}{2}\(A_{11} -A_{22} +\sqrt{(A_{11} -A_{22} )^2 + 4A_{12}^2 }\) = \left\langle A _{11}-A _{22}  \right\rangle
\end{equation}
where $\left\langle \bullet \right\rangle$ denotes the positive part operator.
Hence, if $A_{12} =0$ and $A_{11} \neq A_{22} $, one has $\psi^f_1 \rightarrow \infty$ and $\psi^s_1 = 0$, so that the stress path in the ($\sigma_{11},\sigma_{12}$) plane are horizontal (\textit{resp. vertical}) through a fast (\textit{resp. slow}) wave. 
On the other hand, if $A_{11}  = A_{22} $, both components of the eigenvectors vanish and the functions $\psi^f_1$ and $\psi^s_1$ are undetermined.
At last, it follows from equation \eqref{eq:diff_celerities} that the simultaneous of conditions \eqref{eq:first_root} and \eqref{eq:second_root} leads to characteristic speeds of simple waves that are identical. Hence, the situation $c_f=c_s$ corresponds to a loss of hyperbolicity of the system.


The above discussion is now specified to plane stress and plane strain, for which loading conditions leading to $A_{12} =0$ and $A _{11}-A _{22}=0$ are identified.
\subsection{The plane stress case}
The case of plane stress is first considered by using the tengent modulus of equation \eqref{eq:CP_constitutive}.
\begin{subequations}
  \begin{alignat}{1}
    \label{eq:CP_A11}
    & \tilde{A}_{11}^{ep}= \tilde{C}^{ep}_{1111} - \frac{(\tilde{C}^{ep}_{1133})^2}{\tilde{C}^{ep}_{3333}} = \lambda + 2\mu -\beta s_{11}^2 -\frac{\(\lambda -\beta s_{11}s_{33}\)^2}{\lambda + 2\mu - \beta s_{33}^2} \\
    \label{eq:CP_A22}
    & \tilde{A}_{22}^{ep}= \tilde{C}^{ep}_{1212} - \frac{(\tilde{C}^{ep}_{1233})^2}{\tilde{C}^{ep}_{3333}}= \mu - \beta s_{12}^2 -\frac{\(\beta s_{12}s_{33}\)^2}{\lambda + 2\mu - \beta s_{33}^2} \\
    \label{eq:CP_A12}
    & \tilde{A}_{12}^{ep} = \tilde{C}^{ep}_{1112} - \frac{\tilde{C}^{ep}_{1133}\tilde{C}^{ep}_{1233}}{\tilde{C}^{ep}_{3333}} =\beta s_{12} \frac{\lambda s_{33} - (\lambda + 2\mu)s_{11} }{\lambda + 2\mu - \beta s_{33}^2} 
  \end{alignat}
\end{subequations}

First, in order to ensure the hyperbolicity of the system, the component of the acoustic tensor also have to be defined, that is $\tilde{C}^{ep}_{3333}\neq 0$. This condition leads to:
\begin{equation*}
  \lambda + 2\mu - \beta s_{33}^2 \neq 0 \quad \Leftrightarrow \quad s_{33}\neq \frac{\lambda + 2\mu}{\beta}
\end{equation*}

Second, from equation \eqref{eq:CP_A12}, $\tilde{A}_{12}^{ep}$ admits two roots in terms of the components of the deviatoric stress tensor, namely: 
\begin{equation}
  s_{12}=0 \quad ; \quad s_{11}= \frac{\lambda}{\lambda+2\mu}s_{33}
\end{equation}
In terms of the components of Cauchy stress tensor, those conditions read:
% \begin{align}
%   & \frac{2}{3}\sigma_{11}-\frac{1}{3}\sigma_{22} = -\frac{\lambda}{3\lambda+6\mu}(\sigma_{11}+\sigma_{22}) \\
%   & 2\sigma_{11}-\sigma_{22} = -\frac{\lambda}{\lambda+2\mu}(\sigma_{11}+\sigma_{22}) \\
%   & \sigma_{11}(2 +\frac{\lambda}{\lambda+2\mu})=\sigma_{22}(1-\frac{\lambda}{\lambda+2\mu})\\
%   & \sigma_{11}\frac{3\lambda+4\mu}{\lambda+2\mu}=\sigma_{22}\frac{2\mu}{\lambda+2\mu}\\
%   & \sigma_{11}=\sigma_{22}\frac{2\mu}{3\lambda+4\mu}
% \end{align}
\begin{equation}
  \label{eq:CP_roots}
  \sigma_{12}=0 \quad ; \quad \sigma_{11}=\frac{2\mu}{3\lambda+4\mu}\sigma_{22}
  % \sigma_{12}=0 \quad ; \quad \sigma_{11}=\frac{1-2\nu}{2-\nu} \sigma_{22}
\end{equation}
Hence, the loading path through a fast simple wave is vertical, that is $\psi^f_1 = 0$, for stress values satisfying \eqref{eq:CP_roots}, providing that the $\tilde{A}_{11}^{ep}$ and $\tilde{A}_{22}^{ep}$ are not equal.
Conversely, such stress states yield horizontal path through a slow wave.
\begin{remark}
  \label{rq:loading_paths_CP1}
  The latter result is particularly interesting if considering a loading path starting from a point of the plane $\sigma_{12}=0$.
  Indeed, the loading path followed through a slow simple wave (equation \eqref{eq:sigSlow_n=e1}) is such that $d\sigma_{12}=0$ so that no change in that component of stress occurs.
\end{remark}
%% Attempt to show that if s12=0, cs=c2 but depends on the sign of A11-A22
% \begin{align}
%   &\omega_2=\frac{1}{2}\(A_{11}+A_{22} - \abs{A_{11}-A_{22}}\)\\
%   &\omega_2=\frac{1}{2}\(A_{11}+A_{22} - A_{11}+A_{22}\) \quad ;\quad \omega_2=\frac{1}{2}\(A_{11}+A_{22} + A_{11}-A_{22}\) \\
%   &\omega_2=A_{22} \quad ;\quad \omega_2=A_{11}
% \end{align}


% \paragraph*{Case $s_{12}=0$ :}
% \begin{align}
%   & \rho c_s^2 =\frac{1}{2}\(\tilde{A}_{11}^{ep}+\tilde{A}_{22}^{ep} - \abs{\tilde{A}_{11}^{ep}-\tilde{A}_{22}^{ep}}\) \\
%   & \rho c_f^2 =\frac{1}{2}\(\tilde{A}_{11}^{ep}+\tilde{A}_{22}^{ep} + \abs{\tilde{A}_{11}^{ep}-\tilde{A}_{22}^{ep}}\)
% \end{align}

If on the other hand, on considers the vector $\vect{n}=\vect{e}_2$, the same procedure as above yields:

????????????????????????????????????????????????????????

\textit{Brief summary of the results.}


The complexity introduced by the plane stress tangent modulus prevent the finding of other singular configurations for the hyperbolic system. 
In particular, it is difficult to deal with the equation $\tilde{A}^{ep}_{11}=\tilde{A}^{ep}_{22}$ with the expressions given in equations \eqref{eq:CP_A11} and \eqref{eq:CP_A22}.
As we shall see below, more singular behavior can be identified for plane strain.




\subsection{The plane strain case}
% The expressions of the tangent modulus and the acoustic tensors are recalled here for convenience:
% \begin{align}
%   & C^{ep}_{ijkl} = \lambda \delta_{ij}\delta_{kl} + \mu \(\delta_{il}\delta_{jk} + \delta_{ik}\delta_{jl}\) - \beta s_{ij}s_{kl} \\
%   & A^{ep}_{ij} = \lambda n_i n_j + \mu \(n_k n_k \delta_{ij} +n_i n_j \) - \beta s_{ip}n_p s_{jq}n_q
% \end{align}
% where $s_{ij}$ are the components of the deviatoric part of Cauchy stress tensor, that is $s_{ij}=\sigma_{ij} - \frac{1}{3}\sigma_{kk}\delta_{ij}$. 
The elastoplastic tangent modulus under consideration is now that given in equation \eqref{eq:elastoplastic_tangent}, so that the components of the acoustic tensor read: 
\begin{subequations}
  \begin{alignat}{1}
    \label{eq:DP_A11}
    & A_{11}^{ep}= C_{1111}^{ep} = \lambda + 2\mu -\beta s_{11}^2 \\
    \label{eq:DP_A22}
    & A_{22}^{ep}= C_{1212}^{ep}= \mu -\beta s_{12}^2 \\
    \label{eq:DP_A12}
    & A_{12}^{ep}= C_{1112}^{ep}=-\beta s_{11}s_{12}
  \end{alignat}
\end{subequations}
and the associated eigenvalues are:
\begin{subequations}
  \label{eq:eigen_acc_DP}
  \begin{alignat}{1}
    \label{eq:eigen_acc_DP1}
    & \rho c_s^2 = \frac{1}{2}\( \lambda +3\mu -\beta (s_{11}^2+ s_{12}^2) - \sqrt{(\lambda + \mu -\beta (s_{11}^2-s_{12}^2) )^2 +4(\beta s_{11}s_{12})^2} \) \\
    \label{eq:eigen_acc_DP2}
    & \rho c_f^2 = \frac{1}{2}\( \lambda +3\mu -\beta (s_{11}^2+ s_{12}^2) + \sqrt{(\lambda + \mu -\beta (s_{11}^2-s_{12}^2) )^2 +4(\beta s_{11}s_{12})^2}  \)
  \end{alignat}
\end{subequations}
From equation \eqref{eq:DP_A12}, we see that $A_{12}^{ep}$ vanishes for $s_{12}=0$ and $s_{11}=0$, each solution being studied in more details hereinafter.

%% Sign of one of the functions psi... but not used afterwards
% We first study the sign of the functions $\psi^f$ by noticing that $\mu=\rho c_2^2$ so that $A_{22}^{ep}$ may be rewritten to yield:
% \begin{equation*}
%   \psi^f = -\frac{A_{12}^{ep}}{A_{22}-\rho c_f^2}= -\frac{\beta s_{11}s_{12}}{\rho c_f^2-\rho c_2^2 +\beta s_{12}^2 }
% \end{equation*}
% Since the denominator is positive for $c_f \geq c_2$, it comes out that $\sign (\psi^f) = - \sign(s_{12}) \sign(s_{11})$. Moreover, two roots of the loading function $\psi^f$ can be identified.

%Next, from the expressions of the acoustic tensor components, we see that particular cases leading to $\psi^f_1\rightarrow \infty$ or $\psi^f_1\rightarrow 0$ arise when $s_{11}=0$ and $s_{12}=0$. In particular, such stress values yield respectively $\rho c_s^2(s_{12}=0) = \mu/\rho = \rho c_2^2$ and $\rho c_f^2(s_{11}=0) = (\lambda + 2\mu)/\rho = \rho c_1^2$ according to equations \eqref{eq:eigen_acc_planeStrain1} and \eqref{eq:eigen_acc_planeStrain2}. (supposing $\abs{s_{11}}\leq \sqrt{\frac{\lambda+\mu}{\beta}}$)

\paragraph*{Condition $s_{12}=0$:} 
According to equations \eqref{eq:eigen_acc_DP}, the eigenvalues of the acoustic tensor read:
\begin{align*}
  & \rho c_s^2 = \frac{1}{2}\( \lambda +3\mu -\beta s_{11}^2 - \abs{\lambda + \mu -\beta s_{11}^2 } \) \\
  & \rho c_f^2 = \frac{1}{2}\( \lambda +3\mu -\beta s_{11}^2 + \abs{\lambda + \mu -\beta s_{11}^2 } \)\end{align*}
Then, assuming that $\beta s_{11}^2 < \lambda + \mu$, the expression further reduces to:
\begin{align*}
  & \rho c_s^2 = \mu \\
  & \rho c_f^2 = \lambda +2\mu -\beta s_{11}^2 
\end{align*}
so that the characteristic speed of slow waves reduces to that of elastic shear waves $c_s=c_2=\sqrt{\mu/\rho}$. 
If in contrast $ \lambda + \mu - \beta s_{11}^2$ was to be negative, the characteristic speeds would be: 
% Conversely, assuming that $\beta s_{11}^2 > \lambda + \mu$, one gets from equation \eqref{eq:eigen_acc_DP2}:
\begin{align*}
  & \rho c_s^2 = \lambda +2\mu -\beta s_{11}^2  \\
  & \rho c_f^2 =  \mu 
\end{align*}
Note however that for the characteristic speed of slow waves remains real, $\beta s_{11}^2 > \lambda +2\mu$ is required.
At last, the equality $\beta s_{11}^2 = \lambda + \mu$ leads to $A_{11}^{ep}-A_{22}^{ep}=0$. 
$s_{11} \in ]-\infty,-\sqrt{\frac{\lambda + \mu}{\beta}}[\: \cup\: ]-\sqrt{\frac{\lambda + \mu}{\beta}},\sqrt{\frac{\lambda + \mu}{\beta}}[\: \cup \:]\sqrt{\frac{\lambda + \mu}{\beta}} ,\infty[$
It then appears that $\abs{s_{11}} \neq \sqrt{\frac{\lambda+\mu}{\beta}}$ and $\abs{s_{11}} > \sqrt{\frac{\lambda+2\mu}{\beta}}$ must be satisfied in order to ensure the hyperbolicity of the problem.
%We first look at the shear-free state for which the subtraction of the acoustic tensor diagonal entries reads: $A_{11}^{ep}-A_{22}^{ep}=\lambda + \mu -\beta s_{11}^2$. 
%Hence, the \textbf{undeterminancy} of the functions $\psi^$ arises for $s_{11} = \pm \sqrt{\frac{\lambda+\mu}{\beta}}$ (on the dowstream side), while if $s_{11} \neq \pm \sqrt{\frac{\lambda+\mu}{\beta}}$, $\psi^f_1 \rightarrow \infty$ and $\psi^s_1 \rightarrow 0$.

Recall that $\psi^f_1$ tending to infinity implies that the loading path are horizontal in $(\sigma_{11},\sigma_{12})$ plane and hence, the fast wave has no influence on the shear stress if, and only if, $\sigma_{12}=0$ downstream. Conversely, the stress paths through slow simple waves are vertical. Moreover, with regard the last row of table \ref{tab:simpleWavesEquations}, $\sigma_{22}$ is also unchanged in that case. As a consequence, if the initial state is shear-free the solution no longer contain combined waves, but longitudinal stress and shear stress simple waves.

\paragraph*{Condition $s_{11}=0$ :} The functions $\psi$ cannot be undetermined in the case $s_{11}=0$ since the equation $A_{11}^{ep}-A_{22}^{ep}=\lambda + \mu + \beta s_{12}^2$ does not admit real solutions. Considering the relation \eqref{eq:plane_strain_stress33} between stress components for plane strain, one has:
%However, one can try to give a "physical meaning" to the condition $s_{11}=0$ by considering the relation \eqref{eq:plane_strain_stress33} between stress components for plane strain:
\begin{equation*}
  s_{11}= \frac{2}{3}\sigma_{11}-\frac{1}{3}(\sigma_{22}+\nu(\sigma_{11}+\sigma_{22})-E\eps^p_{33})
\end{equation*}
so that the previous condition is equivalent to:
\begin{equation}
  \label{eq:plane_strain_s11=0}
  \sigma_{11}=\frac{1+\nu}{2-\nu}\sigma_{22}-E\eps^p_{33}
\end{equation}
In such a stress state, the characteristic speeds are of the form:
\begin{align*}
  & \rho c_s^2 = \mu -\beta s_{12}^2 \\
  & \rho c_f^2 = \lambda +2\mu 
\end{align*}
so that the celerity of fast waves identifies to that of elastic pressure wave $c_f=\sqrt{(\lambda + 2\mu)/\rho}=c_1$.

Solution of A11=A22 ????
\subsection{Numerical integration of loading}

\subsubsection*{Plane stress}
\begin{figure}[h!]
  \centering
  \subcaptionbox{Stress path in $(\sigma_{11},\sigma_{12})$ plane}{\begin{tikzpicture}[scale=0.9]
  \begin{axis}[ymajorgrids=true,xmajorgrids=true,ylabel=$\sigma_{12}$,xlabel=$\sigma_{11}$,xmax=2.e8]
    %%
    \addplot[Green,mark=x,only marks,mark repeat=15,very thick] table [x=sigma_11,y=sigma_12] {chapter5/pgfFigures/pgf_thinWalledTubeSlowWave/slowStressPlane_Stress0.pgf};
    \addplot[Green,thick] table [x=sigma_11,y=sigma_12] {chapter5/pgfFigures/pgf_thinWalledTubeSlowWave/TWslowStressPlane_Stress0.pgf};
    %%
    \addplot[Duck,mark=x,only marks,mark repeat=15,very thick] table [x=sigma_11,y=sigma_12] {chapter5/pgfFigures/pgf_thinWalledTubeSlowWave/slowStressPlane_Stress1.pgf};
    \addplot[Duck,thick] table [x=sigma_11,y=sigma_12] {chapter5/pgfFigures/pgf_thinWalledTubeSlowWave/TWslowStressPlane_Stress1.pgf};
    %%
    \addplot[Red,mark=x,only marks,mark repeat=15,very thick] table [x=sigma_11,y=sigma_12] {chapter5/pgfFigures/pgf_thinWalledTubeSlowWave/slowStressPlane_Stress2.pgf};
    \addplot[Red,thick] table [x=sigma_11,y=sigma_12] {chapter5/pgfFigures/pgf_thinWalledTubeSlowWave/TWslowStressPlane_Stress2.pgf};
    %%
    \addplot[Purple,mark=x,only marks,mark repeat=15,very thick] table [x=sigma_11,y=sigma_12] {chapter5/pgfFigures/pgf_thinWalledTubeSlowWave/slowStressPlane_Stress3.pgf};
    \addplot[Purple,thick] table [x=sigma_11,y=sigma_12] {chapter5/pgfFigures/pgf_thinWalledTubeSlowWave/TWslowStressPlane_Stress3.pgf};
    %%
    \addplot[Blue,mark=x,only marks,mark repeat=15,very thick] table [x=sigma_11,y=sigma_12] {chapter5/pgfFigures/pgf_thinWalledTubeSlowWave/slowStressPlane_Stress4.pgf};
    \addplot[Blue,thick] table [x=sigma_11,y=sigma_12] {chapter5/pgfFigures/pgf_thinWalledTubeSlowWave/TWslowStressPlane_Stress4.pgf};
    %%
    \addplot[Orange,mark=x,only marks,mark repeat=15,very thick] table [x=sigma_11,y=sigma_12] {chapter5/pgfFigures/pgf_thinWalledTubeSlowWave/slowStressPlane_Stress5.pgf};
    \addplot[Orange,thick] table [x=sigma_11,y=sigma_12] {chapter5/pgfFigures/pgf_thinWalledTubeSlowWave/TWslowStressPlane_Stress5.pgf};
    %%
    \addplot[Yellow,mark=x,only marks,mark repeat=5,very thick] table [x=sigma_11,y=sigma_12] {chapter5/pgfFigures/pgf_thinWalledTubeSlowWave/slowStressPlane_Stress6.pgf};
    \addplot[Yellow,thick] table [x=sigma_11,y=sigma_12] {chapter5/pgfFigures/pgf_thinWalledTubeSlowWave/TWslowStressPlane_Stress6.pgf};
    %% Yield surface
    \addplot[black,dashed] table  [x=sigma_11,y=sigma_12] {chapter5/pgfFigures/pgf_thinWalledTubeSlowWave/TWslow_yield0.pgf};
  \end{axis}
\end{tikzpicture}

%%% Local Variables:
%%% mode: latex
%%% TeX-master: "../../mainManuscript"
%%% End:} \qquad
  \subcaptionbox{Stress path in deviatoric plane}{\tikzset{cross/.style={cross out, draw=black, minimum size=2*(#1-\pgflinewidth), inner sep=0pt, outer sep=0pt},
%default radius will be 1pt. 
cross/.default={2.5pt}}
\begin{tikzpicture}[scale=0.9]
  \begin{axis}[width=.75\textwidth,view={135}{35.2643},xlabel=$s_1 $,
    ylabel=$s_2 $,zlabel=$s_3$,xmin=-1.e8,xmax=1.e8,ymin=-1.e8,ymax=1.e8,axis equal,axis lines=center,axis on top,xtick=\empty,ytick=\empty,ztick=\empty,
    every axis y label/.style={at={(rel axis cs:0.,.5,-0.65)}, anchor=west},
    every axis x label/.style={at={(rel axis cs:0.5,.,-0.65)}, anchor=east},
    every axis z label/.style={at={(rel axis cs:0.,.0,.18)}, anchor=north}
    ]
    \node[below] at (1.1e8,0.,0.) {$\sigma^y$};
    \node[above] at (-1.1e8,0.,0.) {$-\sigma^y$};
    \draw (1.e8,0.,0.) node[cross,rotate=10] {};
    \draw (-1.e8,0.,0.) node[cross,rotate=10] {};
    \node[white]  at (0,0.,1.42e8) {};
    %%
    \addplot3[Green,dashed,very thick] file {chapter5/pgfFigures/pgf_thinWalledTubeSlowWave/slowDevPlane_Stress0.pgf};
    \addplot3[Green,very thin] file {chapter5/pgfFigures/pgf_thinWalledTubeSlowWave/slowDevPlane_Stress0.pgf};
    %%
    \addplot3[Duck,dashed,very thick] file {chapter5/pgfFigures/pgf_thinWalledTubeSlowWave/slowDevPlane_Stress1.pgf};
    \addplot3[Duck,very thin] file {chapter5/pgfFigures/pgf_thinWalledTubeSlowWave/slowDevPlane_Stress1.pgf};
    %%
    \addplot3[Red,dashed,very thick] file {chapter5/pgfFigures/pgf_thinWalledTubeSlowWave/slowDevPlane_Stress2.pgf};
    \addplot3[Red,very thin] file {chapter5/pgfFigures/pgf_thinWalledTubeSlowWave/slowDevPlane_Stress2.pgf};
    %%
    \addplot3[Purple,dashed,very thick] file {chapter5/pgfFigures/pgf_thinWalledTubeSlowWave/slowDevPlane_Stress3.pgf};
    \addplot3[Purple,very thin] file {chapter5/pgfFigures/pgf_thinWalledTubeSlowWave/slowDevPlane_Stress3.pgf};
    %%
    \addplot3[Blue,dashed,very thick] file {chapter5/pgfFigures/pgf_thinWalledTubeSlowWave/slowDevPlane_Stress4.pgf};
    \addplot3[Blue,very thin] file {chapter5/pgfFigures/pgf_thinWalledTubeSlowWave/slowDevPlane_Stress4.pgf};
    %% 
    \addplot3[Orange,dashed,very thick] file {chapter5/pgfFigures/pgf_thinWalledTubeSlowWave/slowDevPlane_Stress5.pgf};
    \addplot3[Orange,very thin] file {chapter5/pgfFigures/pgf_thinWalledTubeSlowWave/slowDevPlane_Stress5.pgf};
    %% 
    \addplot3[Yellow,dashed,very thick] file {chapter5/pgfFigures/pgf_thinWalledTubeSlowWave/slowDevPlane_Stress6.pgf};
    \addplot3[Yellow,very thin] file {chapter5/pgfFigures/pgf_thinWalledTubeSlowWave/slowDevPlane_Stress6.pgf};
    %% Yield surface
    \addplot3[black,dashed] file {chapter5/pgfFigures/pgf_thinWalledTubeSlowWave/TWCylindreDevPlane.pgf};
  \end{axis}
\end{tikzpicture}

%%% Local Variables:
%%% mode: latex
%%% TeX-master: "../../mainManuscript"
%%% End:}
  \caption{comparison between Clifton and own solution}
\end{figure}

\begin{figure}[h!]
  \centering
  \subcaptionbox{Projections of loading paths in ($\sigma_{11},\sigma_{12}$) and ($\sigma_{22},\sigma_{12}$) planes}{\begin{tikzpicture}[scale=0.9]
\begin{groupplot}[group style={group size=2 by 1,
ylabels at=edge left, yticklabels at=edge left,horizontal sep=3.ex,
xticklabels at=edge bottom,xlabels at=edge bottom},
ymajorgrids=true,xmajorgrids=true,ylabel=$\sigma_{12} \: (Pa)$,
axis on top,scale only axis,width=0.4\linewidth,ymin=0,ymax=63499406.78820015
, every x tick scale label/.style={at={(xticklabel* cs:1.05,0.75cm)},anchor=near yticklabel},colormap name=viridis]
, every x tick scale label/.style={at={(xticklabel* cs:1.05,0.75cm)},anchor=near yticklabel},colormap name=viridis]
\nextgroupplot[xlabel=$\sigma_{11} (Pa)$]
\addplot[arrows along my path,black,thick] table[x=sigma_11,y=sigma_12] {chapter5/pgfFigures/pgf_fastWavesPlaneStress/CPfastStressPlane_frame0_Stress0.pgf};
\addplot[mesh,point meta = \thisrow{p},very thick,no markers] table[x=sigma_11,y=sigma_12] {chapter5/pgfFigures/pgf_fastWavesPlaneStress/CPfastStressPlane_frame0_Stress0.pgf} node[above right,black] {$\textbf{1}$};
\addplot[arrows along my path,black,thick] table[x=sigma_11,y=sigma_12] {chapter5/pgfFigures/pgf_fastWavesPlaneStress/CPfastStressPlane_frame1_Stress0.pgf};
\addplot[mesh,point meta = \thisrow{p},very thick,no markers] table[x=sigma_11,y=sigma_12] {chapter5/pgfFigures/pgf_fastWavesPlaneStress/CPfastStressPlane_frame1_Stress0.pgf} node[above right,black] {$\textbf{2}$};
\addplot[gray,dashed,thin] table[x=sigma_11,y=sigma_12] {chapter5/pgfFigures/pgf_fastWavesPlaneStress/CPfast_yield0_s11s12_Stress0.pgf};

\nextgroupplot[colorbar,colorbar style={title= {$c_f \: (m/s)$},every y tick scale label/.style={at={(2.,-.1125)}} },xlabel=$\sigma_{22}  (Pa)$]
\addplot[arrows along my path,black,thick] table[x=sigma_22,y=sigma_12] {chapter5/pgfFigures/pgf_fastWavesPlaneStress/CPfastStressPlane_frame0_Stress0.pgf};
\addplot[mesh,point meta = \thisrow{p},very thick,no markers] table[x=sigma_22,y=sigma_12] {chapter5/pgfFigures/pgf_fastWavesPlaneStress/CPfastStressPlane_frame0_Stress0.pgf} node[above right,black] {$\textbf{1}$};
\addplot[arrows along my path,black,thick] table[x=sigma_22,y=sigma_12] {chapter5/pgfFigures/pgf_fastWavesPlaneStress/CPfastStressPlane_frame1_Stress0.pgf};
\addplot[mesh,point meta = \thisrow{p},very thick,no markers] table[x=sigma_22,y=sigma_12] {chapter5/pgfFigures/pgf_fastWavesPlaneStress/CPfastStressPlane_frame1_Stress0.pgf} node[above right,black] {$\textbf{2}$};
\end{groupplot}
\end{tikzpicture}
%%% Local Variables:
%%% mode: latex
%%% TeX-master: "../../mainManuscript"
%%% End:
}
  \subcaptionbox{Loading path in deviatoric plane}{\begin{tikzpicture}[scale=0.9]
\begin{axis}[width=.75\textwidth,view={135}{35.2643},xlabel=$s_1 $,ylabel=$s_2 $,zlabel=$s_3$,xmin=-1.e8,xmax=1.e8,ymin=-1.e8,ymax=1.e8,axis equal,axis lines=center,axis on top,ztick=\empty,legend style={at={(0.225,.59)}}]
\addplot3+[Red,very thick,no markers] file {chapter5/pgfFigures/pgf_fastWavesPlaneStress/CPfastDevPlane_frame0_Stress0.pgf};
\addlegendentry{loading path 1}
\addplot3+[Blue,very thick,no markers] file {chapter5/pgfFigures/pgf_fastWavesPlaneStress/CPfastDevPlane_frame1_Stress0.pgf};
\addlegendentry{loading path 2}
\addplot3+[Orange,very thick,no markers] file {chapter5/pgfFigures/pgf_fastWavesPlaneStress/CPfastDevPlane_frame2_Stress0.pgf};
\addlegendentry{loading path 3}
\addplot3+[Purple,very thick,no markers] file {chapter5/pgfFigures/pgf_fastWavesPlaneStress/CPfastDevPlane_frame3_Stress0.pgf};
\addlegendentry{loading path 4}
\addplot3+[gray,dashed,thin,no markers] file {chapter5/pgfFigures/pgf_fastWavesPlaneStress/CPCylindreDevPlane.pgf};
\end{axis}
\end{tikzpicture}
%%% Local Variables:
%%% mode: latex
%%% TeX-master: "../../mainManuscript"
%%% End:
}
  \caption{Loading paths through a fast simple wave with initial condition $\sigma_{22}=0$ for different starting points on the initial yield surface. Stresses in Pa}
  \label{fig:fast_path_plane_strains}
\end{figure}


\begin{figure}[h!]
  \centering
  \subcaptionbox{Slice ($\sigma_{11},\sigma_{12}$) plane}{\input{chapter5/pgfFigures/CPslowWaves1.tex}}
  \subcaptionbox{Deviatoric plane}{\input{chapter5/pgfFigures/CPslowWaves_deviator1.tex}}
  \caption{loading paths through slow simple waves. Stresses in Pa (if required)}
  \label{fig:slow_path_plane_strains}
\end{figure}

\begin{figure}[h!]
  \centering
  \subcaptionbox{Slice ($\sigma_{11},\sigma_{12}$) plane}{\begin{tikzpicture}[scale=0.9]
\begin{groupplot}[group style={group size=2 by 1,
ylabels at=edge left, yticklabels at=edge left,horizontal sep=3.ex,
xticklabels at=edge bottom,xlabels at=edge bottom},
ymajorgrids=true,xmajorgrids=true,ylabel=$\sigma_{12} \: (Pa)$,
axis on top,scale only axis,width=0.45\linewidth,ymin=0,ymax=79249729.4832
, every x tick scale label/.style={at={(xticklabel* cs:1.05,0.75cm)},anchor=near yticklabel}]
\nextgroupplot[xlabel=$\sigma_{11} (Pa)$]
\addplot[mesh,point meta = \thisrow{p},very thick,no markers] table[x=sigma_11,y=sigma_12] {chapter5/pgfFigures/pgf_slowWavesPlaneStress/CPslowStressPlane_frame0_Stress2.pgf} node[above right] {$\textbf{1}$};
\addplot[mesh,point meta = \thisrow{p},very thick,no markers] table[x=sigma_11,y=sigma_12] {chapter5/pgfFigures/pgf_slowWavesPlaneStress/CPslowStressPlane_frame1_Stress2.pgf} node[above right] {$\textbf{2}$};
\addplot[mesh,point meta = \thisrow{p},very thick,no markers] table[x=sigma_11,y=sigma_12] {chapter5/pgfFigures/pgf_slowWavesPlaneStress/CPslowStressPlane_frame2_Stress2.pgf} node[above right] {$\textbf{3}$};
\addplot[mesh,point meta = \thisrow{p},very thick,no markers] table[x=sigma_11,y=sigma_12] {chapter5/pgfFigures/pgf_slowWavesPlaneStress/CPslowStressPlane_frame3_Stress2.pgf} node[above right] {$\textbf{4}$};
\addplot[gray,thin] table[x=sigma_11,y=sigma_12] {chapter5/pgfFigures/pgf_slowWavesPlaneStress/CPslow_yield0_s11s12_Stress2.pgf};

\nextgroupplot[colorbar,colorbar style={title= {$p$},every y tick scale label/.style={at={(2.,-.1125)}} },xlabel=$\sigma_{22}  (Pa)$]
\addplot[mesh,point meta = \thisrow{p},very thick,no markers] table[x=sigma_22,y=sigma_12] {chapter5/pgfFigures/pgf_slowWavesPlaneStress/CPslowStressPlane_frame0_Stress2.pgf} node[above right] {$\textbf{1}$};
\addplot[mesh,point meta = \thisrow{p},very thick,no markers] table[x=sigma_22,y=sigma_12] {chapter5/pgfFigures/pgf_slowWavesPlaneStress/CPslowStressPlane_frame1_Stress2.pgf} node[above right] {$\textbf{2}$};
\addplot[mesh,point meta = \thisrow{p},very thick,no markers] table[x=sigma_22,y=sigma_12] {chapter5/pgfFigures/pgf_slowWavesPlaneStress/CPslowStressPlane_frame2_Stress2.pgf} node[above right] {$\textbf{3}$};
\addplot[mesh,point meta = \thisrow{p},very thick,no markers] table[x=sigma_22,y=sigma_12] {chapter5/pgfFigures/pgf_slowWavesPlaneStress/CPslowStressPlane_frame3_Stress2.pgf} node[above right] {$\textbf{4}$};
\end{groupplot}
\end{tikzpicture}
%%% Local Variables:
%%% mode: latex
%%% TeX-master: "../../mainManuscript"
%%% End:
}
  \subcaptionbox{Deviatoric plane}{\tikzset{cross/.style={cross out, draw=black, minimum size=2*(#1-\pgflinewidth), inner sep=0pt, outer sep=0pt},cross/.default={2.5pt}}
\begin{tikzpicture}[scale=0.9]
  \begin{axis}[width=.75\textwidth,view={135}{35.2643},xlabel=$s_1 $,ylabel=$s_2 $,zlabel=$s_3$,xmin=-1.e8,xmax=1.e8,ymin=-1.e8,ymax=1.e8,axis equal,axis lines=center,axis on top,xtick=\empty,ytick=\empty,ztick=\empty,every axis y label/.style={at={(rel axis cs:0.,.5,-0.65)}, anchor=west}, every axis x label/.style={at={(rel axis cs:0.5,.,-0.65)}, anchor=east}, every axis z label/.style={at={(rel axis cs:0.,.0,.18)}, anchor=north},legend columns= 2, %legend style={at={(.765,0.2)}}
    legend style={at={(1.6,0.6)}}
    ]
\draw (1.e8,0.,0.) node[cross,rotate=10] {};
\draw (-1.e8,0.,0.) node[cross,rotate=10] {};
\node[white]  at (0,0.,1.42e8) {};


\addplot3[Red,arrows along my path,very thick] file {pgfFigures/pgf_HslowWavesPlaneStres/CPslowDevPlane_Stress1.pgf};
\addlegendentry{\footnotesize path 1};
\addplot3[Blue,arrows along my path,very thick] file {pgfFigures/pgf_HslowWavesPlaneStres/CPslowDevPlane_Stress2.pgf};
\addlegendentry{\footnotesize path 2};
\addplot3[Orange,arrows along my path,very thick] file {pgfFigures/pgf_HslowWavesPlaneStres/CPslowDevPlane_Stress3.pgf};
\addlegendentry{\footnotesize path 3};
\addplot3[Purple,arrows along my path,very thick] file {pgfFigures/pgf_HslowWavesPlaneStres/CPslowDevPlane_Stress4.pgf};
\addlegendentry{\footnotesize path 4};
\addplot3[Yellow,arrows along my path,very thick] file {pgfFigures/pgf_HslowWavesPlaneStres/CPslowDevPlane_Stress5.pgf};
\addlegendentry{\footnotesize path 5};
\addplot3[Duck,arrows along my path,very thick] file {pgfFigures/pgf_HslowWavesPlaneStres/CPslowDevPlane_Stress6.pgf};
\addlegendentry{\footnotesize path 6};
\addplot3+[gray,dashed,thin,no markers] file {pgfFigures/pgf_HslowWavesPlaneStres/CPCylindreDevPlane.pgf};
\addlegendentry{initial yield surface}
\node[below] at (1.1e8,0.,0.) {$\sqrt{\frac{2}{3}}\sigma^y$};
\node[above] at (-1.1e8,0.,0.) {$-\sqrt{\frac{2}{3}}\sigma^y$};


\newcommand\radius{1.*0.82e8}
\addplot3[dotted,thick] coordinates {(0.75*\radius,-0.75*\radius,0.) (-0.75*\radius,0.75*\radius,0.)};
\addplot3[dotted,thick] coordinates {(0.,-0.75*\radius,0.75*\radius) (0.,0.75*\radius,-0.75*\radius)};
\addplot3[dotted,thick] coordinates {(-0.75*\radius,0.,0.75*\radius) (0.75*\radius,0.,-0.75*\radius)};
\end{axis}
\end{tikzpicture}
%%% Local Variables:
%%% mode: latex
%%% TeX-master: "../manuscript"
%%% End:
}
  \caption{loading paths through slow simple waves. Stresses in Pa (if required)}
  \label{fig:slow_path_plane_strains}
\end{figure}


\begin{figure}[h!]
  \centering
  \subcaptionbox{Slice ($\sigma_{11},\sigma_{12}$) plane}{\input{chapter5/pgfFigures/CPslowWaves3.tex}}
  \subcaptionbox{Deviatoric plane}{\input{chapter5/pgfFigures/CPslowWaves_deviator3.tex}}
  \caption{loading paths through slow simple waves. Stresses in Pa (if required)}
  \label{fig:slow_path_plane_strains}
\end{figure}



\subsubsection*{Plane strain}
It is assumed that the stress $\sigma_{22}$ in initially zero everywhere in the domain. Several stress paths followed through a fast simple wave and starting from an arbitrary point of the initial yield surface are plotted in figure \ref{fig:fast_path_plane_strains}. Figure \ref{fig:fast_path_plane_strains}\subref{subfig:fastDP_stress} shows the projections in ($\sigma_{11},\sigma_{12}$) and ($\sigma_{22},\sigma_{12}$) planes while figure \ref{fig:fast_path_plane_strains}\subref{subfig:fastDP_dev} shows the stress path in the principal deviatoric stress components space. Note that the projection in that space is orthogonal to the hydrostatic axis $s_1+s_2+s_3=0$ so that the von-Mises yield surface is a circle. Rather, the von-Mises yield surface in that space is a cylindre which axis is used to look at the stress paths in the deviator plane.

Remark, the characteristic speeds are supposed to decrease along the integral curves. It is not the case for all the stress paths depicted in the figures below. In addition, both slow and fast waves lead to loading paths restricted to the yield surface until the direction of pure shear is reached.
\begin{figure}[h!]
  \centering
  \subcaptionbox{Projections of loading paths in ($\sigma_{11},\sigma_{12}$) and ($\sigma_{22},\sigma_{12}$) planes \label{subfig:fastDP_stress}}{\begin{tikzpicture}[scale=0.9]
\begin{groupplot}[group style={group size=2 by 1,
ylabels at=edge left, yticklabels at=edge left,horizontal sep=3.ex,
xticklabels at=edge bottom,xlabels at=edge bottom},
ymajorgrids=true,xmajorgrids=true,ylabel=$\sigma_{12} \: (Pa)$,
axis on top,scale only axis,width=0.45\linewidth,ymin=0,ymax=100000000.0
, every x tick scale label/.style={at={(xticklabel* cs:1.05,0.75cm)},anchor=near yticklabel},colormap={bw}{gray(0cm)=(1); gray(1cm)=(0.05)}]
\nextgroupplot[xlabel=$\sigma_{11} (Pa)$]
\addplot[mesh,point meta = \thisrow{p},very thick,no markers] table[x=sigma_11,y=sigma_12] {chapter5/pgfFigures/pgf_fastWavesPlaneStrain/DPfastStressPlane_frame0_Stress0.pgf} node[above right,black] {$\textbf{1}$};
\addplot[mesh,point meta = \thisrow{p},very thick,no markers] table[x=sigma_11,y=sigma_12] {chapter5/pgfFigures/pgf_fastWavesPlaneStrain/DPfastStressPlane_frame1_Stress0.pgf} node[above right,black] {$\textbf{2}$};
\addplot[mesh,point meta = \thisrow{p},very thick,no markers] table[x=sigma_11,y=sigma_12] {chapter5/pgfFigures/pgf_fastWavesPlaneStrain/DPfastStressPlane_frame2_Stress0.pgf} node[above right,black] {$\textbf{3}$};
\addplot[mesh,point meta = \thisrow{p},very thick,no markers] table[x=sigma_11,y=sigma_12] {chapter5/pgfFigures/pgf_fastWavesPlaneStrain/DPfastStressPlane_frame3_Stress0.pgf} node[above right,black] {$\textbf{4}$};
\addplot[gray,thin] table[x=sigma_11,y=sigma_12] {chapter5/pgfFigures/pgf_fastWavesPlaneStrain/DPfast_yield0_s11s12_Stress0.pgf};

\nextgroupplot[colorbar,colorbar style={title= {$ c_f \: (m/s)$},every y tick scale label/.style={at={(2.,-.1125)}} },xlabel=$\sigma_{22}  (Pa)$]
\addplot[mesh,point meta = \thisrow{p},very thick,no markers] table[x=sigma_22,y=sigma_12] {chapter5/pgfFigures/pgf_fastWavesPlaneStrain/DPfastStressPlane_frame0_Stress0.pgf} node[above right,black] {$\textbf{1}$};
\addplot[mesh,point meta = \thisrow{p},very thick,no markers] table[x=sigma_22,y=sigma_12] {chapter5/pgfFigures/pgf_fastWavesPlaneStrain/DPfastStressPlane_frame1_Stress0.pgf} node[above right,black] {$\textbf{2}$};
\addplot[mesh,point meta = \thisrow{p},very thick,no markers] table[x=sigma_22,y=sigma_12] {chapter5/pgfFigures/pgf_fastWavesPlaneStrain/DPfastStressPlane_frame2_Stress0.pgf} node[above right,black] {$\textbf{3}$};
\addplot[mesh,point meta = \thisrow{p},very thick,no markers] table[x=sigma_22,y=sigma_12] {chapter5/pgfFigures/pgf_fastWavesPlaneStrain/DPfastStressPlane_frame3_Stress0.pgf} node[above right,black] {$\textbf{4}$};
\end{groupplot}
\end{tikzpicture}
%%% Local Variables:
%%% mode: latex
%%% TeX-master: "../../mainManuscript"
%%% End:
}
  \subcaptionbox{Loading path in deviatoric plane \label{subfig:fastDP_dev}}{\tikzset{cross/.style={cross out, draw=black, minimum size=2*(#1-\pgflinewidth), inner sep=0pt, outer sep=0pt},cross/.default={2.5pt}}
\begin{tikzpicture}[spy using outlines={rectangle, magnification=3, size=2.cm, connect spies}]
\begin{axis}[width=.75\textwidth,view={135}{35.2643},xlabel=$s_1 $,ylabel=$s_2 $,zlabel=$s_3$,xmin=-1.e8,xmax=1.e8,ymin=-1.e8,ymax=1.e8,axis equal,axis lines=center,axis on top,xtick=\empty,ytick=\empty,ztick=\empty,every axis y label/.style={at={(rel axis cs:0.,.5,-0.65)}, anchor=west}, every axis x label/.style={at={(rel axis cs:0.5,.,-0.65)}, anchor=east}, every axis z label/.style={at={(rel axis cs:0.,.0,.18)}, anchor=north},legend columns=2,legend style={at={(1.3,0.55)}}]
\node[below] at (1.1e8,0.,0.) {$\sqrt{\frac{2}{3}}\sigma^y$};
\node[above] at (-1.1e8,0.,0.) {$-\sqrt{\frac{2}{3}}\sigma^y$};
\draw (1.e8,0.,0.) node[cross,rotate=10] {};
\draw (-1.e8,0.,0.) node[cross,rotate=10] {};
\node[white]  at (0,0.,1.1e8) {};
\addplot3[Red,thick,arrows along my path] file {pgfFigures/pgf_fastWavesPlaneStrain/DPfastDevPlane_Stress1.pgf};
\addlegendentry{\footnotesize path 1}
\addplot3[Blue,thick,arrows along my path] file {pgfFigures/pgf_fastWavesPlaneStrain/DPfastDevPlane_Stress2.pgf};
\addlegendentry{\footnotesize path 2}
\addplot3[Orange,thick,arrows along my path] file {pgfFigures/pgf_fastWavesPlaneStrain/DPfastDevPlane_Stress3.pgf};
\addlegendentry{\footnotesize path 3}
\addplot3[Purple,thick,arrows along my path] file {pgfFigures/pgf_fastWavesPlaneStrain/DPfastDevPlane_Stress4.pgf};
\addlegendentry{\footnotesize path 4}
\addplot3+[gray,dashed,thin,no markers] file {pgfFigures/pgf_fastWavesPlaneStrain/CylindreDevPlane.pgf};
\addlegendentry{\footnotesize initial yield surface}
\newcommand\radius{1.*0.82e8}
\addplot3[dotted,thick] coordinates {(0.75*\radius,-0.75*\radius,0.) (-0.75*\radius,0.75*\radius,0.01)};
\addplot3[dotted,thick] coordinates {(0.,-0.75*\radius,0.75*\radius) (0.,0.75*\radius,-0.75*\radius)};
\addplot3[dotted,thick] coordinates {(-0.75*\radius,0.,0.75*\radius) (0.75*\radius,0.,-0.75*\radius)};
% \begin{scope}
% \spy[black,size=1.75cm] on (6.7,3.2) in node [fill=none] at (9.5,5.5);
% \end{scope}
\end{axis}
\end{tikzpicture}
%%% Local Variables:
%%% mode: latex
%%% TeX-master: "../manuscript"
%%% End:
}
  \caption{Loading paths through a fast simple wave with initial condition $\sigma_{22}=0$ for different starting points on the initial yield surface.}
  \label{fig:fast_path_plane_strains}
\end{figure}


\begin{figure}[h!]
  \centering
  \subcaptionbox{Slice ($\sigma_{11},\sigma_{12}$) plane}{\input{chapter5/pgfFigures/DPslowWaves1.tex}}
  \subcaptionbox{Deviatoric plane}{\tikzset{cross/.style={cross out, draw=black, minimum size=2*(#1-\pgflinewidth), inner sep=0pt, outer sep=0pt},cross/.default={2.5pt}}
\begin{tikzpicture}[scale=0.9]
\begin{axis}[width=.75\textwidth,view={135}{35.2643},xlabel=$s_1 $,ylabel=$s_2 $,zlabel=$s_3$,xmin=-1.e8,xmax=1.e8,ymin=-1.e8,ymax=1.e8,axis equal,axis lines=center,axis on top,xtick=\empty,ytick=\empty,ztick=\empty,every axis y label/.style={at={(rel axis cs:0.,.5,-0.65)}, anchor=west}, every axis x label/.style={at={(rel axis cs:0.5,.,-0.65)}, anchor=east}, every axis z label/.style={at={(rel axis cs:0.,.0,.18)}, anchor=north},legend style={at={(.2,.68)}}]
\node[below] at (1.1e8,0.,0.) {$\sigma^y$};
\node[above] at (-1.1e8,0.,0.) {$-\sigma^y$};
\draw (1.e8,0.,0.) node[cross,rotate=10] {};
\draw (-1.e8,0.,0.) node[cross,rotate=10] {};
\node[white]  at (0,0.,1.1e8) {};
\addplot3[arrows along my path,Red,very thick] file {chapter5/pgfFigures/pgf_slowWavesPlaneStrain/DPslowDevPlane_frame0_Stress1.pgf};\addlegendentry{loading path 1}
\addplot3[arrows along my path,Blue,very thick] file {chapter5/pgfFigures/pgf_slowWavesPlaneStrain/DPslowDevPlane_frame1_Stress1.pgf};\addlegendentry{loading path 2}
\addplot3[arrows along my path,Orange,very thick] file {chapter5/pgfFigures/pgf_slowWavesPlaneStrain/DPslowDevPlane_frame2_Stress1.pgf};\addlegendentry{loading path 3}
\addplot3[arrows along my path,Purple,very thick] file {chapter5/pgfFigures/pgf_slowWavesPlaneStrain/DPslowDevPlane_frame3_Stress1.pgf};\addlegendentry{loading path 4}
\addplot3[arrows along my path,Green,very thick] file {chapter5/pgfFigures/pgf_slowWavesPlaneStrain/DPslowDevPlane_frame4_Stress1.pgf};\addlegendentry{loading path 5}
\addplot3+[gray,dashed,thin,no markers] file {chapter5/pgfFigures/pgf_slowWavesPlaneStrain/CylindreDevPlane.pgf};\addlegendentry{initial yield surface}
\newcommand\radius{1.*0.82e8}
\addplot3[dotted,thick] coordinates {(0.75*\radius,-0.75*\radius,0.) (-0.75*\radius,0.75*\radius,0.)};
\addplot3[dotted,thick] coordinates {(0.,-0.75*\radius,0.75*\radius) (0.,0.75*\radius,-0.75*\radius)};
\addplot3[dotted,thick] coordinates {(-0.75*\radius,0.,0.75*\radius) (0.75*\radius,0.,-0.75*\radius)};
\end{axis}
\end{tikzpicture}
%%% Local Variables:
%%% mode: latex
%%% TeX-master: "../../mainManuscript"
%%% End:
}
  \caption{loading paths through slow simple waves. Stresses in Pa (if required)}
  \label{fig:slow_path_plane_strains}
\end{figure}

\begin{figure}[h!]
  \centering
  \subcaptionbox{Slice ($\sigma_{11},\sigma_{12}$) plane}{\input{chapter5/pgfFigures/DPslowWaves2.tex}}
  \subcaptionbox{Deviatoric plane}{\input{chapter5/pgfFigures/DPslowWaves_deviator2.tex}}
  \caption{loading paths through slow simple waves. Stresses in Pa (if required)}
  \label{fig:slow_path_plane_strains}
\end{figure}


\begin{figure}[h!]
  \centering
  \subcaptionbox{Slice ($\sigma_{11},\sigma_{12}$) plane}{\input{chapter5/pgfFigures/DPslowWaves3.tex}}
  \subcaptionbox{Deviatoric plane}{\begin{tikzpicture}[scale=0.9]
\begin{axis}[width=.75\textwidth,view={135}{35.2643},xlabel=$s_1 $,ylabel=$s_2 $,zlabel=$s_3$,xmin=-1.e8,xmax=1.e8,ymin=-1.e8,ymax=1.e8,axis equal,axis lines=center,axis on top,ztick=\empty]
\addplot3+[Red,very thick,no markers] file {chapter5/pgfFigures/pgf_slowWavesPlaneStrain/DPslowDevPlane_frame0_Stress3.pgf};
\addplot3+[Blue,very thick,no markers] file {chapter5/pgfFigures/pgf_slowWavesPlaneStrain/DPslowDevPlane_frame1_Stress3.pgf};
\addplot3+[Orange,very thick,no markers] file {chapter5/pgfFigures/pgf_slowWavesPlaneStrain/DPslowDevPlane_frame2_Stress3.pgf};
\addplot3+[Purple,very thick,no markers] file {chapter5/pgfFigures/pgf_slowWavesPlaneStrain/DPslowDevPlane_frame3_Stress3.pgf};
\addplot3+[gray,dashed,thin,no markers] file {chapter5/pgfFigures/pgf_slowWavesPlaneStrain/CylindreDevPlane.pgf};
\end{axis}
\end{tikzpicture}
%%% Local Variables:
%%% mode: latex
%%% TeX-master: "../../mainManuscript"
%%% End:
}
  \caption{loading paths through slow simple waves. Stresses in Pa (if required)}
  \label{fig:slow_path_plane_strains}
\end{figure}





%%%% REMARQUES A LA VOLEE
% It is shown in \cite{Ting73} that the plastic celerities only depends on $\tens{\sigma}/\norm{\tens{\sigma}}$ so that they are constant along in ray of the stress space $(\sigma_{11}, \sigma_{22}, \sigma_{12})$. Thus, look at the loading path along integral curves and see the evolution of celerities.

% For now, it is assumed that the characteristic speeds satisfy: $c_1 \geq c_f \geq c_2 \geq c_s \geq 0$ and that the plastic celerities are monotonically decreasing functions of the stress. The latter assumption is in particular satisfied in the quarter-space $(\sigma_{11}\geq 0, \sigma_{22}\geq 0, \sigma_{12}\geq 0)$ for in that case, every elements of the acoustic tensor $\tens{A}^{ep}$ decrease with increasing stress (pas assez général. Vrai en écrouissage isotrope. Dépend de la normale. Vrai pour un état de contrainte donnée mais dépend du trajet de chargement. Peut-être qu'il faut ommettre ça pour le moment).

%This is in particular true if we restrict our attention to the quarter-space $(\sigma_{11} \geq 0, \sigma_{22} \geq 0 , \sigma_{12}\geq 0)$ in which every components of the tensor  
%Assuming that no shock occurs, the integration of ODEs \eqref{eq:ch5_ODEs} yields simple wave solutions of the problem.
%This assumption seems to be valid with the convex flux function used in equation \eqref{eq:ch5_conservative} that leads to monotonically decreasing wave speeds with respect to the stress tensor. Furthermore, the medium is homogeneous 
%% Ne pas regarder genuinely non-linear car ça n'apporte rien. Ca donne juste une indication sur la variation des vitesses le long des courbes intégrales mais pas en fonction de la contrainte.


%This is in paticular true when looking at the normal vectors $\vect{n} = \vect{e}_1$ and $\vect{n} = \vect{e}_2$ that yield an acoustic tensor $A_{ij}^{ep}=A_{ij}^{elas} - \beta m_{pi}m_{jq}n_p n_q\deta_{pq}$.



%%% Local Variables:
%%% mode: latex
%%% TeX-master: "../mainManuscript"
%%% End:


\section{Towards an elastoplastic approximate Riemann solver}
\label{sec:ep_Riemman_solver}

Parler du solver de Lin et Ballman qui cause problème dans le cas général mais je ne sais plus pourquoi...(je crois que c'est en lien avec des problèmes de Picard et non Riemann)
On propose ici d'approximer toutes les ondes simples par des discontinuités (motivé par les trajets de chargement tracés au-dessus)
\cite{Lin_et_Ballman}
%%% Local Variables:
%%% mode: latex
%%% TeX-master: "../mainManuscript"
%%% End:
