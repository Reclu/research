\chapter{Contribution to the solution elastic-plastic problems in two space dimensions}
%% Faire un historique des formulations faites.
%% Formuler le problème à notre sauce et identifier les cas évoqués en intro en particularisont les modules tangents etc. a ce moment, parler des trajets de chargement et des types d'ondes

\section*{Introduction}
Elastic solver used previously

\cite{Clifton_thesis} development of a method of characteristics in 3 independent variables that is then numerically approximated (Notion of bicharacteristics). Nous on n'en a pas besoin puisqu'on a déja notre schéma numérique. En ravanche, on cherche à résoudre le problème dans une direction donnée pour lequel la méthode des caractéristiques s'applique.

This is for instance the case for the simulation of forming technics that cannot, in general, be modeled in a one-dimensional setting.

Assumptions (isothermal, linear hardenings though extendable, small deformation, cartesian,concave constitutive model,rate independent)

Un intérêt de développer ce qui est fait là est de s'autoriser à limiter les ondes plastique dans des méthodes de haut ordre.
\section{State of the art}
% Researches done on the elastic-plastic bahavior of material at high strain rates for characterization purposes. 
% To this end, uni-axial stress or strain, pure bending or pure torsion problems have been investigated until the late 50's (vérifier ça). 
% Rakmathulin et Cristescu ont ouvert la voie à des problèmes plus complexes impliquant des chargements combinés.

% Elastic solver used previously

% \cite{Clifton_thesis} development of a method of characteristics in 3 independent variables that is then numerically approximated (Notion of bicharacteristics). Nous on n'en a pas besoin puisqu'on a déjà notre schéma numérique. En ravanche, on cherche à résoudre le problème dans une direction donnée pour lequel la méthode des caractéristiques s'applique.

% This is for instance the case for the simulation of forming technics that cannot, in general, be modeled in a one-dimensional setting.

Until the 50s, researches on dynamic problems in plastic solids were focused on uni-axial stress or strain, pure bending or pure torsion loading conditions \cite{Taylor,vonKarman}, and were carried out for materials characterization purposes.
The first references that brought some understanding about the response of linearly hardening solids to combined shear and pressure loads are those of Rakhmatulin \cite{Rakhmatulin} and Cristescu \cite{CRISTESCU19591605}.
These early analytical investigations on plane stress impacts in the plastic regime led to the conclusion that elastic waves, as well as plastic combined-stress simple waves, can propagate in two-dimensional solids. 
While the former were well-known, the latter were shown to fall into the two \textit{fast waves} and \textit{slow waves} families.
The maximal value of fast waves (\textit{resp. slow waves}) is higher than that of pressure (\textit{resp. shear}) plastic discontinuity occuring in one-dimensional problems, for a given compression (\textit{resp. shear}) load amplitude.

Later, Bleich and Nelson \cite{Bleich} considered sumperimposed plane and shear waves in an ideally elastic-plastic materials submitted to step loads.
It has thus been highlighted that different loading cases yield different characteristic structures of the solution of a Picard problem, thus revealing the complexity of plastic flows in more than one dimension.
% Distinguer un peu plus ces deux contributions.
%\thomas{see \cite[p.56 pdf]{Nowacki},\cite{Goel}}. 
The same conclusions have been drawn by Clifton \cite{Clifton} for hardening materials under tension-torsion, who furthermore studied the influence of plastic pre-loading on the solution.
This contribution established the existence of loading paths through the simple waves arising from the characteristic analysis of the hyperbolic system.
Indeed, the combined-stress wave nature lies in ODEs which govern the evolution of stress components within the simple waves.
The integration of these equations of the form $d\sigma_{11}=\psi d\sigma_{12}$ allows the building of curves which connect the applied stress state of the Picard problem $(\sigma^d_{11},\sigma^d_{12})$ to the initial state of the medium.
% Indeed, the study mathematical properties of relations between stress components of the form $d\sigma_{11}=\psi d\sigma_{12}$, satisifed inside fast and slow simple waves, allows to connect the applied stress state of the Picard problem $(\sigma^d_{11},\sigma^d_{12})$ to the initial state of the medium.
It has been for instance shown that if a solid is acted upon by a traction force such that $\sigma^d_{11}=0$ and $\sigma^d_{12}$ lies outside the elastic convex, only an elastic shear discontinuity, followed by a slow simple wave, propagates.
Conversely, other loading conditions may lead to the combination of elastic pressure discontinuity and a fast wave, possibly followed by a slow wave.
Another notable conclusion is that the combined loading paths followed inside simple waves may lead to plastic unloading, while only elastic unloading occurs in the one-dimensional theory.
%In addition, it is possible to meet unloading plastic simple waves with contrast to the one-dimensional theory in which the unloading waves propagate at elastic speeds (c'est pas vraiment ça attention).
%Such loading paths are supplemented by ODEs satisfied by the velocity components so that a closed form of the solution of the problem can be derived.

Experimental data collected on a thin-walled tube submitted to a dynamic tensile load \cite{Clifton_exp,Clifton_exp2} confirmed the existence of two distinct families of  simple waves, both involving combined stress paths.
Those works nevertheless exhibited some discrepancies with the theory which have been attributed to the assumption made on the von-Mises yield surface.
As a matter of fact, a constant strain region lying between the fast and slow waves that is predicted by the theory \cite{Clifton} could not be seen in experimental results.
However, by following the endochronic theory of plasticity \cite{Valanis} which does not require the introduction of a yield surface, Wu and Lin \cite{Wu_experimental} obtained numerical results that better fitted the experimental data provided by Lipkin and Clifton \cite{Clifton_exp2}.
The good agreement showed between numerical and experimental results \cite{Wu_experimental} thus confirmed the theory.

Ting and Nan \cite{Ting68} then generalized the work of Bleich and Nelson to hardening materials and Ting \cite{Ting69} widened this of Clifton to more complex loadings, that is a superimposition of one plane wave and two shear waves states.
Once again, the mathematical study of the ODE system governing the stresses evolution inside fast and slow simple waves led to the construction of loading paths in the stress space that depend on the external loads. A review of governing equations for all the cases depending on one space dimension considered above can be found in \cite{Nowacki}.

The information on characteristic structures thus provided has then be used by Lin and Ballman \cite{Lin_et_Ballman} for the development of an iterative Riemann solver.
This procedure is based on successive guesses on the stress state lying in the stationary region so that the loading paths preticted by the theory of Clifton \cite{Clifton} can integrated numerically until convergence.
The implementation of this solver within a second-order Godunov scheme provided results that were in good agreement the exact solutions.
Nevertheless, the theoretical investigations mentioned above restrict the development of such numerical tools to problems that depend on one space dimension.
%%
Clifton tackled the solution of plane strain problems in elastic-plastic solids by looking for bi-characteristics \cite{Clifton_thesis} in order to build finite difference schemes that account for plastic waves.
The point of view adopted here is that one can benefit from the simplifications introduced by the writing of Riemann problems in an arbitray direction of space.
Indeed, the method of characteristics rather than the more complex method of bi-characteristics can be employed with the quasilinear forms presented in chapter \ref{chap:chap2}.




%%% Local Variables:
%%% mode: latex
%%% TeX-master: "../mainManuscript"
%%% End:


\subsection{Characteristic analysis}
We assume the infinitesimal strain tensor can be additively decomposed into an elastic part and a plastic part according to:
\begin{equation}
  \label{eq:ch5_partition}
  \tens{\eps}=\tens{\eps}^e+\tens{\eps}^p
\end{equation}
The elastic part of the infinitesimal strain tensor is:
\begin{equation}
  \label{eq:ch5_elastic_inverse}
  \tens{\eps}^e = \frac{1+\nu}{E} \tens{\sigma} - \frac{\nu}{E} \tr \tens{\sigma} \tens{I}
\end{equation}

\subsubsection*{The general case}
The governing equations of dynamics in elastic-plastic solids are written in the following quasi-linear form:
\begin{equation}
  \Qcb_t + \Absf^i \drond{\Qcb}{x_i} = \Scb \qquad \text{with: }\Absf^i = -\matrice{\tens{0}^2 & \frac{1}{\rho}\tens{I}\otimes\vect{e}_i\\ \Cbb^{ep}\cdot \vect{e}_i & \tens{0}^4}  \label{eq:ch5_quasilinear}
\end{equation}
where $\Qcb=\matrice{\vect{v}\\ \tens{\sigma}}$ and $\Cbb^{ep}=\ddroit{\tens{\sigma}}{\tens{\eps}}=\Cbb - \beta\tens{m}\otimes\tens{m}$ is the tangent modulus with $\tens{m}=\frac{\tens{s}-\tens{Y}}{\norm{\tens{s}-\tens{Y}}}$ is the flow direction and $\beta=\frac{6\mu^2}{3\mu +(C+R')}$ depends on the shear modulus $\mu$ and kinematic or isotropic hardening modulus $C$ and $R'$ (see section \ref{sec:constitutive-equations}). In particular in the arbitrary direction $\vect{n}$:
\begin{equation}
  \Qcb_t + \Jbsf \drond{\Qcb}{x_n} = \Scb  \label{eq:ch5_quasilinear_normal}
\end{equation}
where $x_n=\vect{x}\cdot\vect{n}$ and the Jacobian matrix $\Jbsf=\Absf^in_i$ arises. The left characteristic fields $\{c_K;\Lcb^K\}$ satisfy the following equation:
\begin{equation}
  \label{eq:ch5_eigen_system}
  \vect{\Lc}^K \(\Jbsf - c_K \Ibsf\) = \vect{0}
\end{equation}
As seen in section \ref{sec:characteristic_analysis}, $6$ couples of characteristic speeds $c_K$ and left eigenvectors $\Lcb^K= \[ \vect{v}^K \: , \: \tens{S}^K \]$ are determined based on those of the acoustic tensor $\tens{A}=\vect{n}\cdot\Cbb^{ep}\cdot \vect{n}$, that are $\{\omega^p;\vect{l}^p\}$ for $p=1,2,3$:
\begin{equation}
  \label{eq:ch5_left_eigenfields}
  \left\lbrace \pm \sqrt{\frac{\omega_p}{\rho_0}} ; \quad \[\: \pm \rho_0\sqrt{\frac{\omega_p}{\rho_0}} \vect{l}^p , -\vect{l}^p\otimes \vect{N} \:\]  \right\rbrace ,\quad p=1,2,3
\end{equation}
In addition, three independent left eigenvectors associated to the zero eigenvalue of system \eqref{eq:ch5_quasilinear_normal}, which is of multiplicity $3$, are found by solving:
\begin{equation}
  \label{eq:ch5_null_eigen}
  \tens{\sigma}^K:\(\Cbb^{ep}\cdot  \vect{n}\) =\vect{0},\quad K=1,2,3
\end{equation}

\subsubsection*{Problems in two space dimensions}
We now focus on the solid domain bounded by $x_1 \times x_2 \times x_3 \in [0,\infty[ \times ]-infty,infty[ \times [-e,e]$ in a Cartesian coordinates system, where $e$ is an arbitrary length.
The solid is subject on the plane $x_1=0$ to a traction force $\vect{T}$ restricted to the $(\vect{e}_1,\vect{e}_2)$ plane, that is $T_3=0$. It is moreover assumed that all quantities except the velocity component $v_3$ depend solely on $x_1$ and $x_2$.

First, the solid is under plane strain, that is $\tens{\eps}\cdot\vect{e}_3=\vect{0}$, if the velocity $v_3$ vanishes on both ends $x_3=\pm h$. Thus, combination of equations \eqref{eq:ch5_partition} and \eqref{eq:ch5_elastic_inverse}, along with the kinematic condition $\eps_{33}=0$, allows to write a relation between $\sigma_{33}$ and other stress components:
\begin{equation}
  \label{eq:plane_strain_stress33}
  \sigma_{33}=\nu\(\sigma_{11}+\sigma_{22}\) - E\eps^p_{33}
\end{equation}

Conversely, if the planes $x_3=\pm e$ are traction free, a plane stress state reading $\tens{\sigma}\cdot\vect{e}_3=\vect{0}$ holds in the solid. For both plane strain and plane stress problems, the stress component $\sigma_{33}$ can then be removed from system \eqref{eq:ch5_quasilinear_normal}, leading to the unknown vector $\Qcb=[v_1,v_2,\sigma_{11},\sigma_{22},\sigma_{12}]$. The problem can then be solved in a two-dimensional setting, for which the acoustic tensor admits two distinct real eigenvalues:
\begin{subequations}
  \begin{alignat}{1}
    \label{eq:ch5_eigenAcc1}
    &\omega_1 = \frac{1}{2}\(A_{11}+A_{22} - \sqrt{(A_{11}-A_{22})^2+4A_{12}}\) \\
    \label{eq:ch5_eigenAcc2}
    &\omega_2 = \frac{1}{2}\(A_{11}+A_{22} + \sqrt{(A_{11}-A_{22})^2+4A_{12}}\) 
  \end{alignat}
\end{subequations}
with associated left eigenvector:
\begin{equation}
  \label{eq:ch5_eigenvectAcc}
  \vect{l}^p=\matrice{-A_{12} \\ A_{11}-\omega_p} = \matrice{ A_{22}-\omega_p \\ -A_{12}}
\end{equation}
From equation \eqref{eq:ch5_left_eigenfields}, one gets that the problem involves two families of waves travelling at speeds $c_1 = \pm \sqrt{\omega_1/\rho}$ and $c_2 = \pm \sqrt{\omega_2/\rho}$. For elastic evolutions, th

that are referred to as \textit{slow} and \textit{fast} waves, travelling respectively at speeds $c_s = \pm \sqrt{\omega_1/\rho}$ and $c_f = \pm \sqrt{\omega_2/\rho}$.
Considering equation \eqref{eq:ch5_left_eigenfields}, the problems involve 









% Orthogonalité des loading paths \cite{Clifton,Ting68}

%%% Local Variables:
%%% mode: latex
%%% TeX-master: "../mainManuscript"
%%% End:



\section{Towards an elastoplastic approximate Riemann solver}
Parler du solver de Lin et Ballman qui cause problème dans le cas général mais je ne sais plus pourquoi...(je crois que c'est en lien avec des problèmes de Picard et non Riemann)
On propose ici d'approximer toutes les ondes simples par des discontinuités (motivé par les trajets de chargement tracés au-dessus)
\cite{Lin_et_Ballman}
%%% Local Variables:
%%% mode: latex
%%% TeX-master: "../mainManuscript"
%%% End:
