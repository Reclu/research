%% Faire un historique des formulations faites.
%% Formuler le problème à notre sauce et identifier les cas évoqués en intro en particularisant les modules tangents etc. a ce moment, parler des trajets de chargement et des types d'ondes

\section*{Introduction}
It has been shown throughout this manuscript that hyperbolic problems in solid mechanics are solved in a different manner depending on the numerical explicit method employed. 
In particular, irreversible deformations which are usually numerically computed based on well-known constitutive integrators, may greatly differ from one scheme to another even for one-dimensional problems.
However, the accurate assessment of residual stresses and strains are of major importance for many industrial applications such as, among others, high-speed metal forming, crash-proof design or the study of the impact of earthquakes on structures.
The simulations performed in chapter \ref{chap:chap4} emphasized the improvements enabled by the knowledge of the characteristic structure of the solutions of conservation laws systems, especially for elastoplastic solids.
Nevertheless, the introduction of the exact solution by means of approximate Riemann solvers is so far only possible for problems in one space dimension in elastic-plastic solids.

The purpose of this chapter is to identify typical behaviors of the solutions of two-dimensional elastoplasticity problems under small strains.
%The purpose of this chapter is to provide solutions and clues for future works for general two-dimensional elastoplasticity problems under small strains. 
The knowledge one can get about these solutions allows a better understanding of the physical phenomena occuring and hence, the ability to accurately deal with them numerically.
% More information on the structure of solutions to these problems allow a better understanding of physical phenomena occuring in media on the one hand, and the ability of accurately deal with them numerically on the other hand.

The chapter is organized as follows.
A brief historical review of the solution of dynamic problems in two-dimensional elastic-plastic solids is made in section \ref{sec:review}.
%A brief historical review of the solution of plastic waves in two-dimension space is made in section \ref{sec:review}.
Then, the equations of plasticity are recalled in section \ref{sec:charac_plast} so that the characteristic analysis, followed by the application of the method of characteristics, can be carried out.
In section \ref{sec:stress_paths}, attention is paid to the evolution of stress components inside the simple waves that might propagate by means of a mathematical study of the ODEs satisfied within the simple waves.
Since the developments rapidly become cumbersome, the analysis is supplemented with numerical results in section \ref{sec:stress_paths_num}.
%Attention is next paid in section \ref{sec:stress_paths} to the evolution of stress components through simple waves possibly arising in the solution. 
At last, some identified trends are discussed at the end of the chapter in order to use them for the building of a dedicated Riemann solver. 

\section{Historical review}
\label{sec:review}
Until the 50s, researches on dynamic problems in elastic-plastic solids were focused on uni-axial stress or strain, pure bending or pure torsion loading conditions \cite{Taylor,vonKarman}, and were carried out for materials characterization purposes.
The first references that brought some understanding about the response of linearly hardening solids to combined shear and pressure loads are those of Rakhmatulin \cite{Rakhmatulin} and Cristescu \cite{CRISTESCU19591605}.
These early analytical investigations on plane stress impacts in the plastic regime led to the conclusion that elastic waves, as well as plastic combined-stress simple waves, can propagate in two-dimensional solids. 
While the former were well-known, the latter were shown to fall into the two \textit{fast waves} and \textit{slow waves} families.
% It has been moreover shown that the fast waves propagate faster than the plastic pressure discontinuity in uni-axial problems would, for a given compression load amplitude.
% Likewise, slow waves propa
% The maximal value of fast waves (\textit{resp. slow waves}) is higher than that of pressure (\textit{resp. shear}) plastic discontinuity occuring in one-dimensional problems, for a given compression (\textit{resp. shear}) load amplitude.

Later, Bleich and Nelson \cite{Bleich} considered superimposed plane and shear waves in an ideally elastic-plastic materials submitted to step loads.
It has then been highlighted that different loading cases yield different characteristic structures of the solution of a Picard problem, thus revealing the complexity of plastic flows in more than one dimension.
% Distinguer un peu plus ces deux contributions.
%\thomas{see \cite[p.56 pdf]{Nowacki},\cite{Goel}}. 
The same conclusions have been drawn by Clifton \cite{Clifton} for hardening materials under tension-torsion, who furthermore studied the influence of plastic pre-loading on the solution.
This contribution established the existence of loading paths through the simple waves arising from the characteristic analysis of the hyperbolic system.
Indeed, the combined-stress wave nature lies in ODEs which govern the evolution of stress components within the simple waves.
The integration of these equations of the form $d\sigma_{11}=\psi d\sigma_{12}$ allows the building of curves that connect the applied stress state of the Picard problem $(\sigma^d_{11},\sigma^d_{12})$ to the initial state of the medium.
% Indeed, the study mathematical properties of relations between stress components of the form $d\sigma_{11}=\psi d\sigma_{12}$, satisifed inside fast and slow simple waves, allows to connect the applied stress state of the Picard problem $(\sigma^d_{11},\sigma^d_{12})$ to the initial state of the medium.
It has been for instance shown that if a solid is acted upon by a traction force such that $\sigma^d_{11}=0$ and $\sigma^d_{12}$ lies outside the elastic convex, only an elastic shear discontinuity, followed by a slow simple wave, propagates.
Conversely, other loading conditions may lead to the combination of an elastic pressure discontinuity and a fast wave, possibly followed by a slow wave.
Another notable conclusion is that the combined loading paths followed inside simple waves may lead to plastic unloading, whereas only elastic unloading occurs in the one-dimensional theory.
%In addition, it is possible to meet unloading plastic simple waves with contrast to the one-dimensional theory in which the unloading waves propagate at elastic speeds (c'est pas vraiment ça attention).
%Such loading paths are supplemented by ODEs satisfied by the velocity components so that a closed form of the solution of the problem can be derived.

Experimental data collected on a thin-walled tube submitted to a dynamic tensile load \cite{Clifton_exp,Clifton_exp2} confirmed the existence of two distinct families of  simple waves, both involving combined stress paths.
Those works nevertheless exhibited some discrepancies with the theory which have been attributed to the assumption made on the von-Mises yield surface.
As a matter of fact, a constant strain region lying between the fast and slow waves that is predicted by the theory \cite{Clifton} could not be seen in experimental results.
However, by following the endochronic theory of plasticity \cite{Valanis} which does not require the introduction of a yield surface, Wu and Lin \cite{Wu_experimental} obtained numerical results that better fitted the experimental data provided by Lipkin and Clifton \cite{Clifton_exp2}.
The good agreement showed between numerical and experimental results \cite{Wu_experimental} thus confirmed the theory.

Ting and Nan \cite{Ting68} then generalized the work of Bleich and Nelson to hardening materials and Ting \cite{Ting69} widened this of Clifton to more complex loadings, that is a superimposition of one plane wave and two shear waves states.
Once again, the mathematical study of the ODE system governing the stresses evolution inside fast and slow simple waves led to the construction of loading paths in the stress space that depend on the external loads. A review of governing equations for all the cases depending on one space dimension considered above can be found in \cite{Nowacki}.

The information on characteristic structures thus provided has then be used by Lin and Ballman \cite{Lin_et_Ballman} for the development of an iterative Riemann solver.
This procedure is based on successive guesses on the stress state lying in the stationary region so that the loading paths predicted by the theory of Clifton \cite{Clifton} can be integrated numerically until convergence.
The implementation of this solver within a second-order Godunov scheme provided results that were in good agreement with the exact solutions.
Nevertheless, the theoretical investigations mentioned above restrict the development of such numerical tools to problems that depend on one space dimension.
%%
Clifton tackled the solution of plane strain problems in elastic-plastic solids by looking for bi-characteristics \cite{Clifton_thesis} in order to build finite difference schemes that account for plastic waves.
The point of view adopted here is that one can benefit from the simplifications introduced by the writing of Riemann problems in an arbitrary direction of space.
Indeed, the method of characteristics rather than the more complex method of bi-characteristics can be employed with the quasi-linear forms presented in chapter \ref{chap:chap2}.

$\newline$
On the other hand, the existence of plastic shocks in solids under the plane wave assumptions has been investigated by several authors \cite{Mandel1,Germain_shock,Mandel2,Claude}.
In those references high pressure impacts were considered so that the problem may be approximated as falling into the hydrodynamics theory.
Hence, the non-monotonic state law describing the evolution of the hydrostatic pressure is responsible for the creation of shock waves due to colliding characteristics.
%%%%%
This work gave rise to discussions on the internal structure of the shock and more precisely to the nature of the load paths followed in the infinitely thin layer in the vicinity of the shock.
%% Mandel suppose un trajet radial dans le choc et donc s'affranchit de l'analyse de la structure interne.
%% Germain utilise le travail plastic comme variable d'écrouissage. + regarde la structure interne du choc pour compléter les courbes d'hugoniot ?
%% Désaccord sur la variable d'écrouissage
%% Et puis Claude

%% Trajets radiaux + hypothèses sur l'écrouissage

% Structure interne + trajet radial de Mandel



%%% Local Variables:
%%% mode: latex
%%% TeX-master: "../mainManuscript"
%%% End:


\section{Elastic-plastic wave structure in two space dimensions}
\label{sec:charac_plast}
%% Comment the possibility of widening the approach to kinematic hardening ?
\subsection{Governing equations}
We are concerned with linear isotropic hardening materials which elastic domain is given by the von-Mises yield surface, under isothermal deformations in the linearized geometrical framework.
The balance equation of linear momentum with neglected body forces, and the geometrical balance equations are:  
\begin{equation}
  \label{eq:ch5_balance_equations}
  \left\lbrace \begin{aligned}
    %\label{eq:ch5_linear_momentum}
    & \rho \dot{\vect{v}} - \nablav \cdot \tens{\sigma} = \vect{0} \\
    %\label{eq:ch5_linear_momentum}
    &  \dot{\tens{\eps}} - \nablav \cdot \(\frac{\vect{v}\otimes \tens{I} + \tens{I} \otimes \vect{v}}{2}\) = \vect{0} 
  \end{aligned} \right.
\end{equation}
In addition, the elastic-plastic constitutive equations derived from thermodynamics in section \ref{sec:constitutive-equations} are recalled here:
\begin{subequations}
  \label{eq:ch5_plasticity_equations}
  \begin{empheq}[left=\empheqlbrace]{align}
    \label{eq:ch5_von-Mises_yield}
    & f\(\tens{\sigma},\Acb \)= \sqrt{\frac{3}{2}}\norm{\tens{s}} - \(R(p)+\sigma^y\) \equiv 0,\quad \text{with }\tens{s}=\tens{\sigma}-\frac{1}{3}\tr \tens{\sigma} \tens{I} \\
    % \label{eq:ch5_kin_hard}
    % & \tens{Y}=\frac{2}{3}C\tens{\eps}^p \\
    \label{eq:ch5_iso_hard}
    & R(p)=C \: p \\
    \label{eq:elastoplastic_tangent}
    & \tens{\dot{\sigma}}=\(\Cbb^{elast} - \beta\:\tens{s}\otimes\tens{s} \):\tens{\dot{\eps}} = \Cbb^{ep}:\tens{\dot{\eps}} \\
    \label{eq:ch5_plastic_flow}
    & \beta = \frac{6\mu^2}{3\mu +C}\times\frac{1}{\tens{s}:\tens{s}}\\
    \label{eq:ch5_EP_acoustic}
    & A_{ij}^{ep}=  A_{ij}^{elast} -  \beta (n_k s_{ki})(s_{jl}n_l)
  \end{empheq}
\end{subequations}
In the expression of von-Mises yield function \eqref{eq:ch5_von-Mises_yield}, the (positive) linear isotropic hardening law \eqref{eq:ch5_iso_hard} is considered.
Moreover, the elastoplastic acoustic tensor \eqref{eq:ch5_EP_acoustic} is decomposed as an elastic part $A_{ij}^{elast}$ and a plastic part depending on the direction of the plastic flow through the coefficient $\beta$ \eqref{eq:ch5_plastic_flow}.
%At last, the (isotropic)elasticity tensor $\Cbb$ in equation \eqref{eq:elastoplastic_tangent} can be inverted to yield the following elastic law:
By inverting the (isotropic)elasticity tensor $\Cbb$ involved in equation \eqref{eq:elastoplastic_tangent}, the following elastic law is written in the isotropic case:
\begin{equation}
  \label{eq:ch5_elastic_inverse}
  \tens{\eps}^e = \frac{1+\nu}{E} \tens{\sigma} - \frac{\nu}{E} \tr \tens{\sigma} \tens{I}
\end{equation}
with Young's modulus $E$ and Poisson's ration $\nu$.

The quasi-linear form of the sets of equations \eqref{eq:ch5_balance_equations} and \eqref{eq:ch5_plasticity_equations} in a Cartesian coordinates system and an arbitrary direction $\vect{n}$ is:
\begin{equation}
  \label{eq:ch5_quasilinear_normal}
  \Qcb_t + \Jbsf \drond{\Qcb}{x_n} = \vect{0} 
\end{equation}
where $x_n=\vect{x}\cdot\vect{n}$, $\Qcb=\matrice{\vect{v}\\ \tens{\sigma}}$, and $\Jbsf$ is the Jacobian matrix.
It has been shown in section \ref{sec:characteristic_analysis} that the $3$ eigenvalues $\omega^p$ and eigenvectors $\vect{l}^p$ of the acoustic tensor lead to $6$ left characteristic fields of the Jacobian matrix $\{c_K;\Lcb^K\}$ according to:
\begin{equation}
  \label{eq:ch5_left_eigenfields}
  \left\lbrace \pm \sqrt{\frac{\omega_p}{\rho}} ; \quad \[\: \pm \rho\sqrt{\frac{\omega_p}{\rho}} \vect{l}^p , -\vect{l}^p\otimes \vect{n} \:\]  \right\rbrace ,\quad p=1,2,3
\end{equation}
In addition, three independent left eigenvectors associated to the zero eigenvalue of system \eqref{eq:ch5_quasilinear_normal}, which is of multiplicity $3$, are found by solving:
\begin{equation}
  \label{eq:ch5_null_eigen}
  \tens{\sigma}^K:\(\Cbb^{ep}\cdot  \vect{n}\) =\vect{0},\quad K=1,2,3
\end{equation}

The present formulation differs from those of \textsc{Bleich} \cite{Bleich}, \textsc{Clifton} \cite{Clifton}, and hence these of \textsc{Ting} and \textsc{Nan} \cite{Ting68} and \textsc{Ting} \cite{Ting69}, in that equation  \eqref{eq:ch5_quasilinear_normal} is based on the elastoplastic stiffnesses rather than softenesses.
As a consequence, it will be seen in what follows that the equations can be easily specialized to plane strain and plane stress cases.
%In what follows, the above equations are specified to plane strain and plane stress cases.
\subsection{Problems in two space dimensions}
We now focus on the solid domain $x_1 \times x_2 \times x_3 \in [0,\infty[ \times [-h,h] \times [-e,e]$ in a Cartesian coordinates system, where $e$ and $h$ are arbitrary lengths.
%The solid is subject in the plane $x_1=0$ to a traction force $\vect{T}^1$ restricted to the $(\vect{e}_1,\vect{e}_2)$ plane, that is $T_3=0$.
It is assumed that all quantities depend solely on $x_1$ and $x_2$ except the velocity component $v_3$ that may depend on $x_3$.
In particular, it is the case for $e \ll h$.

The solid is under plane strain conditions, that is $\tens{\eps}\cdot\vect{e}_3=\vect{0}$, if the velocity $\vect{v}$ does not depend on $x_3$ and if $v_3$ vanishes.
Thus, combining the additive partition of the infinitesimal strain tensor: $\tens{\eps}=\tens{\eps}^e+\tens{\eps}^p$, with the elastic law \eqref{eq:ch5_elastic_inverse} and the kinematic condition $\eps_{33}=0$, one gets:
\begin{equation}
  \label{eq:plane_strain_stress33}
  \sigma_{33}=\nu\(\sigma_{11}+\sigma_{22}\) - E\eps^p_{33}
\end{equation}
Hence, the quasi-linear form \eqref{eq:ch5_quasilinear_normal} reduces for plane strain problems to a system of dimension $5$ with unknowns $v_1,v_2, \sigma_{11},\sigma_{12}$, and $\sigma_{22}$.


Alternatively, a plane stress state ($\tens{\sigma}\cdot\vect{e}_3=\vect{0}$) is assumed if the planes $x_3=\pm h$ are traction free and $e\ll h$.
As a result, the stress component $\sigma_{33}$ can be removed from the system \eqref{eq:ch5_quasilinear_normal}.
Nevertheless, the tangent modulus must account for the vanishing out-of-plane stress component.
Specialization of equation \eqref{eq:elastoplastic_tangent} to $\sigma_{33}$ yields:
\begin{equation*}
  \dot{\sigma}_{33}=C^{ep}_{33ij} \dot{\eps}_{ij} =0
\end{equation*}
from which one writes:
\begin{equation*}
  C^{ep}_{3333} \dot{\eps}_{33} = - C^{ep}_{33ij}\dot{\eps}_{ij} \quad i,j=\{1,2\}
\end{equation*}
Hence, the constitutive equations are rewritten by means of a two-dimensional tangent modulus $\widetilde{\Cbb}^{ep}$:
\begin{equation}
  \label{eq:CP_constitutive}
  \dot{\sigma}_{ij}=C^{ep}_{ijkl} \dot{\eps}_{kl} - \frac{C^{ep}_{ij33}C^{ep}_{33kl}}{C^{ep}_{3333}}\dot{\eps}_{kl}= \widetilde{C}^{ep}_{ijkl} \dot{\eps}_{kl}\qquad i,j,k,l=\{1,2\} 
\end{equation}
The characteristic structure of the problem is then given by this of the associated acoustic tensor $\tens{\widetilde{A}}^{ep}=\vect{n}\cdot\widetilde{\Cbb}^{ep}\cdot \vect{n}$.

The removal of $\sigma_{33}$ from system \eqref{eq:ch5_quasilinear_normal} for both plane strains and plane stresses allows to solve the problem in a two-dimensional setting.
Then, generically denoting the acoustic tensor by $\tens{A}$, the characteristic structures are given by the eigenvalues:
\begin{subequations}
  \begin{alignat}{1}
    \label{eq:ch5_eigenAcc1}
    &\omega_1 = \frac{1}{2}\(A_{11}+A_{22} + \sqrt{(A_{11}-A_{22})^2+{4A_{12}}^2}\) \\
    \label{eq:ch5_eigenAcc2}
    &\omega_2 = \frac{1}{2}\(A_{11}+A_{22} - \sqrt{(A_{11}-A_{22})^2+{4A_{12}}^2}\)     
  \end{alignat}
\end{subequations}
and the associated eigenvectors:
\begin{equation}
  \label{eq:ch5_eigenvectAcc}
   \vect{l}^1=[ A_{22}-  \omega_1 \:,\: -A_{12}] \qquad ;\qquad  \vect{l}^2=[ -A_{12} \:,\:A_{11}- \omega_2 ]
\end{equation}
From equation \eqref{eq:ch5_left_eigenfields}, we see that two families of waves with celerities $c_f=\pm \sqrt{\omega_1/\rho}$ and $c_s = \pm \sqrt{\omega_2/\rho}$ may travel in the domain.
Those waves are respectively referred to as fast and slow waves.
Note that subtracting equations \eqref{eq:ch5_eigenAcc1} and \eqref{eq:ch5_eigenAcc2} leads to:
\begin{equation}
  \label{eq:diff_celerities}
  \rho c_f^2 - \rho c_s^2 = \sqrt{(A_{11}-A_{22})^2+{4A_{12}}^2} \geq 0
\end{equation}
Hence, the characteristic speed associated to fast waves is always greater than of equal to that of slow waves.

The four left eigenfields of the Jacobian matrix thus read:
\begin{subequations}
  \begin{alignat}{1}
    \label{eq:ch5_Jac_eigenfield_fast}
    &\left\lbrace \pm c_f ; \quad \Lcb^{c_f^\pm}=\[\: \pm \rho c_f \vect{l}^1 , -\vect{l}^1\otimes \vect{n} \:\]  \right\rbrace \\
  \label{eq:ch5_Jac_eigenfield_slow}
    &\left\lbrace \pm c_s ; \quad \Lcb^{c_s^\pm}=\[\: \pm \rho c_s \vect{l}^2 , -\vect{l}^2\otimes \vect{n} \:\]  \right\rbrace
  \end{alignat}
\end{subequations}
where $\Lcb^{c_f^+}$ and $\Lcb^{c_f^-}$ are associated to the right-going and left-going fast waves respectively.
The same goes for $\Lcb^{c_s^+}$ and $\Lcb^{c_s^-}$.
Furthermore, one stationary wave associated to the zero eigenvalue of the Jacobian matrix, and which left eigenvector satisfies equation \eqref{eq:ch5_null_eigen}, has to be added:
\begin{equation}
  \label{eq:ch5_null_left_eigen}
  {\Lcb^0}^T = \matrice{v_1^0 \\[5.pt] v_2^0 \\[5.pt] \sigma_{11}^0 \\[5.pt] \sigma^0_{22} \\[5.pt] \sigma^0_{12} }= \matrice{0 \\[5.pt] 0 \\[5.pt] \(C_{121i}C_{222j}-C_{221i}C_{122j}\)n_in_j \\[5.pt] \(C_{111i}C_{122j}-C_{112i}C_{121j}\)n_in_j \\[5.pt] \(C_{112i}C_{221j}-C_{111i}C_{222j}\)\frac{n_in_j}{2}} = \matrice{0 \\ 0 \\ \alpha_{11} \\ \alpha_{22} \\ \alpha_{12} }
\end{equation}
with $\Cbb=\Cbb^{ep}$ for plain strain and $\Cbb=\widetilde{\Cbb}^{ep}$ for plane stress.
%In particular, $\Cbb$ reduces to the elastic stiffness tensor when no plastic flow occurs so that the characteristic structure involves the speeds of elastic pressure waves $c_1$ and shear waves $c_2$. 

It has been seen in section \ref{sec:SVK_solution} that the solution of non-linear problems may contain shock and/or simple waves.
Nevertheless, we restrict here to simple waves by assuming that: (i) the characteristic speeds satisfy $c_1 \geq c_f \geq c_2 \geq c_s $, where $c_1$ and $c_2$ are the speeds of elastic pressure and shear discontinuities respectively; (ii) $c_f$ and $c_s$ monotonically decrease with the hardening of the material; (ii) the computational domain is in an initial natural, plastic strain free state.

The characteristic equations $\Lcb^K \cdot d\Qcb = 0$ are then written:
\begin{subequations}
  %\label{eq:ch5_ODEs}
  \begin{alignat}{3}
    \label{eq:charac_fr}
    & \rho c_f \vect{l}^1 \cdot d\vect{v} - l^1_i n_j d\sigma_{ij} =0 \qquad && \text{along }\: dx/dt = c_f\\
    \label{eq:charac_fl}
    -& \rho c_f \vect{l}^1 \cdot d\vect{v} - l^1_i n_j d\sigma_{ij} =0 \qquad && \text{along }\: dx/dt = - c_f \\
    \label{eq:charac_sr}
    & \rho c_s \vect{l}^2 \cdot d\vect{v} - l^2_i n_j d\sigma_{ij} =0 \qquad  && \text{along }\: dx/dt =  c_s \\
    \label{eq:charac_sl}
    -& \rho c_s \vect{l}^2 \cdot d\vect{v} - l^2_i n_j d\sigma_{ij} =0 \qquad  && \text{along }\: dx/dt = - c_s \\
    \label{eq:charac_contact}
    &\alpha_{11}d\sigma_{11} + \alpha_{12}d\sigma_{12} + \alpha_{22}d\sigma_{22}=0 \qquad && \text{along }\: dx/dt =0 
  \end{alignat}
\end{subequations}
Integration of equations \eqref{eq:charac_fr} to \eqref{eq:charac_contact} leads to integral curves through simple waves in which several stress components vary, hence the name of combined-stress simple waves \cite{CRISTESCU19591605}.
Following \cite{Clifton}, the method of characteristic is applied by combining equations \eqref{eq:charac_fr} to \eqref{eq:charac_contact}.
\begin{figure}[h!]
  \centering
  \subcaptionbox{Slow simple wave \label{subfig:slowWave}}{\begin{tikzpicture}[scale=1.,>=stealth] 
  \newcommand\shift{5.}
  %% Slow
  \draw[thick,->] (0,0) -- (4.,0) node[right] {$x_n$};
  \draw[thick,->] (0,0) -- (0.,4) node[above] {$t$};
  % Slope = 0.75
  \draw (0,0) -- (4,3.) node [right] {$c_s^0 \quad (Head)$};
  % Slope = 4./3.
  \draw (0,0) -- (3.,4.) node [right] {$c_s \quad (Tail)$};

  \fill[black] (1.5,1.5*4./3.) circle (0.05) node [above] {$P$};
  %% Other characteristics
  % stationary
  \draw[dashed] (1.5,0.75*1.5) -- (1.5,1.5*4./3.);
  % fast plus (slope =+-0.25)
  \newcommand\px{1.5}
  \newcommand\py{1.5*4./3.}
  \draw (2.*\py-0.5*\px,1.5*\py-3.*\px/8.) node [below right] {$c_f$}-- (1.5,1.5*4./3.) ;
  % fast minus
  \draw (\py+0.25*\px,0.75*\py+3.*\px/16.) node [right] {$-c_f$} -- (1.5,1.5*4./3.) ;
  % slow minus (slope=-4./3.)
  \draw (12.*\py/25.+16.*\px/25.,9.*\py/25.+36.*\px/75.) node [right] {$-c_s$} -- (1.5,1.5*4./3.) ;
\end{tikzpicture}


%%% Local Variables:
%%% mode: latex
%%% TeX-master: "../../mainManuscript"
%%% End:
} \qquad
  \subcaptionbox{Fast simple wave \label{subfig:fastWave}}{\begin{tikzpicture}[scale=1.3] 
  \newcommand\shift{5.}
  %% Fast
  \draw[thick,->] (0+\shift,0) -- (4.+\shift,0) node[right] {$x_n$};
  \draw[thick,->] (0+\shift,0) -- (0.+\shift,4) node[above] {$t$};
  % Slope = 0.25
  \draw (0+\shift,0) -- (4+\shift,1.) node [right] {$c_f^0$};
  % Slope = 5./8.
  \draw (0+\shift,0) -- (4+\shift,2.5) node [right] {$c_f$};
  
  \fill[black] (1.5+\shift,1.5*5./8.) circle (0.05) node [above] {$P$};
  %% Other characteristics
  \newcommand\pxx{1.5}
  \newcommand\pyy{1.5*5./8.}
  % stationary
  \draw[dashed] (1.5+\shift,1.5/4.) -- (\pxx+\shift,\pyy);
  % fast minus
  \draw (8.*\pyy/7.+5.*\pxx/7.+\shift,2.*\pyy/7.+10.*\pxx/56.) node [below right] {$c_f$}-- (\pxx+\shift,\pyy);
  % slow plus
  \draw (-12.0*\pyy/13.+16.*\pxx/13.+\shift,-3.*\pyy/13.+4.*\pxx/13.) -- (\pxx+\shift,\pyy);
  \node at (-12.0*\pyy/13.+16.*\pxx/13.+\shift+0.02,-3.*\pyy/13.+4.*\pxx/13.-0.12) {$c_s$};
  % slow minus
  \draw (12.0*\pyy/19.+16.*\pxx/19.+\shift,3.*\pyy/19.+4.*\pxx/19.)node [below] {$-c_s$} -- (\pxx+\shift,\pyy);
\end{tikzpicture}


%%% Local Variables:
%%% mode: latex
%%% TeX-master: "../../mainManuscript"
%%% End:
}
  \caption{The method of characteristics through slow and fast simple waves in the $(x_n,t)$ plane.}
  \label{fig:ch5_charac_method}
\end{figure}
The approach consists in tracing every characteristics from some downstream point of a wave where the state vector $\Qcb$ is known, to an upstream point where the solution is seeked. Figures \ref{fig:ch5_charac_method}\subref{subfig:slowWave} and \ref{fig:ch5_charac_method}\subref{subfig:fastWave} schematically illustrate the method for slow and fast simple waves in which the state is known along the head wave and is looked for at point $P$ lying on the tail wave. 
The integral curves through slow and fast simple waves are derived in the next section.

\subsection{Integral curves through simple waves}
The right-going slow waves are first looked at by adding equations \eqref{eq:charac_fr} and \eqref{eq:charac_fl}:
\begin{equation}
  l_i^1 n_j d\sigma_{ij}=0
\end{equation}
Given the geometry of the problem, the vector $\vect{n}$ may be reduced to $\vect{e}_1$ or $\vect{e}_2$.
It therefore comes out:
%In particular, for a vector $\vect{n}$ that is restricted to the axis of the $(\vect{e}_1,\vect{e}_2)$ plane, one gets:
\begin{subequations}
  \begin{alignat}{2}
    \label{eq:sigSlow_n=e1}
    & d\sigma_{11} = - \frac{l^1_2}{l_1^1} d\sigma_{12} = \psi^s_{1}d\sigma_{12} && \qquad \text{for } \:\vect{n}=\vect{e}_1 \\
    \label{eq:sigSlow_n=e2}
    & d\sigma_{22}=- \frac{l_1^1}{l_2^1}  d\sigma_{12} = \psi^s_{2}d\sigma_{12} && \qquad \text{for } \:\vect{n}=\vect{e}_2
  \end{alignat}
\end{subequations}
where $\psi^s_1$ and $\psi^s_2$ are functions of all components of $\tens{\sigma}$. 
Moreover, the $s$ and $f$ superscripts stand for slow and fast waves respectively in the remainder of the manuscript.
With the above equations, the characteristic equation related to the contact wave \eqref{eq:charac_contact} reads:
%Next, the characteristic equation related to the contact wave \eqref{eq:charac_contact} yields:
\begin{subequations}
  \begin{alignat}{2}
    \label{eq:sigContact_n=e1}
    & d\sigma_{22} = -\frac{\psi^s_{1}\alpha_{11}+\alpha_{12}}{\alpha_{22}}d\sigma_{12} && \qquad \text{for } \:\vect{n}=\vect{e}_1 \\
    \label{eq:sigContact_n=e2}
    & d\sigma_{11}= -\frac{\psi^s_{2}\alpha_{22}+\alpha_{12}}{\alpha_{11}} d\sigma_{12} && \qquad \text{for } \:\vect{n}=\vect{e}_2
  \end{alignat}
\end{subequations}
The sets of equations \eqref{eq:sigSlow_n=e1}-\eqref{eq:sigContact_n=e1} and \eqref{eq:sigSlow_n=e2}-\eqref{eq:sigContact_n=e2} show the combined-stress nature of slow simple waves.
% Hence, one stress component may be used as a driving parameter for the two others, as it is the case for $\sigma_{12}$ in equations \eqref{eq:sigSlow_n=e1}, \eqref{eq:sigSlow_n=e2}, \eqref{eq:sigContact_n=e1} and \eqref{eq:sigContact_n=e2}.
On the other hand, the subtraction of equations \eqref{eq:charac_fr} and \eqref{eq:charac_fl} leads to:
\begin{equation*}
  dv_1 = \psi^s_{1}dv_2 = \frac{1}{\psi^s_2}dv_2
\end{equation*}
which, once combined with equations \eqref{eq:sigSlow_n=e1}-\eqref{eq:sigSlow_n=e2} and introduced in \eqref{eq:charac_sl}, yields after simplifications:
\begin{subequations}
  \begin{alignat}{2}
    \label{eq:vSlow_n=e1}
    & dv_1 = -\frac{d\sigma_{11}}{\rho c_s^2} \quad ;\quad  dv_2 = -\frac{d\sigma_{12}}{\rho c_s^2} \qquad & \text{for } \vect{n}=\vect{e}_1\\
    \label{eq:vSlow_n=e2}
    & dv_1 = -\frac{d\sigma_{12}}{\rho c_s^2} \quad ;\quad  dv_2 = -\frac{d\sigma_{22}}{\rho c_s^2} \qquad & \text{for } \vect{n}=\vect{e}_2
  \end{alignat}
\end{subequations}

\begin{remark}
  The integral curves through a left-going slow wave result from the combination of equations \eqref{eq:sigSlow_n=e1}-\eqref{eq:sigSlow_n=e2} introduced in \eqref{eq:charac_sr} rather than \eqref{eq:charac_sl}.
  Therefore, the only difference lies in the signs in equations \eqref{eq:vSlow_n=e1} and \eqref{eq:vSlow_n=e2}
\end{remark}

Similar results are obtained for right-going fast simple waves by using $\vect{l}^2$ instead of $\vect{l}^1$ and $c_f$ rather than $c_s$.
%However, the integral curves involve $\vect{l}^2$ and $c_f$ instead of $\vect{l}^1$ and $c_s$. 
Hence, the evolution in slow and fast waves is governed by the \textit{loading functions}:
\begin{equation}
  \label{eq:loading_func}
  \psi^s_{1}=- \left.\frac{l^1_2}{l_1^1}\right\rvert_{\vect{n}=\vect{e}_1}\quad ,\quad  \psi^s_{2}=- \left.\frac{l_1^1}{l_2^1}\right\rvert_{\vect{n}=\vect{e}_2} \quad ,\quad \psi^f_1=-\left.\frac{l_2^2}{l_1^2}\right\rvert_{\vect{n}=\vect{e}_1} \quad ,\quad \psi^f_2=-\left.\frac{l_1^2}{l_2^2}\right\rvert_{\vect{n}=\vect{e}_2}
\end{equation}
The equations satisfied across right-going slow and fast simple waves are summarized in table \ref{tab:simpleWavesEquations}.
\begin{table}[h!]
  \centering
  \begin{tabular}{cc|ccN}
    \hline
    \multicolumn{2}{c}{Right-going slow wave} \vline& \multicolumn{2}{c}{Right-going fast wave} & \\
    $\vect{n}=\vect{e}_1$ & $\vect{n}=\vect{e}_2$ & $\vect{n}=\vect{e}_1$ & $\vect{n}=\vect{e}_2$&\\
    \hline
    \hline
    $dv_1 = -\frac{d\sigma_{11}}{\rho c_s^2}$ &  $dv_1 = -\frac{d\sigma_{12}}{\rho c_s^2}$ &$dv_1 = -\frac{d\sigma_{11}}{\rho c_f^2}$ &  $dv_1 = -\frac{d\sigma_{12}}{\rho c_f^2}$ &\\ [8pt]
    $dv_2 = -\frac{d\sigma_{12}}{\rho c_s^2}$ & $dv_2 = -\frac{d\sigma_{22}}{\rho c_s^2}$ & $dv_2 = -\frac{d\sigma_{12}}{\rho c_f^2}$ & $dv_2 = -\frac{d\sigma_{22}}{\rho c_f^2}$& \\ [8pt]
    $d\sigma_{11} = \psi^s_{1}d\sigma_{12}$&$d\sigma_{11}= -\frac{\psi^s_{2}\alpha_{22}+\alpha_{12}}{\alpha_{11}} d\sigma_{12}$ &  $d\sigma_{11} = \psi^f_{1}d\sigma_{12}$&$d\sigma_{11}= -\frac{\psi^f_{2}\alpha_{22}+\alpha_{12}}{\alpha_{11}} d\sigma_{12}$ & \\[8pt]
    $d\sigma_{22} = -\frac{\psi^s_{1}\alpha_{11}+\alpha_{12}}{\alpha_{22}}d\sigma_{12}$ & $d\sigma_{22}= \psi^s_{2}d\sigma_{12}$ & $d\sigma_{22} = -\frac{\psi^f_{1}\alpha_{11}+\alpha_{12}}{\alpha_{22}}d\sigma_{12}$ & $d\sigma_{22}= \psi^f_{2}d\sigma_{12}$ & \\[8pt]
    % & & \\
    % & & \\    
    \hline
\end{tabular}
%%% Local Variables:
%%% mode: latex
%%% TeX-master: "../manuscript"
%%% End:

  \caption{Summary of the ODEs satisfied inside right-going slow and fast simple waves.}
  \label{tab:simpleWavesEquations}
\end{table}

The integration of equations gathered in table \ref{tab:simpleWavesEquations} should provide the complete solution of a given problem by means of integral curves or loading paths.
For instance, the velocity resulting from the passage of right-going waves in the direction $\vect{e}_1$ obeys:
\begin{equation}
  \label{eq:integral_example}
  v_1 = v_1^0 - \int_{\tens{\sigma}^0}^{\tens{\sigma}} \frac{d \sigma_{11}}{\rho c^2} \quad ;\quad v_2 = v_2^0 - \int_{\tens{\sigma}^0}^{\tens{\sigma}} \frac{d \sigma_{12}}{\rho c^2}
\end{equation}
where the zero superscript denotes the downstream state.
Nevertheless, \textsc{Clifton} \cite{Clifton} emphasized that depending on the loading conditions, only one simple waves or both may arise in the solution.
Therefore, it is crucial to identify the stress path followed to properly compute integrals \eqref{eq:integral_example}.
%It is thus important to qualify the stress paths through fast and slow waves in order to properly define the upper bound of the integrals of the form \eqref{eq:integral_example}.
This is the purpose of the next section.



%%% Local Variables:
%%% mode: latex
%%% TeX-master: "../mainManuscript"
%%% End:


\section{Loading paths through simple waves}
\label{sec:stress_paths}
% On ne regarde qu'une dimension spatiale en faisant des hypothèse sur les champs alors que nous on se limite à une direction particulière $\vect{n}$.
% En plus, on se limite à l'étude d'ondes simples alors que des chocs peuvent exister (voir Mandell car il semble y etre démontré que les shock n'arrivent que pour $\tau=0$).
% Il y a la question des vitesses charactéristiques plastiques... sont-elles collées aux vitesses élastiques ?
% dependance des vitesses caractéristiques à l'angle entre la direction principale de sigma et la direction de propagation, c'est dit dans la thèse de Clifou en page 90.

\subsection{Properties of the loading paths}
The stress paths followed within slow and fast simple waves are governed by the mathematical properties of the loading functions \eqref{eq:loading_func}.
Before specializing the discussion to plane stress and plane strain cases, some general properties holding regardless of the loading conditions are highlighted.
%The analysis is here carried out for the special case $\vect{n}=\vect{e}_1$, similar results being obtained for the other situation $\vect{n}=\vect{e}_2$.

First, the functions satisfy the orthogonality properties: $\psi^s_1\psi^f_1=-1$ and $\psi^s_2\psi^f_2=-1$.
Indeed, considering the left eigenvectors of the acoustic tensor given in equation \eqref{eq:ch5_eigenvectAcc}, the product $\psi^s_1\psi^f_1$ reads:
\begin{equation*}
  \psi^s_1\psi^f_1 = \frac{l^1_2}{l^1_1}\: \frac{l_2^2}{l^2_1} = \frac{(A_{11}-\omega_2)A_{12}}{(A_{22}-\omega_1)A_{12}}
\end{equation*}
Introduction of the expressions of eigenvalues $\omega_i$ from equations \eqref{eq:ch5_eigenAcc1} and \eqref{eq:ch5_eigenAcc1} further leads to:
\begin{equation*}
  \psi^s_1\psi^f_1 = \frac{A_{11} -A_{22} +\sqrt{(A_{11} -A_{22} )^2 + 4A_{12}^2 }}{A_{22} -A_{11} -\sqrt{(A_{11} -A_{22} )^2 + 4A_{12}^2 }}=-1
\end{equation*}
or equivalently, $\vect{l}^1 \cdot \vect{l}^2=0$.
%as expected by the symmetry of $\tens{A}$.
This orthogonality has already been noticed for particular plane strain and plane stress cases \cite{Clifton,Ting68}. % but now obviously appears as true for all problems in two space dimensions.
However, the eigenvectors of symmetric second-order tensors all satisfy this property in such a way that it is valid for all problems in two space dimensions.
As a result, the study can be restricted to one function in each direction, say $\psi_1^s$ and $\psi_2^s$.

Second, if the function $\psi_1^s$ vanishes at some point of the stress space, the projection of the loading path followed inside a slow wave in the $(\sigma_{11},\sigma_{12})$ plane is vertical according to the ODE \eqref{eq:sigSlow_n=e1}.
Conversely, if $\psi_1^s\rightarrow \infty$, the loading path is horizontal in the $(\sigma_{11},\sigma_{12})$ plane.
%Looking for vanishing $\psi^f_1$ or $1/\psi^f_1$ amounts to finding roots of the components of $\vect{l}^2$:
These situations respectively correspond to:
\begin{subequations}
  \begin{alignat}{1}
    \label{eq:first_root}
    \psi_1^s = 0   & \Leftrightarrow A_{12} =0  \\
    \label{eq:second_root}
    \psi_1^s \rightarrow \infty & \Leftrightarrow A_{22} -\omega_1 =0
  \end{alignat}
\end{subequations}
In particular, if $A_{12}=0$ the second equation reads:
\begin{equation}
  A_{22} -\omega_1 = \frac{1}{2}\(A_{22} -A_{11} -\sqrt{(A_{11} -A_{22} )^2 + 4A_{12}^2 }\) = -\left\langle A _{11}-A _{22}  \right\rangle
\end{equation}
where $\left\langle \bullet \right\rangle$ denotes the positive part operator.
Hence, if $A_{12} =0$ and $A_{11} \neq A_{22} $, one has $\psi^s_1 =0$ and hence $\psi^f_1 \rightarrow -\infty $.
If moreover $A_{11}  = A_{22} $, both components of the eigenvectors vanish and the functions $\psi^s_1$ and $\psi^f_1$ are undetermined.
At last, it follows from equation \eqref{eq:diff_celerities} that the simultaneous satisfaction of conditions \eqref{eq:first_root} and \eqref{eq:second_root} leads to characteristic speeds of simple waves that are identical. Hence, the situation $c_f=c_s$ corresponds to a loss of hyperbolicity of the system.
% Hence, if $A_{12} =0$ and $A_{11} \neq A_{22} $, one has $\psi^s_1 =0$ and $\psi^f_1 \rightarrow \infty $, so that the stress path in the ($\sigma_{11},\sigma_{12}$) plane are vertical (\textit{resp. horizontal}) through a slow (\textit{resp. fast}) wave. 

Analogously, the function $\psi_2^s$ is such that:
\begin{subequations}
  \begin{alignat}{1}
    \label{eq:first_root_psi2cp}
    \psi_2^s \rightarrow \infty  & \Leftrightarrow A_{12} =0  \\
    \label{eq:second_root_psi2cp}
    \psi_2^s =0 & \Leftrightarrow A_{22} -\omega_1 =0
  \end{alignat}
\end{subequations}
Therefore, if both conditions \eqref{eq:first_root_psi2cp} and \eqref{eq:second_root_psi2cp} are satisfied, the system is no longer hyperbolic with characteristic speeds of fast and slow waves that are identical.

According to the ODEs of table \ref{tab:simpleWavesEquations}, the particular values of the loading functions $\psi_i^{s,f}$ through the simple waves propagating in direction $\vect{e}_i$ for $i=\{1,2\}$, provide information about the loading paths in stress space.
First, $\psi_i =0$ leads to $d\sigma_{ii}=0$ (no sum on $i$) so that the longitudinal stress is constant within the simple wave.
Conversely, with loading functions tending to infinity, the stress $\sigma_{12}$ does not vary.
Notice that the coefficients $\alpha_{ij}$ of the left eigenvector of the Jacobian matrix associated to the zero eigenvalue \eqref{eq:ch5_null_left_eigen} also have to be regarded.
Nevertheless, those terms resulting from products of the components of the elastoplastic tangent modulus have complex expressions and are assumed to have non-zero values in the remainder of the manuscript.

The above discussions are now specified to plane strain and plane stress, for which loading conditions leading to $A_{12} =0$ and $A _{11}-A _{22}=0$ are identified.



\subsection{The plane strain case}
The case of plane strain is first considered by using the elastoplastic tangent modulus so that the components of the acoustic tensor for $\vect{n}=\vect{e}_1$ read:
%The elastoplastic tangent modulus under consideration is now that given in equation \eqref{eq:elastoplastic_tangent}, so that the components of the acoustic tensor for $\vect{n}=\vect{e}_1$ read: 
\begin{subequations}
  \begin{alignat}{1}
    \label{eq:DP_A11}
    & A_{11}^{ep}= C_{1111}^{ep} = \lambda + 2\mu -\beta s_{11}^2 \\
    \label{eq:DP_A22}
    & A_{22}^{ep}= C_{2121}^{ep}= \mu -\beta s_{12}^2 \\
    \label{eq:DP_A12}
    & A_{12}^{ep}= C_{1121}^{ep}=-\beta s_{11}s_{12}
  \end{alignat}
\end{subequations}
The associated eigenvalues are then:
\begin{subequations}
  \label{eq:eigen_acc_DP}
  \begin{alignat}{1}
    \label{eq:eigen_acc_DP1}
    & \rho c_s^2 = \frac{1}{2}\( \lambda +3\mu -\beta (s_{11}^2+ s_{12}^2) - \sqrt{(\lambda + \mu -\beta (s_{11}^2-s_{12}^2) )^2 +4(\beta s_{11}s_{12})^2} \) \\
    \label{eq:eigen_acc_DP2}
    & \rho c_f^2 = \frac{1}{2}\( \lambda +3\mu -\beta (s_{11}^2+ s_{12}^2) + \sqrt{(\lambda + \mu -\beta (s_{11}^2-s_{12}^2) )^2 +4(\beta s_{11}s_{12})^2}  \)
  \end{alignat}
\end{subequations}
Subtracting equations \eqref{eq:DP_A11} and \eqref{eq:DP_A22}, one gets: $A_{11}^{ep}-A_{22}^{ep}= \lambda + \mu -\beta \(s_{11}^2-s_{12}^2\)$.
Hence, the equation $A_{11}^{ep}-A_{22}^{ep}=0$ admits a set of solutions in the deviatoric stress space.
On the other hand, we see from equation \eqref{eq:DP_A12} that $A_{12}^{ep}$ vanishes for $s_{12}=0$ and $s_{11}=0$.
% Recall that $A^{ep}_{12}=0$ leads to vertical and horizontal loading paths across slow and fast waves respectively. 
Each solution is studied in more details below.

%% Sign of one of the functions psi... but not used afterwards
% We first study the sign of the functions $\psi^f$ by noticing that $\mu=\rho c_2^2$ so that $A_{22}^{ep}$ may be rewritten to yield:
% \begin{equation*}
%   \psi^f = -\frac{A_{12}^{ep}}{A_{22}-\rho c_f^2}= -\frac{\beta s_{11}s_{12}}{\rho c_f^2-\rho c_2^2 +\beta s_{12}^2 }
% \end{equation*}
% Since the denominator is positive for $c_f \geq c_2$, it comes out that $\sign (\psi^f) = - \sign(s_{12}) \sign(s_{11})$. Moreover, two roots of the loading function $\psi^f$ can be identified.

\paragraph*{Condition $s_{12}=0$:} 
According to equations \eqref{eq:eigen_acc_DP1} and \eqref{eq:eigen_acc_DP2}, the eigenvalues of the acoustic tensor become:
\begin{align*}
  & \rho c_s^2 = \frac{1}{2}\( \lambda +3\mu -\beta s_{11}^2 - \abs{\lambda + \mu -\beta s_{11}^2 } \) \\
  & \rho c_f^2 = \frac{1}{2}\( \lambda +3\mu -\beta s_{11}^2 + \abs{\lambda + \mu -\beta s_{11}^2 } \)
\end{align*}
Thus, assuming that $\beta s_{11}^2 < \lambda + \mu$, the expression further reduces to:
\begin{align*}
  & \rho c_s^2 = \mu \\
  & \rho c_f^2 = \lambda +2\mu -\beta s_{11}^2 
\end{align*}
The characteristic speed of slow waves therefore identifies with this of elastic shear waves for plane strain $c_s=c_2=\sqrt{\mu/\rho}$. 
If conversely $ \lambda + \mu - \beta s_{11}^2$ is negative, the characteristic speeds read: 
\begin{align*}
  & \rho c_s^2 = \lambda +2\mu -\beta s_{11}^2  \\
  & \rho c_f^2 =  \mu 
\end{align*}
As a result, in that case it is the celerity of fast waves which reduces to that of elastic shear waves.
Note however that for the characteristic speed of slow waves remains real, the stress $s_{11}$ must satisfy $\beta s_{11}^2 < \lambda +2\mu$.
At last, the equality $\beta s_{11}^2 = \lambda + \mu$ leads to $A_{11}^{ep}-A_{22}^{ep}=0$ and hence, to undetermined loading functions. 
%% Set of admissible values for s11 (depends on s itself)
% It then appears that the values of $s_{11}$ ensuring hyperbolicity of the system are:
% \begin{equation}
%   s_{11} \in ]-\infty,-\sqrt{\frac{\lambda + \mu}{\beta}}[\: \cup\: ]-\sqrt{\frac{\lambda + \mu}{\beta}},\sqrt{\frac{\lambda + \mu}{\beta}}[\: \cup \:]\sqrt{\frac{\lambda + \mu}{\beta}} ,\infty[
% \end{equation}

%% Discussion about the loading path direction
% Recall that $\psi^f_1$ tending to infinity implies that the loading path are horizontal in $(\sigma_{11},\sigma_{12})$ plane and hence, the fast wave has no influence on the shear stress if, and only if, $\sigma_{12}=0$ downstream.
% Conversely, the stress paths through slow simple waves are vertical.
% Moreover, with regard the last row of table \ref{tab:simpleWavesEquations}, $\sigma_{22}$ is also unchanged in that case.
% As a consequence, if the initial state is shear-free the solution no longer contain combined waves, but longitudinal stress and shear stress simple waves.

\paragraph*{Condition $s_{11}=0$:}
Considering the relation \eqref{eq:plane_strain_stress33} between stress components for plane strain, one writes:
\begin{equation*}
  s_{11}= \frac{2}{3}\sigma_{11}-\frac{1}{3}(\sigma_{22}+\nu(\sigma_{11}+\sigma_{22})-E\eps^p_{33})
\end{equation*}
so that $s_{11}=0$ is equivalent to:
\begin{equation}
  \label{eq:plane_strain_s11=0}
  \sigma_{11}=\frac{1+\nu}{2-\nu}\sigma_{22}-E\eps^p_{33}
\end{equation}
In contrast to what has been seen previously, the functions $\psi$ cannot be undetermined in the case $s_{11}=0$ since the equation $A_{11}^{ep}-A_{22}^{ep}=\lambda + \mu + \beta s_{12}^2$ does not admit real solutions.
However, the stress state \eqref{eq:plane_strain_s11=0} yields the following characteristic speeds:
\begin{align*}
  & \rho c_s^2 = \mu -\beta s_{12}^2 \\
  & \rho c_f^2 = \lambda +2\mu 
\end{align*}
so that the celerity of fast waves identifies with that of elastic pressure waves under plane strains $c_f=\sqrt{(\lambda + 2\mu)/\rho}=c_1$.

%%%% n=e2
$\newline$
The same analysis can be carried out in the direction $\vect{n}=\vect{e}_2$ by considering the following acoustic tensor components:
\begin{subequations}
  \begin{alignat}{1}
    \label{eq:DP_A11_n2}
    & A_{11}^{ep}= C_{1212}^{ep} = \mu -\beta s_{12}^2 \\
    \label{eq:DP_A22_n2}
    & A_{22}^{ep}= C_{2222}^{ep}= \lambda + 2\mu -\beta s_{22}^2 \\
    \label{eq:DP_A12_n2}
    & A_{12}^{ep}= C_{1222}^{ep}=-\beta s_{22}s_{12}
  \end{alignat}
\end{subequations}
The charecteristic speeds are then:
\begin{subequations}
  \label{eq:eigen_acc_DP_n2}
  \begin{alignat}{1}
    \label{eq:eigen_acc_DP1_n2}
    & \rho c_s^2 = \frac{1}{2}\( \lambda +3\mu -\beta (s_{22}^2+ s_{12}^2) - \sqrt{(\lambda +\mu -\beta (s_{22}^2-s_{12}^2) )^2 +4(\beta s_{22}s_{12})^2} \) \\
    \label{eq:eigen_acc_DP2_n2}
    & \rho c_f^2 = \frac{1}{2}\( \lambda +3\mu -\beta (s_{22}^2+ s_{12}^2) + \sqrt{(\lambda + \mu -\beta (s_{22}^2-s_{12}^2) )^2 +4(\beta s_{22}s_{12})^2}  \)
  \end{alignat}
\end{subequations}
With the above expressions, the same remarks than for $\vect{n}=\vect{e}_1$ can obviously be made by replacing $s_{11}$ with $s_{22}$.

Among the above results, the most significant arises from the condition $s_{12}=0$.
Indeed, it has been seen that $A_{12}^{ep}=0$ leads to $\psi_1^s=0$ and $\psi^s_2\rightarrow \infty$ in such a way that the corresponding loading paths in the $(\sigma_{11},\sigma_{12})$ plane are respectively vertical and horizontal.
In virtue of the orthogonality property of the loading functions, the stress path followed in a fast wave propagating in the direction $\vect{e}_1$ is horizontal in the same plane.
Hence, if the path through a fast wave intersects the plane $\sigma={12}=0$, the shear stress component remains constant afterwards.
The same result holds for the slow wave propagating in the direction $\vect{e}_2$.

\subsection{The plane stress case}
The elastoplastic tangent modulus under consideration is now that given in equation \eqref{eq:CP_constitutive}.
We propose to first consider $\psi_1^s$ related to the vector $\vect{n}=\vect{e}_1$.
Thus:
\begin{subequations}
  \label{eq:CP_Acoustic}
  \begin{alignat}{1}
    \label{eq:CP_A11}
    & \widetilde{A}_{11}^{ep}= \widetilde{C}^{ep}_{1111} - \frac{(\widetilde{C}^{ep}_{1133})^2}{\widetilde{C}^{ep}_{3333}} = \lambda + 2\mu -\beta s_{11}^2 -\frac{\(\lambda -\beta s_{11}s_{33}\)^2}{\lambda + 2\mu - \beta s_{33}^2} \\
    \label{eq:CP_A22}
    & \widetilde{A}_{22}^{ep}= \widetilde{C}^{ep}_{2121} - \frac{(\widetilde{C}^{ep}_{2133})^2}{\widetilde{C}^{ep}_{3333}}= \mu - \beta s_{12}^2 -\frac{\(\beta s_{12}s_{33}\)^2}{\lambda + 2\mu - \beta s_{33}^2} \\
    \label{eq:CP_A12}
    & \widetilde{A}_{12}^{ep} = \widetilde{C}^{ep}_{1121} - \frac{\widetilde{C}^{ep}_{1133}\widetilde{C}^{ep}_{1233}}{\widetilde{C}^{ep}_{3333}} =\beta s_{12} \frac{\lambda s_{33} - (\lambda + 2\mu)s_{11} }{\lambda + 2\mu - \beta s_{33}^2} 
  \end{alignat}
\end{subequations}
In order to ensure the hyperbolicity of the system, the component of the acoustic tensor also have to be defined, that is $\widetilde{C}^{ep}_{3333}\neq 0$. This condition leads to:
\begin{equation*}
  \lambda + 2\mu - \beta s_{33}^2 \neq 0 \quad \Leftrightarrow \quad s_{33}\neq \frac{\lambda + 2\mu}{\beta}
\end{equation*}
Second, from equation \eqref{eq:CP_A12}, $\widetilde{A}_{12}^{ep}$ admits two roots in terms of the components of the deviatoric stress tensor, namely: 
\begin{equation}
  s_{12}=0 \quad ; \quad s_{11}= \frac{\lambda}{\lambda+2\mu}s_{33}
\end{equation}
In terms of the components of Cauchy stress tensor, those conditions read:
% \begin{align}
%   & \frac{2}{3}\sigma_{11}-\frac{1}{3}\sigma_{22} = -\frac{\lambda}{3\lambda+6\mu}(\sigma_{11}+\sigma_{22}) \\
%   & 2\sigma_{11}-\sigma_{22} = -\frac{\lambda}{\lambda+2\mu}(\sigma_{11}+\sigma_{22}) \\
%   & \sigma_{11}(2 +\frac{\lambda}{\lambda+2\mu})=\sigma_{22}(1-\frac{\lambda}{\lambda+2\mu})\\
%   & \sigma_{11}\frac{3\lambda+4\mu}{\lambda+2\mu}=\sigma_{22}\frac{2\mu}{\lambda+2\mu}\\
%   & \sigma_{11}=\sigma_{22}\frac{2\mu}{3\lambda+4\mu}
% \end{align}
\begin{equation}
  \label{eq:CP_roots}
  \sigma_{12}=0 \quad ; \quad \sigma_{11}=\frac{2\mu}{3\lambda+4\mu}\sigma_{22}
  % \sigma_{12}=0 \quad ; \quad \sigma_{11}=\frac{1-2\nu}{2-\nu} \sigma_{22}
\end{equation}
% Hence, the loading path through a slow simple wave is vertical, that is $\psi^s_1 = 0$, for stress values satisfying \eqref{eq:CP_roots}, providing that the $\widetilde{A}_{11}^{ep}$ and $\widetilde{A}_{22}^{ep}$ are not equal.
% Conversely, such stress states yield horizontal path through a fast wave.

%% Attempt to show that if s12=0, cs=c2 but depends on the sign of A11-A22
% \begin{align}
%   &\omega_2=\frac{1}{2}\(A_{11}+A_{22} - \abs{A_{11}-A_{22}}\)\\
%   &\omega_2=\frac{1}{2}\(A_{11}+A_{22} - A_{11}+A_{22}\) \quad ;\quad \omega_2=\frac{1}{2}\(A_{11}+A_{22} + A_{11}-A_{22}\) \\
%   &\omega_2=A_{22} \quad ;\quad \omega_2=A_{11}
% \end{align}


% \paragraph*{Case $s_{12}=0$ :}
% \begin{align}
%   & \rho c_s^2 =\frac{1}{2}\(\widetilde{A}_{11}^{ep}+\widetilde{A}_{22}^{ep} - \abs{\widetilde{A}_{11}^{ep}-\widetilde{A}_{22}^{ep}}\) \\
%   & \rho c_f^2 =\frac{1}{2}\(\widetilde{A}_{11}^{ep}+\widetilde{A}_{22}^{ep} + \abs{\widetilde{A}_{11}^{ep}-\widetilde{A}_{22}^{ep}}\)
% \end{align}

If on the other hand the vector $\vect{n}=\vect{e}_2$ is considered, the acoustic tensors components read:
\begin{subequations}
  \begin{alignat}{1}
    \label{eq:CP_A11_n=e2}
    & \widetilde{A}_{11}^{ep}= \widetilde{C}^{ep}_{1212} - \frac{(\widetilde{C}^{ep}_{1233})^2}{\widetilde{C}^{ep}_{3333}} = \mu -\beta s_{12}^2 -\frac{\(\lambda -\beta s_{12}s_{33}\)^2}{\lambda + 2\mu - \beta s_{33}^2} \\
    \label{eq:CP_A22_n=e2}
    & \widetilde{A}_{22}^{ep}= \widetilde{C}^{ep}_{2222} - \frac{(\widetilde{C}^{ep}_{2233})^2}{\widetilde{C}^{ep}_{3333}}= \lambda +2\mu - \beta s_{22}^2 -\frac{\(\beta s_{22}s_{33}\)^2}{\lambda + 2\mu - \beta s_{33}^2} \\
    \label{eq:CP_A12_n=e2}
    % \widetilde{A}_{12}^{ep}    = -\beta s_{12}s_{22} - \frac{(-\beta s_{12}s_{33})(\lambda - \beta s_{22}s_{33})}{\lambda + 2\mu - \beta s_{33}^2}
    % \widetilde{A}_{12}^{ep}    = \beta s_{12}\( s_{33}\frac{\lambda - \beta s_{22}s_{33}}{\lambda + 2\mu - \beta s_{33}^2}-s_{22}\)
    % \widetilde{A}_{12}^{ep}    = \beta s_{12}\( \frac{\lambda s_{33}  -s_{22}(\lambda + 2\mu ) }{\lambda + 2\mu - \beta s_{33}^2}\)
    & \widetilde{A}_{12}^{ep} = \widetilde{C}^{ep}_{1222} - \frac{\widetilde{C}^{ep}_{1233}\widetilde{C}^{ep}_{2233}}{\widetilde{C}^{ep}_{3333}} =\beta s_{12} \frac{\lambda s_{33} - (\lambda + 2\mu)s_{22} }{\lambda + 2\mu - \beta s_{33}^2}
  \end{alignat}
\end{subequations}
Which are similar to these obtained for $\vect{n}=\vect{e}_1$ \eqref{eq:CP_Acoustic} with $s_{22}$ instead of $s_{11}$.
It comes out that $\widetilde{A}_{12}^{ep}$ admits to roots:
\begin{equation}
  \label{eq:CP_roots_n=e2}
  \sigma_{12}=0 \quad ; \quad \sigma_{22}=\frac{2\mu}{3\lambda+4\mu}\sigma_{11}
\end{equation}

The complexity introduced by the plane stress tangent modulus prevents the finding of other singular configurations for the hyperbolic system. 
In particular, it is difficult to deal with the equation $\widetilde{A}^{ep}_{11}=\widetilde{A}^{ep}_{22}$ due to the expressions given in equations \eqref{eq:CP_A11} and \eqref{eq:CP_A22}.
Nevertheless, since the stress state $s_{12}=0$ also constitutes a singular point for plane stresses, the same remarks on the loading paths than for plane strains hold.
Namely, if $\sigma_{12}$ falls to zero along the loading path followed inside a fast (\textit{resp. slow}) wave propagating in direction $\vect{e}_1$ (\textit{resp. $\vect{e}_2$}), it is restricted to that value.
%As we shall see below, more singular behaviors can be identified for plane strain.





%%% Local Variables:
%%% mode: latex
%%% TeX-master: "../mainManuscript"
%%% End:



\section{Numerical integration of stress paths}
\label{sec:stress_paths_num}
Although some properties of the simple waves have been given in section \ref{sec:stress_paths}, the complexity of the equations prevents the complete characterization of the loading paths followed.
In order to get additional information about the evolution of the stress states within, the systems of ODEs gathered in table \ref{tab:simpleWavesEquations} are numerically integrated in this section for plane stress and plane strain loadings, based on the material parameters used in chapter \ref{chap:chap3}.
In particular, the thin-walled tube problem considered by Clifton \cite{Clifton} is first looked at so that the above developments can be validated.
Next, the plane stress and plane strain cases are treated.
The identified behaviors should provide some simplification assumptions for the building of a procedure that lead to approximate solutions of the problems.
The values of the elastic properties considered here are those use in the previous chapter (see table \ref{tab:material}).
On the other hand, the tensile yield stress $\sigma^y=10^{8} \: Pa$ and the hardening modulus $C=1\times10^8 \: Pa$ are set here arbitrarily.
Finally, we restrict here to positive shear stress $\sigma_{12}\geq 0$.
\subsection{Thin-walled tube problem}
%% Hypothèses du problème
Consider the semi-infinite domain in Cartesian coordinate system: $x_1 \times x_2 \times x_3 \in [0,\infty[ \times ]-\infty,\infty[ \times [-\infty,\infty]$, being acted upon by a traction vector $\vect{T}^d$ at $x_1=0 $.
Only the first two components of $\vect{T}^d$ are non-null so that the stress and strain tensor within the medium are of the form:
\begin{equation}
  \tens{\sigma} = \matrice{\sigma_{11} & \sigma_{12} & \\ \sigma_{12} & 0 & \\ & & 0} \quad ;\quad\tens{\eps} = \matrice{\eps_{11} & \eps_{12} & \\ \eps_{12} & \eps_{22}& \\ & & \eps_{33}}
\end{equation}
Such a state corresponds to that holding in a hollow cylinder with radius and length much bigger that its thickness, submitted to combined longitudinal and torsional loads.
Hence the name of thin-walled tube problem. 
As a particular plane stress case, the set of ODEs along characteristic derived in section \ref{sec:stress_paths} applies with nevertheless, taking into account the vanishing stress component $\sigma_{22}$.
Indeed, for plane stress one has:
\begin{equation*}
  \dot{\sigma}_{22}=\widetilde{C}^{ep}_{22ij} \dot{\eps}_{ij} =0 \quad i,j=\{1,2\}
\end{equation*}
where $\widetilde{\Cbb}^{ep}$ is the plane stress tangent modulus \eqref{eq:CP_constitutive}.
\begin{equation*}
  \widetilde{C}^{ep}_{2222} \dot{\eps}_{22} = - \widetilde{C}^{ep}_{22ij}\dot{\eps}_{ij} \quad ij=\{11,12,21\}
\end{equation*}
Thus, inverting the above equation and introducing it in the constitutive equation, we are left with the following law:
\begin{equation}
  \label{eq:ch5_TW_tangent}
  \dot{\sigma}_{ij}=\widetilde{C}^{ep}_{ijkl} \dot{\eps}_{kl} - \frac{\widetilde{C}^{ep}_{ij22}\widetilde{C}^{ep}_{22kl}}{\widetilde{C}^{ep}_{2222}}\dot{\eps}_{kl}= \widehat{C}^{ep}_{ijkl} \dot{\eps}_{kl}\qquad ij,kl=\{11,12,21\} 
\end{equation}
where $\widetilde{\Cbb}^{ep}$ is referred to as the thin-walled tube tangent modulus.
The characteristic analysis of the hyperbolic system based on this tangent modulus also leads to loading paths followed across slow and fast waves, involving nevertheless two components of stress only. For the sake of simplicity, the stress components are denoted by $\sigma_{11}=\sigma$ and $\sigma_{12}=\tau$ whereas the velocity components reads $v_1=u$ and $v_2=v$ in what follows.

Those equations as well as these of Clifton \cite{Clifton} have been numerically integrated numerically, starting from several points lying on the initial yield surface.
The loading functions are odd functions of $\sigma$ and $\tau$ \cite{Clifton} so that the loading paths have axial symmetries.
Hence, the study is restricted to the quarter-plane $\sigma>0,\tau>0$.

\begin{figure}[h!]
  \centering
  \subcaptionbox{Stress path in $(\sigma,\tau)$ plane \label{subfig:tw_fast_stress}}{\begin{tikzpicture}[scale=0.9]
  \begin{axis}[ymajorgrids=true,xmajorgrids=true,ylabel=$\tau \: (Pa)$,xlabel=$\sigma \: (Pa)$,legend style={legend pos=south west}]
    %%
    \addplot[Blue,mark=x,only marks,mark repeat=10,very thick,mark size=3pt] table [x=sigma_11,y=sigma_12] {chapter5/pgfFigures/pgf_thinWalledTubeFastWave/fastStressPlane_Stress.pgf};
    \addlegendentry{Present work}
    \addplot[arrows along my path,Red,thick] table [x=sigma_11,y=sigma_12] {chapter5/pgfFigures/pgf_thinWalledTubeFastWave/TWfastStressPlane_Stress.pgf};
    \addlegendentry{Clifton}
    %% Yield surface
    \addplot[black,dashed] table  [x=sigma_11,y=sigma_12] {chapter5/pgfFigures/pgf_thinWalledTubeSlowWave/TWslow_yield0.pgf};
    \addlegendentry{initial yield surface}
  \end{axis}
\end{tikzpicture}

%%% Local Variables:
%%% mode: latex
%%% TeX-master: "../../mainManuscript"
%%% End:} \qquad
  \subcaptionbox{Stress path in deviatoric plane\label{subfig:tw_fast_dev}}{\tikzset{cross/.style={cross out, draw=black, minimum size=2*(#1-\pgflinewidth), inner sep=0pt, outer sep=0pt},
%default radius will be 1pt. 
cross/.default={2.5pt}}
\begin{tikzpicture}[scale=0.8]
  \begin{axis}[width=.75\textwidth,view={135}{35.2643},xlabel=$s_1 $,
    ylabel=$s_2 $,zlabel=$s_3$,xmin=-1.e8,xmax=1.e8,ymin=-1.e8,ymax=1.e8,axis equal,axis lines=center,axis on top,xtick=\empty,ytick=\empty,ztick=\empty,
    every axis y label/.style={at={(rel axis cs:0.,.5,-0.65)}, anchor=west},
    every axis x label/.style={at={(rel axis cs:0.5,.,-0.65)}, anchor=east},
    every axis z label/.style={at={(rel axis cs:0.,.0,.18)}, anchor=north},
    legend style={at={(1.125,.59)}}
    ]
    \node[below] at (axis cs:1.1e8,0.,0.) {$\sigma^y$};
    \node[above] at (axis cs:-1.1e8,0.,0.) {$-\sigma^y$};
    \draw (axis cs:1.e8,0.,0.) node[cross,rotate=10] {};
    \draw (axis cs:-1.e8,0.,0.) node[cross,rotate=10] {};
    \node[white]  at (axis cs:0,0.,1.42e8) {};
    %%
    \addplot3[black,mark=x,only marks,mark repeat=20,thick,mark size=3pt] file {section7/pgfFigures/pgf_thinWalledTubeFastWave/TWfastDevPlane_Stress.pgf};
    \addplot3[black,arrows along my path,thick] file {section7/pgfFigures/pgf_thinWalledTubeFastWave/fastDevPlane_Stress.pgf};
    \addlegendentry{Clifton}
    \addlegendentry{This work}
    %% Yield surface
    \addplot3[black,dashed] file {section7/pgfFigures/pgf_thinWalledTubeSlowWave/TWCylindreDevPlane.pgf};
    \addlegendentry{Initial yield surface}
    \newcommand\radius{0.82e8}
    \addplot3[dotted,thick] coordinates {(0.75*\radius,-0.75*\radius,0.) (-0.75*\radius,0.75*\radius,0.)};
    \addplot3[dotted,thick] coordinates {(0.,-0.75*\radius,0.75*\radius) (0.,0.75*\radius,-0.75*\radius)};
    \addplot3[dotted,thick] coordinates {(-0.75*\radius,0.,0.75*\radius) (0.75*\radius,0.,-0.75*\radius)};

  \end{axis}
  \begin{scope}[shift={(8.5,0.)}]
    \begin{axis}[width=.75\textwidth,view={135}{35.2643},xlabel=$s_1 $,
    ylabel=$s_2 $,zlabel=$s_3$,xmin=-1.e8,xmax=1.e8,ymin=-1.e8,ymax=1.e8,axis equal,axis lines=center,axis on top,xtick=\empty,ytick=\empty,ztick=\empty,
    every axis y label/.style={at={(rel axis cs:0.,.5,-0.65)}, anchor=west},
    every axis x label/.style={at={(rel axis cs:0.5,.,-0.65)}, anchor=east},
    every axis z label/.style={at={(rel axis cs:0.,.0,.18)}, anchor=north}
    ]
    \node[below] at (axis cs:1.1e8,0.,0.) {$\sigma^y$};
    \node[above] at (axis cs:-1.1e8,0.,0.) {$-\sigma^y$};
    \draw (axis cs:1.e8,0.,0.) node[cross,rotate=10] {};
    \draw (axis cs:-1.e8,0.,0.) node[cross,rotate=10] {};
    \node[white]  at (axis cs:0,0.,1.42e8) {};
    %%
    \addplot3[black,mark=x,only marks,mark repeat=30,thick] file {section7/pgfFigures/pgf_thinWalledTubeSlowWave/TWslowDevPlane_Stress0.pgf};
    \addplot3[black,arrows along my path,thick] file {section7/pgfFigures/pgf_thinWalledTubeSlowWave/slowDevPlane_Stress0.pgf};
    %%
    \addplot3[black,mark=x,only marks,mark repeat=30,thick] file {section7/pgfFigures/pgf_thinWalledTubeSlowWave/TWslowDevPlane_Stress1.pgf};
    \addplot3[black,arrows along my path,thick] file {section7/pgfFigures/pgf_thinWalledTubeSlowWave/slowDevPlane_Stress1.pgf};
    %%
    \addplot3[black,mark=x,only marks,mark repeat=30,thick] file {section7/pgfFigures/pgf_thinWalledTubeSlowWave/TWslowDevPlane_Stress2.pgf};
    \addplot3[black,arrows along my path,thick] file {section7/pgfFigures/pgf_thinWalledTubeSlowWave/slowDevPlane_Stress2.pgf};
    %%
    \addplot3[black,mark=x,only marks,mark repeat=30,thick] file {section7/pgfFigures/pgf_thinWalledTubeSlowWave/TWslowDevPlane_Stress3.pgf};
    \addplot3[black,arrows along my path,thick] file {section7/pgfFigures/pgf_thinWalledTubeSlowWave/slowDevPlane_Stress3.pgf};
    %%
    \addplot3[black,mark=x,only marks,mark repeat=30,thick] file {section7/pgfFigures/pgf_thinWalledTubeSlowWave/TWslowDevPlane_Stress4.pgf};
    \addplot3[black,arrows along my path,thick] file {section7/pgfFigures/pgf_thinWalledTubeSlowWave/slowDevPlane_Stress4.pgf};
    %% 
    \addplot3[black,mark=x,only marks,mark repeat=30,thick] file {section7/pgfFigures/pgf_thinWalledTubeSlowWave/TWslowDevPlane_Stress5.pgf};
    \addplot3[black,arrows along my path,thick] file {section7/pgfFigures/pgf_thinWalledTubeSlowWave/slowDevPlane_Stress5.pgf};
    %% 
    \addplot3[black,mark=x,only marks,mark repeat=5,thick] file {section7/pgfFigures/pgf_thinWalledTubeSlowWave/TWslowDevPlane_Stress6.pgf};
    \addplot3[black,arrows along my path,thick] file {section7/pgfFigures/pgf_thinWalledTubeSlowWave/slowDevPlane_Stress6.pgf};
    %% Yield surface
    \addplot3[black,dashed] file {section7/pgfFigures/pgf_thinWalledTubeSlowWave/TWCylindreDevPlane.pgf};
    \newcommand\radius{0.82e8}
    \addplot3[dotted,thick] coordinates {(0.75*\radius,-0.75*\radius,0.) (-0.75*\radius,0.75*\radius,0.)};
    \addplot3[dotted,thick] coordinates {(0.,-0.75*\radius,0.75*\radius) (0.,0.75*\radius,-0.75*\radius)};
    \addplot3[dotted,thick] coordinates {(-0.75*\radius,0.,0.75*\radius) (0.75*\radius,0.,-0.75*\radius)};

    % \newcommand\radius{0.82e8}
    % \addplot3[dotted,very thick] coordinates {(1.05*\radius,-1.05*\radius,0.) (-1.05*\radius,1.05*\radius,0.)};

  \end{axis}
\end{scope}
\node at (4.95,6.75) {\text{Fast wave}};
\node at (14.35,6.75) {\text{Slow waves}};
\end{tikzpicture}

%%% Local Variables:
%%% mode: latex
%%% TeX-master: "../../presentation"
%%% End:}
  \caption{Stress path followed in a fast simple wave for the thin-walled tube problem. Comparison between the equations of table \ref{tab:simpleWavesEquations} and these of \cite{Clifton}.}
  \label{fig:fast_clifton}
\end{figure}
Figure \ref{fig:fast_clifton} shows one stress path resulting from the integration of the right-going fast wave with $\sigma$ used as a driving parameter.
The initial stress state lies on the yield surface at $\sigma=0$ and the fast wave ODE is discretized by means of backward Euler method, the integration being performed until the stress reaches the value $\sigma=\sigma^y$.
The path is respectively depicted in the stress space and in the deviatoric plane in figures \ref{fig:fast_clifton}\subref{subfig:tw_fast_stress} and \ref{fig:fast_clifton}\subref{subfig:tw_fast_dev}.
The deviatoric plane projection is obtained by drawing the paths in the principal deviatoric stress space and projecting them onto the plane perpendicular to the hydrostatic axis $s_1+s_2+s_3=0$.
In this plane, the von-Mises yield surface is a cricle drawn with dashed lines.
The ODEs derived in section \ref{sec:stress_paths} for plane stress problems thus allow to retrieve the solution originally proposed for thin-walled tubes undergoing combined longitudinal and torsional loading.
%Furthermore, the direction of the path is given by the arrows in figure \ref{sec:stress_paths}.
Furthermore, as observed by Clifton, the path inside fast waves first follows the initial yield surface until the intersection of $\sigma=0$ axis.
Then, the loading path is such that $d\tau=0$ while $\sigma$ increase as far as hyperbolicity holds, that is for $c_f < c_2 = \sqrt{\mu/\rho} $ \cite{Clifton}.

Adopting the same approach with $\tau$ as driving parameter, some stress paths through slow waves have been reported in figure \ref{fig:tw_slow}.
\begin{figure}[h!]
  \centering
  \subcaptionbox{Stress path in $(\sigma,\tau)$ plane \label{subfig:tw_slow_stress}}{\begin{tikzpicture}[scale=0.7]
  \begin{axis}[ymajorgrids=true,xmajorgrids=true,ylabel=$\tau \: (Pa)$,xlabel=$\sigma \: (Pa)$,xmin=-0.1e8,xmax=2.e8,ymin=0.,ymax=7.5e7]
    %%
    \addplot[very thick] table [x=sigma_11,y=sigma_12] {section5/pgfFigures/pgf_thinWalledTubeSlowWave/TWslowStressPlane_Stress0.pgf};
    %%
    \addplot[very thick] table [x=sigma_11,y=sigma_12] {section5/pgfFigures/pgf_thinWalledTubeSlowWave/TWslowStressPlane_Stress1.pgf};
    %%
    \addplot[very thick] table [x=sigma_11,y=sigma_12] {section5/pgfFigures/pgf_thinWalledTubeSlowWave/TWslowStressPlane_Stress2.pgf};
    %%
    \addplot[very thick] table [x=sigma_11,y=sigma_12] {section5/pgfFigures/pgf_thinWalledTubeSlowWave/TWslowStressPlane_Stress3.pgf};
    %%
    \addplot[very thick] table [x=sigma_11,y=sigma_12] {section5/pgfFigures/pgf_thinWalledTubeSlowWave/TWslowStressPlane_Stress4.pgf};
    %%
    \addplot[very thick] table [x=sigma_11,y=sigma_12] {section5/pgfFigures/pgf_thinWalledTubeSlowWave/TWslowStressPlane_Stress5.pgf};
    %%
    \addplot[very thick] table [x=sigma_11,y=sigma_12] {section5/pgfFigures/pgf_thinWalledTubeSlowWave/TWslowStressPlane_Stress6.pgf};
    %% Yield surface
    \addplot[black,dashed] table  [x=sigma_11,y=sigma_12] {section5/pgfFigures/pgf_thinWalledTubeSlowWave/TWslow_yield0.pgf};

    %\addplot[very thick,Orange,restrict y to domain=4.e7:6.75e7] table [x=sigma_11,y=sigma_12]{section5/pgfFigures/pgf_thinWalledTubeSlowWave/TWslowStressPlane_Stress1.pgf};


  \end{axis}
\end{tikzpicture}

%%% Local Variables:
%%% mode: latex
%%% TeX-master: "../../presentation"
%%% End:} \qquad
  \subcaptionbox{Stress path in deviatoric plane \label{subfig:tw_slow_dev}}{\tikzset{cross/.style={cross out, draw=black, minimum size=2*(#1-\pgflinewidth), inner sep=0pt, outer sep=0pt},
%default radius will be 1pt. 
cross/.default={2.5pt}}
\begin{tikzpicture}[scale=0.9]
  \begin{axis}[width=.75\textwidth,view={135}{35.2643},xlabel=$s_1 $,
    ylabel=$s_2 $,zlabel=$s_3$,xmin=-1.e8,xmax=1.e8,ymin=-1.e8,ymax=1.e8,axis equal,axis lines=center,axis on top,xtick=\empty,ytick=\empty,ztick=\empty,
    every axis y label/.style={at={(rel axis cs:0.,.5,-0.65)}, anchor=west},
    every axis x label/.style={at={(rel axis cs:0.5,.,-0.65)}, anchor=east},
    every axis z label/.style={at={(rel axis cs:0.,.0,.18)}, anchor=north}
    ]
    \node[below] at (1.1e8,0.,0.) {$\sigma^y$};
    \node[above] at (-1.1e8,0.,0.) {$-\sigma^y$};
    \draw (1.e8,0.,0.) node[cross,rotate=10] {};
    \draw (-1.e8,0.,0.) node[cross,rotate=10] {};
    \node[white]  at (0,0.,1.42e8) {};
    %%
    \addplot3[Green,mark=x,only marks,mark repeat=20,very thick] file {chapter5/pgfFigures/pgf_thinWalledTubeSlowWave/slowDevPlane_Stress0.pgf};
    \addplot3[Green,thick] file {chapter5/pgfFigures/pgf_thinWalledTubeSlowWave/slowDevPlane_Stress0.pgf};
    %%
    \addplot3[Duck,mark=x,only marks,mark repeat=20,very thick] file {chapter5/pgfFigures/pgf_thinWalledTubeSlowWave/slowDevPlane_Stress1.pgf};
    \addplot3[Duck,thick] file {chapter5/pgfFigures/pgf_thinWalledTubeSlowWave/slowDevPlane_Stress1.pgf};
    %%
    \addplot3[Red,mark=x,only marks,mark repeat=20,very thick] file {chapter5/pgfFigures/pgf_thinWalledTubeSlowWave/slowDevPlane_Stress2.pgf};
    \addplot3[Red,thick] file {chapter5/pgfFigures/pgf_thinWalledTubeSlowWave/slowDevPlane_Stress2.pgf};
    %%
    \addplot3[Purple,mark=x,only marks,mark repeat=20,very thick] file {chapter5/pgfFigures/pgf_thinWalledTubeSlowWave/slowDevPlane_Stress3.pgf};
    \addplot3[Purple,thick] file {chapter5/pgfFigures/pgf_thinWalledTubeSlowWave/slowDevPlane_Stress3.pgf};
    %%
    \addplot3[Blue,mark=x,only marks,mark repeat=20,very thick] file {chapter5/pgfFigures/pgf_thinWalledTubeSlowWave/slowDevPlane_Stress4.pgf};
    \addplot3[Blue,thick] file {chapter5/pgfFigures/pgf_thinWalledTubeSlowWave/slowDevPlane_Stress4.pgf};
    %% 
    \addplot3[Orange,mark=x,only marks,mark repeat=20,very thick] file {chapter5/pgfFigures/pgf_thinWalledTubeSlowWave/slowDevPlane_Stress5.pgf};
    \addplot3[Orange,thick] file {chapter5/pgfFigures/pgf_thinWalledTubeSlowWave/slowDevPlane_Stress5.pgf};
    %% 
    \addplot3[Yellow,mark=x,only marks,mark repeat=5,very thick] file {chapter5/pgfFigures/pgf_thinWalledTubeSlowWave/slowDevPlane_Stress6.pgf};
    \addplot3[Yellow,thick] file {chapter5/pgfFigures/pgf_thinWalledTubeSlowWave/slowDevPlane_Stress6.pgf};
    %% Yield surface
    \addplot3[black,dashed] file {chapter5/pgfFigures/pgf_thinWalledTubeSlowWave/TWCylindreDevPlane.pgf};
    \newcommand\radius{0.82e8}
    \addplot3[dotted,thick] coordinates {(0.75*\radius,-0.75*\radius,0.) (-0.75*\radius,0.75*\radius,0.)};
    \addplot3[dotted,thick] coordinates {(0.,-0.75*\radius,0.75*\radius) (0.,0.75*\radius,-0.75*\radius)};
    \addplot3[dotted,thick] coordinates {(-0.75*\radius,0.,0.75*\radius) (0.75*\radius,0.,-0.75*\radius)};

    % \newcommand\radius{0.82e8}
    % \addplot3[dotted,very thick] coordinates {(1.05*\radius,-1.05*\radius,0.) (-1.05*\radius,1.05*\radius,0.)};

  \end{axis}
\end{tikzpicture}

%%% Local Variables:
%%% mode: latex
%%% TeX-master: "../../mainManuscript"
%%% End:}
  \caption{Stress paths followed in a slow simple wave for the thin-walled tube problem. Comparison between the equations of table \ref{tab:simpleWavesEquations} (cross markers) and these of \cite{Clifton} (solid lines).}
  \label{fig:tw_slow}
\end{figure}
Starting from several stress values along the initial yield surface, the orthogonality of the loading functions leads to stresses moving away from the elastic convex.
Since the stress path in a fast wave follow the yield surface, those of a slow wave are perpendicular to the yield surface in figure \ref{fig:tw_slow}\subref{subfig:tw_slow_stress}.
This is however not the case in the deviatoric plane (\ref{fig:tw_slow}\subref{subfig:tw_slow_dev}).
Furthermore, we see that the initial condition $\sigma=0$ leads to a stress path following the direction of pure shear in the deviatoric plane (horizontal dotted line in figure \ref{fig:tw_slow}\subref{subfig:tw_slow_dev}).

The behavior highlighted above allows the solution of the Picard problem in a thin-walled cylinder, that is:
\begin{itemize}
\item initial conditions $\tens{\sigma}(\vect{x},t=0)=\tens{0}$, $\vect{v}(\vect{x},t=0)=\vect{0}$
\item step-loading boundary conditions $\sigma(x_1=0,t)=\sigma^d$ and $\tau(x_1=0,t)=\tau^d$
\end{itemize}
Indeed, with given $\sigma^d,\tau^d$ outside of the initial yield surface, one can integrate backward the loading path through a simple wave since it is the last, because the slowest, wave that can be met in the solution.
Then, if the integration leads to some point of the initial yield surface, which can be reached by elastic discontinuities, the solution is complete.
Conversely, if the slow wave connects $\sigma^d,\tau^d$ to the $\sigma$-axis at some point lying outside of the initial yield surface, then a fast wave must be integrated backward to the initial elastic convex.
At last, the cases $\tau^d=0$ and $\sigma^d=0$ respectively lead to one single fast wave and one single slow wave.
Once the characteristic structure of the problem has been determined (\textit{i.e. one fast wave, one slow wave, or both}), the complete set of ODEs can be integrated so that a solution is found.
It is worth emphasizing the complexity introduced by waves of combined stress since the characteristic structure of the solution of a Picard problem now depends on the boundary conditions.
Hence, for developing a Riemann solver that would provide the stationary solution, additional computational effort must be made.

Lin and Ballman \cite{Lin_et_Ballman} proposed an iterative procedure to solve Riemann problems with the stress states considered above.
The left and right initial conditions of that problem satisfy equations similar to \eqref{eq:integral_example}:
\begin{subequations}
  \label{eq:lin_ballman}
  \begin{alignat}{1}
    \label{eq:lin_ballman_left}
    & u^* = u^L + \int_{\tens{\sigma}^L}^{\tens{\sigma}^*} \frac{d\sigma}{\rho c} \quad ; \quad v^* = v^L + \int_{\tens{\sigma}^L}^{\tens{\sigma}^*} \frac{d\tau}{\rho c} \\
    \label{eq:lin_ballman_right}
    & u^* = u^R - \int_{\tens{\sigma}^R}^{\tens{\sigma}^*} \frac{d\sigma}{\rho c} \quad ; \quad v^* = v^R - \int_{\tens{\sigma}^R}^{\tens{\sigma}^*} \frac{d\tau}{\rho c}
  \end{alignat}
\end{subequations}
where the asterisk denotes the stationary state of the Riemann problem.
First, a stress state ($\bar{\sigma},\bar{\tau}$) is assumed to be connected to $\tens{\sigma}^L$ and $\tens{\sigma}^R$ (see figure \ref{fig:lin_et_ballman} for the illustration of the method).
\begin{figure}[h!]
  \centering
  % \begin{tikzpicture}[scale=1.5]
  %   \draw[->] (0,0) --(3,0);
  %   \draw[->] (0,0) --(0,3);
  %   \node[below] at (1,0) {$v^L$};
  % \end{tikzpicture}
  \begin{tikzpicture}[scale=1.5]
  %% (u,sigma) plane
  \draw[->,thick] (0,0)-- (3,0) node [right] {$u$};
  \draw[->,thick] (0,0)-- (0,3) node [above] {$\sigma$};
  \fill[black] (0.25,0.3) circle (0.05) ;
  \fill[black] (2.5,0.5) circle (0.05) ;
  \fill[black] (1.,2.8) circle (0.05) ;
  \fill[black] (1.75,2.8) circle (0.05) ;
  %% Left states
  \draw[dotted] (0.25,0.3) -- (0.25,0.) node [below] {$u^L$};
  \draw[dotted] (1.75,2.8) -- (1.75,0) node [below] {$u^1$};
  %% Right states
  \draw[dotted] (2.5,0.5) -- (2.5,0.) node [below] {$u^R$};
  \draw[dotted] (1,2.8) -- (1.,0) node [below] {$u^2$};
  \draw[dotted] (0,2.8) node [left] {$\bar{\sigma}_{11}$} -- (1.75,2.8) ;
  \draw[dashed] (0.25,0.3) .. controls (0.3,0.33) and (1.5,2.8) .. (1.75,2.8);
  \draw[dashed] (2.5,0.5) .. controls (2.,0.5) and (1,2.5) .. (1,2.8);
  %% Intersection of integral curves
  \draw[dotted] (0.,2.125) node [left] {$\widehat{\sigma}$}-- (2.5,2.125);

  %% (v,tau) plane
  \newcommand\shift{6}
  \draw[->,thick] (0+\shift,0)-- (3+\shift,0) node [right] {$v$};
  \draw[->,thick] (0+\shift,0)-- (0+\shift,3) node [above] {$\tau$};
  \fill[black] (0.25+\shift,2.3) circle (0.05) ;
  \fill[black] (1.75+\shift,1.5) circle (0.05) ;
  \fill[black] (2.8+\shift,0.8) circle (0.05) ;
  \fill[black] (1.+\shift,0.8) circle (0.05) ;
  %% Left states
  \draw[dotted] (0.25+\shift,2.3) -- (0.25+\shift,0.) node [below] {$v^L$};
  \draw[dotted] (2.8+\shift,0.8) -- (2.8+\shift,0) node [below] {$v^1$};
  %% Right states
  \draw[dotted] (1.75+\shift,1.5) -- (1.75+\shift,0.) node [below] {$v^R$};
  \draw[dotted] (1.+\shift,0.8) -- (1.+\shift,0.0) node [below] {$v^2$};
  \draw[dotted] (0+\shift,0.8) node [left] {$\bar{\sigma}_{12}$} -- (3.+\shift,0.8) ;
  %% integral curves
  \draw[dashed] (0.25+\shift,2.3) .. controls (0.75+\shift,1.3) and (2.5+\shift,0.8) .. (2.8+\shift,0.8);
  \draw[dashed] (1.75+\shift,1.5) .. controls (1.5+\shift,1.5) and (1+\shift,1.5) .. (1.+\shift,0.8);
  %% Intersection of integral curves
  \draw[dotted] (0.+\shift,1.38) node [left] {$\widehat{\tau}$}-- (1.55+\shift,1.38);
\end{tikzpicture}



%%% Local Variables:
%%% mode: latex
%%% TeX-master: "../../mainManuscript"
%%% End:
 
  \caption{Schematic representation of the iterative Riemann solver proposed in \cite{Lin_et_Ballman}.}
  \label{fig:lin_et_ballman}
\end{figure}
The considerations made above enable to identify the loading paths followed so that equations \eqref{eq:lin_ballman_left} and \eqref{eq:lin_ballman_right} can be integrated in order to determined velocities $\vect{v}^1$ and $\vect{v}^2$.
Thus, virtual integral curves are built in ($u,\sigma$) and ($v,\tau$) planes as depicted with dashed lines in figure \ref{fig:lin_et_ballman}.
Second, the intersection of the curves joining respectively $\vect{v}^L$ to $\vect{v}^1$ and $\vect{v}^R$ to $\vect{v}^2$ gives a stress state ($\widehat{\sigma},\widehat{\tau}$) that is used to apply the procedure again until some criterion $\norm{\vect{v}^1-\vect{v}^2}\leq \epsilon $ is achieved.
At last, the sate obtained $(\widehat{\vect{v}},\widehat{\tens{\sigma}})$ corresponds to the stationary state of the Riemann problem and can be used to compute numerical fluxes.
Notice that in this procedure, the intersection of integral curves is found by means of the tangent lines approximation so that this solver does not fully account for the exact solution.

\subsection{Plane stress}
We now move on to a more general plane stress case for which the stress $\sigma_{22} $ is not zero.
Although the equations of section \ref{sec:stress_paths} have been derived for two directions of propagation $\vect{n}=\vect{e}_1$ and $\vect{n}=\vect{e}_2 $, attention is paid here to the first one only.
Indeed, it has been seen that similar properties of the loading paths inside the simple waves hold for both directions.

One path through a fast simple wave is first looked at by assuming an initially free-stress state, brought to the yield surface at the point $ \sigma_{11}=\sigma_{22}=0 $.
The ODEs of table \ref{tab:simpleWavesEquations} are thus integrated implicitly with $\sigma_{12}$ as driving parameter by means of the backward Euler algorithm, until the shear component $\sigma_{12}$ vanishes.
Two situations are considered for which the stress $\sigma_{11}$ increases or decreases.
The resulting loading paths are depicted in figure \ref{fig:fast_path_plane_stress}\subref{subfig:CP_fast_stress} in $(\sigma_{11},\sigma_{12})$ and $(\sigma_{22},\sigma_{12})$ planes, while the projection in the deviatoric plane can be seen in figure \ref{fig:fast_path_plane_stress}\subref{subfig:CP_fast_dev}.
In addition, figure \ref{fig:fast_path_plane_stress}\subref{subfig:CP_fast_stress} shows the evolution of the characteristic speed associated to the fast wave along the path by means of a colored gradient.
Thus, it can be seen that for the loadings under consideration, the waves celerities are decreasing functions of the stress so that the simple wave solutions are valid.
Second, it appears that the paths have axial symmetry, although the property has not been shown mathematically.
At last, analagously to the thin-walled cylinder solution, the stress components follow the initial yield surface, which is obvious in the deviatoric plane (figure \ref{fig:fast_path_plane_stress}\subref{subfig:CP_fast_dev}).
Furthermore, according to the property \eqref{eq:CP_roots} the stress path must be horizontal in the $(\sigma_{11},\sigma_{12})$ and $(\sigma_{22},\sigma_{12})$ planes, once the intersection of with the $\sigma_{11}$-axis is reached.
As depicted in figure \ref{fig:fast_path_plane_stress}\subref{subfig:CP_fast_dev}, this point correspond to a direction of pure shear in the deviatoric plane.
%As a result, the plastic flow becomes significant once a direction of pure shear is reached.
Nevertheless, the numerical integration of ODEs once the shear stress $\sigma_{12}$ vanishes is not possible owing to an indeterminacy of the loading function $\psi_1^f$ that has not been identified so far.
%condition \eqref{eq:CP_roots} yields $\psi^f_1 \rightarrow \infty$ and associated numerical issues so that the loading path cannot be integrated further.
%The integration performed here nevertheless stops once $\sigma_{12}=0$ due to a singularity that has not been identified and leads to an undefined loading function $\psi_1^f$.
\begin{figure}[h!]
  \centering
  \subcaptionbox{Projections of loading paths in ($\sigma_{11},\sigma_{12}$) and ($\sigma_{22},\sigma_{12}$) planes \label{subfig:CP_fast_stress}}{\begin{tikzpicture}[scale=0.9]
\begin{groupplot}[group style={group size=2 by 1,
ylabels at=edge left, yticklabels at=edge left,horizontal sep=3.ex,
xticklabels at=edge bottom,xlabels at=edge bottom},
ymajorgrids=true,xmajorgrids=true,ylabel=$\sigma_{12} \: (Pa)$,
axis on top,scale only axis,width=0.4\linewidth,ymin=0,ymax=63499406.78820015
, every x tick scale label/.style={at={(xticklabel* cs:1.05,0.75cm)},anchor=near yticklabel},colormap name=viridis]
\nextgroupplot[xlabel=$\sigma_{11} (Pa)$]
\addplot[arrows along my path,black,thick] table[x=sigma_11,y=sigma_12] {chapter5/pgfFigures/pgf_fastWavesPlaneStress/CPfastStressPlane_frame0_Stress0.pgf};\addplot[mesh,point meta = \thisrow{p},very thick,no markers] table[x=sigma_11,y=sigma_12] {chapter5/pgfFigures/pgf_fastWavesPlaneStress/CPfastStressPlane_frame0_Stress0.pgf} node[above right,black] {$\textbf{1}$};
\addplot[arrows along my path,black,thick] table[x=sigma_11,y=sigma_12] {chapter5/pgfFigures/pgf_fastWavesPlaneStress/CPfastStressPlane_frame1_Stress0.pgf};\addplot[mesh,point meta = \thisrow{p},very thick,no markers] table[x=sigma_11,y=sigma_12] {chapter5/pgfFigures/pgf_fastWavesPlaneStress/CPfastStressPlane_frame1_Stress0.pgf} node[above right,black] {$\textbf{2}$};
\addplot[gray,dashed,thin] table[x=sigma_11,y=sigma_12] {chapter5/pgfFigures/pgf_fastWavesPlaneStress/CPfast_yield0_s11s12_Stress0.pgf};

\nextgroupplot[colorbar,colorbar style={title= {$c_f \: (m/s)$},every y tick scale label/.style={at={(2.,-.1125)}} },xlabel=$\sigma_{22}  (Pa)$]
\addplot[arrows along my path,black,thick] table[x=sigma_22,y=sigma_12] {chapter5/pgfFigures/pgf_fastWavesPlaneStress/CPfastStressPlane_frame0_Stress0.pgf};\addplot[mesh,point meta = \thisrow{p},very thick,no markers] table[x=sigma_22,y=sigma_12] {chapter5/pgfFigures/pgf_fastWavesPlaneStress/CPfastStressPlane_frame0_Stress0.pgf} node[above right,black] {$\textbf{1}$};
\addplot[arrows along my path,black,thick] table[x=sigma_22,y=sigma_12] {chapter5/pgfFigures/pgf_fastWavesPlaneStress/CPfastStressPlane_frame1_Stress0.pgf};\addplot[mesh,point meta = \thisrow{p},very thick,no markers] table[x=sigma_22,y=sigma_12] {chapter5/pgfFigures/pgf_fastWavesPlaneStress/CPfastStressPlane_frame1_Stress0.pgf} node[above right,black] {$\textbf{2}$};
\end{groupplot}
\end{tikzpicture}
%%% Local Variables:
%%% mode: latex
%%% TeX-master: "../../mainManuscript"
%%% End:
}
  \subcaptionbox{Loading paths in deviatoric plane \label{subfig:CP_fast_dev}}{\tikzset{cross/.style={cross out, draw=black, minimum size=2*(#1-\pgflinewidth), inner sep=0pt, outer sep=0pt},cross/.default={2.5pt}}
\begin{tikzpicture}[scale=0.9]
\begin{axis}[width=.75\textwidth,view={135}{35.2643},xlabel=$s_1 $,ylabel=$s_2 $,zlabel=$s_3$,xmin=-1.e8,xmax=1.e8,ymin=-1.e8,ymax=1.e8,axis equal,axis lines=center,axis on top,xtick=\empty,ytick=\empty,ztick=\empty,every axis y label/.style={at={(rel axis cs:0.,.5,-0.65)}, anchor=west}, every axis x label/.style={at={(rel axis cs:0.5,.,-0.65)}, anchor=east}, every axis z label/.style={at={(rel axis cs:0.,.0,.18)}, anchor=north},legend style={at={(.225,.59)}}]
\node[below] at (1.1e8,0.,0.) {$\sigma^y$};
\node[above] at (-1.1e8,0.,0.) {$-\sigma^y$};
\draw (1.e8,0.,0.) node[cross,rotate=10] {};
\draw (-1.e8,0.,0.) node[cross,rotate=10] {};
\node[white]  at (0,0.,1.1e8) {};
\addplot3[gray,dashed,thin,no markers] file {chapter5/pgfFigures/pgf_fastWavesPlaneStress/CPCylindreDevPlane.pgf};\addlegendentry{initial yield surface}
%\addplot3[Red,mark=star,mark repeat=20,mark size=3pt,very thick] file {chapter5/pgfFigures/pgf_fastWavesPlaneStress/CPfastDevPlane_frame0_Stress0.pgf};
\addplot3[arrows along my path,Red,very thick] file {chapter5/pgfFigures/pgf_fastWavesPlaneStress/CPfastDevPlane_frame0_Stress0.pgf};
\addlegendentry{loading path 1}
%\addplot3[Blue,mark=asterisk,mark repeat=20,mark size=3pt,very thick] file {chapter5/pgfFigures/pgf_fastWavesPlaneStress/CPfastDevPlane_frame1_Stress0.pgf};
\addplot3[arrows along my path,Blue,very thick] file {chapter5/pgfFigures/pgf_fastWavesPlaneStress/CPfastDevPlane_frame1_Stress0.pgf};
\addlegendentry{loading path 2}
\newcommand\radius{1.*0.82e8}
\addplot3[dotted,thick] coordinates {(0.75*\radius,-0.75*\radius,0.) (-0.75*\radius,0.75*\radius,0.)};
\addplot3[dotted,thick] coordinates {(0.,-0.75*\radius,0.75*\radius) (0.,0.75*\radius,-0.75*\radius)};
\addplot3[dotted,thick] coordinates {(-0.75*\radius,0.,0.75*\radius) (0.75*\radius,0.,-0.75*\radius)};
\end{axis}
\end{tikzpicture}
%%% Local Variables:
%%% mode: latex
%%% TeX-master: "../../mainManuscript"
%%% End:
}
  \caption{Loading paths through a fast simple wave starting from the initial yield surface with initial condition $\sigma_{11}=\sigma_{22}=0$ in directions of decreasing and increasing $\sigma_{11}$.}
  \label{fig:fast_path_plane_stress}
\end{figure}

%% integration jusqu'à \tau=2\sigma^y pour \sigma_{22}=\{\pm 57735026.919,0\}
We now focus on the stress evolution in slow waves.
The same procedure is followed for several starting points on the initial yield surface.
In addition, various initial values are considered for $\sigma_{22}$ since, even for a solid in a free stress state at $t=0$, a fast wave may lead, as seen above, to $\sigma_{22}\neq 0$.
The loading paths thus obtained for the arbitrary initial values $\sigma_{22}=-5.8\times 10^7 \: Pa$, $\sigma_{22}=0$ and $\sigma_{22}=5.8\times 10^7 \: Pa$, are respecticely depicted in figures \ref{fig:slow_path_plane_stress1}, \ref{fig:slow_path_plane_stress2} and \ref{fig:slow_path_plane_stress3}.
The projections in the stress space and the deviatoric plane are shown.
\begin{figure}[h!]
  \centering
  \subcaptionbox{Projections of loading paths in ($\sigma_{11},\sigma_{12}$) and ($\sigma_{22},\sigma_{12}$) planes \label{subfig:CP_slow_stress1}}{\begin{tikzpicture}[scale=0.9]
\begin{groupplot}[group style={group size=2 by 1,
ylabels at=edge left, yticklabels at=edge left,horizontal sep=3.ex,
xticklabels at=edge bottom,xlabels at=edge bottom},
ymajorgrids=true,xmajorgrids=true,ylabel=$\sigma_{12} \: (Pa)$,
axis on top,scale only axis,width=0.4\linewidth,ymin=0,ymax=109528891.78848386
, every x tick scale label/.style={at={(xticklabel* cs:1.05,0.75cm)},anchor=near yticklabel},colormap name=viridis]
, every x tick scale label/.style={at={(xticklabel* cs:1.05,0.75cm)},anchor=near yticklabel},colormap name=viridis]
\nextgroupplot[xlabel=$\sigma_{11} \: (Pa)$]
\addplot[arrows along my path,black,thick] table[x=sigma_11,y=sigma_12] {chapter5/pgfFigures/pgf_slowWavesPlaneStress/CPslowStressPlane_frame0_Stress1.pgf};
\addplot[mesh,point meta = \thisrow{p},very thick,no markers] table[x=sigma_11,y=sigma_12] {chapter5/pgfFigures/pgf_slowWavesPlaneStress/CPslowStressPlane_frame0_Stress1.pgf} node[above right,black] {$\textbf{1}$};
\addplot[arrows along my path,black,thick] table[x=sigma_11,y=sigma_12] {chapter5/pgfFigures/pgf_slowWavesPlaneStress/CPslowStressPlane_frame1_Stress1.pgf};
\addplot[mesh,point meta = \thisrow{p},very thick,no markers] table[x=sigma_11,y=sigma_12] {chapter5/pgfFigures/pgf_slowWavesPlaneStress/CPslowStressPlane_frame1_Stress1.pgf} node[above right,black] {$\textbf{2}$};
\addplot[arrows along my path,black,thick] table[x=sigma_11,y=sigma_12] {chapter5/pgfFigures/pgf_slowWavesPlaneStress/CPslowStressPlane_frame2_Stress1.pgf};
\addplot[mesh,point meta = \thisrow{p},very thick,no markers] table[x=sigma_11,y=sigma_12] {chapter5/pgfFigures/pgf_slowWavesPlaneStress/CPslowStressPlane_frame2_Stress1.pgf} node[above right,black] {$\textbf{3}$};
\addplot[arrows along my path,black,thick] table[x=sigma_11,y=sigma_12] {chapter5/pgfFigures/pgf_slowWavesPlaneStress/CPslowStressPlane_frame3_Stress1.pgf};
\addplot[mesh,point meta = \thisrow{p},very thick,no markers] table[x=sigma_11,y=sigma_12] {chapter5/pgfFigures/pgf_slowWavesPlaneStress/CPslowStressPlane_frame3_Stress1.pgf} node[above right,black] {$\textbf{4}$};
\addplot[gray,dashed,thin] table[x=sigma_11,y=sigma_12] {chapter5/pgfFigures/pgf_slowWavesPlaneStress/CPslow_yield0_s11s12_Stress1.pgf};

\nextgroupplot[colorbar,colorbar style={title= {$ c_s \: (m/s)$},every y tick scale label/.style={at={(2.,-.1125)}} },xlabel=$\sigma_{22} \: (Pa)$]
\addplot[arrows along my path,black,thick] table[x=sigma_22,y=sigma_12] {chapter5/pgfFigures/pgf_slowWavesPlaneStress/CPslowStressPlane_frame0_Stress1.pgf};
\addplot[mesh,point meta = \thisrow{p},very thick,no markers] table[x=sigma_22,y=sigma_12] {chapter5/pgfFigures/pgf_slowWavesPlaneStress/CPslowStressPlane_frame0_Stress1.pgf} node[above right,black] {$\textbf{1}$};
\addplot[arrows along my path,black,thick] table[x=sigma_22,y=sigma_12] {chapter5/pgfFigures/pgf_slowWavesPlaneStress/CPslowStressPlane_frame1_Stress1.pgf};
\addplot[mesh,point meta = \thisrow{p},very thick,no markers] table[x=sigma_22,y=sigma_12] {chapter5/pgfFigures/pgf_slowWavesPlaneStress/CPslowStressPlane_frame1_Stress1.pgf} node[above right,black] {$\textbf{2}$};
\addplot[arrows along my path,black,thick] table[x=sigma_22,y=sigma_12] {chapter5/pgfFigures/pgf_slowWavesPlaneStress/CPslowStressPlane_frame2_Stress1.pgf};
\addplot[mesh,point meta = \thisrow{p},very thick,no markers] table[x=sigma_22,y=sigma_12] {chapter5/pgfFigures/pgf_slowWavesPlaneStress/CPslowStressPlane_frame2_Stress1.pgf} node[above right,black] {$\textbf{3}$};
\addplot[arrows along my path,black,thick] table[x=sigma_22,y=sigma_12] {chapter5/pgfFigures/pgf_slowWavesPlaneStress/CPslowStressPlane_frame3_Stress1.pgf};
\addplot[mesh,point meta = \thisrow{p},very thick,no markers] table[x=sigma_22,y=sigma_12] {chapter5/pgfFigures/pgf_slowWavesPlaneStress/CPslowStressPlane_frame3_Stress1.pgf} node[above right,black] {$\textbf{4}$};
\end{groupplot}
\end{tikzpicture}
%%% Local Variables:
%%% mode: latex
%%% TeX-master: "../../mainManuscript"
%%% End:
}
  \subcaptionbox{Loading paths in deviatoric plane \label{subfig:CP_slow_dev1}}{\begin{tikzpicture}[scale=0.9]
\begin{axis}[width=.75\textwidth,view={135}{35.2643},xlabel=$s_1 $,ylabel=$s_2 $,zlabel=$s_3$,xmin=-1.e8,xmax=1.e8,ymin=-1.e8,ymax=1.e8,axis equal,axis lines=center,axis on top,ztick=\empty,legend style={at={(.225,.59)}}]
\addplot3+[Red,mark=star,mark repeat=20,mark size=3pt,very thick] file {chapter5/pgfFigures/pgf_slowWavesPlaneStress/CPslowDevPlane_frame0_Stress1.pgf};
\addlegendentry{loading path 1}
\addplot3+[Blue,mark=asterisk,mark repeat=20,mark size=3pt,very thick] file {chapter5/pgfFigures/pgf_slowWavesPlaneStress/CPslowDevPlane_frame1_Stress1.pgf};
\addlegendentry{loading path 2}
\addplot3+[Orange,mark=+,mark repeat=20,mark size=3pt,very thick] file {chapter5/pgfFigures/pgf_slowWavesPlaneStress/CPslowDevPlane_frame2_Stress1.pgf};
\addlegendentry{loading path 3}
\addplot3+[Purple,mark=x,mark repeat=20,mark size=3pt,very thick] file {chapter5/pgfFigures/pgf_slowWavesPlaneStress/CPslowDevPlane_frame3_Stress1.pgf};
\addlegendentry{loading path 4}
\addplot3+[gray,dashed,thin,no markers] file {chapter5/pgfFigures/pgf_slowWavesPlaneStress/CPCylindreDevPlane.pgf};\addlegendentry{initial yield surface}
\end{axis}
\end{tikzpicture}
%%% Local Variables:
%%% mode: latex
%%% TeX-master: "../../mainManuscript"
%%% End:
}
  \caption{Stress paths in a slow simple wave for various starting point lying on the initial yield surface for $\sigma_{22}=-5.8\times 10^7 \: Pa$. Projections in the stress space (figure \subref{subfig:CP_slow_stress2})  and deviatoric plane (figure \subref{subfig:CP_slow_dev1}).}
  \label{fig:slow_path_plane_stress1}
\end{figure}
The evolution of the characteristic speed associated to the slow wave is also depicted by means of a color gradient.
Once again the simple wave solution appears to be valid with the considered loading conditions.
One can see that the stress paths are more complex in those case.
For instance, for a negative initial value of the stress $\sigma_{22}$ (figure \ref{fig:slow_path_plane_stress1}), the projection in the $(\sigma_{11},\sigma_{12})$ plane does not allow to identify some symmetry.
Moreover, the behavior is even more complex in the $(\sigma_{22},\sigma_{12})$ plane where first, the variations rather concern $\sigma_{22}$ and next, the slopes of curves roughly change so that the path are almost vertical.
This sharp change in slopes is also notable in the deviatoric plane in figure \ref{fig:slow_path_plane_stress1}\subref{subfig:CP_slow_dev1}.
%However, this figure does not help in finding some interpretations. 

On the other hand, the observations made above no longer hold if the initial stress $\sigma_{22}$ is zero.
Indeed, figure \ref{fig:slow_path_plane_stress2}\subref{subfig:CP_slow_stress2} shows that in this case, $\sigma_{22}$ is identically null.
Hence, the results in figure \ref{fig:slow_path_plane_stress2} are these of the thin-walled cylinder problem already met in figure \ref{fig:tw_slow}.
No break in slopes are then visible in stress plane as well as in the deviatoric plane and the paths are symmetrical in the $(\sigma_{11},\sigma_{12})$ plane.
At last, the values taken by the characteristic speed $c_s$ are much lower for $\sigma_{22}=0$ than for the results depicted in figures \ref{fig:slow_path_plane_stress1} and \ref{fig:slow_path_plane_stress3}.
Since the magnitudes of the initial stresses $\sigma_{11}$ and $\sigma_{12}$ are quite similar for the three simulations, namely those of figures \ref{fig:slow_path_plane_stress1}, \ref{fig:slow_path_plane_stress2} and \ref{fig:slow_path_plane_stress3}, this indicates that $\sigma_{22}$ has a great influence on the slow wave celerity.
\begin{figure}[h!]
  \centering
  \subcaptionbox{Projections of loading paths in ($\sigma_{11},\sigma_{12}$) and ($\sigma_{22},\sigma_{12}$) planes \label{subfig:CP_slow_stress2}}{\begin{tikzpicture}[scale=0.9]
\begin{groupplot}[group style={group size=2 by 1,
ylabels at=edge left, yticklabels at=edge left,horizontal sep=3.ex,
xticklabels at=edge bottom,xlabels at=edge bottom},
ymajorgrids=true,xmajorgrids=true,ylabel=$\sigma_{12} \: (Pa)$,
axis on top,scale only axis,width=0.4\linewidth,ymin=0,ymax=126473070.316
, every x tick scale label/.style={at={(xticklabel* cs:1.05,0.75cm)},anchor=near yticklabel}
,colormap name =viridis]
\nextgroupplot[xlabel=$\sigma_{11} \: (Pa)$]
%\addplot[arrows along my path,black,thick] table[x=sigma_11,y=sigma_12] {chapter5/pgfFigures/pgf_slowWavesPlaneStress/CPslowStressPlane_frame0_Stress2.pgf};
\addplot[mesh,point meta = \thisrow{p},very thick,no markers] table[x=sigma_11,y=sigma_12] {chapter5/pgfFigures/pgf_slowWavesPlaneStress/CPslowStressPlane_frame0_Stress2.pgf} node[above,black] {$\textbf{1}$};
%\addplot[arrows along my path,black,thick] table[x=sigma_11,y=sigma_12] {chapter5/pgfFigures/pgf_slowWavesPlaneStress/CPslowStressPlane_frame1_Stress2.pgf};
\addplot[mesh,point meta = \thisrow{p},very thick,no markers] table[x=sigma_11,y=sigma_12] {chapter5/pgfFigures/pgf_slowWavesPlaneStress/CPslowStressPlane_frame1_Stress2.pgf} node[above,black] {$\textbf{2}$};
%\addplot[arrows along my path,black,thick] table[x=sigma_11,y=sigma_12] {chapter5/pgfFigures/pgf_slowWavesPlaneStress/CPslowStressPlane_frame2_Stress2.pgf};
\addplot[mesh,point meta = \thisrow{p},very thick,no markers] table[x=sigma_11,y=sigma_12] {chapter5/pgfFigures/pgf_slowWavesPlaneStress/CPslowStressPlane_frame2_Stress2.pgf} node[above,black] {$\textbf{3}$};
%\addplot[arrows along my path,black,thick] table[x=sigma_11,y=sigma_12] {chapter5/pgfFigures/pgf_slowWavesPlaneStress/CPslowStressPlane_frame3_Stress2.pgf};
\addplot[mesh,point meta = \thisrow{p},very thick,no markers] table[x=sigma_11,y=sigma_12] {chapter5/pgfFigures/pgf_slowWavesPlaneStress/CPslowStressPlane_frame3_Stress2.pgf} node[above,black] {$\textbf{4}$};
\addplot[gray,dashed,thin] table[x=sigma_11,y=sigma_12] {chapter5/pgfFigures/pgf_slowWavesPlaneStress/CPslow_yield0_s11s12_Stress2.pgf};

\nextgroupplot[colorbar,colorbar style={title= {$ c_s \: (m/s)$},every y tick scale label/.style={at={(2.,-.1125)}} },xlabel=$\sigma_{22} \: (Pa)$]
\addplot[arrows along my path,black!70,thick] table[x=sigma_22,y=sigma_12] {chapter5/pgfFigures/pgf_slowWavesPlaneStress/CPslowStressPlane_frame0_Stress2.pgf};\addplot[mesh,point meta = \thisrow{p},very thick,no markers] table[x=sigma_22,y=sigma_12] {chapter5/pgfFigures/pgf_slowWavesPlaneStress/CPslowStressPlane_frame0_Stress2.pgf};
\addplot[arrows along my path,black!70,thick] table[x=sigma_22,y=sigma_12] {chapter5/pgfFigures/pgf_slowWavesPlaneStress/CPslowStressPlane_frame1_Stress2.pgf};\addplot[mesh,point meta = \thisrow{p},very thick,no markers] table[x=sigma_22,y=sigma_12] {chapter5/pgfFigures/pgf_slowWavesPlaneStress/CPslowStressPlane_frame1_Stress2.pgf} ;
\addplot[arrows along my path,black!70,thick] table[x=sigma_22,y=sigma_12] {chapter5/pgfFigures/pgf_slowWavesPlaneStress/CPslowStressPlane_frame2_Stress2.pgf};\addplot[mesh,point meta = \thisrow{p},very thick,no markers] table[x=sigma_22,y=sigma_12] {chapter5/pgfFigures/pgf_slowWavesPlaneStress/CPslowStressPlane_frame2_Stress2.pgf} ;
\addplot[arrows along my path,black!70,thick] table[x=sigma_22,y=sigma_12] {chapter5/pgfFigures/pgf_slowWavesPlaneStress/CPslowStressPlane_frame3_Stress2.pgf};\addplot[mesh,point meta = \thisrow{p},very thick,no markers] table[x=sigma_22,y=sigma_12] {chapter5/pgfFigures/pgf_slowWavesPlaneStress/CPslowStressPlane_frame3_Stress2.pgf} ;
\end{groupplot}
\end{tikzpicture}
%%% Local Variables:
%%% mode: latex
%%% TeX-master: "../../mainManuscript"
%%% End:
}
  \subcaptionbox{Loading paths in deviatoric plane  \label{subfig:CP_slow_dev2}}{\tikzset{cross/.style={cross out, draw=black, minimum size=2*(#1-\pgflinewidth), inner sep=0pt, outer sep=0pt},cross/.default={2.5pt}}
\begin{tikzpicture}[scale=0.9]
\begin{axis}[width=.75\textwidth,view={135}{35.2643},xlabel=$s_1 $,ylabel=$s_2 $,zlabel=$s_3$,xmin=-1.e8,xmax=1.e8,ymin=-1.e8,ymax=1.e8,axis equal,axis lines=center,axis on top,xtick=\empty,ytick=\empty,ztick=\empty,every axis y label/.style={at={(rel axis cs:0.,.5,-0.65)}, anchor=west}, every axis x label/.style={at={(rel axis cs:0.5,.,-0.65)}, anchor=east}, every axis z label/.style={at={(rel axis cs:0.,.0,.18)}, anchor=north},legend style={at={(.225,.59)}}]
\node[below] at (1.1e8,0.,0.) {$\sigma^y$};
\node[above] at (-1.1e8,0.,0.) {$-\sigma^y$};
\draw (1.e8,0.,0.) node[cross,rotate=10] {};
\draw (-1.e8,0.,0.) node[cross,rotate=10] {};
\node[white]  at (0,0.,1.42e8) {};
\addplot3+[Red,mark=star,mark repeat=20,mark size=3pt,very thick] file {chapter5/pgfFigures/pgf_slowWavesPlaneStress/CPslowDevPlane_frame0_Stress2.pgf};
\addlegendentry{loading path 1}
\addplot3+[Blue,mark=asterisk,mark repeat=20,mark size=3pt,very thick] file {chapter5/pgfFigures/pgf_slowWavesPlaneStress/CPslowDevPlane_frame1_Stress2.pgf};
\addlegendentry{loading path 2}
\addplot3+[Orange,mark=+,mark repeat=20,mark size=3pt,very thick] file {chapter5/pgfFigures/pgf_slowWavesPlaneStress/CPslowDevPlane_frame2_Stress2.pgf};
\addlegendentry{loading path 3}
\addplot3+[Purple,mark=x,mark repeat=20,mark size=3pt,very thick] file {chapter5/pgfFigures/pgf_slowWavesPlaneStress/CPslowDevPlane_frame3_Stress2.pgf};
\addlegendentry{loading path 4}
\addplot3+[gray,dashed,thin,no markers] file {chapter5/pgfFigures/pgf_slowWavesPlaneStress/CPCylindreDevPlane.pgf};\addlegendentry{initial yield surface}
\newcommand\radius{0.82e8}
\addplot3[dotted,thick] coordinates {(0.75*\radius,-0.75*\radius,0.) (-0.75*\radius,0.75*\radius,0.)};
\addplot3[dotted,thick] coordinates {(0.,-0.75*\radius,0.75*\radius) (0.,0.75*\radius,-0.75*\radius)};
\addplot3[dotted,thick] coordinates {(-0.75*\radius,0.,0.75*\radius) (0.75*\radius,0.,-0.75*\radius)};
\end{axis}
\end{tikzpicture}
%%% Local Variables:
%%% mode: latex
%%% TeX-master: "../../mainManuscript"
%%% End:
}
  \caption{Stress paths in a slow simple wave for various starting point lying on the initial yield surface for $\sigma_{22}=0$. Projections in the stress space (figure \subref{subfig:CP_slow_stress2}) and deviatoric plane (figure \subref{subfig:CP_slow_dev2}).}
  \label{fig:slow_path_plane_stress2}
\end{figure}

The last set of results obtained with the initial data $\sigma_{22}<0$, which can be see in figure \ref{fig:slow_path_plane_stress3}, shows a similar behavior to that of figure \ref{fig:slow_path_plane_stress1}.
However, the paths now follow a direction opposite to the previous case in the $(\sigma_{22},\sigma_{12})$ plane, since $\sigma_{22}$ decreases rather than increases before the slopes of paths break (see figure \ref{fig:slow_path_plane_stress3}\subref{subfig:CP_slow_stress3}).
The previous remark is also valid in the deviatoric plane in figure \ref{fig:slow_path_plane_stress3}\subref{subfig:CP_slow_dev3}.
Indeed, the integral curves first describe clock-wise curved lines until the break in slopes occurs, after which a behavior closed to straight lines is seen.
\begin{figure}[h!]
  \centering
  \subcaptionbox{Projections of loading paths in ($\sigma_{11},\sigma_{12}$) and ($\sigma_{22},\sigma_{12}$) planes \label{subfig:CP_slow_stress3}}{\begin{tikzpicture}[scale=0.9]
\begin{groupplot}[group style={group size=2 by 1,
ylabels at=edge left, yticklabels at=edge left,horizontal sep=3.ex,
xticklabels at=edge bottom,xlabels at=edge bottom},
ymajorgrids=true,xmajorgrids=true,ylabel=$\sigma_{12} \: (Pa)$,
axis on top,scale only axis,width=0.4\linewidth,ymin=0,ymax=109528891.788
, every x tick scale label/.style={at={(xticklabel* cs:1.05,0.75cm)},anchor=near yticklabel}
,colormap name =viridis]
\nextgroupplot[xlabel=$\sigma_{11} \: (Pa)$]
%\addplot[arrows along my path,black,thick] table[x=sigma_11,y=sigma_12] {chapter5/pgfFigures/pgf_slowWavesPlaneStress/CPslowStressPlane_frame0_Stress3.pgf};
\addplot[mesh,point meta = \thisrow{p},very thick,no markers] table[x=sigma_11,y=sigma_12] {chapter5/pgfFigures/pgf_slowWavesPlaneStress/CPslowStressPlane_frame0_Stress3.pgf} node[above,black] {$\textbf{1}$};
%\addplot[arrows along my path,black,thick] table[x=sigma_11,y=sigma_12] {chapter5/pgfFigures/pgf_slowWavesPlaneStress/CPslowStressPlane_frame1_Stress3.pgf};
\addplot[mesh,point meta = \thisrow{p},very thick,no markers] table[x=sigma_11,y=sigma_12] {chapter5/pgfFigures/pgf_slowWavesPlaneStress/CPslowStressPlane_frame1_Stress3.pgf} node[above,black] {$\textbf{2}$};
%\addplot[arrows along my path,black,thick] table[x=sigma_11,y=sigma_12] {chapter5/pgfFigures/pgf_slowWavesPlaneStress/CPslowStressPlane_frame2_Stress3.pgf};
\addplot[mesh,point meta = \thisrow{p},very thick,no markers] table[x=sigma_11,y=sigma_12] {chapter5/pgfFigures/pgf_slowWavesPlaneStress/CPslowStressPlane_frame2_Stress3.pgf} node[above,black] {$\textbf{3}$};
%\addplot[arrows along my path,black,thick] table[x=sigma_11,y=sigma_12] {chapter5/pgfFigures/pgf_slowWavesPlaneStress/CPslowStressPlane_frame3_Stress3.pgf};
\addplot[mesh,point meta = \thisrow{p},very thick,no markers] table[x=sigma_11,y=sigma_12] {chapter5/pgfFigures/pgf_slowWavesPlaneStress/CPslowStressPlane_frame3_Stress3.pgf} node[above,black] {$\textbf{4}$};
\addplot[gray,dashed,thin] table[x=sigma_11,y=sigma_12] {chapter5/pgfFigures/pgf_slowWavesPlaneStress/CPslow_yield0_s11s12_Stress3.pgf};

\nextgroupplot[colorbar,colorbar style={title= {$ c_s \: (m/s)$},every y tick scale label/.style={at={(2.,-.1125)}} },xlabel=$\sigma_{22} \: (Pa)$]
\addplot[arrows along my path,black!70,thick] table[x=sigma_22,y=sigma_12] {chapter5/pgfFigures/pgf_slowWavesPlaneStress/CPslowStressPlane_frame0_Stress3.pgf};\addplot[mesh,point meta = \thisrow{p},very thick,no markers] table[x=sigma_22,y=sigma_12] {chapter5/pgfFigures/pgf_slowWavesPlaneStress/CPslowStressPlane_frame0_Stress3.pgf} node[above,black] {$\textbf{1}$};
\addplot[arrows along my path,black!70,thick] table[x=sigma_22,y=sigma_12] {chapter5/pgfFigures/pgf_slowWavesPlaneStress/CPslowStressPlane_frame1_Stress3.pgf};\addplot[mesh,point meta = \thisrow{p},very thick,no markers] table[x=sigma_22,y=sigma_12] {chapter5/pgfFigures/pgf_slowWavesPlaneStress/CPslowStressPlane_frame1_Stress3.pgf} node[above,black] {$\textbf{2}$};
\addplot[arrows along my path,black!70,thick] table[x=sigma_22,y=sigma_12] {chapter5/pgfFigures/pgf_slowWavesPlaneStress/CPslowStressPlane_frame2_Stress3.pgf};\addplot[mesh,point meta = \thisrow{p},very thick,no markers] table[x=sigma_22,y=sigma_12] {chapter5/pgfFigures/pgf_slowWavesPlaneStress/CPslowStressPlane_frame2_Stress3.pgf} node[above,black] {$\textbf{3}$};
\addplot[arrows along my path,black!70,thick] table[x=sigma_22,y=sigma_12] {chapter5/pgfFigures/pgf_slowWavesPlaneStress/CPslowStressPlane_frame3_Stress3.pgf};\addplot[mesh,point meta = \thisrow{p},very thick,no markers] table[x=sigma_22,y=sigma_12] {chapter5/pgfFigures/pgf_slowWavesPlaneStress/CPslowStressPlane_frame3_Stress3.pgf} node[above,black] {$\textbf{4}$};
\end{groupplot}
\end{tikzpicture}
%%% Local Variables:
%%% mode: latex
%%% TeX-master: "../../mainManuscript"
%%% End:
}
  \subcaptionbox{Loading paths in deviatoric plane  \label{subfig:CP_slow_dev3}}{\tikzset{cross/.style={cross out, draw=black, minimum size=2*(#1-\pgflinewidth), inner sep=0pt, outer sep=0pt},cross/.default={2.5pt}}
\begin{tikzpicture}[scale=0.9]
\begin{axis}[width=.75\textwidth,view={135}{35.2643},xlabel=$s_1 $,ylabel=$s_2 $,zlabel=$s_3$,xmin=-1.e8,xmax=1.e8,ymin=-1.e8,ymax=1.e8,axis equal,axis lines=center,axis on top,xtick=\empty,ytick=\empty,ztick=\empty,every axis y label/.style={at={(rel axis cs:0.,.5,-0.65)}, anchor=west}, every axis x label/.style={at={(rel axis cs:0.5,.,-0.65)}, anchor=east}, every axis z label/.style={at={(rel axis cs:0.,.0,.18)}, anchor=north},legend style={at={(.225,.59)}}]
\node[below] at (1.1e8,0.,0.) {$\sigma^y$};
\node[above] at (-1.1e8,0.,0.) {$-\sigma^y$};
\draw (1.e8,0.,0.) node[cross,rotate=10] {};
\draw (-1.e8,0.,0.) node[cross,rotate=10] {};
\node[white]  at (0,0.,1.42e8) {};
\addplot3+[Red,mark=star,mark repeat=20,mark size=3pt,very thick] file {chapter5/pgfFigures/pgf_slowWavesPlaneStress/CPslowDevPlane_frame0_Stress3.pgf};
\addlegendentry{loading path 1}
\addplot3+[Blue,mark=asterisk,mark repeat=20,mark size=3pt,very thick] file {chapter5/pgfFigures/pgf_slowWavesPlaneStress/CPslowDevPlane_frame1_Stress3.pgf};
\addlegendentry{loading path 2}
\addplot3+[Orange,mark=+,mark repeat=20,mark size=3pt,very thick] file {chapter5/pgfFigures/pgf_slowWavesPlaneStress/CPslowDevPlane_frame2_Stress3.pgf};
\addlegendentry{loading path 3}
\addplot3+[Purple,mark=x,mark repeat=20,mark size=3pt,very thick] file {chapter5/pgfFigures/pgf_slowWavesPlaneStress/CPslowDevPlane_frame3_Stress3.pgf};
\addlegendentry{loading path 4}
\addplot3+[gray,dashed,thin,no markers] file {chapter5/pgfFigures/pgf_slowWavesPlaneStress/CPCylindreDevPlane.pgf};\addlegendentry{initial yield surface}
\newcommand\radius{1.*0.82e8}
\addplot3[dotted,thick] coordinates {(0.75*\radius,-0.75*\radius,0.) (-0.75*\radius,0.75*\radius,0.)};
\addplot3[dotted,thick] coordinates {(0.,-0.75*\radius,0.75*\radius) (0.,0.75*\radius,-0.75*\radius)};
\addplot3[dotted,thick] coordinates {(-0.75*\radius,0.,0.75*\radius) (0.75*\radius,0.,-0.75*\radius)};
\end{axis}
\end{tikzpicture}
%%% Local Variables:
%%% mode: latex
%%% TeX-master: "../../mainManuscript"
%%% End:
}
  \caption{Stress paths in a slow simple wave for various starting point lying on the initial yield surface for $\sigma_{22}=5.8\times 10^7 \: Pa$. Projections in the stress space (figure \subref{subfig:CP_slow_stress3}) and deviatoric plane (figure \subref{subfig:CP_slow_dev3}).}
  \label{fig:slow_path_plane_stress3}
\end{figure}

More generally, the loading paths resulting from the integration of ODEs governing the behavior inside simple waves in plane stress can be summarized as follows.
Whereas the integral curves inside a fast wave first exhibits a phase in which the stress is restricted to the initial yield surface, the passage of a slow wave make the stress leave the elastic convex quasi-instantaneously.
It is moreover noteworthy that the shear stress component $\sigma_{12}$ undergoes the biggest variation through a slow wave, though the visible combined-stress nature of the corresponding paths.
In addition, rough change in the slopes of the integral curves associated to slow waves occur.
However, such phenomena may also be observed for fast waves once the shear waves vanishes but have not been highlighted in figure \ref{fig:fast_path_plane_stress} due to numerical issues in integrating a function tending to infinity.
%Indeed, the theory (equation \eqref{eq:CP_roots}) predicts that this situation would lead to loading paths (locally) perpendicular to the yield surface rather than parallel.
Furthermore, after the slopes of slow waves integral cruves broke, the paths are straight in both $(\sigma_{22},\sigma_{12})$ and deviatoric planes.

\newpage
\subsection{Plane strain}
Assuming that a solid initially at rest undergoes external loads leading to a plane strain case, the above approach is now repeated.
However, the derivation of the hyperbolic system in a two-dimensional setting lies on the writing of the out-of-plane stress component as a function of plastic strain.
Hence, the integral curves associated to simple waves are integrated implicitly, along with the plastic flow.

Since a fast wave propagates faster than a slow one, a material particle is first acted upon by the effects of the former. 
Thus, figure \ref{fig:fast_path_plane_strains} shows the evolution of stress resulting from integration using $\sigma_{11}$ as driving parameter, for several starting points on the yield surface.
The evolution of the celerity of fast waves along the integral curve confirms the validity of the simple wave solution.
Furthermore, the starting are chosen in such a way that a symmetry of the loading path with respect to $\sigma_{11}=0$ and $\sigma_{22}=0$ planes is notable.
Although the stress path depicted in figure \ref{fig:fast_path_plane_strains}\subref{subfig:fastDP_stress} are rather different to these resulting from a fast wave in plane strain, the behavior in the deviatoric plane is the same as can be seen in figure \ref{fig:fast_path_plane_strains}\subref{subfig:fastDP_dev}. 
Indeed, the fact that the computed loading paths through a fast wave is parallel to the initial yield surface is obvious when looking at the deviatoric plane.
More specifically, the von-Mises circle is traced by the integral curve even once the shear component $\sigma_{12}$ is zero (see paths $1$ and $6$ in figure \ref{fig:fast_path_plane_strains})).
Notice that the integration of loading path in the plane $\sigma_{12}=0$ is here possible with contrast to plane stress.
The integral curves of figure \ref{fig:fast_path_plane_strains} however exhibit a cusp that cannot be explained (see the two external arrows pointing toward axes of pure tensile/compression).
%Nevertheless, the results of figure \ref{fig:fast_path_plane_strains} must be taken carefully since the loading paths exhibits a cusp that is not explained so far (see the two external arrows pointing toward axes of pure tensile/compression).
\begin{figure}[h!]
  \centering
  \subcaptionbox{Projections of loading paths in ($\sigma_{11},\sigma_{12}$) and ($\sigma_{22},\sigma_{12}$) planes \label{subfig:fastDP_stress}}{\begin{tikzpicture}[scale=0.9]
\begin{groupplot}[group style={group size=2 by 1,
ylabels at=edge left, yticklabels at=edge left,horizontal sep=3.ex,
xticklabels at=edge bottom,xlabels at=edge bottom},
ymajorgrids=true,xmajorgrids=true,ylabel=$\sigma_{12} \: (Pa)$,
axis on top,scale only axis,width=0.4\linewidth,ymin=0,ymax=100000000.0
, every x tick scale label/.style={at={(xticklabel* cs:1.05,0.75cm)},anchor=near yticklabel},colormap={ry}{rgb255(0cm)=(255,255,0);rgb255(1cm)=(255,0,0)}]
\nextgroupplot[xlabel=$\sigma_{11} (Pa)$]
\addplot[mesh,point meta = \thisrow{p},very thick,no markers] table[x=sigma_11,y=sigma_12] {chapter5/pgfFigures/pgf_fastWavesPlaneStrain/DPfastStressPlane_frame0_Stress0.pgf} node[above right,black] {$\textbf{1}$};
\addplot[mesh,point meta = \thisrow{p},very thick,no markers] table[x=sigma_11,y=sigma_12] {chapter5/pgfFigures/pgf_fastWavesPlaneStrain/DPfastStressPlane_frame1_Stress0.pgf} node[above right,black] {$\textbf{2}$};
\addplot[mesh,point meta = \thisrow{p},very thick,no markers] table[x=sigma_11,y=sigma_12] {chapter5/pgfFigures/pgf_fastWavesPlaneStrain/DPfastStressPlane_frame2_Stress0.pgf} node[above right,black] {$\textbf{3}$};
\addplot[mesh,point meta = \thisrow{p},very thick,no markers] table[x=sigma_11,y=sigma_12] {chapter5/pgfFigures/pgf_fastWavesPlaneStrain/DPfastStressPlane_frame3_Stress0.pgf} node[above right,black] {$\textbf{4}$};
\addplot[mesh,point meta = \thisrow{p},very thick,no markers] table[x=sigma_11,y=sigma_12] {chapter5/pgfFigures/pgf_fastWavesPlaneStrain/DPfastStressPlane_frame4_Stress0.pgf} node[above right,black] {$\textbf{5}$};
\addplot[mesh,point meta = \thisrow{p},very thick,no markers] table[x=sigma_11,y=sigma_12] {chapter5/pgfFigures/pgf_fastWavesPlaneStrain/DPfastStressPlane_frame5_Stress0.pgf} node[above right,black] {$\textbf{6}$};
\addplot[gray,dashed,thin] table[x=sigma_11,y=sigma_12] {chapter5/pgfFigures/pgf_fastWavesPlaneStrain/DPfast_yield0_s11s12_Stress0.pgf};

\nextgroupplot[colorbar,colorbar style={title= {$ c_f \: (m/s)$},every y tick scale label/.style={at={(2.,-.1125)}} },xlabel=$\sigma_{22}  (Pa)$]
\addplot[mesh,point meta = \thisrow{p},very thick,no markers] table[x=sigma_22,y=sigma_12] {chapter5/pgfFigures/pgf_fastWavesPlaneStrain/DPfastStressPlane_frame0_Stress0.pgf} node[above right,black] {$\textbf{1}$};
\addplot[mesh,point meta = \thisrow{p},very thick,no markers] table[x=sigma_22,y=sigma_12] {chapter5/pgfFigures/pgf_fastWavesPlaneStrain/DPfastStressPlane_frame1_Stress0.pgf} node[above right,black] {$\textbf{2}$};
\addplot[mesh,point meta = \thisrow{p},very thick,no markers] table[x=sigma_22,y=sigma_12] {chapter5/pgfFigures/pgf_fastWavesPlaneStrain/DPfastStressPlane_frame2_Stress0.pgf} node[above right,black] {$\textbf{3}$};
\addplot[mesh,point meta = \thisrow{p},very thick,no markers] table[x=sigma_22,y=sigma_12] {chapter5/pgfFigures/pgf_fastWavesPlaneStrain/DPfastStressPlane_frame3_Stress0.pgf} node[above right,black] {$\textbf{4}$};
\addplot[mesh,point meta = \thisrow{p},very thick,no markers] table[x=sigma_22,y=sigma_12] {chapter5/pgfFigures/pgf_fastWavesPlaneStrain/DPfastStressPlane_frame4_Stress0.pgf} node[above right,black] {$\textbf{5}$};
\addplot[mesh,point meta = \thisrow{p},very thick,no markers] table[x=sigma_22,y=sigma_12] {chapter5/pgfFigures/pgf_fastWavesPlaneStrain/DPfastStressPlane_frame5_Stress0.pgf} node[above right,black] {$\textbf{6}$};
\addplot[gray,dashed,thin] table[x=sigma_22,y=sigma_12] {chapter5/pgfFigures/pgf_fastWavesPlaneStrain/DPfast_yield0_s22s12_frame0_Stress0.pgf};

\addplot[gray,dashed,thin] table[x=sigma_22,y=sigma_12] {chapter5/pgfFigures/pgf_fastWavesPlaneStrain/DPfast_yield0_s22s12_frame1_Stress0.pgf};

\addplot[gray,dashed,thin] table[x=sigma_22,y=sigma_12] {chapter5/pgfFigures/pgf_fastWavesPlaneStrain/DPfast_yield0_s22s12_frame2_Stress0.pgf};

\addplot[gray,dashed,thin] table[x=sigma_22,y=sigma_12] {chapter5/pgfFigures/pgf_fastWavesPlaneStrain/DPfast_yield0_s22s12_frame3_Stress0.pgf};

\addplot[gray,dashed,thin] table[x=sigma_22,y=sigma_12] {chapter5/pgfFigures/pgf_fastWavesPlaneStrain/DPfast_yield0_s22s12_frame4_Stress0.pgf};

\addplot[gray,dashed,thin] table[x=sigma_22,y=sigma_12] {chapter5/pgfFigures/pgf_fastWavesPlaneStrain/DPfast_yield0_s22s12_frame5_Stress0.pgf};

\end{groupplot}
\end{tikzpicture}
%%% Local Variables:
%%% mode: latex
%%% TeX-master: "../../mainManuscript"
%%% End:
}
  \subcaptionbox{Loading path in deviatoric plane \label{subfig:fastDP_dev}}{\tikzset{cross/.style={cross out, draw=black, minimum size=2*(#1-\pgflinewidth), inner sep=0pt, outer sep=0pt},cross/.default={2.5pt}}
\begin{tikzpicture}[spy using outlines={rectangle, magnification=3, size=2.cm, connect spies},scale=0.9]
\begin{axis}[width=.75\textwidth,view={135}{35.2643},xlabel=$s_1 $,ylabel=$s_2 $,zlabel=$s_3$,xmin=-1.e8,xmax=1.e8,ymin=-1.e8,ymax=1.e8,axis equal,axis lines=center,axis on top,xtick=\empty,ytick=\empty,ztick=\empty,every axis y label/.style={at={(rel axis cs:0.,.5,-0.65)}, anchor=west}, every axis x label/.style={at={(rel axis cs:0.5,.,-0.65)}, anchor=east}, every axis z label/.style={at={(rel axis cs:0.,.0,.18)}, anchor=north},legend style={at={(.2,.68)}}]
\node[below] at (1.1e8,0.,0.) {$\sigma^y$};
\node[above] at (-1.1e8,0.,0.) {$-\sigma^y$};
\draw (1.e8,0.,0.) node[cross,rotate=10] {};
\draw (-1.e8,0.,0.) node[cross,rotate=10] {};
\node[white]  at (0,0.,1.42e8) {};
\addplot3[Red,thick,arrows along my path] file {chapter5/pgfFigures/pgf_fastWavesPlaneStrain/DPfastDevPlane_frame0_Stress0.pgf};\addlegendentry{loading path 1}
\addplot3[Blue,thick,arrows along my path] file {chapter5/pgfFigures/pgf_fastWavesPlaneStrain/DPfastDevPlane_frame1_Stress0.pgf};\addlegendentry{loading path 2}
\addplot3[Orange,thick,arrows along my path] file {chapter5/pgfFigures/pgf_fastWavesPlaneStrain/DPfastDevPlane_frame2_Stress0.pgf};\addlegendentry{loading path 3}
\addplot3[Purple,thick,arrows along my path] file {chapter5/pgfFigures/pgf_fastWavesPlaneStrain/DPfastDevPlane_frame3_Stress0.pgf};\addlegendentry{loading path 4}
\addplot3[Green,thick,arrows along my path] file {chapter5/pgfFigures/pgf_fastWavesPlaneStrain/DPfastDevPlane_frame4_Stress0.pgf};\addlegendentry{loading path 5}
\addplot3[Duck,thick,arrows along my path] file {chapter5/pgfFigures/pgf_fastWavesPlaneStrain/DPfastDevPlane_frame5_Stress0.pgf};\addlegendentry{loading path 6}
\addplot3+[gray,dashed,thin,no markers] file {chapter5/pgfFigures/pgf_fastWavesPlaneStrain/CylindreDevPlane.pgf};\addlegendentry{initial yield surface}
\newcommand\radius{1.*0.82e8}
\addplot3[dotted,thick] coordinates {(0.75*\radius,-0.75*\radius,0.) (-0.75*\radius,0.75*\radius,0.)};
\addplot3[dotted,thick] coordinates {(0.,-0.75*\radius,0.75*\radius) (0.,0.75*\radius,-0.75*\radius)};
\addplot3[dotted,thick] coordinates {(-0.75*\radius,0.,0.75*\radius) (0.75*\radius,0.,-0.75*\radius)};
\begin{scope}
\spy[black,size=1.75cm] on (6.75,3.2) in node [fill=none] at (9.5,5.5);
\end{scope}

\end{axis}
\end{tikzpicture}
%%% Local Variables:
%%% mode: latex
%%% TeX-master: "../../mainManuscript"
%%% End:
}
  \caption{Loading paths through a fast simple wave with initial condition $\sigma_{22}=0$ for different starting points on the initial yield surface.}
  \label{fig:fast_path_plane_strains}
\end{figure}

%% For slow waves, integrated by driving with a decreasing \tau yields an increasing celerity (it goes for fast waves driven with tau in plane stress)
On the other hand, some loading paths resulting from the integration of slow waves ODEs are depicted in figures \ref{fig:slow_path_plane_strains1}, \ref{fig:slow_path_plane_strains2} and \ref{fig:slow_path_plane_strains3}.
Analogously to plane stress cases, three initial values $\sigma_{22}=-1.3 \times 10^{8} \: Pa$, $\sigma_{22}=0$ and $\sigma_{22}=1.3 \times 10^8 \: Pa$ are considered since a fast wave may modify the stress.
The integration of loading paths through slow waves in plane strain is performed by using $\sigma_{12}$ as driving parameter.
However, numerical difficulties arise since the characteristic speed associated to slow waves may start increasing rather than decreasing at some point along the path (it was the case for every results presented below).
In order to circumvent this issue, the last stress state leading to a decreasing celerity was used as an initial condition for a second integration driven by means of $\sigma_{11}$.
The final value of $\sigma_{11}$ was set so that the path followed up to that singularity is continued.
This strategy enables to carry on the integration further.
Nevertheless, the same problem of increasing characteristic speed again occurs and the computation must be aborted.

The integral curves depicted in figure \ref{fig:slow_path_plane_strains1} results from the negative initial value $\sigma_{22}=-1.3 \times 10^{8} \: Pa$ for several starting points on the initial yield surface.
While the stress $\sigma_{11}$ varies little, as shown by the projection of the path in the ($\sigma_{11},\sigma_{12}$) plane, it is not the case for $\sigma_{22}$ (see figure \ref{fig:slow_path_plane_strains1}\subref{subfig:slowDP_stress1}).
Indeed, the projections of the integral curves in ($\sigma_{22},\sigma_{12}$) plane exhibit complex paths.
In addition, it can be seen in figure \ref{fig:slow_path_plane_strains1}\subref{subfig:slowDP_dev1} that, as for fast waves in plane strain, the integral curves are first restricted to the initial yield surface.
%A negative initial value of the transverse stress $\sigma_{22}=-1.3 \times 10^{8} \: Pa$ has been considered for obtaining the curves depicted in figure \ref{fig:slow_path_plane_strains}.
%% Symmetry of the integral curve
% The projections of integral curves in the ($\sigma_{11},\sigma_{12}$) plane in figure \ref{fig:slow_path_plane_strains}\subref{subfig:slowDP_stress1} show, roughly speaking, linear by part curves.
% Conversely, the loading paths exhibit more complex shapes in the ($\sigma_{22},\sigma_{12}$) plane.
% Analogously to the plane stress case, the slope of integral curves break at some point and then seems to be vertical in that plane.
% On the other hand, it appears in figure \ref{fig:slow_path_plane_strains}\subref{subfig:slowDP_dev1} that as for fast waves, the stress state remains on the initial yield surface until the direction of pure shear is reached.
% The direction next changes so that the loading paths are radial regardless of the initial conditions considered.
%% Trajectoire hors plan du déviateur donc on n'est pas purement déviatoric s12-s3 et s12-p pour voir ça
% The Lode angle $\Theta$, based the second and third invariants of the stress deviator $J_2 = \frac{1}{2} \tens{s} : \tens{s}$ and $J_3 = \frac{1}{3}\det(\tens{s})$, and defined as:
% \begin{equation*}
%   \cos(3\Theta)=\frac{J_3}{2}\(\frac{3}{J_2}\)^{3/2}
% \end{equation*}
% can be used if the breaks in slopes arising in both ($\sigma_{22},\sigma_{12}$) and deviatoric planes are linked.
% Indeed, Lode angle $\Theta$ measures the angle to the middle eigenvalue $s_2$ in the deviatoric plane so that it should reach a constant value of $30^o$ for the stress paths depicted in figure \ref{fig:slow_path_plane_strains}.
% \begin{figure}[h!]
%   \centering
%   \subcaptionbox{Projections of loading paths in ($\sigma_{11},\sigma_{12}$) and ($\sigma_{22},\sigma_{12}$) planes \label{subfig:slowDP_stress1}}{\begin{tikzpicture}[scale=0.9]
\begin{groupplot}[group style={group size=2 by 1,
ylabels at=edge left, yticklabels at=edge left,horizontal sep=3.ex,
xticklabels at=edge bottom,xlabels at=edge bottom},
ymajorgrids=true,xmajorgrids=true,ylabel=$\sigma_{12} \: (Pa)$,
axis on top,scale only axis,width=0.45\linewidth,ymin=0,ymax=68618075.3103
, every x tick scale label/.style={at={(xticklabel* cs:1.05,0.75cm)},anchor=near yticklabel}]
\nextgroupplot[xlabel=$\sigma_{11} (Pa)$]
\addplot[mesh,point meta = \thisrow{p},very thick,no markers] table[x=sigma_11,y=sigma_12] {chapter5/pgfFigures/pgf_slowWavesPlaneStrain/DPslowStressPlane_frame0_Stress1.pgf};
\addplot[mesh,point meta = \thisrow{p},very thick,no markers] table[x=sigma_11,y=sigma_12] {chapter5/pgfFigures/pgf_slowWavesPlaneStrain/DPslowStressPlane_frame1_Stress1.pgf};
\addplot[mesh,point meta = \thisrow{p},very thick,no markers] table[x=sigma_11,y=sigma_12] {chapter5/pgfFigures/pgf_slowWavesPlaneStrain/DPslowStressPlane_frame2_Stress1.pgf};
\addplot[mesh,point meta = \thisrow{p},very thick,no markers] table[x=sigma_11,y=sigma_12] {chapter5/pgfFigures/pgf_slowWavesPlaneStrain/DPslowStressPlane_frame3_Stress1.pgf};
\addplot[gray,thin] table[x=sigma_11,y=sigma_12] {chapter5/pgfFigures/pgf_slowWavesPlaneStrain/DPslow_yield0_s11s12_Stress1.pgf};

\nextgroupplot[colorbar,colorbar style={title= {$\rho c^2$},every y tick scale label/.style={at={(2.,-.1125)}} },xlabel=$\sigma_{22}  (Pa)$]
\addplot[mesh,point meta = \thisrow{p},very thick,no markers] table[x=sigma_22,y=sigma_12] {chapter5/pgfFigures/pgf_slowWavesPlaneStrain/DPslowStressPlane_frame0_Stress1.pgf};
\addplot[mesh,point meta = \thisrow{p},very thick,no markers] table[x=sigma_22,y=sigma_12] {chapter5/pgfFigures/pgf_slowWavesPlaneStrain/DPslowStressPlane_frame1_Stress1.pgf};
\addplot[mesh,point meta = \thisrow{p},very thick,no markers] table[x=sigma_22,y=sigma_12] {chapter5/pgfFigures/pgf_slowWavesPlaneStrain/DPslowStressPlane_frame2_Stress1.pgf};
\addplot[mesh,point meta = \thisrow{p},very thick,no markers] table[x=sigma_22,y=sigma_12] {chapter5/pgfFigures/pgf_slowWavesPlaneStrain/DPslowStressPlane_frame3_Stress1.pgf};
\end{groupplot}
\end{tikzpicture}
%%% Local Variables:
%%% mode: latex
%%% TeX-master: "../../mainManuscript"
%%% End:
}
%   \subcaptionbox{Loading path in deviatoric plane \label{subfig:slowDP_dev1}}{\begin{tikzpicture}[scale=0.9]
\begin{axis}[width=.75\textwidth,view={135}{35.2643},xlabel=$s_1 $,ylabel=$s_2 $,zlabel=$s_3$,xmin=-1.e8,xmax=1.e8,ymin=-1.e8,ymax=1.e8,axis equal,axis lines=center,axis on top,ztick=\empty]
\addplot3+[Red,very thick,no markers] file {chapter5/pgfFigures/pgf_slowWavesPlaneStrain/DPslowDevPlane_frame0_Stress1.pgf};
\addplot3+[Blue,very thick,no markers] file {chapter5/pgfFigures/pgf_slowWavesPlaneStrain/DPslowDevPlane_frame1_Stress1.pgf};
\addplot3+[Orange,very thick,no markers] file {chapter5/pgfFigures/pgf_slowWavesPlaneStrain/DPslowDevPlane_frame2_Stress1.pgf};
\addplot3+[Purple,very thick,no markers] file {chapter5/pgfFigures/pgf_slowWavesPlaneStrain/DPslowDevPlane_frame3_Stress1.pgf};
\addplot3+[gray,dashed,thin,no markers] file {chapter5/pgfFigures/pgf_slowWavesPlaneStrain/CylindreDevPlane.pgf};
\end{axis}
\end{tikzpicture}
%%% Local Variables:
%%% mode: latex
%%% TeX-master: "../../mainManuscript"
%%% End:
}
%   \caption{Loading paths through slow simple waves with initial condition $\sigma_{22}=-1.3\times 10^{8} \: Pa$ for different starting points on the initial yield surface.}
%   \label{fig:slow_path_plane_strains}
% \end{figure}
\begin{figure}[h!]
  \centering
  \subcaptionbox{Projections of loading paths in ($\sigma_{11},\sigma_{12}$) and ($\sigma_{22},\sigma_{12}$) planes \label{subfig:slowDP_stress1}}{\begin{tikzpicture}[scale=0.9]
\begin{groupplot}[group style={group size=2 by 1,
ylabels at=edge left, yticklabels at=edge left,horizontal sep=3.ex,
xticklabels at=edge bottom,xlabels at=edge bottom},
ymajorgrids=true,xmajorgrids=true,ylabel=$\sigma_{12} \: (Pa)$,
axis on top,scale only axis,width=0.45\linewidth,ymin=0,ymax=68618075.3103
, every x tick scale label/.style={at={(xticklabel* cs:1.05,0.75cm)},anchor=near yticklabel}]
\nextgroupplot[xlabel=$\sigma_{11} (Pa)$]
\addplot[mesh,point meta = \thisrow{p},very thick,no markers] table[x=sigma_11,y=sigma_12] {chapter5/pgfFigures/pgf_slowWavesPlaneStrain/DPslowStressPlane_frame0_Stress1.pgf};
\addplot[mesh,point meta = \thisrow{p},very thick,no markers] table[x=sigma_11,y=sigma_12] {chapter5/pgfFigures/pgf_slowWavesPlaneStrain/DPslowStressPlane_frame1_Stress1.pgf};
\addplot[mesh,point meta = \thisrow{p},very thick,no markers] table[x=sigma_11,y=sigma_12] {chapter5/pgfFigures/pgf_slowWavesPlaneStrain/DPslowStressPlane_frame2_Stress1.pgf};
\addplot[mesh,point meta = \thisrow{p},very thick,no markers] table[x=sigma_11,y=sigma_12] {chapter5/pgfFigures/pgf_slowWavesPlaneStrain/DPslowStressPlane_frame3_Stress1.pgf};
\addplot[gray,thin] table[x=sigma_11,y=sigma_12] {chapter5/pgfFigures/pgf_slowWavesPlaneStrain/DPslow_yield0_s11s12_Stress1.pgf};

\nextgroupplot[colorbar,colorbar style={title= {$\rho c^2$},every y tick scale label/.style={at={(2.,-.1125)}} },xlabel=$\sigma_{22}  (Pa)$]
\addplot[mesh,point meta = \thisrow{p},very thick,no markers] table[x=sigma_22,y=sigma_12] {chapter5/pgfFigures/pgf_slowWavesPlaneStrain/DPslowStressPlane_frame0_Stress1.pgf};
\addplot[mesh,point meta = \thisrow{p},very thick,no markers] table[x=sigma_22,y=sigma_12] {chapter5/pgfFigures/pgf_slowWavesPlaneStrain/DPslowStressPlane_frame1_Stress1.pgf};
\addplot[mesh,point meta = \thisrow{p},very thick,no markers] table[x=sigma_22,y=sigma_12] {chapter5/pgfFigures/pgf_slowWavesPlaneStrain/DPslowStressPlane_frame2_Stress1.pgf};
\addplot[mesh,point meta = \thisrow{p},very thick,no markers] table[x=sigma_22,y=sigma_12] {chapter5/pgfFigures/pgf_slowWavesPlaneStrain/DPslowStressPlane_frame3_Stress1.pgf};
\end{groupplot}
\end{tikzpicture}
%%% Local Variables:
%%% mode: latex
%%% TeX-master: "../../mainManuscript"
%%% End:
}
  \subcaptionbox{Loading path in deviatoric plane \label{subfig:slowDP_dev1}}{\begin{tikzpicture}[scale=0.9]
\begin{axis}[width=.75\textwidth,view={135}{35.2643},xlabel=$s_1 $,ylabel=$s_2 $,zlabel=$s_3$,xmin=-1.e8,xmax=1.e8,ymin=-1.e8,ymax=1.e8,axis equal,axis lines=center,axis on top,ztick=\empty]
\addplot3+[Red,very thick,no markers] file {chapter5/pgfFigures/pgf_slowWavesPlaneStrain/DPslowDevPlane_frame0_Stress1.pgf};
\addplot3+[Blue,very thick,no markers] file {chapter5/pgfFigures/pgf_slowWavesPlaneStrain/DPslowDevPlane_frame1_Stress1.pgf};
\addplot3+[Orange,very thick,no markers] file {chapter5/pgfFigures/pgf_slowWavesPlaneStrain/DPslowDevPlane_frame2_Stress1.pgf};
\addplot3+[Purple,very thick,no markers] file {chapter5/pgfFigures/pgf_slowWavesPlaneStrain/DPslowDevPlane_frame3_Stress1.pgf};
\addplot3+[gray,dashed,thin,no markers] file {chapter5/pgfFigures/pgf_slowWavesPlaneStrain/CylindreDevPlane.pgf};
\end{axis}
\end{tikzpicture}
%%% Local Variables:
%%% mode: latex
%%% TeX-master: "../../mainManuscript"
%%% End:
}
  \caption{Loading paths through slow simple waves for different starting points on the initial yield surface for the initial condition $\sigma_{22}=-1.3 \times 10^{8} \: Pa$.}
  \label{fig:slow_path_plane_strains1}
\end{figure}
A second phase then occurs once the direction of pure shear in the deviatoric plane is reached, during which the loading path is radial.
For the stress paths $1$ and $4$ in figure \ref{fig:slow_path_plane_strains1}\subref{subfig:slowDP_dev1}, the integration goes well in the radial direction and then stops owing to the singularities mentioned above.

The same behavior is observed in figures \ref{fig:slow_path_plane_strains2} and \ref{fig:slow_path_plane_strains3} which respectively show the loading paths resulting form the zero and the positive initial values of $\sigma_{22}$.
Nevertheless, the integral curves of figure \ref{fig:slow_path_plane_strains2} reveals that numerical issues occur faster than in the previous case.
Indeed, the characteristic speeds quickly start increasing so that the stress paths depicted are short.
Furthermore, the projection in ($\sigma_{22},\sigma_{12}$) plane of the integral curves $2$ and $3$ show that the slow wave mainly influence $\sigma_{22}$.
The projections in deviatoric plane in figure \ref{fig:slow_path_plane_strains2} however show that the stress paths first follow the initial yield surface until the direction of pure shear is reached, and next follow tha radial direction.
In addition, considering figures \ref{fig:slow_path_plane_strains1}, \ref{fig:slow_path_plane_strains2} and \ref{fig:slow_path_plane_strains3}, it can be seen that the loading paths in $(\sigma_{22},\sigma_{12})$ plane lead to an increasing $\sigma_{22}$ if the initial condition is such that $\sigma_{11}$ is greater than that corresponding to the maximum shear on the initial yield surface in projection in the $(\sigma_{11},\sigma_{12})$ plane.
Conversely, an initial condition $\sigma_{11}$ lower than that associated to the maximum shear stress on the initial yield surface in the plane $(\sigma_{11},\sigma_{12})$ results in a decreasing $\sigma_{22}$ along the integral curves.
Note that the same goes for $\sigma_{11}$.
Thus, it seems that the property $\sign(d\sigma_{22})=\sign(d\sigma_{11})$ holds along the loading path followed inside a slow simple wave in plane strain, though this has not been shown mathematically.

Generally speaking, it appears that for the ranges of stress considered here, the hardening of the material is mainly due to slow simple waves for plane strain cases.
Indeed, the latter may lead to radial loading paths that greatly increase the radius of the von-Mises cylinder in principal stress space, whereas the integral curves corresponding to fast waves are restricted to the initial yield surface. Notice, however, that the above results have been obtained by using a rather low hardening modulus.

\begin{figure}[h!]
  \centering
  % \subcaptionbox{Projections of loading paths in ($\sigma_{11},\sigma_{12}$) and ($\sigma_{22},\sigma_{12}$) planes \label{subfig:slowDP_stress2}}{\begin{tikzpicture}[scale=0.9]
\begin{groupplot}[group style={group size=2 by 1,
ylabels at=edge left, yticklabels at=edge left,horizontal sep=3.ex,
xticklabels at=edge bottom,xlabels at=edge bottom},
ymajorgrids=true,xmajorgrids=true,ylabel=$\sigma_{12} \: (Pa)$,
axis on top,scale only axis,width=0.4\linewidth,ymin=0,ymax=94979909.10761759
, every x tick scale label/.style={at={(xticklabel* cs:1.05,0.75cm)},anchor=near yticklabel},colormap name=viridis]
\nextgroupplot[xlabel=$\sigma_{11} (Pa)$]
\addplot[mesh,point meta = \thisrow{p},very thick,no markers] table[x=sigma_11,y=sigma_12] {chapter5/pgfFigures/pgf_slowWavesPlaneStrain/DPslowStressPlane_frame0_Stress2.pgf} node[above,black] {$\textbf{1}$};
\addplot[mesh,point meta = \thisrow{p},very thick,no markers] table[x=sigma_11,y=sigma_12] {chapter5/pgfFigures/pgf_slowWavesPlaneStrain/DPslowStressPlane_frame1_Stress2.pgf} node[above,black] {$\textbf{2}$};
\addplot[mesh,point meta = \thisrow{p},very thick,no markers] table[x=sigma_11,y=sigma_12] {chapter5/pgfFigures/pgf_slowWavesPlaneStrain/DPslowStressPlane_frame2_Stress2.pgf} node[above,black] {$\textbf{3}$};
\addplot[mesh,point meta = \thisrow{p},very thick,no markers] table[x=sigma_11,y=sigma_12] {chapter5/pgfFigures/pgf_slowWavesPlaneStrain/DPslowStressPlane_frame3_Stress2.pgf} node[above,black] {$\textbf{4}$};
\addplot[gray,dashed,thin] table[x=sigma_11,y=sigma_12] {chapter5/pgfFigures/pgf_slowWavesPlaneStrain/DPslow_yield0_s11s12_Stress2.pgf};

\addplot[gray,dashed,thin] table[x=sigma_11,y=sigma_12] {chapter5/pgfFigures/pgf_slowWavesPlaneStrain/DPslow_yieldfin_s11s12_frame0_Stress2.pgf};

\addplot[gray,dashed,thin] table[x=sigma_11,y=sigma_12] {chapter5/pgfFigures/pgf_slowWavesPlaneStrain/DPslow_yieldfin_s11s12_frame1_Stress2.pgf};

\addplot[gray,dashed,thin] table[x=sigma_11,y=sigma_12] {chapter5/pgfFigures/pgf_slowWavesPlaneStrain/DPslow_yieldfin_s11s12_frame2_Stress2.pgf};

\addplot[gray,dashed,thin] table[x=sigma_11,y=sigma_12] {chapter5/pgfFigures/pgf_slowWavesPlaneStrain/DPslow_yieldfin_s11s12_frame3_Stress2.pgf};

\nextgroupplot[colorbar,colorbar style={title= {$ c_s \: (m/s)$},every y tick scale label/.style={at={(2.,-.1125)}} },xlabel=$\sigma_{22}  (Pa)$]
\addplot[mesh,point meta = \thisrow{p},very thick,no markers] table[x=sigma_22,y=sigma_12] {chapter5/pgfFigures/pgf_slowWavesPlaneStrain/DPslowStressPlane_frame0_Stress2.pgf} node[above,black] {$\textbf{1}$};
\addplot[mesh,point meta = \thisrow{p},very thick,no markers] table[x=sigma_22,y=sigma_12] {chapter5/pgfFigures/pgf_slowWavesPlaneStrain/DPslowStressPlane_frame1_Stress2.pgf} node[above,black] {$\textbf{2}$};
\addplot[mesh,point meta = \thisrow{p},very thick,no markers] table[x=sigma_22,y=sigma_12] {chapter5/pgfFigures/pgf_slowWavesPlaneStrain/DPslowStressPlane_frame2_Stress2.pgf} node[above,black] {$\textbf{3}$};
\addplot[mesh,point meta = \thisrow{p},very thick,no markers] table[x=sigma_22,y=sigma_12] {chapter5/pgfFigures/pgf_slowWavesPlaneStrain/DPslowStressPlane_frame3_Stress2.pgf} node[above,black] {$\textbf{4}$};
\addplot[gray,dashed,thin] table[x=sigma_22,y=sigma_12] {chapter5/pgfFigures/pgf_slowWavesPlaneStrain/DPslow_yieldfin_s22s12_frame0_Stress2.pgf};

\addplot[gray,dashed,thin] table[x=sigma_22,y=sigma_12] {chapter5/pgfFigures/pgf_slowWavesPlaneStrain/DPslow_yieldfin_s22s12_frame1_Stress2.pgf};

\addplot[gray,dashed,thin] table[x=sigma_22,y=sigma_12] {chapter5/pgfFigures/pgf_slowWavesPlaneStrain/DPslow_yieldfin_s22s12_frame2_Stress2.pgf};

\addplot[gray,dashed,thin] table[x=sigma_22,y=sigma_12] {chapter5/pgfFigures/pgf_slowWavesPlaneStrain/DPslow_yieldfin_s22s12_frame3_Stress2.pgf};

\end{groupplot}
\end{tikzpicture}
%%% Local Variables:
%%% mode: latex
%%% TeX-master: "../../mainManuscript"
%%% End:
}
  % \subcaptionbox{Loading path in deviatoric plane \label{subfig:slowDP_dev2}}{\tikzset{cross/.style={cross out, draw=black, minimum size=2*(#1-\pgflinewidth), inner sep=0pt, outer sep=0pt},cross/.default={2.5pt}}
\begin{tikzpicture}[scale=0.9]
\begin{axis}[width=.75\textwidth,view={135}{35.2643},xlabel=$s_1 $,ylabel=$s_2 $,zlabel=$s_3$,xmin=-1.e8,xmax=1.e8,ymin=-1.e8,ymax=1.e8,axis equal,axis lines=center,axis on top,xtick=\empty,ytick=\empty,ztick=\empty,every axis y label/.style={at={(rel axis cs:0.,.5,-0.65)}, anchor=west}, every axis x label/.style={at={(rel axis cs:0.5,.,-0.65)}, anchor=east}, every axis z label/.style={at={(rel axis cs:0.,.0,.18)}, anchor=north},legend style={at={(.2,.68)}}]
\node[below] at (1.1e8,0.,0.) {$\sigma^y$};
\node[above] at (-1.1e8,0.,0.) {$-\sigma^y$};
\draw (1.e8,0.,0.) node[cross,rotate=10] {};
\draw (-1.e8,0.,0.) node[cross,rotate=10] {};
\node[white]  at (0,0.,1.1e8) {};
\addplot3[arrows along my path,Red,very thick] file {chapter5/pgfFigures/pgf_slowWavesPlaneStrain/DPslowDevPlane_frame0_Stress2.pgf};\addlegendentry{loading path 1}
\addplot3[arrows along my path,Blue,very thick] file {chapter5/pgfFigures/pgf_slowWavesPlaneStrain/DPslowDevPlane_frame1_Stress2.pgf};\addlegendentry{loading path 2}
\addplot3[arrows along my path,Orange,very thick] file {chapter5/pgfFigures/pgf_slowWavesPlaneStrain/DPslowDevPlane_frame2_Stress2.pgf};\addlegendentry{loading path 3}
\addplot3[arrows along my path,Purple,very thick] file {chapter5/pgfFigures/pgf_slowWavesPlaneStrain/DPslowDevPlane_frame3_Stress2.pgf};\addlegendentry{loading path 4}
\addplot3[arrows along my path,Green,very thick] file {chapter5/pgfFigures/pgf_slowWavesPlaneStrain/DPslowDevPlane_frame4_Stress2.pgf};\addlegendentry{loading path 5}
\addplot3+[gray,dashed,thin,no markers] file {chapter5/pgfFigures/pgf_slowWavesPlaneStrain/CylindreDevPlane.pgf};\addlegendentry{initial yield surface}
\newcommand\radius{1.*0.82e8}
\addplot3[dotted,thick] coordinates {(0.75*\radius,-0.75*\radius,0.) (-0.75*\radius,0.75*\radius,0.)};
\addplot3[dotted,thick] coordinates {(0.,-0.75*\radius,0.75*\radius) (0.,0.75*\radius,-0.75*\radius)};
\addplot3[dotted,thick] coordinates {(-0.75*\radius,0.,0.75*\radius) (0.75*\radius,0.,-0.75*\radius)};
\end{axis}
\end{tikzpicture}
%%% Local Variables:
%%% mode: latex
%%% TeX-master: "../../mainManuscript"
%%% End:
}
  {\begin{tikzpicture}[scale=0.9]
\begin{groupplot}[group style={group size=2 by 1,
ylabels at=edge left, yticklabels at=edge left,horizontal sep=3.ex,
xticklabels at=edge bottom,xlabels at=edge bottom},
ymajorgrids=true,xmajorgrids=true,ylabel=$\sigma_{12} \: (Pa)$,
axis on top,scale only axis,width=0.4\linewidth,ymin=0,ymax=94979909.10761759
, every x tick scale label/.style={at={(xticklabel* cs:1.05,0.75cm)},anchor=near yticklabel},colormap name=viridis]
\nextgroupplot[xlabel=$\sigma_{11} (Pa)$]
\addplot[mesh,point meta = \thisrow{p},very thick,no markers] table[x=sigma_11,y=sigma_12] {chapter5/pgfFigures/pgf_slowWavesPlaneStrain/DPslowStressPlane_frame0_Stress2.pgf} node[above,black] {$\textbf{1}$};
\addplot[mesh,point meta = \thisrow{p},very thick,no markers] table[x=sigma_11,y=sigma_12] {chapter5/pgfFigures/pgf_slowWavesPlaneStrain/DPslowStressPlane_frame1_Stress2.pgf} node[above,black] {$\textbf{2}$};
\addplot[mesh,point meta = \thisrow{p},very thick,no markers] table[x=sigma_11,y=sigma_12] {chapter5/pgfFigures/pgf_slowWavesPlaneStrain/DPslowStressPlane_frame2_Stress2.pgf} node[above,black] {$\textbf{3}$};
\addplot[mesh,point meta = \thisrow{p},very thick,no markers] table[x=sigma_11,y=sigma_12] {chapter5/pgfFigures/pgf_slowWavesPlaneStrain/DPslowStressPlane_frame3_Stress2.pgf} node[above,black] {$\textbf{4}$};
\addplot[gray,dashed,thin] table[x=sigma_11,y=sigma_12] {chapter5/pgfFigures/pgf_slowWavesPlaneStrain/DPslow_yield0_s11s12_Stress2.pgf};

\addplot[gray,dashed,thin] table[x=sigma_11,y=sigma_12] {chapter5/pgfFigures/pgf_slowWavesPlaneStrain/DPslow_yieldfin_s11s12_frame0_Stress2.pgf};

\addplot[gray,dashed,thin] table[x=sigma_11,y=sigma_12] {chapter5/pgfFigures/pgf_slowWavesPlaneStrain/DPslow_yieldfin_s11s12_frame1_Stress2.pgf};

\addplot[gray,dashed,thin] table[x=sigma_11,y=sigma_12] {chapter5/pgfFigures/pgf_slowWavesPlaneStrain/DPslow_yieldfin_s11s12_frame2_Stress2.pgf};

\addplot[gray,dashed,thin] table[x=sigma_11,y=sigma_12] {chapter5/pgfFigures/pgf_slowWavesPlaneStrain/DPslow_yieldfin_s11s12_frame3_Stress2.pgf};

\nextgroupplot[colorbar,colorbar style={title= {$ c_s \: (m/s)$},every y tick scale label/.style={at={(2.,-.1125)}} },xlabel=$\sigma_{22}  (Pa)$]
\addplot[mesh,point meta = \thisrow{p},very thick,no markers] table[x=sigma_22,y=sigma_12] {chapter5/pgfFigures/pgf_slowWavesPlaneStrain/DPslowStressPlane_frame0_Stress2.pgf} node[above,black] {$\textbf{1}$};
\addplot[mesh,point meta = \thisrow{p},very thick,no markers] table[x=sigma_22,y=sigma_12] {chapter5/pgfFigures/pgf_slowWavesPlaneStrain/DPslowStressPlane_frame1_Stress2.pgf} node[above,black] {$\textbf{2}$};
\addplot[mesh,point meta = \thisrow{p},very thick,no markers] table[x=sigma_22,y=sigma_12] {chapter5/pgfFigures/pgf_slowWavesPlaneStrain/DPslowStressPlane_frame2_Stress2.pgf} node[above,black] {$\textbf{3}$};
\addplot[mesh,point meta = \thisrow{p},very thick,no markers] table[x=sigma_22,y=sigma_12] {chapter5/pgfFigures/pgf_slowWavesPlaneStrain/DPslowStressPlane_frame3_Stress2.pgf} node[above,black] {$\textbf{4}$};
\addplot[gray,dashed,thin] table[x=sigma_22,y=sigma_12] {chapter5/pgfFigures/pgf_slowWavesPlaneStrain/DPslow_yieldfin_s22s12_frame0_Stress2.pgf};

\addplot[gray,dashed,thin] table[x=sigma_22,y=sigma_12] {chapter5/pgfFigures/pgf_slowWavesPlaneStrain/DPslow_yieldfin_s22s12_frame1_Stress2.pgf};

\addplot[gray,dashed,thin] table[x=sigma_22,y=sigma_12] {chapter5/pgfFigures/pgf_slowWavesPlaneStrain/DPslow_yieldfin_s22s12_frame2_Stress2.pgf};

\addplot[gray,dashed,thin] table[x=sigma_22,y=sigma_12] {chapter5/pgfFigures/pgf_slowWavesPlaneStrain/DPslow_yieldfin_s22s12_frame3_Stress2.pgf};

\end{groupplot}
\end{tikzpicture}
%%% Local Variables:
%%% mode: latex
%%% TeX-master: "../../mainManuscript"
%%% End:
}
  {\tikzset{cross/.style={cross out, draw=black, minimum size=2*(#1-\pgflinewidth), inner sep=0pt, outer sep=0pt},cross/.default={2.5pt}}
\begin{tikzpicture}[scale=0.9]
\begin{axis}[width=.75\textwidth,view={135}{35.2643},xlabel=$s_1 $,ylabel=$s_2 $,zlabel=$s_3$,xmin=-1.e8,xmax=1.e8,ymin=-1.e8,ymax=1.e8,axis equal,axis lines=center,axis on top,xtick=\empty,ytick=\empty,ztick=\empty,every axis y label/.style={at={(rel axis cs:0.,.5,-0.65)}, anchor=west}, every axis x label/.style={at={(rel axis cs:0.5,.,-0.65)}, anchor=east}, every axis z label/.style={at={(rel axis cs:0.,.0,.18)}, anchor=north},legend style={at={(.2,.68)}}]
\node[below] at (1.1e8,0.,0.) {$\sigma^y$};
\node[above] at (-1.1e8,0.,0.) {$-\sigma^y$};
\draw (1.e8,0.,0.) node[cross,rotate=10] {};
\draw (-1.e8,0.,0.) node[cross,rotate=10] {};
\node[white]  at (0,0.,1.1e8) {};
\addplot3[arrows along my path,Red,very thick] file {chapter5/pgfFigures/pgf_slowWavesPlaneStrain/DPslowDevPlane_frame0_Stress2.pgf};\addlegendentry{loading path 1}
\addplot3[arrows along my path,Blue,very thick] file {chapter5/pgfFigures/pgf_slowWavesPlaneStrain/DPslowDevPlane_frame1_Stress2.pgf};\addlegendentry{loading path 2}
\addplot3[arrows along my path,Orange,very thick] file {chapter5/pgfFigures/pgf_slowWavesPlaneStrain/DPslowDevPlane_frame2_Stress2.pgf};\addlegendentry{loading path 3}
\addplot3[arrows along my path,Purple,very thick] file {chapter5/pgfFigures/pgf_slowWavesPlaneStrain/DPslowDevPlane_frame3_Stress2.pgf};\addlegendentry{loading path 4}
\addplot3[arrows along my path,Green,very thick] file {chapter5/pgfFigures/pgf_slowWavesPlaneStrain/DPslowDevPlane_frame4_Stress2.pgf};\addlegendentry{loading path 5}
\addplot3+[gray,dashed,thin,no markers] file {chapter5/pgfFigures/pgf_slowWavesPlaneStrain/CylindreDevPlane.pgf};\addlegendentry{initial yield surface}
\newcommand\radius{1.*0.82e8}
\addplot3[dotted,thick] coordinates {(0.75*\radius,-0.75*\radius,0.) (-0.75*\radius,0.75*\radius,0.)};
\addplot3[dotted,thick] coordinates {(0.,-0.75*\radius,0.75*\radius) (0.,0.75*\radius,-0.75*\radius)};
\addplot3[dotted,thick] coordinates {(-0.75*\radius,0.,0.75*\radius) (0.75*\radius,0.,-0.75*\radius)};
\end{axis}
\end{tikzpicture}
%%% Local Variables:
%%% mode: latex
%%% TeX-master: "../../mainManuscript"
%%% End:
}
  \caption{Loading paths through slow simple waves for different starting points on the initial yield surface for the initial condition $\sigma_{22}=0$.}
  \label{fig:slow_path_plane_strains2}
\end{figure}
\begin{figure}[h!]
  \centering
  % \subcaptionbox{Projections of loading paths in ($\sigma_{11},\sigma_{12}$) and ($\sigma_{22},\sigma_{12}$) planes \label{subfig:slowDP_stress3}}{\begin{tikzpicture}[scale=0.9]
\begin{groupplot}[group style={group size=2 by 1,
ylabels at=edge left, yticklabels at=edge left,horizontal sep=3.ex,
xticklabels at=edge bottom,xlabels at=edge bottom},
ymajorgrids=true,xmajorgrids=true,ylabel=$\sigma_{12} \: (Pa)$,
axis on top,scale only axis,width=0.4\linewidth,ymin=0,ymax=68618075.3102588
, every x tick scale label/.style={at={(xticklabel* cs:1.05,0.75cm)},anchor=near yticklabel},colormap={ry}{rgb255(0cm)=(255,255,0);rgb255(1cm)=(255,0,0)}]
\nextgroupplot[xlabel=$\sigma_{11} (Pa)$]
\addplot[mesh,point meta = \thisrow{p},very thick,no markers] table[x=sigma_11,y=sigma_12] {chapter5/pgfFigures/pgf_slowWavesPlaneStrain/DPslowStressPlane_frame0_Stress3.pgf} node[above right,black] {$\textbf{1}$};
\addplot[mesh,point meta = \thisrow{p},very thick,no markers] table[x=sigma_11,y=sigma_12] {chapter5/pgfFigures/pgf_slowWavesPlaneStrain/DPslowStressPlane_frame1_Stress3.pgf} node[above right,black] {$\textbf{2}$};
\addplot[mesh,point meta = \thisrow{p},very thick,no markers] table[x=sigma_11,y=sigma_12] {chapter5/pgfFigures/pgf_slowWavesPlaneStrain/DPslowStressPlane_frame2_Stress3.pgf} node[above right,black] {$\textbf{3}$};
\addplot[mesh,point meta = \thisrow{p},very thick,no markers] table[x=sigma_11,y=sigma_12] {chapter5/pgfFigures/pgf_slowWavesPlaneStrain/DPslowStressPlane_frame3_Stress3.pgf} node[above right,black] {$\textbf{4}$};
\addplot[gray,dashed,thin] table[x=sigma_11,y=sigma_12] {chapter5/pgfFigures/pgf_slowWavesPlaneStrain/DPslow_yield0_s11s12_Stress3.pgf};

\nextgroupplot[colorbar,colorbar style={title= {$ c_s \: (m/s)$},every y tick scale label/.style={at={(2.,-.1125)}} },xlabel=$\sigma_{22}  (Pa)$]
\addplot[mesh,point meta = \thisrow{p},very thick,no markers] table[x=sigma_22,y=sigma_12] {chapter5/pgfFigures/pgf_slowWavesPlaneStrain/DPslowStressPlane_frame0_Stress3.pgf} node[above right,black] {$\textbf{1}$};
\addplot[mesh,point meta = \thisrow{p},very thick,no markers] table[x=sigma_22,y=sigma_12] {chapter5/pgfFigures/pgf_slowWavesPlaneStrain/DPslowStressPlane_frame1_Stress3.pgf} node[above right,black] {$\textbf{2}$};
\addplot[mesh,point meta = \thisrow{p},very thick,no markers] table[x=sigma_22,y=sigma_12] {chapter5/pgfFigures/pgf_slowWavesPlaneStrain/DPslowStressPlane_frame2_Stress3.pgf} node[above right,black] {$\textbf{3}$};
\addplot[mesh,point meta = \thisrow{p},very thick,no markers] table[x=sigma_22,y=sigma_12] {chapter5/pgfFigures/pgf_slowWavesPlaneStrain/DPslowStressPlane_frame3_Stress3.pgf} node[above right,black] {$\textbf{4}$};
\addplot[gray,dashed,thin] table[x=sigma_22,y=sigma_12] {chapter5/pgfFigures/pgf_slowWavesPlaneStrain/DPslow_yield0_s22s12_frame0_Stress3.pgf};

\addplot[gray,dashed,thin] table[x=sigma_22,y=sigma_12] {chapter5/pgfFigures/pgf_slowWavesPlaneStrain/DPslow_yield0_s22s12_frame1_Stress3.pgf};

\addplot[gray,dashed,thin] table[x=sigma_22,y=sigma_12] {chapter5/pgfFigures/pgf_slowWavesPlaneStrain/DPslow_yield0_s22s12_frame2_Stress3.pgf};

\addplot[gray,dashed,thin] table[x=sigma_22,y=sigma_12] {chapter5/pgfFigures/pgf_slowWavesPlaneStrain/DPslow_yield0_s22s12_frame3_Stress3.pgf};

\end{groupplot}
\end{tikzpicture}
%%% Local Variables:
%%% mode: latex
%%% TeX-master: "../../mainManuscript"
%%% End:
}
  % \subcaptionbox{Loading path in deviatoric plane \label{subfig:slowDP_dev3}}{\begin{tikzpicture}[scale=0.9]
\begin{axis}[width=.75\textwidth,view={135}{35.2643},xlabel=$s_1 $,ylabel=$s_2 $,zlabel=$s_3$,xmin=-1.e8,xmax=1.e8,ymin=-1.e8,ymax=1.e8,axis equal,axis lines=center,axis on top,ztick=\empty]
\addplot3+[Red,very thick,no markers] file {chapter5/pgfFigures/pgf_slowWavesPlaneStrain/DPslowDevPlane_frame0_Stress3.pgf};
\addplot3+[Blue,very thick,no markers] file {chapter5/pgfFigures/pgf_slowWavesPlaneStrain/DPslowDevPlane_frame1_Stress3.pgf};
\addplot3+[Orange,very thick,no markers] file {chapter5/pgfFigures/pgf_slowWavesPlaneStrain/DPslowDevPlane_frame2_Stress3.pgf};
\addplot3+[Purple,very thick,no markers] file {chapter5/pgfFigures/pgf_slowWavesPlaneStrain/DPslowDevPlane_frame3_Stress3.pgf};
\addplot3+[gray,dashed,thin,no markers] file {chapter5/pgfFigures/pgf_slowWavesPlaneStrain/CylindreDevPlane.pgf};
\end{axis}
\end{tikzpicture}
%%% Local Variables:
%%% mode: latex
%%% TeX-master: "../../mainManuscript"
%%% End:
}
  {\begin{tikzpicture}[scale=0.9]
\begin{groupplot}[group style={group size=2 by 1,
ylabels at=edge left, yticklabels at=edge left,horizontal sep=3.ex,
xticklabels at=edge bottom,xlabels at=edge bottom},
ymajorgrids=true,xmajorgrids=true,ylabel=$\sigma_{12} \: (Pa)$,
axis on top,scale only axis,width=0.4\linewidth,ymin=0,ymax=68618075.3102588
, every x tick scale label/.style={at={(xticklabel* cs:1.05,0.75cm)},anchor=near yticklabel},colormap={ry}{rgb255(0cm)=(255,255,0);rgb255(1cm)=(255,0,0)}]
\nextgroupplot[xlabel=$\sigma_{11} (Pa)$]
\addplot[mesh,point meta = \thisrow{p},very thick,no markers] table[x=sigma_11,y=sigma_12] {chapter5/pgfFigures/pgf_slowWavesPlaneStrain/DPslowStressPlane_frame0_Stress3.pgf} node[above right,black] {$\textbf{1}$};
\addplot[mesh,point meta = \thisrow{p},very thick,no markers] table[x=sigma_11,y=sigma_12] {chapter5/pgfFigures/pgf_slowWavesPlaneStrain/DPslowStressPlane_frame1_Stress3.pgf} node[above right,black] {$\textbf{2}$};
\addplot[mesh,point meta = \thisrow{p},very thick,no markers] table[x=sigma_11,y=sigma_12] {chapter5/pgfFigures/pgf_slowWavesPlaneStrain/DPslowStressPlane_frame2_Stress3.pgf} node[above right,black] {$\textbf{3}$};
\addplot[mesh,point meta = \thisrow{p},very thick,no markers] table[x=sigma_11,y=sigma_12] {chapter5/pgfFigures/pgf_slowWavesPlaneStrain/DPslowStressPlane_frame3_Stress3.pgf} node[above right,black] {$\textbf{4}$};
\addplot[gray,dashed,thin] table[x=sigma_11,y=sigma_12] {chapter5/pgfFigures/pgf_slowWavesPlaneStrain/DPslow_yield0_s11s12_Stress3.pgf};

\nextgroupplot[colorbar,colorbar style={title= {$ c_s \: (m/s)$},every y tick scale label/.style={at={(2.,-.1125)}} },xlabel=$\sigma_{22}  (Pa)$]
\addplot[mesh,point meta = \thisrow{p},very thick,no markers] table[x=sigma_22,y=sigma_12] {chapter5/pgfFigures/pgf_slowWavesPlaneStrain/DPslowStressPlane_frame0_Stress3.pgf} node[above right,black] {$\textbf{1}$};
\addplot[mesh,point meta = \thisrow{p},very thick,no markers] table[x=sigma_22,y=sigma_12] {chapter5/pgfFigures/pgf_slowWavesPlaneStrain/DPslowStressPlane_frame1_Stress3.pgf} node[above right,black] {$\textbf{2}$};
\addplot[mesh,point meta = \thisrow{p},very thick,no markers] table[x=sigma_22,y=sigma_12] {chapter5/pgfFigures/pgf_slowWavesPlaneStrain/DPslowStressPlane_frame2_Stress3.pgf} node[above right,black] {$\textbf{3}$};
\addplot[mesh,point meta = \thisrow{p},very thick,no markers] table[x=sigma_22,y=sigma_12] {chapter5/pgfFigures/pgf_slowWavesPlaneStrain/DPslowStressPlane_frame3_Stress3.pgf} node[above right,black] {$\textbf{4}$};
\addplot[gray,dashed,thin] table[x=sigma_22,y=sigma_12] {chapter5/pgfFigures/pgf_slowWavesPlaneStrain/DPslow_yield0_s22s12_frame0_Stress3.pgf};

\addplot[gray,dashed,thin] table[x=sigma_22,y=sigma_12] {chapter5/pgfFigures/pgf_slowWavesPlaneStrain/DPslow_yield0_s22s12_frame1_Stress3.pgf};

\addplot[gray,dashed,thin] table[x=sigma_22,y=sigma_12] {chapter5/pgfFigures/pgf_slowWavesPlaneStrain/DPslow_yield0_s22s12_frame2_Stress3.pgf};

\addplot[gray,dashed,thin] table[x=sigma_22,y=sigma_12] {chapter5/pgfFigures/pgf_slowWavesPlaneStrain/DPslow_yield0_s22s12_frame3_Stress3.pgf};

\end{groupplot}
\end{tikzpicture}
%%% Local Variables:
%%% mode: latex
%%% TeX-master: "../../mainManuscript"
%%% End:
}
  {\begin{tikzpicture}[scale=0.9]
\begin{axis}[width=.75\textwidth,view={135}{35.2643},xlabel=$s_1 $,ylabel=$s_2 $,zlabel=$s_3$,xmin=-1.e8,xmax=1.e8,ymin=-1.e8,ymax=1.e8,axis equal,axis lines=center,axis on top,ztick=\empty]
\addplot3+[Red,very thick,no markers] file {chapter5/pgfFigures/pgf_slowWavesPlaneStrain/DPslowDevPlane_frame0_Stress3.pgf};
\addplot3+[Blue,very thick,no markers] file {chapter5/pgfFigures/pgf_slowWavesPlaneStrain/DPslowDevPlane_frame1_Stress3.pgf};
\addplot3+[Orange,very thick,no markers] file {chapter5/pgfFigures/pgf_slowWavesPlaneStrain/DPslowDevPlane_frame2_Stress3.pgf};
\addplot3+[Purple,very thick,no markers] file {chapter5/pgfFigures/pgf_slowWavesPlaneStrain/DPslowDevPlane_frame3_Stress3.pgf};
\addplot3+[gray,dashed,thin,no markers] file {chapter5/pgfFigures/pgf_slowWavesPlaneStrain/CylindreDevPlane.pgf};
\end{axis}
\end{tikzpicture}
%%% Local Variables:
%%% mode: latex
%%% TeX-master: "../../mainManuscript"
%%% End:
}
  \caption{Loading paths through slow simple waves for different starting points on the initial yield surface for the initial condition $\sigma_{22}=1.3 \times 10^{8} \: Pa$.}
  \label{fig:slow_path_plane_strains3}
\end{figure}

If on the other hand, the hardening parameter is raised to $C=1\times10^{10} \: Pa$, the loading path slightly differ.
To illustrate this, the same plane strain problems as before are considered by using the same driving parameters and initial conditions for both fast and slow waves.
The resulting integral curves are here depicted in deviatoric plane only in order to highlight particular behaviors.

First, the fast waves integral curves are shown in figure \ref{fig:fast_H}.
As before, the integral curves follow the initial yield surface but then branch off to reach a direction of pure tensile/compression loading.
\begin{figure}[h!]
  \centering
  %\tikzset{cross/.style={cross out, draw=black, minimum size=2*(#1-\pgflinewidth), inner sep=0pt, outer sep=0pt},cross/.default={2.5pt}}
\begin{tikzpicture}[spy using outlines={rectangle, magnification=3, size=2.cm, connect spies},scale=0.9]
\begin{axis}[width=.75\textwidth,view={135}{35.2643},xlabel=$s_1 $,ylabel=$s_2 $,zlabel=$s_3$,xmin=-1.e8,xmax=1.e8,ymin=-1.e8,ymax=1.e8,axis equal,axis lines=center,axis on top,xtick=\empty,ytick=\empty,ztick=\empty,every axis y label/.style={at={(rel axis cs:0.,.5,-0.65)}, anchor=west}, every axis x label/.style={at={(rel axis cs:0.5,.,-0.65)}, anchor=east}, every axis z label/.style={at={(rel axis cs:0.,.0,.18)}, anchor=north},legend style={at={(.2,.68)}}]
\node[below] at (1.1e8,0.,0.) {$\sigma^y$};
\node[above] at (-1.1e8,0.,0.) {$-\sigma^y$};
\draw (1.e8,0.,0.) node[cross,rotate=10] {};
\draw (-1.e8,0.,0.) node[cross,rotate=10] {};
\node[white]  at (0,0.,1.1e8) {};
\addplot3[Red,thick,arrows along my path] file {chapter5/pgfFigures/pgf_HfastWavesPlaneStrai/DPfastDevPlane_frame0_Stress0.pgf};\addlegendentry{loading path 1}
\addplot3[Blue,thick,arrows along my path] file {chapter5/pgfFigures/pgf_HfastWavesPlaneStrai/DPfastDevPlane_frame1_Stress0.pgf};\addlegendentry{loading path 2}
\addplot3[Orange,thick,arrows along my path] file {chapter5/pgfFigures/pgf_HfastWavesPlaneStrai/DPfastDevPlane_frame2_Stress0.pgf};\addlegendentry{loading path 3}
\addplot3[Purple,thick,arrows along my path] file {chapter5/pgfFigures/pgf_HfastWavesPlaneStrai/DPfastDevPlane_frame3_Stress0.pgf};\addlegendentry{loading path 4}
\addplot3[Green,thick,arrows along my path] file {chapter5/pgfFigures/pgf_HfastWavesPlaneStrai/DPfastDevPlane_frame4_Stress0.pgf};\addlegendentry{loading path 5}
\addplot3[Duck,thick,arrows along my path] file {chapter5/pgfFigures/pgf_HfastWavesPlaneStrai/DPfastDevPlane_frame5_Stress0.pgf};\addlegendentry{loading path 6}
\addplot3+[gray,dashed,thin,no markers] file {chapter5/pgfFigures/pgf_HfastWavesPlaneStrai/CylindreDevPlane.pgf};\addlegendentry{initial yield surface}
\newcommand\radius{1.*0.82e8}
\addplot3[dotted,thick] coordinates {(0.75*\radius,-0.75*\radius,0.) (-0.75*\radius,0.75*\radius,0.)};
\addplot3[dotted,thick] coordinates {(0.,-0.75*\radius,0.75*\radius) (0.,0.75*\radius,-0.75*\radius)};
\addplot3[dotted,thick] coordinates {(-0.75*\radius,0.,0.75*\radius) (0.75*\radius,0.,-0.75*\radius)};
\begin{scope}
\spy[black,size=1.75cm] on (6.9,3.3) in node [fill=none] at (9.5,5.5);
\end{scope}
\end{axis}
\end{tikzpicture}
%%% Local Variables:
%%% mode: latex
%%% TeX-master: "../../mainManuscript"
%%% End:

  \caption{Fast simple wave solutions of the plane strain problems $C=1\times10^{10} \: Pa$.}
  \label{fig:fast_H}
\end{figure}
Moreover, the visible cusps in the loading paths $2$, $3$, $4$ and $5$, which already arise in the solutions based on a lower hardening modulus, indicates that the direction of pure tensile/compression is "aimed" inside a fast simple wave under plane strain. 

The increase in hardening parameter allows to eliminate the integration issues that occur for a lower one.
As a result, the stress paths followed inside slow simple waves depicted in figure \ref{fig:slow_H} lead to stress state lying furhter outside of the initial elastic convex.
\begin{figure}[h!]
  \centering
  {\phantomsubcaption \label{subfig:slow_H1}}
  {\phantomsubcaption \label{subfig:slow_H2}}
  {\phantomsubcaption \label{subfig:slow_H3}}
  % \tikzset{cross/.style={cross out, draw=black, minimum size=2*(#1-\pgflinewidth), inner sep=0pt, outer sep=0pt},cross/.default={2.5pt}}
\begin{tikzpicture}
  \newcommand\radius{1.*0.82e8}
  \begin{groupplot}[group style={group size=3 by 1,
      % ylabels at=edge left, yticklabels at=edge left,
      horizontal sep=-90pt,
      % xticklabels at=edge bottom,xlabels at=edge bottom
    },
    ymajorgrids=true,xmajorgrids=true,%enlargelimits=0,
    axis on top,scale only axis,%width=0.4\linewidth,
    view={135}{35.2643},xlabel=$s_1 $,ylabel=$s_2 $,zlabel=$s_3$,
    xmin=-1.e8,xmax=1.e8,ymin=-1.e8,ymax=1.e8,axis equal,axis lines=center,
    xtick=\empty,ytick=\empty,ztick=\empty,
    every axis y label/.style={at={(rel axis cs:0.,.5,-0.65)}, anchor=west},
    every axis x label/.style={at={(rel axis cs:0.5,.,-0.65)}, anchor=east},
    every axis z label/.style={at={(rel axis cs:0.,.0,.18)}, anchor=north}]
    %%%
    \nextgroupplot[width=.5\textwidth,
    % title={(a) $\sigma_{22}=-1.3\times 10^{8} \: Pa$.}
    ]
    \node[below] at (axis cs:1.1e8,0.,0.) {$\sigma^y$};
    \node[above] at (axis cs:-1.1e8,0.,0.) {$-\sigma^y$};
    \draw (axis cs:1.e8,0.,0.) node[cross,rotate=10] {};
    \draw (axis cs:-1.e8,0.,0.) node[cross,rotate=10] {};
    %\node[white]  at (0,0.,.7e8) {};
    \addplot3[Red,thick,arrows along my path] file {section7/pgfFigures/pgf_HslowWavesPlaneStrai/DPslowDevPlane_frame0_Stress1.pgf};
    \addplot3[Blue,thick,arrows along my path] file {section7/pgfFigures/pgf_HslowWavesPlaneStrai/DPslowDevPlane_frame1_Stress1.pgf};
    \addplot3[Orange,thick,arrows along my path] file {section7/pgfFigures/pgf_HslowWavesPlaneStrai/DPslowDevPlane_frame2_Stress1.pgf};
    \addplot3[Purple,thick,arrows along my path] file {section7/pgfFigures/pgf_HslowWavesPlaneStrai/DPslowDevPlane_frame3_Stress1.pgf};
    \addplot3[Duck,thick,arrows along my path] file {section7/pgfFigures/pgf_HslowWavesPlaneStrai/DPslowDevPlane_frame4_Stress1.pgf};
    \addplot3+[gray,dashed,thin,no markers] file {section7/pgfFigures/pgf_HslowWavesPlaneStrai/CylindreDevPlane.pgf};
    \addplot3[dotted,thick] coordinates {(0.75*\radius,-0.75*\radius,0.) (-0.75*\radius,0.75*\radius,0.)};
    \addplot3[dotted,thick] coordinates {(0.,-0.75*\radius,0.75*\radius) (0.,0.75*\radius,-0.75*\radius)};
\addplot3[dotted,thick] coordinates {(-0.75*\radius,0.,0.75*\radius) (0.75*\radius,0.,-0.75*\radius)};
    %%%
\nextgroupplot[width=.5\textwidth,
%title={(b) $\sigma_{22}=0$.},
legend style={at={(1.14,.15)},legend columns=3}]
    \node[below] at (axis cs:1.1e8,0.,0.) {$\sigma^y$};
    \node[above] at (axis cs:-1.1e8,0.,0.) {$-\sigma^y$};
    \draw (axis cs:1.e8,0.,0.) node[cross,rotate=10] {};
    \draw (axis cs:-1.e8,0.,0.) node[cross,rotate=10] {};
    %\node[white]  at (0,0.,1.1e8) {};
    \addplot3[Red,thick,arrows along my path] file {section7/pgfFigures/pgf_HslowWavesPlaneStrai/DPslowDevPlane_frame0_Stress2.pgf};%\âddlegendentry{loading path 1}
    \addplot3[Blue,thick,arrows along my path] file {section7/pgfFigures/pgf_HslowWavesPlaneStrai/DPslowDevPlane_frame1_Stress2.pgf};%\âddlegendentry{loading path 2}
    \addplot3[Orange,thick,arrows along my path] file {section7/pgfFigures/pgf_HslowWavesPlaneStrai/DPslowDevPlane_frame2_Stress2.pgf};%\âddlegendentry{loading path 3}
    \addplot3[Purple,thick,arrows along my path] file {section7/pgfFigures/pgf_HslowWavesPlaneStrai/DPslowDevPlane_frame3_Stress2.pgf};%\âddlegendentry{loading path 4}
    \addplot3[Duck,thick,arrows along my path] file {section7/pgfFigures/pgf_HslowWavesPlaneStrai/DPslowDevPlane_frame4_Stress2.pgf};%\âddlegendentry{loading path 5}
    \addplot3+[gray,dashed,thin,no markers] file {section7/pgfFigures/pgf_HslowWavesPlaneStrai/CylindreDevPlane.pgf};%\âddlegendentry{initial yield surface}
    \addplot3[dotted,thick] coordinates {(0.75*\radius,-0.75*\radius,0.) (-0.75*\radius,0.75*\radius,0.)};
    \addplot3[dotted,thick] coordinates {(0.,-0.75*\radius,0.75*\radius) (0.,0.75*\radius,-0.75*\radius)};
    \addplot3[dotted,thick] coordinates {(-0.75*\radius,0.,0.75*\radius) (0.75*\radius,0.,-0.75*\radius)};
    %%%
    \nextgroupplot[width=.5\textwidth,
    % title={(c) $\sigma_{22}=1.3\times 10^{8} \: Pa$.}
    ]
    \node[below] at (axis cs:1.1e8,0.,0.) {$\sigma^y$};
    \node[above] at (axis cs:-1.1e8,0.,0.) {$-\sigma^y$};
    \draw (axis cs:1.e8,0.,0.) node[cross,rotate=10] {};
    \draw (axis cs:-1.e8,0.,0.) node[cross,rotate=10] {};
    %\node[white]  at (0,0.,1.1e8) {};
    \addplot3[Red,thick,arrows along my path] file {section7/pgfFigures/pgf_HslowWavesPlaneStrai/DPslowDevPlane_frame0_Stress3.pgf};
    \addplot3[Blue,thick,arrows along my path] file {section7/pgfFigures/pgf_HslowWavesPlaneStrai/DPslowDevPlane_frame1_Stress3.pgf};
    \addplot3[Orange,thick,arrows along my path] file {section7/pgfFigures/pgf_HslowWavesPlaneStrai/DPslowDevPlane_frame2_Stress3.pgf};
    \addplot3[Purple,thick,arrows along my path] file {section7/pgfFigures/pgf_HslowWavesPlaneStrai/DPslowDevPlane_frame3_Stress3.pgf};
    \addplot3[Duck,thick,arrows along my path] file {section7/pgfFigures/pgf_HslowWavesPlaneStrai/DPslowDevPlane_frame4_Stress3.pgf};
    \addplot3+[gray,dashed,thin,no markers] file {section7/pgfFigures/pgf_HslowWavesPlaneStrai/CylindreDevPlane.pgf};
    \addplot3[dotted,thick] coordinates {(0.75*\radius,-0.75*\radius,0.) (-0.75*\radius,0.75*\radius,0.)};
    \addplot3[dotted,thick] coordinates {(0.,-0.75*\radius,0.75*\radius) (0.,0.75*\radius,-0.75*\radius)};
    \addplot3[dotted,thick] coordinates {(-0.75*\radius,0.,0.75*\radius) (0.75*\radius,0.,-0.75*\radius)};
  \end{groupplot}
\end{tikzpicture}






























%%% Local Variables:
%%% mode: latex
%%% TeX-master: "../../mainManuscript"
%%% End:
  \caption{Slow simple wave solutions of the previous problems with $C=1\times10^{10} \: Pa$.}
  \label{fig:slow_H}
\end{figure}
Moreover, for every initial values of $\sigma_{22}$ considered in figures \ref{fig:slow_H}\subref{subfig:slow_H1}, \ref{fig:slow_H}\subref{subfig:slow_H2} and \ref{fig:slow_H}\subref{subfig:slow_H3}, the slopes of integral curves no longer break but smoothly reach the direction of pure shear in the deviatoric plane.

\begin{remark}
  The use of a higher hardening parameter for plane stress problem also leads to smoother paths in the deviatoric plane for slow waves.
  On the other hand, the integral curves associated to fast waves under plane stress (slightly) move away from the yield surface up to a direction of pure shear.
  Nevertheless, at that point numerical difficulties occur due to an indeterminacy of the loading functions.
\end{remark}
%%%% REMARQUES A LA VOLEE
% It is shown in \cite{Ting73} that the plastic celerities only depends on $\tens{\sigma}/\norm{\tens{\sigma}}$ so that they are constant along rays of the stress space $(\sigma_{11}, \sigma_{22}, \sigma_{12})$. Thus, look at the loading path along integral curves and see the evolution of celerities.


%%% Local Variables:
%%% mode: latex
%%% TeX-master: "../mainManuscript"
%%% End:


% \section{Towards a two-dimensional elastoplastic Riemann solver}
% \label{sec:ep_Riemman_solver}
% %%%%%%%%%%%%%%%%%%%%%%%%%%%%%%%%%%%%%%%%%%%
% Revenir sur ce qui a été fait dans le chapitre du point de vue analytique: ok
% Parler de Lin et Ballman (dire à la fin que c'est la seule tentative à notre connaissance pour prendre en compte la structure dans un schéma volume finis)
% Parler de l'importance de connaitre la solution analytique

% Parler des résultats numériques qui donnent quelques indices sur la structure de la solution

% Parler de pistes qui peuvent être suivies en vue de prendre en compte ces informations
%% Utiliser une forme quasi-linéaire en valeur propres de cauchy
%% Lin et Ballman
%%%%%%%%%%%%%%%%%%%%%%%%%%%%%%%%%%%%%%%%%%%


% | prendre en compte ce qu'on vient de trouver pour construire un solveur à la Lin et Ballman
% v si on cherche la solution d'un problème de Picard.
The physical structures emphasized in this chapter enable a better understanding of the propagation of waves in two-dimensional elastoplastic medium, although further investigations are required.
On the other hand, the loading paths followed in fast and slow simple waves can be used in order to improve the numerical simulation of those problems.

%% Lin et Ballman
First, the approach proposed by Lin and Ballman \cite{Lin_et_Ballman} can be generalized by identifying elementary stress paths which allow the solution of Picard problems.
Namely, for a given stationary stress state, one should be able to trace backward integral curves corresponding to slow and fast waves so that initial data are recovered.
Thus, an interative procedure for the solution of Riemann problems could be constructed.

%% Approximate Riemann solver
Second, one can imagine a non-iterative approximate Riemann solver based on the behaviors emphasized above.
Indeed, the analytical and numerical results for both plane strain and plane stress indicate that the fast waves significantly harden the material once the shear stress $\sigma_{12}$ is zero.
In addition, slow waves more notably tend to make the stress state "move away" from the elastic convex.
Therefore, one possibility for the building of an approximate Riemann solver would be to first make an elastic prediction which leads to two situations depending on its projection onto the yield surface:
\begin{itemize}
\item[(i)] if the intersection of the straight line joining the initial stress state $\tens{\sigma}^0$ to the trial stress and the yield surface, say $\widetilde{\tens{\sigma}}$, is such that $\widetilde{\sigma}_{12}\neq 0$, the plastic flow is assumed to be only due to a slow wave.
  Then, the wave can be approximate as a plastic discontinuity operating on several stress components and moving at the constant speed $c_s(\widetilde{\tens{\sigma}})$.
  This situation is illustrated in figure \ref{fig:approx_RP_EP_slow}.
  \begin{figure}[h!]
    \centering
    \tikzset{cross/.style={cross out, draw=black, minimum size=2*(#1-\pgflinewidth), inner sep=0pt, outer sep=0pt},cross/.default={2.5pt}}
\begin{tikzpicture}
  \draw[thick,->] (0.,0.) -- (4.,0.) node [right] {$\sigma_{11}$};
  \draw[thick,->] (0.,0.) -- (0.,4.) node [above] {$\sigma_{12}$};
  % \draw (\x,{sqrt((9.-\x*\x)/3)) -- (\x,{sqrt((9.-\x*\x)/3)});
  \draw [black, very thick, domain=0:3, samples=50] plot (\x,{sqrt((9.-\x*\x)/3)})  ;
  \draw[dashed] (.5,1.) node[below] {$\tens{\sigma}^0$} -- (1.,2.5)node[above] {$\tens{\sigma}^{trial}$};
  \draw (0.72,1.68) node[cross,rotate=10] {};
  \node[above left] at (0.75,1.7) {$\widetilde{\tens{\sigma}}$};
  \draw (0.72,1.68) .. controls (0.82,1.8) and (2.5,2.) .. (3.,3.5) node[above] {\text{loading path}};
  \draw[dotted] (0.72,1.68) -- (3.,2.5) node[right] {\text{approximate path}} ;
  %%
  %%
  \newcommand\shift{8.}
  \draw[thick,->] (0.+\shift,0.) -- (4.+\shift,0.) node [right] {$x_1$};
  \draw[thick,->] (0.+\shift,0.) -- (0.+\shift,4.) node [above] {$t$};
  \draw[very thick] (0+\shift,0) -- (4+\shift,1) node [right] {$c_1$};
  \draw[very thick] (0+\shift,0) -- (4+\shift,2.5) node [right] {$c_2$};
  \draw[thick] (0+\shift,0) -- (4+\shift,3.25) node [right] {$c_s(\widetilde{\tens{\sigma}})$};
  
\end{tikzpicture}
%%% Local Variables:
%%% mode: latex
%%% TeX-master: "../../mainManuscript"
%%% End:

    %\subcaptionbox{Approximation of the characteristic structure with only one fast wave \label{subfig:approx_fast}}{\tikzset{cross/.style={cross out, draw=black, minimum size=2*(#1-\pgflinewidth), inner sep=0pt, outer sep=0pt},cross/.default={2.5pt}}
\begin{tikzpicture}
  \draw[thick,->] (0.,0.) -- (4.,0.) node [right] {$\sigma_{11}$};
  \draw[thick,->] (0.,0.) -- (0.,4.) node [above] {$\sigma_{12}$};
  % \draw (\x,{sqrt((9.-\x*\x)/3)) -- (\x,{sqrt((9.-\x*\x)/3)});
  \draw [black, very thick, domain=0:3, samples=50] plot (\x,{sqrt((9.-\x*\x)/3)})  ;
  \draw (2.,.5) node[left] {$\tens{\sigma}^0$} -- (4.,-.5)node[right] {$\tens{\sigma}^{trial}$};
  \draw (3,0.) node[cross,rotate=10] {};
  \node[below] at (3,0.) {$\widetilde{\tens{\sigma}}$};
  %% paths
  \draw (3.,0.) .. controls (3.5,1.) and (2.5,2.) .. (4.,3.5) node[above] {\text{loading path}};
  \draw[dotted] (3.,0.) -- (3.85,2.5) node[right] {\text{approximate path}} ;
  %%
  %%
  \newcommand\shift{8.}
  \draw[thick,->] (0.+\shift,0.) -- (4.+\shift,0.) node [right] {$x_1$};
  \draw[thick,->] (0.+\shift,0.) -- (0.+\shift,4.) node [above] {$t$};
  \draw[very thick] (0+\shift,0) -- (4+\shift,1) node [right] {$c_1$};
  \draw[thick] (0+\shift,0) -- (4+\shift,1.5) node [right] {$c_f(\widetilde{\tens{\sigma}})$};
  \draw[very thick] (0+\shift,0) -- (4+\shift,2.5) node [right] {$c_2$};
\end{tikzpicture}
%%% Local Variables:
%%% mode: latex
%%% TeX-master: "../../mainManuscript"
%%% End:
}
    \caption{Procedure for the building of an elastic-plastic approximate Riemann-solver for a propagation in the direction $\vect{e}_1$.}
    \label{fig:approx_RP_EP_slow}
  \end{figure}
  Since across the pressure wave one has $\saut{\sigma_{11}}\neq 0$ only, the state lying in the region bounded by this discontinuity and the shear wave is known and is such that  $\saut{\sigma_{11}} = \widetilde{\sigma}_{11}-\sigma_{11}^0$.
  Thus, one can remove the pressure discontinuity and add one plastic discontinuity from the characteristic structure.
  
  %% Revoir cette hypothèse
\item[(ii)] alternatively, if $\widetilde{\sigma}_{12}= 0$ one can assume that only a fast wave occurs, which can be approximated in a similar manner with the celerity $c_f(\widetilde{\tens{\sigma}})$ (see figure \ref{fig:approx_RP_EP_fast}).
  \begin{figure}[h!]
    \centering
    \tikzset{cross/.style={cross out, draw=black, minimum size=2*(#1-\pgflinewidth), inner sep=0pt, outer sep=0pt},cross/.default={2.5pt}}
\begin{tikzpicture}
  \draw[thick,->] (0.,0.) -- (4.,0.) node [right] {$\sigma_{11}$};
  \draw[thick,->] (0.,0.) -- (0.,4.) node [above] {$\sigma_{12}$};
  % \draw (\x,{sqrt((9.-\x*\x)/3)) -- (\x,{sqrt((9.-\x*\x)/3)});
  \draw [black, very thick, domain=0:3, samples=50] plot (\x,{sqrt((9.-\x*\x)/3)})  ;
  \draw (2.,.5) node[left] {$\tens{\sigma}^0$} -- (4.,-.5)node[right] {$\tens{\sigma}^{trial}$};
  \draw (3,0.) node[cross,rotate=10] {};
  \node[below] at (3,0.) {$\widetilde{\tens{\sigma}}$};
  %% paths
  \draw (3.,0.) .. controls (3.5,1.) and (2.5,2.) .. (4.,3.5) node[above] {\text{loading path}};
  \draw[dotted] (3.,0.) -- (3.85,2.5) node[right] {\text{approximate path}} ;
  %%
  %%
  \newcommand\shift{8.}
  \draw[thick,->] (0.+\shift,0.) -- (4.+\shift,0.) node [right] {$x_1$};
  \draw[thick,->] (0.+\shift,0.) -- (0.+\shift,4.) node [above] {$t$};
  \draw[very thick] (0+\shift,0) -- (4+\shift,1) node [right] {$c_1$};
  \draw[thick] (0+\shift,0) -- (4+\shift,1.5) node [right] {$c_f(\widetilde{\tens{\sigma}})$};
  \draw[very thick] (0+\shift,0) -- (4+\shift,2.5) node [right] {$c_2$};
\end{tikzpicture}
%%% Local Variables:
%%% mode: latex
%%% TeX-master: "../../mainManuscript"
%%% End:

    \caption{Procedure for the building of an elastic-plastic approximate Riemann-solver for a propagation in the direction $\vect{e}_1$.}
    \label{fig:approx_RP_EP_fast}
  \end{figure}
  In a similar manner, the state lying between the pressure wave and the plastic disontinuity is known since the elastic wave only carries a jump of $\sigma_{11}$, so that the pressure wave may be removed.
\end{itemize}

Applying the procedures $(i)$ and/or $(ii)$ from both sides of the interface $x_1=0$, the Riemann problem then reduces to a linear problem involving two left-going and two right-going discontinuities which can easily solved.
 
The development of such a heuristic approach, which is based on strong assumptions, enables nevertheless some improvements numerically speaking.
Indeed, rather than considering the propagation of elastic waves, an elastic-plastic Riemann solver for problems in two space dimensions account for both elastic and plastic characteristics.
More specifically, the error introduced by the plastic approximation may be partially balanced by using limiters for elastic and plastic waves.






%%% Local Variables:
%%% mode: latex
%%% TeX-master: "../mainManuscript"
%%% End:



\section{Conclusion}

\subsection{Summary of the chapter}

In this chapter, the characteristic structure of the solution of hyperbolic problems in elastic-plastic solids in two space dimensions has been highlighted.
It is known since the 50s that plastic flow in two-dimensional solids yields two families of waves which speeds depend on the stress state, the slow and fast waves.
As a result, shock and simple waves may occur in an elastoplastic medium even for linear hardening material.
In addition, these plastic waves may have an impact on several stresses in contrast to elastic discontinuities across which one stress component varies, hence the name of combined stress waves.
During the 60s, attention has been paid to simple waves in particular two-dimensional problems thus providing, among other, solutions of Picard problems in elastic-plastic medium undergoing step loadings \cite{Clifton,Ting68,Ting73}. % Idem pour ting ? c'est dit dans l'intro ? voir ce qui est fait dans le 73
The singular nature of such problems lies in the fact that the characteristic structure of the solution depends on the external loading undergone.
Indeed, it has been shown \cite{Clifton} that boundary conditions can lead to plastic flow involving one fast, one slow, or both simple waves.
Therefore, it is crucial to be able to identify typical stress paths followed in each simple waves in order to link the initial data to a given stress state, and subsequently to determine the occurring wave pattern.

$\newline$
%% Lin et Ballman
Based on those works, an iterative Riemann solver \cite{Lin_et_Ballman}, which procedure has been recalled in section \ref{sec:stress_paths_num}, has been developed for the numerical solution of the thin-walled tube problem. 
% L'idée ici c'est de généraliser cette approche pour tous les problèmes 2D
Following this approach, identifying characteristic wave patterns for general elastoplastic problems in two space dimensions should allow to enrich the numerical solution with the knowledge of physics.
For that purpose, the characteristic analysis of two-dimensional problems in elastic-plastic materials with linear isotropic hardening under plane strain and plane stress, in projection in an arbitrary direction of space, has been carried out in section \ref{sec:charac_plast}.
Fast and slow waves are also involved in the solution so that applying the method of characteristic through the simple waves provides a system of ODE.
Integration of this system leads to combined stress paths that are followed between initial and final stress states on the one hand, and to the integral curves in terms of velocity components involving integral along those loading paths on the other hand.
Specializing the ODEs to one direction of a Cartesian grid, it has been shown in section \ref{sec:stress_paths} that the loading paths satisfied through slow and fast waves are perpendicular in the stress space for both plane strain and plaen stress.
Moreover, it has been established that the stress paths exhibit particular behavior in the space $(\sigma_{11},\sigma_{22},\sigma_{12})$, that is $d\sigma_{11}=0$, $d\sigma_{12}=0$ or $d\sigma_{22}=0$, for special values of the components of the acoustic tensor.
These situations are achieved for different stress states depending on wheter the problem involves plane stresses of plane strains as shown in section \ref{sec:stress_paths}.

$\newline$
The complexity of the ODEs derived in section \ref{sec:charac_plast} prevents from identifying all the singularities which may occur along the loading paths.
Hence, the mathematical analysis has been supplemented with numerical results consisting in the integration of stress paths from arbitrary initial stress values lying on the initial yield surface, for the particular direction of space $\vect{e}_1$.

%% Thin-walled tube
First, in section \ref{sec:num_thin-walled} the loading paths resulting from the integration of the ODEs derived in section \ref{sec:charac_plast} have been compared to those of Clifton \cite{Clifton}.
The two different formulations, respectively based on elastoplastic stiffnesses and softnesses, show a good agreement.

%% Cont. planes
Second, some the evolution of stress components across fast and slow waves under plane stress has been looked at in section \ref{sec:num_plane_stress}.
It appears that though the loading paths are rather complex in stress space through a fast wave, the stress evolution in the deviatoric plane is restricted to the initial yield surface until one direction of pure shear is reached.
A singularity then occurs so that the numerical integration cannot be pursued.
On the other hand, the loading paths resulting from the integration of ODEs satisfied inside a slow wave, except that $\sigma_{11}$ varies much less than the other stress components.

%% Def. planes
Third, the plane strain case has been considered in section \ref{sec:num_plane_strain}.
Once again, the integral curves inside a fast wave show complex shapes in stress space, and an evolution restricted to the initial yield surface in the deviatoric plane.
In that case, however, the paths may follow a direction of pure tensile/compression in the latter plane so that the plastic flow is radial. 
In contrast, the slow waves lead to a stress state of pure shear in the deviatoric plane.

\subsection{Towards a two-dimensional elastoplastic Riemann solver}
The physical structures emphasized in this chapter enable a better understanding of the propagation of waves in two-dimensional elastoplastic medium, although further investigations are required.
On the other hand, the loading paths followed in fast and slow simple waves can be used in order to improve the numerical simulation of those problems.

%% Lin et Ballman
One possibility is to generalized the approach proposed by Lin and Ballman \cite{Lin_et_Ballman} based on the clues provided above.
The idea would be to successively assume stationary states of the Riemann problem in terms of stress $\sigma_{11}$, $\sigma_{12}$ and $\sigma_{22}$ in order to build stress paths starting from the initial data.
Namely, considering the direction $\vect{e}_1$, the loading paths followed through a fast wave can be integrated backward starting from the guess state.
Then, different situations may occur:
\begin{itemize}
\item[(1-a)] the curve thus obtained crosses the initial yield surface at a point where $\sigma_{22}$ satisfy the initial data.
  In that case, the elastic dicontinuities led to that stress state so that the characterisitc structure correspond to this depicted in figure \ref{fig:charac}\subref{subfig:charac1}.
\item[(1-b)] if on the other hand the point reached on the initial yield does not satisfy the initial stress $\sigma_{22}$, a fast wave is added in order to browse the initial yield surface until the initial data is recovered.
  This situation is depicted in figure \ref{fig:charac}\subref{subfig:charac2}.
\item[(2-a)] the curve resulting from the reverse integration across a slow wave intersects the plane $\sigma_{12}=0$.
  Then, assuming that the paths of slow waves are symmetric with respect to that plane, a fast wave is added in order to reach the initial yield surface at the initial value of $\sigma_{22}$.
  Indeed, the fast waves have been shown to yield horizontal path in the ($\sigma_{11},\sigma_{12}$) plane in such a way that only that type of wave enables to achieve the initial elastic convex.
  This also corresponds to figure \ref{fig:charac}\subref{subfig:charac2}.
\item[(2-b)] if at last, the guessed state is such that $\sigma_{12}=0$, a fast wave enables to reach the initial yield surface as depicted in figure \ref{fig:charac}\subref{subfig:charac3}.
\end{itemize}

\begin{figure}[h!]
  \centering
  \subcaptionbox{One slow wave \label{subfig:charac1}}{\begin{tikzpicture}
    \draw[thick, ->] (0,0) -- (3.,0) node [right] {$x_1$};
    \draw[thick, ->] (0,0) -- (0,3) node [above] {$t$};
    \draw[very thick] (0,0) -- (3,1) node [right] {$c_1$};
    \draw[very thick] (0,0) -- (3,2) node [right] {$c_2$};
    \foreach \x in {0.1,0.2,...,0.9}
    \draw (0,0) -- (3,2+\x);
    \draw (0,0) -- (3,3) node [right] {$c_s$};
\end{tikzpicture}} \qquad 
  \subcaptionbox{Both simple waves \label{subfig:charac2}}{\begin{tikzpicture}
    \draw[thick, ->] (0,0) -- (3.,0) node [right] {$x_1$};
    \draw[thick, ->] (0,0) -- (0,3) node [above] {$t$};
    \draw[very thick] (0,0) -- (3,1) node [right] {$c_1$};
    \foreach \x in {0.1,0.2,...,0.5}
    \draw (0,0) -- (3,1+\x);
    \draw (0,0) -- (3,1.6) node [right] {$c_f$};
    \draw[very thick] (0,0) -- (3,2) node [right] {$c_2$};
    \foreach \x in {0.1,0.2,...,0.9}
    \draw (0,0) -- (3,2+\x);
    \draw (0,0) -- (3,3) node [right] {$c_s$};
\end{tikzpicture}} \qquad 
  \subcaptionbox{One fast wave \label{subfig:charac3}}{\begin{tikzpicture}
    \draw[thick, ->] (0,0) -- (3.,0) node [right] {$x_1$};
    \draw[thick, ->] (0,0) -- (0,3) node [above] {$t$};
    \draw[very thick] (0,0) -- (3,1) node [right] {$c_1$};
    \foreach \x in {0.1,0.2,...,0.5}
    \draw (0,0) -- (3,1+\x);
    \draw (0,0) -- (3,1.6) node [right] {$c_f$};
    \draw[very thick]  (0,0) -- (3,2) node [right] {$c_2$};
    
\end{tikzpicture}}
  \caption{Characteristic stuctures possibly occuring in two-dimensional elastic-plastic solids.}
  \label{fig:charac}
\end{figure}
Notice however that the above elementary loading paths are based on strong assumptions about the symmetry of the loading paths that have not been shown so far.
As a result, additional work must be performed in order to develop this approach and to introduce it in numerical methods.
Moreover, the hardening of the material may modifiy the behavior of the loading paths and have not been considered yet.
On the other hand, generalize the approach followed in this chapter to more complex hardening models (kinematic, nonlinear etc.) and other yield surfaces would be very interesting for the understanding of physics.
%On the other hand, generalized the approach followed in this chapter to more complex hardening models (kinematic, nonlinear etc.) and other yield surface would be of major interest for the understanding of physics.


%%% Local Variables:
%%% mode: latex
%%% TeX-master: "../mainManuscript"
%%% End:
