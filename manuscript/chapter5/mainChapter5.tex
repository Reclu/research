\chapter{Contribution to the solution elastic-plastic problems in two space dimensions}
%% Faire un historique des formulations faites.
%% Formuler le problème à notre sauce et identifier les cas évoqués en intro en particularisont les modules tangents etc. a ce moment, parler des trajets de chargement et des types d'ondes

\section*{Introduction}
Elastic solver used previously

\cite{Clifton_thesis} development of a method of characteristics in 3 independent variables that is then numerically approximated (Notion of bicharacteristics). Nous on n'en a pas besoin puisqu'on a déja notre schéma numérique. En ravanche, on cherche à résoudre le problème dans une direction donnée pour lequel la méthode des caractéristiques s'applique.

This is for instance the case for the simulation of forming technics that cannot, in general, be modeled in a one-dimensional setting.

Assumptions (isothermal, linear hardenings though extendable, small deformation, cartesian,concave constitutive model,rate independent)

Un intérêt de développer ce qui est fait là est de s'autoriser à limiter les ondes plastique dans des méthodes de haut ordre.
\section{State of the art}
Ce remonte à von Karman en 42 mais on va dater la Biblio à partir de Rakhmatulin (ou Clifton ?!).
\begin{itemize}
%% plane stress
\item \cite{Rakhmatulin} ; \cite{CRISTESCU19591605} : Strong discontinuity; Distinction between plastic waves ($c_1,c_2$) of strong discontinuities, and combined waves ($c_f,c_s$)
%% superimposition of plane wave and shear wave
\item \cite{Bleich}: $\tens{\sigma}=\matrice{\sigma & \tau & \\ \tau & \sigma_{22} & \\ & & \sigma_{22}}$;$\tens{\eps}=\matrice{\eps & \gamma/2 & \\ \gamma/2 &  & \\ & & }$ ideal elastic-plastic continuum ; shock front at p.15 and subsection B + exact solutions ; identification of particular loading cases that lead to different characteristic structures.
%% combined shear and longitudinal waves
\item \cite{Clifton}: $\tens{\sigma}=\matrice{\sigma & \tau & \\ \tau & & \\ & & }$;$\tens{\eps}=\matrice{\eps & \gamma/2 & \\ \gamma/2 &  & \\ & & }$ Not a strong discontinuity + hardening + Emphasis unexpected stress structure (prestressed tubes)
%% superimposition of plane wave and shear wave
\item \cite{Ting68}: $\tens{\sigma}=\matrice{\sigma & \tau & \\ \tau & \sigma_{22} & \\ & & \sigma_{22}}$;$\tens{\eps}=\matrice{\eps & \gamma/2 & \\ \gamma/2 &  & \\ & & }$ with linear hardening materials; identification of particular loading cases that lead to different characteristic structures (p.8). Generalization of \cite{Bleich} tp hardening materials and of \cite{Clifton} to more complex loadings ; Emphasis unexpected stress structure (prestressed tubes)
%% superimposition of plane wave and shear waveS
\item \cite{Ting69}:  $\tens{\sigma}=\matrice{\sigma & \tau_y & \tau_z\\ \tau_y & \sigma_{22} & \\\tau_z & & \sigma_{22}}$;$\tens{\eps}=\matrice{\eps & \gamma_y/2 & \gamma_z/2\\ \gamma_y/2 &  & \\ \gamma_z/2& & }$ ; superimposition of plane wave and shear waveS ; Emphasis unexpected stress structure (prestressed tubes)
\item \cite{Ting73}: c'est pas un peu une review des deux papiers d'avant ?
%% superimposition of plane wave and shear wave
\item \cite{Li_planeStress_EP}:$\tens{\sigma}=\matrice{\sigma & \tau & \\ \tau & \sigma_{22} & \\ & & \sigma_{22}}$;$\tens{\eps}=\matrice{\eps & \gamma/2 & \\ \gamma/2 &  & \\ & & }$
\end{itemize}

Orthogonalité des loading paths \cite{Clifton,Ting68}
\subsection{Characteristic analysis}
We assume the infinitesimal strain tensor can be additively decomposed into an elastic part and a plastic part according to:
\begin{equation}
  \label{eq:ch5_partition}
  \tens{\eps}=\tens{\eps}^e+\tens{\eps}^p
\end{equation}
The elastic part of the infinitesimal strain tensor is:
\begin{equation}
  \label{eq:ch5_elastic_inverse}
  \tens{\eps}^e = \frac{1+\nu}{E} \tens{\sigma} - \frac{\nu}{E} \tr \tens{\sigma} \tens{I}
\end{equation}

\subsubsection*{The general case}
The governing equations of dynamics in elastic-plastic solids are written in the following quasi-linear form:
\begin{equation}
  \Qcb_t + \Absf^i \drond{\Qcb}{x_i} = \Scb \qquad \text{with: }\Absf^i = -\matrice{\tens{0}^2 & \frac{1}{\rho}\tens{I}\otimes\vect{e}_i\\ \Cbb^{ep}\cdot \vect{e}_i & \tens{0}^4}  \label{eq:ch5_quasilinear}
\end{equation}
where $\Qcb=\matrice{\vect{v}\\ \tens{\sigma}}$ and $\Cbb^{ep}=\ddroit{\tens{\sigma}}{\tens{\eps}}=\Cbb - \beta\tens{m}\otimes\tens{m}$ is the tangent modulus with $\tens{m}=\frac{\tens{s}-\tens{Y}}{\norm{\tens{s}-\tens{Y}}}$ is the flow direction and $\beta=\frac{6\mu^2}{3\mu +(C+R')}$ depends on the shear modulus $\mu$ and kinematic or isotropic hardening modulus $C$ and $R'$ (see section \ref{sec:constitutive-equations}). In particular in the arbitrary direction $\vect{n}$:
\begin{equation}
  \Qcb_t + \Jbsf \drond{\Qcb}{x_n} = \Scb  \label{eq:ch5_quasilinear_normal}
\end{equation}
where $x_n=\vect{x}\cdot\vect{n}$ and the Jacobian matrix $\Jbsf=\Absf^in_i$ arises. The left characteristic fields $\{c_K;\Lcb^K\}$ satisfy the following equation:
\begin{equation}
  \label{eq:ch5_eigen_system}
  \vect{\Lc}^K \(\Jbsf - c_K \Ibsf\) = \vect{0}
\end{equation}
As seen in section \ref{sec:characteristic_analysis}, $6$ couples of characteristic speeds $c_K$ and left eigenvectors $\Lcb^K= \[ \vect{v}^K \: , \: \tens{S}^K \]$ are determined based on those of the acoustic tensor $\tens{A}=\vect{n}\cdot\Cbb^{ep}\cdot \vect{n}$, that are $\{\omega^p;\vect{l}^p\}$ for $p=1,2,3$:
\begin{equation}
  \label{eq:ch5_left_eigenfields}
  \left\lbrace \pm \sqrt{\frac{\omega_p}{\rho_0}} ; \quad \[\: \pm \rho_0\sqrt{\frac{\omega_p}{\rho_0}} \vect{l}^p , -\vect{l}^p\otimes \vect{N} \:\]  \right\rbrace ,\quad p=1,2,3
\end{equation}
In addition, three independent left eigenvectors associated to the zero eigenvalue of system \eqref{eq:ch5_quasilinear_normal}, which is of multiplicity $3$, are found by solving:
\begin{equation}
  \label{eq:ch5_null_eigen}
  \tens{\sigma}^K:\(\Cbb^{ep}\cdot  \vect{n}\) =\vect{0},\quad K=1,2,3
\end{equation}

\subsubsection*{Problems in two space dimensions}
%%% CARTESIAN COORDINATES SOMEWHERE
We now focus on the solid domain bounded by $x_1 \times \x_2 \times x_3 \in [0,\infty[ \times ]-\infty,\infty[ \times [-e,e]$ in a Cartesian coordinates system, where $e$ is an arbitrary length.
The solid is subject on the plane $x_1=0$ to a traction force $\vect{T}$ restricted to the $(\vect{e}_1,\vect{e}_2)$ plane, that is $T_3=0$.

First, assuming that all quantities depend only on $x_1$ and $x_2$, as it the case for instance if the velocity $v_3$ vanishes on both ends $x_3=\pm e$, leads to a plane strain case ($\eps_{33}=0$). For such loading conditions, combination of equations \eqref{eq:ch5_partition} and \eqref{eq:ch5_elastic_inverse} and considering $\eps_{33}=0$, allows to write a relation between the stress $\sigma_{33}$ and other components as:
\begin{equation}
  \label{eq:plane_strain_stress33}
  \sigma_{33}=\nu\(\sigma_{11}+\sigma_{22}\) - E\eps^p_{33}
\end{equation}



On the other end, if zero traction forces are supposed at $x_3=\pm e$, a plane stress state holds in the solid ($\sigma_{33}=0$). 


% \subsection{The complexity of elastoplasticity}
% % Bibliography about:
% % - what has been done so far (thin-walled ; plane wave + shear wave)
% % - what is missing (solution of more general problems [i.e plane strain or stress] ; development of approximate Riemann solvers)
% % - what it will allow (comparison with experimental data in order to track plastic shocks) internal structure of plastic shock
% % - what is currently done in fluid or solid mechanics numerically
% Biblio, 
% %\section{State of the art}
% \subsection{The thin-walled tube problem}
% \cite{Clifton} + thesis
% \subsection{Superimposition of plane wave and shear waves}
% Citer Ting
% \section{Plane strain and plane stress problem}
% \subsection{General multi-dimensional formulation}
% % Based on the tangent modulus -> independent on the hardening model
% \subsection{Plane strain problems}
% \subsection{Plane stress problems}
% \section{Characteristic analysis}
% dependance des vitesses caractéristiques à l'angle entre la direction principale de sigma et la direction de propagation, c'est dit dans la thèse de Clifou en page 90.
% \subsection{Structure of the solution}
% \subsection{Integral curves and loading paths}
% \subsection{The plane strain case}
% \subsection{The plane stress case}


\section{Towards an elastoplastic approximate Riemann solver}
Parler du solver de Lin et Ballman qui cause problème dans le cas général mais je ne sais plus pourquoi...(je crois que c'est en lien avec des problèmes de Picard et non Riemann)
On propose ici d'approximer toutes les ondes simples par des discontinuités (motivé par les trajets de chargement tracés au-dessus)
\cite{Lin_et_Ballman}
%%% Local Variables:
%%% mode: latex
%%% TeX-master: "../mainManuscript"
%%% End:
