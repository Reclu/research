\documentclass[10pt,a4paper]{report}
\usepackage[latin1]{inputenc}
\usepackage{amsmath}
\usepackage{amsfonts}
\usepackage{amssymb}
\usepackage{graphicx}
\author{Adrien Renaud}

\begin{document}
\tableofcontents
	\chapter{Introduction}

	    \section{General introduction}
	    \section{Numerical methods in solid mechanics}
	    \section{Goals and purposes}
                
	\chapter{Notion of hyperbolic problems -- Analytical developments }
	        
		\section{Conservation laws of solid mechanics}
        % Context, isothermal, thermodynamical framework, standard generalized material etc.
	        \subsection{Classification of PDE's}
	        \subsection{System of conservation laws}
		        \subsubsection{Homogeneous systems}
		        \subsubsection{Non-homogeneous systems}
                                   
        \section{Solution of hyperbolic systems -- Characteristic analysis}
	        \subsection{Elastic viscoplastic solids}
            \subsection{Elastoplastic solids}
	            \subsubsection{Two-dimensional plane strain problem}
                \subsubsection{Two-dimensional plane stress problem}

            \subsection{Hyperelastic solids}
	            \subsubsection{One-dimensional Saint-Venant-Kirchhoff material}
                \subsubsection{Two-dimensional plane strain problems}

                                   
	\chapter{Extension of the Material Point Method}
		
		\section{The material point method}
		    \subsection{Development of the method}
            % PIC - FLIP - MPM - GIMP - HOMPM/BSMPM: introduction text 
			\subsection{Derivation of the mpm}
	            \subsubsection{MPM discrete equations}
	            \subsubsection{Solution scheme summary}
			\subsection{Shortcomings}
			
		
		\section{Derivation of the discontinuous Galerkin material point method}
			\subsection{The discontinuous Galerkin approximation}
            % Introduction text, DG approximation, DGFEM, main  idea of the introduction of this approx into MPM, mootivations ?             
            \subsection{Derivation of the method}
			% Do not explicit interface flxes yet (not required for numerical analysis) // Also, it is an independant tool used in other methods such as FVM, DGFEM.
			\subsection{DGMPM solution scheme}
			
		\section{Numerical analysis of the DGMPM}
			\subsection{Convergence analysis}
			\subsection{One-dimensional stability analysis}
			\subsection{Two-dimensional stability analysis}

	\chapter{Approximate Riemann solvers for solid mechanics problems}
	
		\section{The godunov method}
		\section{Transverse correction -- The CTU method}
		\section{Linearized geometrical framework}
			\subsection{Elastic-viscoplastic solids}
			\subsection{Elastoplastic solids}
				\subsubsection{Plane strain problems}
				\subsubsection{Plane stress problems}
				
		\section{Hyperelastic solids}
		
		\section{Numerical results}
		
	\chapter{Conclusion}
	
	
\end{document}
