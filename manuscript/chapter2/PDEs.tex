We will start this chapter with general considerations about systems of partial differential equations (basic definitions, existence theorem and the notion of characteristics). More specifically, it will be seen that partial differential equations of high order can be reduced to first order systems of partial differential equations, in which fall equations of dynamics in solid mechanics (see section \ref{sec:solidMech_equations}). Then, the general concepts introduced in section \ref{sec:PDEs} will be applied in the context of solid mechanics conservation laws (section \ref{sec:characteristic_analysis}) in order to derive analytical solutions of one-dimensional problems in section \ref{sec:analytical_results}.
\subsection{General concepts}
A \textit{system of partial differential equations} can be written by means of a vector operator $\vect{\Fc}$ of independent and dependent variables $(x_1,...,x_N)$ and $(u_1,...,u_I)$:
\begin{equation}
  \label{eq:diff_operator}
  \vect{\Fc}\(x_1,...,x_N,u_1,...,u_I,\drond{u_1}{x_1},..., \drond{^Mu_I}{x_N^M}\) = \vect{0}
\end{equation}
The dimension of the system is given by the size of the array $\vect{\Uc}^T=[u_1,...,u_I] \in \Rbb^I$, also referred to as the \textit{unknown vector}. The highest derivative of the unknown vector in the system defines the \textit{order of the system}. In equation \ref{eq:diff_operator} and in what follows, sans-serif symbols refer to matrices while calligraphic symbols stand for column array. Furthermore, the partial derivatives of a quantity $u$ with respect to a variable $x$ may be replaced by $u_x$ when there will be no ambiguity. Making use of index notation and convention of implicit summation over repeated indices, a system of partial differential equations can be written:
\begin{equation*}
  \sum_{k=1}^{N}\Asf_{ij}^p \drond{^p\Uc_j}{x_k^p} + \Bc_i = 0
\end{equation*}
or equivalently, in matrix form:
\begin{equation}
  \label{eq:diff_system_matrix}
  \sum_{k=1}^{N}\tens{\Asf}^p \drond{^p\vect{\Uc}}{x_k^p} + \vect{\Bc} =  \vect{0}
\end{equation}
Coefficients matrices $\tens{\Asf}^p$ and the vector $\vect{\Bc}$ may be functions of independent variables and the unknowns vector ($x_1,...,x_N,\vect{\Uc}$) leading to different types of partial differential systems. Namely, whether those terms are functions of the $x_k$ are not leads respectively to a \textit{linear system with variable coefficients} or to a \textit{linear system with constant coefficients}. The system remains \textit{linear} if $\vect{\Bc}$ depends linearly on $\vect{\Uc}$, and is \textit{semi-linear} if the relation is non-linear. Finally, if $\tens{\Asf}^p$ depends on independent variables, $\vect{\Uc}$ and its derivatives up to order $M-1$, the system is called \textit{quasi-linear}.


The \textit{Initial Value Problem (IVP)} or \textit{Cauchy problem} consists in finding a solution $\vect{\Uc}$ of system \ref{eq:diff_system_matrix} that satisfies a set of given initial conditions. Geometrically speaking, the solution of such a problem can be seen as the building of an hyper-surface surface of $\Rbb^{I+N}$, hence the term of \textit{integral surface} for $\vect{\Uc}$. Such a problem can be reduced to that of solving a first order IVP by using suitable change of variable. To illustrate this property of PDEs, we consider a second order partial differential equation of two independent variables $(x,y)$ for the unknown $u$:
\begin{equation}
  \label{eq:2nd_order_pde}
  F(x,y,u,u_x,u_y,u_{xx},u_{yy},u_{xy})=0
\end{equation}
It is assumed that is equation can be rewritten as:
\begin{equation}
  \label{eq:2nd_order_pde_normal}
  u_{xx}= f(x,y,u,u_x,u_y,u_{yy},u_{xy})
\end{equation}
with associated initial conditions:
\begin{equation}
  \label{eq:2nd_order_pde_ICs}
  u(0,y)= g(y) \quad ; \quad u_x(0,y)= g_1(y)
\end{equation}
Then, applying the following change of variables:
\begin{equation*}
  \label{eq:change_of_variables}
  p = u_x \quad  ; \quad   q = u_{y}  \quad   ; \quad    r = u_{xx} \quad   ; \quad  s = u_{yy} \quad;\quad   t = u_{xy} 
\end{equation*}
and differentiating equation \ref{eq:2nd_order_pde_normal} with respect to $x$:
\begin{equation*}
  \label{eq:r_x}
  u_{xxx}= f_x + f_u u_x + f_{u_x}u_{xx} + f_{u_y}u_{yx} + f_{u_{yy}}u_{yyx}+f_{u_{yx}}u_{yxx}
\end{equation*}
yields the following first order partial differential system:
\begin{equation}
  \label{eq:1st_order_quasilinear_system}
  \begin{aligned}
    u_x  = p \quad & ; \quad    q_x  = p_y \\
    p_x  = r \quad &;\quad     t_x  = r_y \\
    s_x  = t_y \quad &;\quad   r_x  = f_x + f_up + f_p r + f_q p_y + f_s t_y + f_t r_y
  \end{aligned}
\end{equation}
Furthermore, from \ref{eq:2nd_order_pde_ICs} and equation \ref{eq:2nd_order_pde_normal}, initial conditions associated to system \ref{eq:1st_order_quasilinear_system} can be determined:
\begin{equation}
  \label{eq:1st_order_quasilinear_system_ICs}
  \begin{aligned}
    u(0,y) = g(y) \quad & ; \quad     p(0,y) = g_1(y) \\
    q(0,y) = g'(y) \quad & ; \quad    s(0,y) = g''(y) \\
    t(0,y) = g_1'(y) \quad & ; \quad    r(0,y) = f(0,y,g(y),g_1(y),g'(y),g''(y),g_1'(y))
  \end{aligned}
\end{equation}
The solution of problem \ref{eq:1st_order_quasilinear_system}--\ref{eq:1st_order_quasilinear_system_ICs}  is equivalent to that of equation \ref{eq:2nd_order_pde} and the procedure can be extended to more than two variables and higher orders \cite[p.54]{PDEs}. Note that the differentiation of the original PDE leads to a quasi-linear equation. Hence, such a reduction will yield a quasi-linear first order system regardless of the nature of initial equations.

\subsection{Notion of characteristics -- Hyperbolic problems}
The theorem of \textit{Cauchy--Kowalevsky} locally ensures the existence of solutions to a Cauchy problem for partial differential systems and is based on the restrictive requirement of analytic coefficients and initial data (see \cite[p.46]{PDEs}). The case of first order systems, however, only requires continuity and differentiability conditions and lies on the concept of \textit{characteristics}, which makes the developement of a solution procedure more flexible.

\subsubsection*{First order quasi-linear equations}
We consider the first order quasi-linear PDE with independent variables $x$ and $t$:
\begin{equation}
  \label{eq:1st_order_pde}
   a u_x + b u_t  = c
\end{equation}
where coefficients $a$ and $b$ are such that $a^2 + b^2 \neq 0$. The solutions $u^{(i)}(x,t)$ of this equations form a set of surfaces in the space $(x,t,u)$ which tangent planes have normal vectors $\vect{n}^{(i)}=[u^{(i)}_x,u^{(i)}_t,-1]$. According to equation \ref{eq:1st_order_pde}, those normal vectors are perpendicular to $\vect{v}=[a,b,c]$. Therefore, the intersection of all those planes is a straight line which direction is given by $\vect{v}$. This defines \textit{characteristic line elements} in the whole space $(x,t,u)$, that is:
\begin{equation}
  \label{eq:monge_axis}
  \matrice{dx \\ dt \\ du} = \matrice{a \\ b \\c}
\end{equation}
Introduction of a parameter $\xi$ and integration of equation \ref{eq:monge_axis} yields a one-parameter family of \textit{characteristic curves} of the PDE:
\begin{equation*}
  x=x(\xi) \quad ; \quad t=t(\xi) \quad ; \quad u=u(\xi)
\end{equation*}
With these considerations, it is obvious that a characteristic curve is tangent at every point to all the integral surfaces. In other words, an infinity of integral surfaces cross one characteristic curve. The following statements then hold \cite[p.63-64]{Courant}:
\begin{theorem}
  \label{th:integral_surface_generated}
  Every surface $u(x,t)$ generated by a one-parameter family of characteristic curves is an integral surface. Conversely, every integral surface is generated by a one-parameter family of characteristic curves.
\end{theorem}
\begin{theorem}
  \label{th:charac_in_integral_surface}
  Every characteristic curve which has one point in common with an integral surface lies entirely on the integral surface.
\end{theorem}

\subsubsection*{The Cauchy problem}
An unique solution of equation \ref{eq:1st_order_pde} may be found by solving the initial value problem. Introducing a parameter $\xi$, one defines a curve $\Cscr$ in the space $(x,t,u)$ along which we are given initial conditions that do not need to be analytic. We assume that along $\Cscr$
\begin{itemize}
\item the curve is regular, that is $x_\xi^2 + t_\xi^2 \neq 0$
\item coefficients $x(\xi)$, $t(\xi)$ and $u(\xi)$ are continuously differentiable
\item it has an unique projection $\Cscr_0$ on the $(x,t)$ plane.
\end{itemize}
The unique solution of Cauchy problem is the integral surface $u(x,t)$ that contains $\Cscr$.
Along that curve, characteristic curves can be built by integrating equation \ref{eq:monge_axis} with respect to a parameter $\eta$, namely:
\begin{equation}
  \label{eq:charac_on_C}
  x=x(\xi,\eta) \quad ; \quad t=t(\xi,\eta) \quad ; \quad u=u(\xi,\eta) \quad \text{for fixed } \xi
\end{equation}
If the Jacobian of the mapping function $\varphi:(\xi,\eta) \mapsto (x,t)$ does not vanish along $\Cscr$ (\textit{i.e. characteristic curves and $\Cscr$ have different projections in the $(x,t)$ plane}\todo{Plots of geometrical considerations about J and characteristics.}), then the set of characteristic curves thus drawn generates the solution $u(x,t)$ of the Cauchy problem. Indeed, if the Jacobian $J=x_\xi t_\eta - x_\eta t_\xi$ is not zero, the mapping $\varphi$ can be inverted so that $u(\xi,\eta)$ becomes a function of $x$ and $t$ that can be differentiated to yield:
\begin{equation*}
  \ddroit{u}{\eta}= u_x \ddroit{x}{\eta} + u_t \ddroit{t}{\eta} =  a u_x + b u_t
\end{equation*}
which, according to equation \ref{eq:monge_axis}, corresponds to the partial differential equation \ref{eq:1st_order_pde}. If however, $J=0$ along the initial curve, $\Cscr$ is a characteristic curve and infinitely many integral surfaces contain it, which involves that Cauchy problem has infinitely many solutions\todo{ensure uniqueness;reduction of PDE to an ODE -> integration involves arbitrary constant ; \cite{Kythe_PDEs}}. These concepts extend to the case of first order non-linear PDEs with more than two independent variables \cite[Chapter~2]{Courant}.

%Hence the method of characteristics !!

\subsubsection*{First order quasi-linear systems}
We now move one the more general case of quasi-linear systems by considering the matrix form:
\todo{see \cite[p.172]{Courant}, notion of free curves if a matrix is non-singular}
\begin{equation}
  \label{eq:1st_order_quasilinear_syst}
  \Absf^t\(x,t,\vect{\Uc}\) \: \vect{\Uc}_t + \Absf^x\(x,t,\vect{\Uc}\)\: \vect{\Uc}_x + \vect{\Bc} = \vect{0}
\end{equation}
in which one of the matrices, say $\Absf^x$, is non-singular. Furthermore, the coefficients $\Asf^{(p)}_{ij}$ are supposed to be continuously differentiable. La question sous-jacente au problème de Cauchy est de savoir s'il est possible, en ayant prescrit des conditions nitiales en U le long d'une courbe, de trouver $u_x$ de sorte à ce que la PDE soit satisfaite le long de C (ou plutot sur la bande ainsi définie puisqu'on traite les dérivées)

System \ref{eq:1st_order_quasilinear_syst} can be solved for the $\vect{\Uc}_x$ at a given point $(x_0,t_0)$ providing that the matrix $\Absf^x$ is non-singular at that point. If it is, then the line $x=x_0$ is a characteristic line otherwise it is \textit{non-characteristic}. 


the Cauchy initial value problem consisting in finding $\vect{\Uc}_x$ for given initial values of $\vect{\Uc}$ on a parametrized curve $\Cscr:\phi(x,t)=0$, such that equation \ref{eq:1st_order_quasilinear_syst} is satisfied along $\Cscr$ (along a strip plutôt). 

The initial curve is also assumed to satisfy the following regularity requirement $\norm{\nablav \phi} \ne 0$, where $\nablav (\bullet)$ denotes the gradient operator in the $(x,t)$ plane (\textit{i.e. one unique projection $\Cscr_0$ in the $(x,t)$ plane}). Moreover, it is also assumed that  $\phi_x\ne 0$.
Domain of dependence, uniqueness.\todo{see \cite[p.424]{Courant}}

%%%%%%%%%%%%%%%%%%%%%%%%%%%% 1st version of the section
$\newline$
Since the unknown vector is viewed as a function of $\phi(x,t)$ along $\Cscr$, the following property holds:% interior derivative
\begin{equation}
  \label{eq:interior_derivative}
  \vect{\Uc}_x\phi_t - \vect{\Uc}_t\phi_x= \vect{0}
\end{equation}
The differentiation of a function $f$ along a curve $\phi(x,t)$ ($\norm{\nablav \phi} \ne 0$, $\phi_x\ne 0$), namely $\alpha f_x + \beta f_t$, is interior if $\alpha \phi_x +\beta\phi_t =0$. In particular:
\begin{equation}
  %\label{eq:interior_derivative}
  \phi_t f_x -\phi_xf_t 
\end{equation}
is an interior derivative $\Rightarrow \phi_t \phi_x -\phi_x \phi_t =0$. Determinant of the Jacobian.
\paragraph{Proof :} If we write the unknown vector as $\vect{\Uc}=\vect{\Wc}(\phi(x,t))$, then by using the chain rule the partial derivatives of $\vect{\Uc}$ read:
\begin{equation*}
  \vect{\Uc}_t = \vect{\Wc}' \: \phi_t \quad ; \quad \vect{\Uc}_x = \vect{\Wc}' \: \phi_x
\end{equation*}
Elimination of $\vect{\Wc}'$ then leads to:
\begin{equation*}
  \vect{\Uc}_t = \vect{\Uc}_x \frac{\phi_t}{\phi_x} 
\end{equation*}
which can be rewritten:
\begin{equation*}
  \vect{\Uc}_t \phi_x = \vect{\Uc}_x \phi_t
\end{equation*}
This ends the proof.

Equation \ref{eq:interior_derivative} defines a relation between $\vect{\Uc}_t$ and $\vect{\Uc}_x$ that allows to write the original system \ref{eq:1st_order_quasilinear_syst} with $\vect{\Uc}_x$ only:
\begin{equation}
  \label{eq:cauchy_IVP_for_ux}
  \( \lambda\tens{\Isf} + \tens{\Asf} \) \vect{\Uc}_x + \vect{\Bc} = \vect{0} 
\end{equation}
where $\lambda = \phi_t/\phi_x$ and $\tens{\Isf}$ is the identity matrix. It is then obvious that equation \ref{eq:cauchy_IVP_for_ux} admits a unique solution if and only if the determinant of the system is non-zero, namely:
\begin{equation}
  \label{eq:characteristic_determinant}
  D=\abs{\lambda\tens{\Isf} + \tens{\Asf}} \ne 0
\end{equation}
where D is called the \textit{characteristic determinant} of system \ref{eq:1st_order_quasilinear_syst}. \cite[Page~172,Page~77]{Courant}
% The complete theory of partial differential equations can be found in \cite{Courant}. In what fillows, we will restrict our attention to first order partial differential equations. 
% Suppose we are interested in describing the variation of a physical quantity depending on two independant variables which can be either space or time variables $u(x,y)$.
% \cite[Chapter~3]{Courant};\cite[Chapter~5]{Courant}
% General form of partial differential equations:
% \begin{equation}
%   \label{eq:general_pde}
%   F\(x,y,...,u,u_x,u_y,...,u_{xx},u_{xy},u_{yy},... \) = 0
% \end{equation}
% in which $F$ denotes a combination of partial derivatives of $u$ up to a given order $n$, and the subscript on $u$ stands for partial differentiation $u_x=\drond{u}{x}$. The partial differential equation \ref{eq:general_pde} is said to be of order $n$ if the highest derivative order is of order $n$. If the function $F$ depends linearly on its variables, the partial differential equation is said to be \textit{linear}. On the other hand, if the funcion $F$ depends linearly on the highest derivatives of $u$ only then the differential equation is \textit{quasi-linear}.

% Definitions: Quasi-linear or Linear equations, order.
% The same definitions holds if the functions $F$ and $u$ are vectors:
% \begin{equation}
%   \label{eq:general_pde_system}
%   \vect{F}\(x,y,...,\vect{u},\vect{u}_x,\vect{u}_y,...,\vect{u}_{xx},\vect{u}_{xy},\vect{u}_{yy},... \) = 0
% \end{equation}
% In matrix form:
% \begin{equation}
%   \label{eq:matrix_pde_system}
%   \tens{A}^x \vect{u}_x + \tens{A}^y \vect{u}_y + \vect{b}= \vect{0}
% \end{equation}
% where the $m \times m$ matrices $\tens{A}^i$ may depend on $\vect{u}$ (quasi-linear).

% In what follows we will restrict to first order partial differential equations. Equation \ref{eq:general_pde} may be \textit{semilinear} or \textit{linear} depending on whether the right-hand side depends on the unknown functions or not.
% Linear,quasi linear, semi linear 
% General pde form - order ?- linear quasi linear - interior differentiation




%p.172 pour les sytèmes linéaires d'EPD elliptiques et hyperboliques à deux variables indépendantes. (p.173 pour n variables)
%\cite{Toro},\cite{Leveque} pour la forme générale des systèmes hyperboliques du premier ordre. Voir \cite{Courant} pour l'étude des problèmes à plus de deux variables indépendantes qui pourrait justifier qu'on se ramène systématiquement à un problème à deux variables indépendantes pour l'analyse caractéristique.



%%% Local Variables:
%%% mode: latex
%%% TeX-master: "../mainManuscript"
%%% End:
