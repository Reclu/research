%As in many other physics suh as electromagnetism, thermodynamics or fluid mechanics, solid mechanics is based on \textit{partial differential equations} or \textit{PDEs} in order to desribe phenomena occuring and that leads to the variation of given quantities (\textit{temperature, pressure etc.}) in space and time. Those equations can be classified into three types which governs the nature of the solution. In this section, we focus on this classification in order to reduce the scope of the study to hyperbolic problems and to develop in the following the construction of solutions to those problems.
\subsection{Generalities}
Let us assume the differential operator:
\begin{equation}
  \label{eq:diff_operator}
  F\(x_1,...,x_N,u_1,...,u_I,\drond{u_1}{x_1},..., \drond{^Mu_I}{x_N^M}\) = 0
\end{equation}
that corresponds to a \textit{system of partial differential equations} of dimension $I$ for the dependent variables $u_i$. This system is said of order $M$ in the $N$ independent variables meaning that the highest derivative of dependent variables with respect to the $x_k$ is of order $M$. With index notation, and by using the convention of implicit summation over repeated indices, system \ref{eq:diff_operator} reads:
\begin{equation}
  \label{eq:diff_system_matrix_index}
  \Asf_{ij}^p \drond{^p\Uc_j}{x_k^p} + \Bc_i = 0
\end{equation}
or equivalently, in matrix form:
\begin{equation}
  \label{eq:diff_system_matrix}
  \tens{\Asf}^p \drond{^p\vect{\Uc}}{x_k^p} + \vect{\Bc} =  \vect{0}
\end{equation}
In equations \ref{eq:diff_system_matrix_index} and \ref{eq:diff_system_matrix} as in what follows, sans-serif symbols refer to matrices while calligraphic symbols stand for column array. Furthermore, the partial derivatives of an arbitrary quantity with respect to a given variable $\drond{q}{y}$ may be replaced by $q_y$ when there will be no ambiguity. The coefficients matrices $\tens{\Asf}^p$ and the vector $\vect{\Bc}$ may be functions of independent variables and the unknowns vector ($x_k,\vect{\Uc}$) leading to different types of partial differential systems. Namely, whether those terms are functions of the $x_k$ are not leads respectively to a \textit{linear system with variable coefficients} or to a \textit{linear system with constant coefficients}. the system remains \textit{linear} if $\vect{\Bc}$ depends linearly on $\vect{\Uc}$, and is \textit{semi-linear} if the relation is non-linear. Finally, if $\tens{\Asf}^p$ depends on independent variables, $\vect{\Uc}$ and its derivatives up to order $M-1$, the system is called \textit{quasi-linear}.

\subsection{The existence theorem of Cauchy and Kowalewsky}
%Due to Cauchy and findable in \cite[Chapter~1]{Courant}
Let us assume that a system of $I$ partial differential equations of order $M$ depending on the independant variables $x_1,...,x_N$ can be written in the form:
\begin{equation}
  \label{eq:normal_form}
  \drond{^M\Uc_i}{x_1^M} = \Fc_i\(x_1,...,x_N,\Uc_1,...,\Uc_I,\drond{\Uc_1}{x_1},..., \drond{^M\Uc_I}{x_N^M} \)
\end{equation}
where the functions $\Fc_i$ do not depend on the left-hand side.

% where $\Fc_i$ are functions that depend analytically on its variables (i.e. they can be expanded into power series in all these variables converging in a suitably small domain, which may be assumed to contain the origin x_k=0 \Uc_i=0)
The \textit{initial value problem} or \textit{Cauchy's problem} consists in constructing a solution $\vect{\Uc}$ of system \ref{eq:normal_form}, that can be seen as a surface of $\Rbb^{N+I}$, for which initial conditions on the other derivatives of $\vect{\Uc}$ in terms of arbitrary functions are satisfied at $x_1=0$. Namely, if one prescribes arbitrary functions $\Gc_i^k(x_2,...,x_N)$ with $(i=1,...,I \: \text{and } k=0,....M-1)$ on the plane $x_1=0$:
\begin{equation}
  \label{eq:IC_Cauchy}
  \begin{aligned}
    & \Uc_i(0,x_2,...,x_N) = \Gc_i(x_2,...,x_N) \\
    & ... \\
    & \drond{^{M-1}}{x_1^{M-1}}\Uc_i(0,x_2,...,x_N) = \Gc_i^{M-1}(x_2,...,x_N) 
  \end{aligned}
\end{equation}
The theorem states the Cauchy's problem admits one and only one solution $\vect{\Uc}$ providing that the functions $\Fc_i$ and initial data $\Gc_i$ are analytical. The proof of this theorem can be found in \cite[Chapter~1]{Courant}. It is worth noticing that the restrictions assumed on the analytical nature of functions $\Fc_i,\Gc_i$ has been relaxed since the early works of Cauchy and Kowalewsky.


%This problem can be seen as the problem of finding a set of solutions depending on constant of integration for ordinary differential equations.
Only mention a suitable change of variable that allows to reduce a partial differential equation of order $N$ to a first order quasi-linear system. 
We now focus on equations of two independant variables $(x,y)$ without loss of generality. In order to illustrate the aforementioned approach, we consider the first order partial differential equation:
\begin{equation}
  \label{eq:1st_order_pde}
  F(x,y,u,u_x,u_y)=0
\end{equation}
and suppose that equation \ref{eq:1st_order_pde} can be solved for $u_x$ with the following form:
\begin{equation}
  \label{eq:1st_order_pde_normal}
  u_x = f(x,y,u,u_y)
\end{equation}
The Cauchy's problem thus consists in finding a solution $u(x,y)$ of equation \ref{eq:1st_order_pde_normal} that coincides with a prescribed function $u(x,y)=g(y)$ at $x=0$. It follows that by differenciating $u(0,y)$ with respect to $y$ yields $u_y=g'(y)$. We then introduce the convenient notation $v=u_x\: ; \: w = u_x$ so that equation \ref{eq:1st_order_pde_normal} reads:
\begin{equation}
  \label{eq:1st_order_pde_wvu}
  v = f(x,y,u,w)
\end{equation}
Then, noting that partial differential equations \ref{eq:1st_order_pde_wvu} allows to evaluate an initial value for $v$, the differentiation of \ref{eq:1st_order_pde_wvu} leads to the following first order quasi-linear system:
\begin{equation}
  \label{eq:1st_order_quasilinear_system}
  \begin{aligned}
    u_x & = v \\
    w_x & = v_y \\
    v_x &= f_x + f_u v + f_w w_x  
  \end{aligned}
\end{equation}
with associated initial conditions:
\begin{equation}
  \label{eq:1st_order_quasilinear_system_IC}
  \begin{aligned}
    u(0,y) &= g(y)  \\
    w(0,y) &= g'(y) \\
    v(0,y) &= f(0,y,g(y),g'(y))
  \end{aligned}
\end{equation}
which is equivalent to the original problem. Indeed, combination of the two first equations of system \ref{eq:1st_order_quasilinear_system} and integration with respect to $x$ yields:
\begin{equation*}
  u_y = w + \alpha(y)  
\end{equation*}
The first initial condition in \ref{eq:1st_order_quasilinear_system_IC} then allows to eliminates the function $\alpha(y)$ since $u_y(0,y)=w(0,y)$. Moreover, the integration of the last equation in \ref{eq:1st_order_quasilinear_system} with respect to $x$ leads to:
\begin{equation*}
  u_x = f(x,y,u,w) + \beta(y)
\end{equation*}
in which the function $\beta$ can also be eliminated by considering that the original partial differnetial equation holds for $x=0$.
An important result comes form the consideration of a second order partial differential equation of two idenpendant variables:
\begin{equation}
  \label{eq:2nd_order_pde}
  F(x_1,x_2,u,\drond{u}{x_1},\drond{u}{x_2},\drond{^2u}{x^2_1},\drond{^2u}{x^2_2},\ddrond{u}{x_1}{ x_2})=0
\end{equation}
that can be solved for $\drond{^2u}{x^2_1}$:
\begin{equation}
  \label{eq:2nd_order_pde_normal}
  \drond{^2u}{x^2_1} = f(x_1,x_2,u,\drond{u}{x_1},\drond{u}{x_2},\drond{^2u}{x^2_2},\ddrond{u}{x_1}{ x_2})
\end{equation}
The initial value problem consists now in determining a surface $u(x_1,x_2)$ of $\Rbb^3$ that satisfies at $x_1=0$:
\begin{equation}
  \label{eq:IC_Cauchy_2ndorder}
  u(0,x_1)=g(x_2) \quad \text{and} \quad \drond{u}{x_1}(0,x_2)=g^1(x_2)
\end{equation}
It is then obvious that in order to solve the Cauchy's problem in the case of second order pde's, more than one arbitrary function sepcifying the initial data are required. 
\subsection{Equivalence of second order pde's and first order quasi-linear systems}
\cite[p.43]{Courant}It follows from the generalization of the above problems
See also paragraph 3. p.47 in [Courant]
\subsection{The notion of characteristics}
\cite[p.55]{Courant}
%%% On se limite aux edp du premier ordre à deux variables indépendantes car comme on le verra dans la suite, sous certainces conditions, les équations de la mécanique des solides se ramènent à ce cas précis. Pour une vue plus générale sur les edp, on peut se référer à [Courant]. Forme générale d'une edp, parler d'un système d'edp pour lequel F et u sont des vecteurs. Définitions de système linéaire, quasi-linéaire et semi-linéaire.
We consider the first order partial differential equations system written for given quantities $\Uc_i$ (for the reduction of high order pde's to a system of 1st order quasi linear equations see \cite[Chapter~1 -- Section~7 -- p.40]{Courant}):
\begin{equation}
  \label{eq:general_pde}
  \drond{\Uc_i}{t} + \Asf^k_{ij}\drond{\Uc_j}{x_k} + \Bc_i= 0 \quad k=1,...,M \:;\: i=1,...,I
\end{equation}
or in matrix form:
\begin{equation}
  \label{eq:general_pde_matrix}
  \drond{\vect{\Uc}}{t} + \tens{\Asf}^k \drond{\vect{\Uc}}{x_k} + \vect{\Bc} = \vect{0}\quad k=1,2,...,M
\end{equation}
where $\vect{\Bc}$ and $\tens{\Asf}^k$ respectively a column vector of length $I$ and $I\times I$ matrices. In what follows, bold sans-serif symbols denote matrices while calligraphic symbols stand for column array. In equations \ref{eq:general_pde} and \ref{eq:general_pde_matrix}, $t,x_1,...,x_M$ and $\Uc_i$ are respectively independent and dependent variables. More specifically, $t$ denotes the time variable and the $x_k$ stand for space variable such that $M$ is the space dimension of the problem. In the general case the matrices $\tens{\Asf}^k$ may be functions of $\vect{\Uc}$ and independent variables, the system is then \textit{quasi-linear}. On the other hand, if $\tens{\Asf}^k=\tens{\Asf}^k(t,x_k)$ and $\vect{\Bc}=\vect{\Bc}(t,x_k)$, the system is \textit{linear with variable coefficients}, otherwise it is \textit{linear with constant coefficients}. The last situation is the one in which $\vect{\Bc}=\vect{\Bc}(\vect{\Uc},t,x_k)$, then if $\vect{\Bc}$ depends linearly on $\vect{\Uc}$, the system remains \textit{linear}, but if it depends non-linearly on $\vect{\Uc}$ the system is called \textit{semi-linear} \cite[Chapter~5]{Courant},\cite[Chapter~2]{Toro}.

$\newline$
We now restrict the problem to the case of \textbf{linear systems} in one space variable in order to introduce the notions of \textit{hyperbolicity} and \textit{characteristics}, system \ref{eq:general_pde_matrix} hence reads:
\begin{equation}
  \label{eq:1d_pde_matrix}
  \vect{\Uc}_t + \tens{\Asf}\vect{\Uc}_x + \vect{\Bc} = \vect{0} 
\end{equation}
Let us consider the Cauchy's initial value problem consisting in finding $\vect{\Uc}_x$ for given initial values of $\vect{\Uc}$ (\textit{i.e. $\vect{\Uc}(x,t=0)$ is known}) on a parametrized curve $\Cc$ of the $(x,t)$ plane, namely by the function $\phi(x,t)=0$, such that equation \ref{eq:1d_pde_matrix} is satisfied on $\Cc$. The curve $\Cc$ is assumed to satisfy the following regularity requirement $\norm{\nablav \phi} \ne 0$, where $\nablav (\bullet)$ denotes the gradient operator. Since the unknown vector is viewed as a function of $\phi(x,t)$, the following property holds (also assume $\phi_x\ne 0$):
\begin{equation}
  \label{eq:interior_derivative}
  \vect{\Uc}_x\phi_t - \vect{\Uc}_t\phi_x= \vect{0}
\end{equation}

\paragraph{Proof :} If we write the unknown vector as $\vect{\Uc}=\vect{\Wc}(\phi(x,t))$, then by using the chain rule the partial derivatives of $\vect{\Uc}$ read:
\begin{equation*}
  \vect{\Uc}_t = \vect{\Wc}' \: \phi_t \quad ; \quad \vect{\Uc}_x = \vect{\Wc}' \: \phi_x
\end{equation*}
Elimination of $\vect{\Wc}'$ then leads to:
\begin{equation*}
  \vect{\Uc}_t = \vect{\Uc}_x \frac{\phi_t}{\phi_x} 
\end{equation*}
which can be rewritten:
\begin{equation*}
  \vect{\Uc}_t \phi_x = \vect{\Uc}_x \phi_t
\end{equation*}
This ends the proof.

Equation \ref{eq:interior_derivative} defines a relation between $\vect{\Uc}_t$ and $\vect{\Uc}_x$ that allows to write the original system \ref{eq:1d_pde_matrix} with $\vect{\Uc}_x$ only:
\begin{equation}
  \label{eq:cauchy_IVP_for_ux}
  \( \lambda\tens{\Isf} + \tens{\Asf} \) \vect{\Uc}_x + \vect{\Bc} = \vect{0} 
\end{equation}
where $\lambda = \phi_t/\phi_x$ and $\tens{\Isf}$ is the identity matrix. It is then obvious that equation \ref{eq:cauchy_IVP_for_ux} admits a unique solution if and only if the determinant of the system is non-zero, namely:
\begin{equation}
  \label{eq:characteristic_determinant}
  D=\abs{\lambda\tens{\Isf} + \tens{\Asf}} \ne 0
\end{equation}
where D is called the \textit{characteristic determinant} of system \ref{eq:1d_pde_matrix}. \cite[Page~172,Page~77]{Courant}
% The complete theory of partial differential equations can be found in \cite{Courant}. In what fillows, we will restrict our attention to first order partial differential equations. 
% Suppose we are interested in describing the variation of a physical quantity depending on two independant variables which can be either space or time variables $u(x,y)$.
% \cite[Chapter~3]{Courant};\cite[Chapter~5]{Courant}
% General form of partial differential equations:
% \begin{equation}
%   \label{eq:general_pde}
%   F\(x,y,...,u,u_x,u_y,...,u_{xx},u_{xy},u_{yy},... \) = 0
% \end{equation}
% in which $F$ denotes a combination of partial derivatives of $u$ up to a given order $n$, and the subscript on $u$ stands for partial differentiation $u_x=\drond{u}{x}$. The partial differential equation \ref{eq:general_pde} is said to be of order $n$ if the highest derivative order is of order $n$. If the function $F$ depends linearly on its variables, the partial differential equation is said to be \textit{linear}. On the other hand, if the funcion $F$ depends linearly on the highest derivatives of $u$ only then the differential equation is \textit{quasi-linear}.

% Definitions: Quasi-linear or Linear equations, order.
% The same definitions holds if the functions $F$ and $u$ are vectors:
% \begin{equation}
%   \label{eq:general_pde_system}
%   \vect{F}\(x,y,...,\vect{u},\vect{u}_x,\vect{u}_y,...,\vect{u}_{xx},\vect{u}_{xy},\vect{u}_{yy},... \) = 0
% \end{equation}
% In matrix form:
% \begin{equation}
%   \label{eq:matrix_pde_system}
%   \tens{A}^x \vect{u}_x + \tens{A}^y \vect{u}_y + \vect{b}= \vect{0}
% \end{equation}
% where the $m \times m$ matrices $\tens{A}^i$ may depend on $\vect{u}$ (quasi-linear).

% In what follows we will restrict to first order partial differential equations. Equation \ref{eq:general_pde} may be \textit{semilinear} or \textit{linear} depending on whether the right-hand side depends on the unknown functions or not.
% Linear,quasi linear, semi linear 
% General pde form - order ?- linear quasi linear - interior differentiation




%p.172 pour les sytèmes linéaires d'EPD elliptiques et hyperboliques à deux variables indépendantes. (p.173 pour n variables)
%\cite{Toro},\cite{Leveque} pour la forme générale des systèmes hyperboliques du premier ordre. Voir \cite{Courant} pour l'étude des problèmes à plus de deux variables indépendantes qui pourrait justifier qu'on se ramène systématiquement à un problème à deux variables indépendantes pour l'analyse caractéristique.



%%% Local Variables:
%%% mode: latex
%%% TeX-master: "../mainManuscript"
%%% End:
