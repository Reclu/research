%As in many other physics suh as electromagnetism, thermodynamics or fluid mechanics, solid mechanics is based on \textit{partial differential equations} or \textit{PDEs} in order to desribe phenomena occuring and that leads to the variation of given quantities (\textit{temperature, pressure etc.}) in space and time. Those equations can be classified into three types which governs the nature of the solution. In this section, we focus on this classification in order to reduce the scope of the study to hyperbolic problems and to develop in the following the construction of solutions to those problems.


Suppose we are interested in describing the variation of a physical quantity depending on two variables which can be either space or time variables $u(x,y)$.

\cite[Chapter~3]{Courant}
\begin{equation}
  \label{eq:general_pde}
  F\(x,y,u_x,u_y,u_{xx},u_{xy},u_{yy},... \) = g
\end{equation}
in which $F$ denotes a combination of partial derivatives of $u$ up to a given order $n$, and the subscript on $u$ stands for partial differentiation $u_x=\drond{u}{x}$. Linear,quasi linear, semi linear 
General pde form - order ?- linear quasi linear - interior differentiation


%%% Local Variables:
%%% mode: latex
%%% TeX-master: "../mainManuscript"
%%% End:
