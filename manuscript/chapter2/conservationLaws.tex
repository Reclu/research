% Context, isothermal, thermodynamical framework, standard generalized material etc.
In this section, the balance equations of dynamic solid mechanics problems ; Lagrangian laws description
\cite{Kluth,Simo}
\subsection{Kinematic laws}
Deformation of a body between a reference state to a deformed state, smooth function $\vect{\phi}(\X,t)$ or $\phi_i(\X,t)$ where $\X$ is the coordinate vector of a given \textit{material point} or \textit{particle}. The velocity is given by the partial derivative $\drond{\phi_i(\X,t)}{t}=\dot{\phi_i}(\X,t)$ since $\X$ does not depend on $t$. The deformation gradient is $\tens{F} = \drond{\vect{\phi}}{\X}$ $F_{i\alpha} = \drond{\phi_i}{X_\alpha}=\phi_{i,\alpha}$ corresponds to the change of an element vector $\vect{dx}=\tens{F}\vect{dX}$. The displacement vector is $\vect{U}= \vect{\phi}(\X,t)-\vect{X}$ and its gradient $\tens{H}=\tens{F}-\tens{I}$ ; right Cauchy-Green tensor $\tens{C}=\tens{F}\tens{F}^T$ ; Green-Lagrange tensor $\tens{E}=\frac{1}{2}(\tens{C}-\tens{I})$. HPP if $\norm{\U}<<1$ and in that case
\begin{equation*}
  \tens{C}=(\H+\I)(\H^T + \I) = \H \H^T +\H + \H^T + \I
\end{equation*}
\begin{equation*}
  \tens{E}=\frac{1}{2}(\H \H^T +\H + \H^T) \approx \frac{1}{2}(\H + \H^T) = \tens{\eps} 
\end{equation*}
by neglecting second order terms. Linearized geometrical framework.
\subsection{Balance equations}
Either start from Newton an identify terms or introduce duality between strains and stress within thermodynamical framework...
\subsection{Thermodynamical framework}
% \subsubsection{Homogeneous systems}
% \subsubsection{Non-homogeneous systems}
\subsection{The general formulation}
Quasi-linear system




%%% Local Variables:
%%% mode: latex
%%% TeX-master: "../mainManuscript"
%%% End:
