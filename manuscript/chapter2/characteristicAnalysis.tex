The eigenspaces of hyperbolic problems defined in the previous section will be now investigated. As we shall see, the characteristic structure of those problems may lead to different type of waves propagating within a medium. Finally, existing analytical solutions of one-dimensional problems \cite{Wang} will be reviewed, and that of a one-dimensional problem involving a hyperelastic \textit{Saint-Venant-Kirchhoff} will be developed in order to illustrate the identified wave structures
\subsection{Characteristic structure of solutions}
Since hyperbolic problems are related to wave propagation phenomena, the quasi-linear form \eqref{eq:quasilinear_pi_v} will be studied in the direction $\vect{N}=\[\vect{e}_1,\vect{e}_2,\vect{e}_3\]$ for finite and infinitesimal deformations. In this direction, equation \eqref{eq:quasilinear_pi_v} reads \todo{Multidimensional spliting \cite[Chapter~3]{Toro};\cite[p.425-426]{Leveque}}:
\begin{equation}
  \label{eq:quasilinear_normal}
  \Absf^t\drond{\Wcb}{t} + N_\alpha \Absf^\alpha \drond{\Wcb}{X_N} = \Scb
\end{equation}
where $X_N = \vect{X}\cdot \vect{N}$ and the matrices expressions are:
\begin{equation}
  \Absf^t=\matrice{\rho_0\tens{I} & \tens{0}^3 \\ \tens{0}^3  & \drond{\tens{F}}{\tens{\Pi}} } \quad ; \quad N_\alpha \Absf^\alpha = -\matrice{ \tens{0}^2 & \tens{I}\otimes\vect{N} \\ \tens{I}\otimes\vect{N} & \tens{0}^4}
\end{equation}
with $\tens{0}^p$, a $p-th$ order zero tensor. Alternatively, the system \eqref{eq:quasilinear_normal} can be multiplied by the inverse matrix of $\Absf^t$:
\begin{equation}
  \label{eq:quasilinear_normal}
  \drond{\Wcb}{t} + \Jbsf_N \drond{\Wcb}{X_N} = {(\Absf^t)}^{-1}\Scb = \tilde{\Scb}
\end{equation}
in which the \textit{Jacobian matrix} $\Jbsf_N$ is:
\begin{equation}
  \label{eq:jacobian_HE}
  \Jbsf_N=-\matrice{\tens{0}^2 & \frac{1}{\rho_0}\tens{I}\otimes \vect{N} \\  \drond{\tens{\Pi}}{\tens{F}}\cdot\vect{N} & \tens{0}^4 }
\end{equation}
for hyperelastic materials and:
\begin{equation}
  \label{eq:jacobian_HPP}
  \Jbsf_N=-\matrice{\tens{0}^2 & \frac{1}{\rho}\(\frac{\tens{I}\otimes \vect{n}+\tens{I}\boxtimes \vect{n}}{2}\) \\  \drond{\tens{\sigma}}{\tens{\eps}}\cdot\vect{n}& \tens{0}^4 }
\end{equation}
for the linearized geometrical framework. In equation \eqref{eq:jacobian_HPP}, the operator $\tens{I}\boxtimes\vect{n}$ is the transpose of second and third index of the classical tensor product arising due to the symmetric gradient of $\vect{v}$: $\tens{I}\boxtimes\vect{n}=\delta_{ij}n_k \vect{e}_i\otimes \vect{e}_k\otimes\vect{e}_j$. The characteristic structure of the problem is given by the eigenvalues $\lambda_k$ and left eigenvectors $\Lcb^k$ of the Jacobian matrix:
\begin{equation}
  \label{eq:eigen_system}
  \vect{\Lc}^k\cdot \(\Jbsf_N - \lambda_k \Ibsf\) = \vect{0}
\end{equation}
where $\Ibsf$ is the identity matrix and $\vect{\Lc}^k= \[ \vect{v}^K \: , \: \tens{\Pi}^K \]$. 
\subsubsection*{Hyperelasticity}
For hyperelastic materials, the eigensystem \eqref{eq:eigen_system} reads:
\begin{equation}
  \label{eq:eigen_left}
  \left\lbrace
  \begin{aligned}
    & -\tens{\Pi}^k:\drond{\tens{\Pi}}{\tens{F}}\cdot  \vect{N} - \lambda_k  \vect{v}^k =\vect{0} \\
    & -\frac{1}{\rho_0}\vect{v}^k\otimes\vect{N} - \lambda_k \tens{\Pi}^k = \tens{0}
  \end{aligned}\right.
\end{equation}
Substitution of $\tens{\Pi}$ in the first equation leads to:
\begin{equation*}
 (\vect{v}^k\otimes\vect{N}):\drond{\tens{\Pi}}{\tens{F}}\cdot\vect{N} - \rho_0\lambda^2_k \vect{v}^k = \tens{0}
\end{equation*}
which is equivalent to the \textit{acoustic tensor} $A_{ij}=N_\alpha H_{i\alpha j \beta}  N_\beta$ left eigensystem:
\begin{equation}
  \label{eq:acoustic_eigen_system}
 v_i^k \(  N_\alpha H_{i\alpha j \beta}  N_\beta - \rho_0 \lambda_k^2 \delta_{ij} \)  =0
\end{equation}
Equation \eqref{eq:acoutsic_eigen_system} leads in the general case to $3$ couples eigenvalues--eigenvectors $(\omega_p,\vect{l}^p) \: p=1,2,3$. Note that the symetry of the acoustic tensor implies that its left and right eigenvectors are identical. Furthermore, it is obvious that $3$ eigenvalues $\omega$ will yield $6$ eigenvalues $\lambda=\pm\sqrt{\frac{\omega}{\rho_0}}$. In addition, the equation equation of system \eqref{eq:eigen_left} gives the expression of the stress component the Jacobian left eigenvectors:
\begin{equation}
  \label{eq:stress_left_eigen}
  \tens{\Pi}^k = -\frac{1}{\lambda_k\rho_0}\vect{v}^k\otimes\vect{N}
\end{equation}
Finally the eigenstructure of the quasi-linear form is defined by the eigenfields:
\begin{equation}
  \label{eq:HE_eigenfields}
    \left\lbrace \pm \sqrt{\frac{\omega^p}{\rho_0}} ; \quad <\: \pm \rho_0\sqrt{\frac{\omega^p}{\rho_0}} \vect{l}^p , -\frac{\vect{l}^p\otimes \vect{n}+\vect{n}\otimes \vect{l}^p}{2} \:>  \right\rbrace ,\quad p=1,2,3
\end{equation}
% Look at eigenvalues // eigen vectors // hyperbolicity condition // Legendre-Hadamard condition etc. -> system of scalar advection equations 
\subsection{Linear problems: contact waves}
\subsection{Non-linearities: simple and shock waves}
\subsection{Integral curves}
\subsection{The Rankine-Hugoniot condition}


%%% Local Variables:
%%% mode: latex
%%% TeX-master: "../mainManuscript"
%%% End:
