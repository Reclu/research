The eigenspaces of hyperbolic problems defined in the previous section will be now investigated. As we shall see, the characteristic structure of those problems may lead to different type of waves propagating within a medium. Finally, existing analytical solutions of one-dimensional problems \cite{Wang} will be reviewed, and that of a one-dimensional problem involving a hyperelastic \textit{Saint-Venant-Kirchhoff} will be developed in order to illustrate the identified wave structures
\subsection{Characteristic structure of solutions}
Since hyperbolic problems are related to wave propagation phenomena, the quasi-linear form \eqref{eq:quasilinear_pi_v} will be studied in the direction $\vect{N}=\[\vect{e}_1,\vect{e}_2,\vect{e}_3\]$ for finite and infinitesimal deformations. In this direction, equation \eqref{eq:quasilinear_pi_v} reads \todo{Multidimensional spliting \cite[Chapter~3]{Toro};\cite[p.425-426]{Leveque}}:
\begin{equation}
  \label{eq:quasilinear_normal}
  \Absf^t\drond{\Wcb}{t} + N_\alpha \Absf^\alpha \drond{\Wcb}{X_N} = \vect{0}
\end{equation}
where $X_N = \vect{X}\cdot \vect{N}$. Give expressions of matrices, find a way to do that in a generic way


% Look at eigenvalues // eigen vectors // hyperbolicity condition // Legendre-Hadamard condition etc. -> system of scalar advection equations 
\subsection{Linear problems: contact waves}
\subsection{Non-linearities: simple and shock waves}
\subsection{Integral curves}
\subsection{The Rankine-Hugoniot condition}


%%% Local Variables:
%%% mode: latex
%%% TeX-master: "../mainManuscript"
%%% End:
