The quasi-linear forms written above enable the particularization of the theory of first order quasi-linear systems developed in section \ref{sec:PDEs} to solid mechanics problems. Once the characteristic analysis of systems \eqref{eq:HE_quasilinear}, \eqref{eq:HPP_quasi-linear} and \eqref{eq:EP_quasilinear} will be carried out, a particular type of IVP which is of particular interest in this manuscript will be introduced.

%The theory of first order quasi-linear system is now applied to solid dynamics through the characteristic analysis of systems \eqref{eq:HE_quasilinear}, \eqref{eq:HPP_quasi-linear} and \eqref{eq:EP_quasilinear} is performed in this section. Second, a particular type of IVP that will be considered in the rest of the manuscript is introduced. 
\subsection{Characteristic structure of solutions}
For the sake of simplicity studies of finite deformation and linearized geometrical frameworks will be condensed in this part by using the generic notation of stress $\tens{S}$ and vectors written in the reference configuration. Furthermore, instead of considering multi-dimensional conservation laws systems, we will focus without loss of generality on quasi-linear form \eqref{eq:HE_quasilinear} projected on an arbitrary direction $\vect{N}=\[\vect{E}_1,\vect{E}_2,\vect{E}_3\]$ \cite[p.425-426]{Leveque}. In this direction, one has:
\begin{equation}
  \label{eq:normal_quasi}
  \Qcb_t + \Jbsf \drond{\Qcb}{X_N} = \Scb
\end{equation}
where $X_N=\vect{X}\cdot\vect{N}$ and the \textit{Jacobian matrix} $\Jbsf = \Absf^\alpha N_\alpha$ of dimension $m$ arise.
In three dimensions, the non-symmetrical PK1 tensor in quasi-linear form \eqref{eq:HE_quasilinear} yields a Jacobian matrix of dimension $m=3+9$ while systems \eqref{eq:HPP_quasi-linear} and \eqref{eq:EP_quasilinear} involving Cauchy tensor, lead to $m=3+6$.
The characteristic analysis of system \eqref{eq:normal_quasi} is therefore equivalent to that of linear combinations of matrices $\Absf^\alpha$. With the previous developments, the Jacobian matrix reads:
\begin{equation}
  \label{eq:jacobian_generic}
  \Jbsf=-\matrice{\tens{0}^2 & \frac{1}{\rho_0}\tens{I}\otimes \vect{N} \\  \tilde{\Hbb}\cdot\vect{N} & \tens{0}^4 }
\end{equation}
in which $\tilde{\Hbb}$ is either the hyperelastic or elastoplastic tangent modulus, or the elastic stiffness tensor depending on the case considered. The characteristic structure of the problem is given by the $m$ eigenvalues $c_K$ and associated left eigenvectors $\Lcb^K= \[ \vect{v}^K \: , \: \tens{S}^K \]$ of the Jacobian matrix satisfying:
\begin{equation}
  \label{eq:eigen_system}
  \vect{\Lc}^K\cdot \(\Jbsf - c_K \Ibsf\) = \vect{0}
\end{equation}
where $\Ibsf$ is the $m\times m$ identity matrix. Thus, for non-zero eigenvalues one gets:
\begin{subequations}
  \begin{alignat}{1}
    \label{eq:eigen_left_stress}
    & -\tens{S}^K:\(\tilde{\Hbb}\cdot  \vect{N}\) - c_K  \vect{v}^K =\vect{0} \\
    \label{eq:eigen_left_velo}
    & -\frac{1}{\rho_0}\vect{v}^K\otimes\vect{N} - c_K \tens{S}^K = \tens{0}
  \end{alignat}
\end{subequations}
Substitution of $\tens{S}^K$ obtained from \eqref{eq:eigen_left_velo} in \eqref{eq:eigen_left_stress} leads to:
\begin{equation}
  \label{eq:acoustic_eigen}
 (\vect{v}^K\otimes\vect{N}):\(\tilde{\Hbb}\cdot  \vect{N}\) - \rho_0c_K^2 \vect{v}^K = \tens{0}
\end{equation}
which is the left eigensystem of the acoustic tensor $A_{ij}=N_\alpha \tilde{H}_{i\alpha j \beta}  N_\beta$. Due to the symmetry of $\tens{A}$, system \eqref{eq:acoustic_eigen} is equivalent to the right eigensystem:
\begin{equation}
  \label{eq:acoustic_eigen_system_lambda}
  \(  N_\alpha \tilde{H}_{i\alpha j \beta}  N_\beta - \rho_0 c_K^2 \delta_{ij} \) v_j^K =0
\end{equation}
or alternatively, with the eigenvalues $\omega_p$ and associated left eigenvectors of the acoustic tensor $\vect{l}^p\: \: (p=1,2,3)$:
\begin{equation}
  \label{eq:acoustic_eigen_system}
  \( \tens{A} - \omega_p \tens{I} \) \vect{l}^p = \vect{0}
\end{equation}
The condition for system \eqref{eq:normal_quasi} to be hyperbolic (real eigenvalues and independent eigenvectors) is thus ensured by the positive definiteness of the acoustic tensor, also known as the \textit{strong ellipticity} condition \cite{Foundation_of_elasticity}:
\begin{equation}
  \label{eq:strong_ellipticity}
  (\vect{n}\otimes \vect{N}): \tilde{\Hbb}: (\vect{n}\otimes \vect{N}) > 0 \quad \forall \vect{N},\vect{n} \in \Rbb^3 \: ; \: \vect{N},\vect{n} \ne \vect{0}
\end{equation}
If the condition holds, the acoustic tensor admits $3$ couples eigenvalues--eigenvectors $\{\omega_p,\vect{l}^p\}$ leading to $6$ couples $\{c_K,\Lcb^K\}$ for the Jacobian matrix, the $6$ other eigenvalues being null \cite{Kluth}. The couples $\{c_K,\Lcb^K\}$ are referred to as \textit{left characteristic fields}. The left eigenvectors associated to non-zero eigenvalues of the Jacobian matrix are obtained by using equation \eqref{eq:eigen_left_velo} so that the following $6$ eigenfields of the quasi-linear form \eqref{eq:normal_quasi} can be defined:
\begin{equation}
  \label{eq:left_eigenfields}
    \left\lbrace \pm \sqrt{\frac{\omega_p}{\rho_0}} ; \quad \[\: \pm \rho_0\sqrt{\frac{\omega_p}{\rho_0}} \vect{l}^p , -\vect{l}^p\otimes \vect{N} \:\]  \right\rbrace ,\quad p=1,2,3
\end{equation}
At last, one has to find six independent left eigenvectors associated to the null eigenvalue of multiplicity $6$ by solving equation \eqref{eq:eigen_left_stress} for the null eigenvalue:
\begin{equation}
  \label{eq:left_null_eigenvectors}
  \tens{S}^K:\(\tilde{\Hbb}\cdot  \vect{N}\) =\vect{0},\quad K=1,...,6
\end{equation}
Following the same procedure for right eigenvectors $\Rcb^K=\matrice{\vect{v}^K \\ \tens{S}^K}$, the Jacobian matrix right eigensystem reads:
\begin{subequations}
  \begin{alignat}{1}
    \label{eq:eigen_right_stress}
    & -\frac{1}{\rho_0}\tens{S}^K\cdot  \vect{N} - c_K  \vect{v}^K =\vect{0} \\
    \label{eq:eigen_right_velo}
    & -\tilde{\Hbb}:\(\vect{v}^K\otimes\vect{N}\) - c_K \tens{S}^K = \tens{0}
  \end{alignat}
\end{subequations}
which leads to the \textit{right characteristic fields} associated to the non-null eigenvalues:
\begin{equation}
  \label{eq:right_eigenfields}
  \left\lbrace \pm \sqrt{\frac{\omega_p}{\rho_0}} ; \quad \[\: \pm \sqrt{\frac{\omega_p}{\rho_0}} \vect{l}^p , -\tilde{\Hbb}:\( \vect{l}^p\otimes \vect{N}\) \:\]  \right\rbrace ,\quad p=1,2,3
\end{equation}
In equation \eqref{eq:right_eigenfields}, $\{\omega_p,\vect{l}^p\}$ still denotes the eigenfields of the acoustic tensor. Moreover, the $6$ independent right eigenvectors associated to the zero eigenvalue required to complete the set of right characteristic fields must satisfy:
\begin{equation}
  \label{eq:right_null_eigenvectors}
  \tens{S}^K \cdot  \vect{N} =\vect{0},\quad K=1,...,6
\end{equation}

\begin{remark}
  Since the right-hand side of equation \eqref{eq:normal_quasi} is not involved in the characteristic analysis, linear elasticity and elasto-viscoplasticity in small strains leads to the same characteristic structure. 
\end{remark}

\begin{remark}
  \label{rq:similarity_solution}
  In the case of a vanishing right-hand side, the specialization of characteristic equations \eqref{eq:PDEs_ODEs} to system \eqref{eq:normal_quasi} leads to:
\begin{equation}
  \label{eq:characteristic_equations_homogeneous}
  \Lcb^K \cdot d\Qcb = \vect{0},\quad K=1,...,6
\end{equation}
meaning that the solution is constant along each characteristic straight line with slope $\xi = c_K$. Such solutions $\Qcb(\xi)$ that only depend on the ray $\xi$ are called \textit{similarity solutions}.
\end{remark}

\subsection{The Riemann problem}
A Riemann problem is a Cauchy problem with piecewise constant initial data. In particular, the Riemann problem based on the conservative form \eqref{eq:general_conservative_HE} for hyperelastic solids, in the arbitrary direction $\vect{N}=N_\alpha \vect{E}_\alpha$, takes the form:
\begin{equation}
  \label{eq:Riemann_problem_HE}
  \begin{aligned}
    &\Wcb_t + \drond{\Fcb\cdot \vect{N}}{X_N} = \Scb, \\
    &\left\lbrace 
      \begin{aligned}
        & \Wcb(X_N,t=0) = \Wcb^L \quad \text{if } X_N< 0\\
        & \Wcb(X_N,t=0) = \Wcb^R \quad \text{if } X_N> 0
      \end{aligned}
    \right.
  \end{aligned}
\end{equation}
Analogously, for small strains one writes the Riemann problem corresponding to conservative forms \eqref{eq:general_conservative} or \eqref{eq:general_conservative_EP} in the direction $\vect{n}=n_i \vect{e}_i$:
\begin{equation}
  \label{eq:Riemann_problem_HPP}
  \begin{aligned}
    &\Qcb_t + \drond{\Fcb\cdot \vect{n}}{x_n} = \Scb, \\
    &\left\lbrace 
      \begin{aligned}
        & \Qcb(x_n,t=0) = \Qcb^L \quad \text{if } x_n< 0\\
        & \Qcb(x_n,t=0) = \Qcb^R \quad \text{if } x_n> 0
      \end{aligned}
    \right.
  \end{aligned}
\end{equation}
where $x_n=\vect{x}\cdot\vect{n}$.
Problems of the form \eqref{eq:Riemann_problem_HE} or \eqref{eq:Riemann_problem_HPP} are considered in the next section, in which exact solutions are recalled or derived.

%%% Local Variables:
%%% mode: latex
%%% TeX-master: "../mainManuscript"
%%% End:





