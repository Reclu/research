The eigenspaces of hyperbolic problems defined in the previous section will be now investigated. As we shall see, the characteristic structure of those problems may lead to different type of waves propagating within a medium. Finally, existing analytical solutions of one-dimensional problems \cite{Wang} will be reviewed, and that of a one-dimensional problem involving a hyperelastic \textit{Saint-Venant-Kirchhoff} will be developed in order to illustrate the identified wave structures
\subsection{Characteristic structure of solutions}
Since hyperbolic problems are related to wave propagation phenomena, the quasi-linear form \eqref{eq:quasilinear_pi_v} will be studied in the direction $\vect{N}=\[\vect{e}_1,\vect{e}_2,\vect{e}_3\]$ for finite and infinitesimal deformations. In this direction, equation \eqref{eq:quasilinear_pi_v} reads \todo{Multidimensional spliting \cite[Chapter~3]{Toro};\cite[p.425-426]{Leveque}}:
\begin{equation}
  \label{eq:quasilinear_normal}
  \Absf^t\drond{\Wcb}{t} + N_\alpha \Absf^\alpha \drond{\Wcb}{X_N} = \vect{0}
\end{equation}
where $X_N = \vect{X}\cdot \vect{N}$. Give expressions of matrices, find a way to do that in a generic way
\begin{equation}
  \label{eq:2}
  \Absf^t=\matrice{\rho_0\tens{I} & \tens{0}^3 \\ \tens{0}^3  & \drond{\tens{F}}{\tens{\Pi}} } \quad ; \quad N_\alpha \Absf^\alpha = -\matrice{ \tens{0}^2 & \tens{I}\otimes\vect{N} \\ \tens{I}\otimes\vect{N} & \tens{0}^4}
\end{equation}
where $\tens{0}^p$ is a zero tensor of order $p$. The eigen system of the system is:
\begin{equation}
  \label{eq:4}
  \vect{\Lc}^k\cdot \(N_\alpha \Absf^\alpha - \lambda_k \Absf^t\) = \vect{0}
\end{equation}
with $\vect{\Lc}^k= \[ \vect{v}^K \: , \: \tens{\Pi}^K \]$.
\begin{equation}
  \label{eq:eigen_left}
  \left\lbrace
  \begin{aligned}
    & -\tr(\tens{\Pi}^k)  \vect{N} - \rho_0\lambda_k  \vect{v}^k =\vect{0} \\
    & -\vect{v}^k\otimes\vect{N} - \lambda_k \tens{\Pi}^k:\drond{\tens{F}}{\tens{\Pi}} = \tens{0}
  \end{aligned}\right.
\end{equation}
Inversion of the second equation and subsitution of $\tens{\Pi}^k$ in the first one yields:
\begin{equation}
  \label{eq:7}
  \( \frac{1}{\lambda_k} (\vect{v}^k\otimes\vect{N}):\drond{\tens{\Pi}}{\tens{F}}\) - \lambda_k \tens{\Pi}^k = \tens{0}
\end{equation}
\begin{equation}
  \label{eq:acousit_eigen_system}
  \(  N_\alpha H_{i\alpha j \beta}  N_\beta - \rho_0 \lambda_k^2 \delta_{ij} \) v_j^k =0
\end{equation}
where $N_\alpha H_{i\alpha j \beta}  N_\beta$ is the symetric \textit{acoustic tensor} $\tens{A}=\drond{\tens{\Pi}\cdot\vect{N}}{\tens{F}\cdot\vect{N}}$. Equation \eqref{eq:acousit_eigen_system} hence corresponds to the acoustic tensor eigensystem, which eigenvalues and eigenvectors are $\rho_0 \lambda_k^2$ and $\vect{v}^k$. Note that the symetry of the acoustic tensor implies that its left and right eigenvectors are identical. Then, the stress component of left eigenvectors is determined from the first equation of system \eqref{eq:eigen_left}. Finally the eigenstructure of the quasi-linear form is defined by the eigenfields:
\begin{equation}
  \label{eq:9}
    \left\lbrace \pm \sqrt{\frac{\omega^p}{\rho_0}} ; \quad <\: \pm \rho_0\sqrt{\frac{\omega^p}{\rho_0}} \vect{l}^p , -\frac{\vect{l}^p\otimes \vect{n}+\vect{n}\otimes \vect{l}^p}{2} \:>  \right\rbrace
\end{equation}
% Look at eigenvalues // eigen vectors // hyperbolicity condition // Legendre-Hadamard condition etc. -> system of scalar advection equations 
\subsection{Linear problems: contact waves}
\subsection{Non-linearities: simple and shock waves}
\subsection{Integral curves}
\subsection{The Rankine-Hugoniot condition}


%%% Local Variables:
%%% mode: latex
%%% TeX-master: "../mainManuscript"
%%% End:
