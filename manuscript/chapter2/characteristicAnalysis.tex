The eigenspaces of conservation laws systems defined above will be now investigated. As we shall see, the characteristic structure of those problems may lead to different type of waves propagating within a medium. Finally, existing analytical solutions of one-dimensional problems \cite{Wang} will be reviewed and that of a one-dimensional problem involving a hyperelastic \textit{Saint-Venant-Kirchhoff} material will be developed in order to illustrate the identified wave structures.
\subsection{Characteristic structure of solutions}
For the sake of simplicity studies of finite deformation and linearized geometrical frameworks will be condensed in this part by using a generic stress measure $\tens{S}$ and vectors written in the reference configuration. Furthermore, instead of studying multi-dimensional conservation laws systems, we will focus without loss of generality on conservative forms \eqref{eq:general_conservative} projected on an arbitrary direction $\vect{N}=\[\vect{e}_1,\vect{e}_2,\vect{e}_3\]$ \cite[p.425-426]{Leveque}. In this direction, the quasi-linear forms determined above are rewritten as:
\begin{equation}
  \label{eq:normal_quasi}
  \Qcb_t + \Jbsf \drond{\Qcb}{X_N} = \Scb
\end{equation}
where $X_N=\vect{X}\cdot\vect{N}$ and the \textit{Jacobian matrix} $\Jbsf = \Absf^\alpha N_\alpha$ arise. Hence, the characteristic analysis of system \eqref{eq:normal_quasi} is equivalent to that of linear combinations of matrices $\Absf^\alpha$. With the previous developments, the Jacobian matrix reads:
\begin{equation}
  \label{eq:jacobian_generic}
  \Jbsf=-\matrice{\tens{0}^2 & \frac{1}{\rho_0}\tens{I}\otimes \vect{N} \\  \tilde{\Hbb}\cdot\vect{N} & \tens{0}^4 }
\end{equation}
in which $\tilde{\Hbb}$ is either the hyperelastic or elastoplastic tangent modulus, or the elastic stifness tensor depending on the case considered. For general three-dimensional case, the characteristic structure of the problem is given by the $12$ eigenvalues $\lambda_k$ and associated left eigenvectors $\Lcb^k$ of the Jacobian matrix:
\begin{equation}
  \label{eq:eigen_system}
  \vect{\Lc}^k\cdot \(\Jbsf - \lambda_k \Ibsf\) = \vect{0}
\end{equation}
where $\Ibsf$ is the identity matrix and $\vect{\Lc}^k= \[ \vect{v}^K \: , \: \tens{S}^K \]$, with $\tens{S}$ standing for the suitable stress mesure. Thus, for non-null eigenvalues one gets:
\begin{subequations}
  \begin{alignat}{1}
    \label{eq:eigen_left_stress}
    & -\tens{S}^k:\(\tilde{\Hbb}\cdot  \vect{N}\) - \lambda_k  \vect{v}^k =\vect{0} \\
    \label{eq:eigen_left_velo}
    & -\frac{1}{\rho_0}\vect{v}^k\otimes\vect{N} - \lambda_k \tens{S}^k = \tens{0}
  \end{alignat}
\end{subequations}
Substitution of $\tens{S}$ obtained from \eqref{eq:eigen_left_velo} in \eqref{eq:eigen_left_stress} leads to:
\begin{equation}
  \label{eq:acoustic_eigen}
 (\vect{v}^k\otimes\vect{N}):\(\tilde{\Hbb}\cdot  \vect{N}\) - \rho_0\lambda^2_k \vect{v}^k = \tens{0}
\end{equation}
System \eqref{eq:acoustic_eigen} is the \textit{acoustic tensor} $A_{ij}=N_\alpha \tilde{H}_{i\alpha j \beta}  N_\beta$ left eigensystem which, due to the symmetry of $\tens{A}$ is equivalent to the right eigensystem:
\begin{equation}
  \label{eq:acoustic_eigen_system_lambda}
  \(  N_\alpha \tilde{H}_{i\alpha j \beta}  N_\beta - \rho_0 \lambda_k^2 \delta_{ij} \) v_j^k =0
\end{equation}
or atlernatively with the eigenvalues $\omega_p$ and associated left eigenvectors of the acoustic tensor $\vect{l}^p\: \: (p=1,2,3)$:
\begin{equation}
  \label{eq:acoustic_eigen_system}
  \( \tens{A} - \omega_p \tens{I} \) \vect{l}^p = \vect{0}
\end{equation}
The condition for system \eqref{eq:normal_quasi} to be hyperbolic and have real eigenvalues and associated eigenvectors is thus ensured by the positive definiteness of the acoustic tensor, also known as the \textit{strong ellipticity} condition \cite{Foundation_of_elasticity}\todo{Peut-être juste parler de la condition de forte ellipticité et faire du cas par cas plus tard + Elastoplastic ones ??}:
\begin{equation}
  \label{eq:strong_ellipticity}
  (\vect{m}\otimes \vect{N}): \tilde{\Hbb}: (\vect{m}\otimes \vect{N}) > 0 \quad \forall \vect{N},\vect{m} \in \Rbb^3 \: ; \: \vect{N},\vect{m} \ne \vect{0}
\end{equation}
If the condition holds, the acoustic tensor admits $3$ couples eigenvalues--eigenvectors $\{\omega_p,\vect{l}^p\}$ leading to $6$ couples $\{\lambda_k,\Lcb^k\}$ for the Jacobian matrix, the $6$ other eigenvalues being null \cite{Kluth}. The left eigenvectors associated to non-zero eigenvalues of the Jacobian matrix are obtained by using equation \eqref{eq:eigen_left_velo} so that the following $6$ eigenfields of quasi-linear form \eqref{eq:normal_quasi} can be defined:
\begin{equation}
  \label{eq:eigenfields}
    \left\lbrace \pm \sqrt{\frac{\omega_p}{\rho_0}} ; \quad \[\: \pm \rho_0\sqrt{\frac{\omega_p}{\rho_0}} \vect{l}^p , -\vect{l}^p\otimes \vect{N} \:\]  \right\rbrace ,\quad p=1,2,3
\end{equation}
At last, one has to find six independent left eigenvectors associated to the null eigenvalue of multiplicity $6$ by solving equation of \eqref{eq:eigen_left_stress} for the null eigenvalue:
\begin{equation}
  \label{eq:null_eigenvectors}
  \tens{S}:\(\tilde{\Hbb}\cdot  \vect{N}\) =\vect{0}
\end{equation}

Note that since the right-hande side of equation \eqref{eq:normal_quasi} is not involved in the characteristic analysis, linear elasticity and elaso-viscoplasticity leads to the same characteristic structure.


\subsection{Linear problems: contact waves}
read Courant p.427;p.486
\subsection{Non-linearities: simple and shock waves}
\subsection{Integral curves}
\subsection{The Rankine-Hugoniot condition}


%%% Local Variables:
%%% mode: latex
%%% TeX-master: "../mainManuscript"
%%% End:
