\documentclass[10pt,a4paper]{report}
\usepackage[latin1]{inputenc}
\usepackage{amsmath}
\usepackage{amsfonts}
\usepackage{amssymb}
\usepackage{graphicx}
\author{Adrien Renaud}

\begin{document}
\tableofcontents
\chapter{Introduction}
\section{General introduction}
Applications + motivations + industrial context
\section{The class of dynamic problems}
Biblio
\section{Numerical methods in solid mechanics}
Biblio
\section{Challenges for the simulation of dynamic problems}
Puproses + Objectives
\chapter{Hyperbolic partial differential equations}
\section{Partial differential equations}
\subsection{General concepts}
\subsection{Notion of characteristics -- Hyperbolic problems}
\subsection{The method of characteristics}

\section{Governing equations of solid mechanics}
% Context, isothermal, thermodynamical framework, standard generalized material etc.
\subsection{Kinematic laws -- Strain measures}
\subsection{Balance equations}
\subsection{Constitutive equations -- Thermodynamics}
GSM
\subsubsection{Examples of constitutive laws}
Neo-Hookean + Saint-Venant-Kirchhoff hyperelastic ; Linear elastic + elastic-viscoplastic + elastoplastic HPP 
\subsection{The general formulation}

\section{Characteristic analysis}
\subsection{Characteristic structure of the solution}
\subsection{Linear problems: contact waves}
\subsection{Non-linearities: simple and shock waves}
\subsection{Integral curves}
\subsection{The Rankine-Hugoniot condition}
\section{Review and derivation of one-dimensional analytical solutions}
\subsection{Linear elastic solids}
\subsubsection{Linear elastic plane bar}
\subsubsection{Linear elastic plane wave}
%\subsubsection{Elastoviscoplasic bar and plane wave}
%\subsubsection{Elastoviscoplasic}
\subsection{Elastoplastic solids}
\subsubsection{Linearly hardening media}
\subsubsection{Decreasingly hardening media}
\subsection{Saint-Venant-Kirchhoff hyperelastic solids}
% \subsubsection{General multi-dimensional solution}
% \subsubsection{One-dimensional Saint-Venant-Kirchhoff material}

% \subsection{Elastic viscoplastic solids}
% \subsection{Elastoplastic solids}
% \subsubsection{The thin-walled tube problem}
% \subsubsection{Two-dimensional plane strain problem}
% \subsubsection{Two-dimensional plane stress problem}
% \subsection{Hyperelastic solids}

%% On a un degr� de libert� sur l'�lastoplasticit�:
%%% Soit on en parle dans le chapitre 2 pour les probl�me 1D uniquement
\chapter{Approximate Riemann solvers}
% Review of existing approximate solver
\section{The Riemann problem}
% See Toro ch.4
\subsection{Exact solution of Riemann problems}
\subsection{Approximate-state Riemann solver}
\subsection{The Godunov flux}
\section{Linearized geometrical framework}
\subsection{Linear elasticity}
\subsection{Elasto-plasticity}
\section{Hyperelastic problems}


\chapter{The Discontinuous Galerkin Material Point Method}
\section{The material point method}
Biblio MPM 
\subsection{Development of the method}
% PIC - FLIP - MPM - GIMP - HOMPM/BSMPM: introduction text 
\subsection{Derivation of the mpm}
\subsubsection{MPM discrete equations}
\subsubsection{Solution scheme summary}
\subsection{Shortcomings}

\section{Extension to discontinuous Galerkin approximation}
\subsection{The discontinuous Galerkin approximation}
Biblio DG
% Introduction text, DG approximation, DGFEM, main  idea of the introduction of this approx into MPM, mootivations ?             
\subsection{Derivation of the method}
% Do not explicit interface fluxes yet (not required for numerical analysis) // Also, it is an independant tool used in other methods such as FVM, DGFEM.
\subsection{DGMPM solution scheme}

\section{Transverse corrections for multidimensional problems}
%The CTU method
\section{Numerical analysis of the DGMPM}
\subsection{Convergence analysis}
\subsection{One-dimensional stability analysis}
Euler + RK2
\subsection{Two-dimensional stability analysis}

\chapter{Numerical Results}
Study the influence of the number of material points per cell in simulations + comparison MPM
\section{Linearized geometrical framework}
\subsection{One-dimensional elasticity and elastoviscoplasticity}
Splitting method for EVP
\subsection{One-dimensional elastoplasticity}
\subsection{Two-dimensional elasticity}

\section{Hyperelastic solids}
\subsection{One-dimensional problems}
\subsection{Two-dimensional problems}

\chapter{Contribution to the solution of plane-strain and plane-stress elastoplastic problems}
\section{The complexity of elastoplasticity}
% Bibliography about:
% - what has been done so far (thin-walled ; plane wave + shear wave ; internal structure of plastic shock)
% - what is missing (solution of more general problems [i.e plane strain or stress] ; development of approximate Riemann solvers)
% - what it will allow (comparison with experimental data in order to track plastic shocks)
% - what is currently done in fluid or solid mechanics numerically
Biblio, 
\section{State of the art}
\subsection{The thin-walled tube problem}
\subsection{Superimposition of plane wave and shear waves}
\section{Formulation of the problem}
\subsection{General multi-dimensional formulation}
% Based on the tangent modulus -> independant on the hardening model
\subsection{Plane strain problems}
\subsection{Plane stress problems}
\section{Characteristic analysis}
\subsection{Structure of the solution}
\subsection{Integral curves and loading paths}
\subsection{The plane strain case}
\subsection{The plane stress case}
\section{Towards an elastoplastic approximate Riemann solver}

\chapter{Conclusion}


\end{document}

%%% Local Variables:
%%% mode: latex
%%% TeX-master: t
%%% End:
