The numerical simulation of initial boundary value problems (IBVP), to which hyperbolic ones belong, has been and is still widely performed in solid mechanics with the Finite Element Method (FEM) \cite{Belytschko}.
This approach is based on the discretization of a computational domain into a set of elements with simple geometries used to solve the equations.
An approximate solution is thus built by means of a combination of polynomial functions which degree defines the approximation order.
FEM has become attractive due to its ability to handle low or high-order approximations, and to easily deal with complex geometries and nonlinear constitutive models as these mentioned above.  %% At integration points
%% Lagrangian (appuyer un peu les avantages et les inconvénients)
Nevertheless, difficulties may be encountered if large deformations occur when the finite elements deform with the domain, according to a material description of the motion (Lagrangian approach).
Indeed, the method is less efficient and accurate when the elements are highly distorted or entangled so that re-meshing techniques and projection steps must be employed.
%% Eulerian (appuyer un peu les avantages et les inconvénients)
Those issues can be avoided by using a spatial description of the motion consisting in viewing the elements vertices as fixed points of the space (Eulerian approach).
However interface tracking techniques and diffusing convection steps are required in order to follow the boundaries and transport internal variables which is less convenient for solid mechanics.
%% ALE (appuyer un peu les avantages et les inconvénients)
Alternatively, arbitrary Lagrangian Eulerian (ALE) methods aim at meeting advantages of both approaches by freeing itself of their respective limits by distinguishing the motion of mesh to this of material points.
This type of strategy nonetheless also requires re-meshing or re-zoning procedures that can be costly for fine meshes. %that can dramatically increase the computational time for fine meshes.
They also require diffusive projection steps of internal variables for solid media.
%% Time integration
In addition to problems caused by finite deformations, classical explicit time integrators used in solid dynamics with FEM introduce high frequency noise in the vicinity of discontinuities.
Such regions of high gradient may be caused by the waves arising in the solutions of hyperbolic problems.
The removal of these spurious numerical oscillations with additional viscosities is difficult to achieve without loss of accuracy, and can be troublesome for the wave tracking.

%% FVM
On the other hand, the Finite Volume Method (FVM) \cite{Leveque}, initially developed for fluid dynamics, provided until the 90s piece-wise constant or piece-wise linear approximate solutions in cells that discretize a continuum.
The extension to very high-order has since been proposed by increasing the stencil of the scheme (see WENO \cite{WENO}).
This class of methods can embed tools based on the Total Variation Diminishing (TVD) stability condition \cite{Harten}, thus ensuring that no numerical spurious oscillations arise in the solutions. 
The formulation moreover lies on a conservative form leading to the same order of accuracy for all components of the unknown vector $\Ucb$.
In particular, one shows that both velocity and gradients arise in $\Ucb$ in solid mechanics.
This point makes a big difference with respect to methods that do not use a formulation written as a differential system of order one, namely FEM, for which the convergence rate of gradients is one order less than that of displacement.
To some extent, the writing of solid mechanics equations in the form of conservation laws amounts to a mixed approach, well-known in FEM procedure.
FVM formulations moreover rely on numerical fluxes allowing to account for the characteristic structure of hyperbolic equations.
Hence, finite volumes enable an accurate tracking of the path of waves although the most widely used approximations are second-order. %the approximation is limited to order two.
Recent studies furthermore extended these approaches to solid mechanics for problems involving history-dependent models \cite{Gavrilyuk,Thomas_EP}, or finite deformations with a Lagrangian formulation \cite{Lee_FVM,Haider_FVM}.
FVM are nevertheless grid-based methods for which the numerical difficulties linked to mesh occur.
%FVM are nevertheless grid-based methods for which the numerical issues linked to mesh-entanglement do not vanish. %% Vertex centered Vs Cell-centered ?

%% DG approximation profiter des avantages des FEM et FVM Comportement non linear
The discontinuous Galerkin (DG) approximation \cite{NeutronDG} allows to build numerical schemes that benefit from both FEM and FVM.
In DG-methods, the approximate solution is seeked as a combination of piece-wise polynomial functions which supports are dictated by the discretization used.
Therefore, the combination of the DG approximation with the finite element formulation (DGFEM) yields a local high approximation order, the same order being achievable for both velocity and gradients if a conservative form is used.
Furthermore, numerical fluxes at the interface between elements, allowing to take into account the characteristic structure of hyperbolic systems, arise from the introduction of DG approximation.
%% Discontinuity
%In addition, the numerical noise arising in FEM solutions can be removed following 
Several development steps, aiming at removing the numerical noise appearing in DGFEM solution, followed the works on FVM in order to reach Total Variation Diminishing in the Means (TVDM) and Total Variation Bounded (TVB) high order schemes that ensure convergence to entropy-satisfying solutions \cite{Cockburn}.
However, in spite of the piece-wise continuous approximation, the methods fail in accurately capture discontinuities owing to the time discretization.
%% TDG - SDG
Adopting a similar approach, the Time-Discontinuous Galerkin (TDG) \cite{Hughes_TDG} and later, the Space-Time Discontinuous Galerkin method (SDG) \cite{ST_DGFEM1}, relaxed the continuity of fields in time domain. 
By discretizing the entire space-time domain as a possibly unstructured mesh, SDG allows to avoid the difficulties related to the time integrators and hence, enables the following of waves.
Nevertheless, although those approaches can easily handle mesh-adaption strategies due to the relaxation of fields continuity, it does not eliminate the mesh tangling problems \cite{FVilar_DG}.

%% Meshless methods
%% Les meshless peuvent être basées sur une forme forte (collocation method) ou un forme faible (Galerkin method)
In order to address the loss of accuracy caused by mesh distortion, another class of numerical methods based on a space discretization using a collection of points has been developed in parallel to the above approaches \cite{Belytschko_Meshless,Meshless}.
In contrast to finite volumes or finite elements, mesh-free methods represent a spatial domain by means of a collection of points that are given a support allowing them to interact with each other.
A wide variety of mesh-free methods such as among others, the Smoothed-Particle Hydrodynamics \cite{SPH} or the Element-Free Galerkin \cite{Belytschko_EFG}, differing by the weight functions used that dictate the domain of influence of particles, have thus been constructed.

%% PIC
More specifically, some hybrid families as this of Particle-In-Cells methods (PIC) \cite{PIC} is based on particles that move in a computational mesh while carrying the fields of a problem.
%More specifically, the Material Point Method (MPM) \cite{Sulsky94} inherits from the family of Particle-in-Cells methods (PIC) \cite{PIC} in which the particles move in a computational mesh made of finite elements while storing the fields of a problem.
The underlying grid is punctually used in order to compute an approximate solution so that it can be discarded at each time step and re-constructed for computational convenience without additional projections.
The application of PIC to solid mechanics led to the Material Point Method (MPM) in which the constitutive equations are managed at particles.
As a result MPM can be seen as a mesh-free extension of FEM with quadrature points moving in elements.
Overcoming the storage of the approximate solutions based on elements or cells enables to remove mesh entanglement instabilities.
It is nevertheless well-known that PIC exhibit numerical diffusion that can be reduced at the cost of spurious oscillations \cite{Mass_Flip}.
% However, meshfree approaches suffer from limits related to interfaces tracking, the enforcement of boundary condition as well as numerical noise arising due to time integrators.

$\newline$
In light of the above brief overview, it appears that in spite of the plenty of existing numerical methods, none can easily deal with all the difficulties involved by hyperbolic problems in finite deforming dissipative solids.
First, the large deformations often met in solid mechanics problems make tricky the employment of mesh-based approaches.
Second, the waves propagating in media can be accurately followed providing that the scheme used computes solutions devoid of spurious oscillations and too much numerical diffusion. 
At last, accounting for the characteristic structure of the solution of hyperbolic problems within a numerical method yields results closer to the expected output of the model.
That is, the solution of the wave structure is in general drowned in solid mechanics solvers in such a way that the approaches are not oriented towards the accurate description of waves.


%% Purposes + Objectives

%%% Local Variables:
%%% mode: latex
%%% TeX-master: "../mainManuscript"
%%% End:
