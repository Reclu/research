\chapter{Introduction}

\section{General context}
%%% Solid mechanics
The work presented in this thesis is concerned with hyperbolic systems of conservation laws in mechanics used to describe the propagation of mechanical waves.
A wide variety of engineering problems including among others acoustics, aerodynamics, impacts, which are of major importance for plenty of applications, are modeled with this class of mathematical equations. 
Although those applications may involve solid and fluid media whose constitutive response differ, the focus is here on solid mechanics.
%%% Specificities
More specifically, an important class of models is that of dissipative solid behavior depending on the loading history undergone.
Such materials are implicated in high-speed forming techniques, crash-proof design or in the reliability of structures.

%% Complexity ofthe modeling
In these solids the waves propagate while carrying the information about the loading, interacting with each other and reflecting on the boundary so that complex structures arise.
The accurate assessment of irreversible deformations in dissipative solids therefore requires the correct description of those waves as well as the ability to account for their interactions.
% Different time and space scales
Furthermore, the time scale governing the propagation of waves may be different from that of other phenomena also involved in a deformation, as for viscous effects in solids for instance.
% Large deformations
At last, the problem may be even more complicated by possibly large displacements, rotations or strains, undergone by the solid.

%% Simulation -- motivations
% In order to better apprehend misunderstood phenomena, the numerical simulation of solid mechanics problems allows to see what cannot be observed due to the time or space scales considered.
% Naturally, the simulation and the experiments are not decoupled but interact with mathematics so as to enrich each other, thus allowing a more and more precise knowledge of the physics.
% On the other hand, by simulating well-known problems, one can optimize industrial processes or design structures while avoiding the waste of material. 
% Hence, the development of numerical methods that reproduce and take into account the physics is crucial to provide qualitative and quantitative results.

Given the complexity of the equations, the numerical simulation provides a framework for approximating the solutions of a model.
It is a way to virtually experiment and highlight phenomena that are implicitly included in a model but not necessarily observed.
The simulation can then be seen as a way to make the theory meaningful.
Moreover, it enables to fill gaps left by experimental works for feasibility reasons and due to the higher amount of information it provides compared to instrumental devices.

%%% Local Variables:
%%% mode: latex
%%% TeX-master: "../mainManuscript"
%%% End:



\section{Numerical methods for hyperbolic problems in solid mechanics}
%The numerical simulation of initial boundary value problems (IBVP), to which hyperbolic ones belong, has been and is still widely performed in solid mechanics with the Finite Element Method (FEM) \cite{Belytschko}.
This approach is based on the discretization of a computational domain into a set of elements with simple geometries used to solve the equations.
An approximate solution is thus built by means of a combination of polynomial functions whose degree defines the approximation order.
FEM became attractive due to its ability to handle low or high-order approximations, and to easily deal with complex geometries and nonlinear constitutive models as those mentioned above.  %% At integration points
%% Lagrangian (appuyer un peu les avantages et les inconvénients)
Nevertheless, difficulties may be encountered if large deformations occur when the finite elements deform with the domain, according to a material description of the motion (Lagrangian approach).
Indeed, the method is less efficient and accurate when the elements are highly distorted or entangled so that re-meshing techniques and projection steps must be employed.
%% Eulerian (appuyer un peu les avantages et les inconvénients)
These issues can be avoided by using a spatial description of the motion consisting in viewing the elements vertices as fixed points of the space (Eulerian approach).
However interface tracking techniques and diffusing convection steps are required in order to follow the boundaries and transport internal variables which is less convenient for solid mechanics.
%% ALE (appuyer un peu les avantages et les inconvénients)
Alternatively, arbitrary Lagrangian Eulerian (ALE) methods aim at meeting advantages of both approaches by freeing itself of their respective limits by distinguishing the motion of the mesh to these of material points.
This type of strategy nonetheless also requires re-meshing or re-zoning procedures that can be costly for fine meshes. %that can dramatically increase the computational time for fine meshes.
They also require diffusive projection steps of internal variables for solid media.
%% Time integration
In addition to problems caused by finite deformations, classical explicit time integrators used in solid dynamics with FEM introduce high frequency noise in the vicinity of discontinuities.
Such regions of high gradient may be caused by the waves arising in the solutions of hyperbolic problems.
The removal of these spurious numerical oscillations with additional viscosities is difficult to achieve without loss of accuracy, and can be troublesome for the wave tracking.

%% FVM
On the other hand, the Finite Volume Method (FVM) \cite{Leveque}, initially developed for fluid dynamics, provided until the 90s piece-wise constant or piece-wise linear approximate solutions in cells that discretize a continuum.
The extension to very high-order has since been proposed by increasing the stencil of the scheme (see WENO \cite{WENO}).
This class of methods can embed tools based on the Total Variation Diminishing (TVD) stability condition \cite{Harten}, thus ensuring that no numerical spurious oscillations arise in the solutions. 
The formulation moreover lies on a conservative form leading to the same order of accuracy for all components of the unknown vector $\Ucb$.
In particular, one shows that both velocity and gradients arise in $\Ucb$ in solid mechanics.
This point makes a big difference with respect to methods that do not use a formulation written as a differential system of order one, namely FEM, for which the convergence rate of gradients is one order less than that of displacement.
To some extent, the writing of solid mechanics equations in the form of conservation laws amounts to a mixed approach, well-known in FEM.
FVM formulations moreover rely on numerical fluxes that enable to account for the characteristic structure of hyperbolic equations.
Hence, finite volumes enable an accurate tracking of the path of waves although the most widely used approximations are second-order. %the approximation is limited to order two.
Recent studies furthermore extended these approaches to solid mechanics for problems involving history-dependent models \cite{Gavrilyuk,Thomas_EP}, or finite deformations with a Lagrangian formulation \cite{Lee_FVM,Haider_FVM}.
The latter methods are nevertheless grid-based techniques for which the numerical difficulties linked to mesh occur.
%FVM are nevertheless grid-based methods for which the numerical difficulties linked to mesh occur.

%% DG approximation profiter des avantages des FEM et FVM Comportement non linear
The discontinuous Galerkin (DG) approximation \cite{NeutronDG} makes possible to build numerical schemes that benefit from both FEM and FVM.
In DG-methods, the approximate solution is sought as a combination of piece-wise polynomial functions whose supports are dictated by the discretization used.
Therefore, the combination of the DG approximation with the finite element formulation (DGFEM) yields a local high approximation order, the same order being achievable for both velocity and gradients if a conservative form is used.
Furthermore, numerical fluxes at the interface between elements, which enable to take into account the characteristic structure of hyperbolic systems, arise from the introduction of DG approximation.
%% Discontinuity
%In addition, the numerical noise arising in FEM solutions can be removed following 
Several development steps, aiming at removing the numerical noise appearing in DGFEM solution, followed the works on FVM in order to reach Total Variation Diminishing in the Means (TVDM) and Total Variation Bounded (TVB) high order schemes that ensure convergence to entropy-satisfying solutions \cite{Cockburn}.
However, in spite of the piece-wise continuous approximation, the methods fail to accurately capture discontinuities owing to the time discretization.
%% TDG - SDG
Adopting a similar approach, the Time-Discontinuous Galerkin (TDG) \cite{Hughes_TDG} and later, the Space-Time Discontinuous Galerkin method (SDG) \cite{ST_DGFEM1}, relaxed the continuity of fields in the time domain. 
By discretizing the entire space-time domain as a possibly unstructured mesh, SDG avoids the difficulties related to the time integrators and hence, enables the following of waves.
Nevertheless, although these approaches can easily handle mesh-adaption strategies due to the relaxation of fields continuity, it does not eliminate the mesh tangling problems for Lagrangian formulations \cite{FVilar_DG}.

%% Meshless methods
%% Les meshless peuvent être basées sur une forme forte (collocation method) ou un forme faible (Galerkin method)
In order to address the loss of accuracy caused by mesh distortion, another class of numerical methods based on a space discretization using a collection of points has been developed in parallel to the above approaches \cite{Belytschko_Meshless,Meshless}.
In contrast to finite volumes or finite elements, mesh-free methods represent a spatial domain by means of a collection of points that are given a support allowing them to interact with each other.
A wide variety of mesh-free methods such as the Smoothed-Particle Hydrodynamics \cite{SPH} or the Element-Free Galerkin \cite{Belytschko_EFG}, have thus been constructed.
%, differing by the weight functions used that dictate the domain of influence of particles, have thus been constructed.
%Mesh-free methods can be classified into two families depending on whether a weak form is used or not. 

%% PIC
%More specifically, some hybrid families as this of Particle-In-Cells methods (PIC) \cite{PIC} is based on particles that move in a computational mesh while carrying the fields of a problem.
Particle-In-Cell methods (PIC) \cite{PIC} are, on the other hand, based on particles that move in a computational mesh while carrying the fields of a problem.
% The underlying grid is punctually used in order to compute an approximate solution so that it can be discarded at each time step and re-constructed for computational convenience without additional projections.
The underlying grid is used in order to compute an approximate solution that is projected and stored at particles.
Hence, the background mesh can be discarded at each time step and re-constructed for computational convenience.
The application of PIC to solid mechanics led to the Material Point Method (MPM) in which the constitutive equations are managed at particles.
As a result MPM can be seen as a mesh-free extension of FEM with quadrature points moving in elements.
Overcoming the storage of the approximate solutions based on elements or cells enables to remove mesh entanglement instabilities.
It is nevertheless well-known that PIC exhibits numerical dissipation that can be reduced at the cost of spurious oscillations \cite{Mass_Flip}.
% However, meshfree approaches suffer from limits related to interfaces tracking, the enforcement of boundary condition as well as numerical noise arising due to time integrators.

$\newline$
In light of the above brief overview, it appears that in spite of the plenty of existing numerical methods, none can easily deal with all the difficulties involved by hyperbolic problems in finite deforming dissipative solids.
First, the large deformations often met in solid mechanics problems make tricky the employment of mesh-based approaches.
Second, the waves propagating in media can be accurately followed providing that the scheme used computes solutions devoid of spurious oscillations and too much numerical diffusion. 
At last, accounting for the characteristic structure of the solution of hyperbolic problems within a numerical method yields results closer to the expected output of the model.
%That is, the solution of the wave structure is in general drowned in solid mechanics solvers in such a way that the approaches are not oriented towards the accurate description of waves.
That is, the solution of the wave structure is in general not directly tackled in solid mechanics solvers in such a way that the approaches are not devoted to the accurate description of waves.
%  these aspects are not directly tackled

%% Purposes + Objectives

%%% Local Variables:
%%% mode: latex
%%% ispell-local-dictionary: "american"
%%% TeX-master: "../mainManuscript"
%%% End:



\section{Approach followed in this work}
%\subsection{Point of view of this thesis}

This work is based on the idea that the accurate numerical solution of hyperbolic problems relies on robust and efficient discretization techniques and the ability to embed information about some particular solution of the model into the numerical scheme.
The underlying concepts arise from Godunov method \cite{Godunov_method} that first proposed to account for the characteristic structure of the solution of hyperbolic problems within numerical schemes.
This approach, as we shall see later, enables significant improvements over some other methods.
Although the amount of information provided is limited by the computational cost, new model reduction tools can now allow to "compress" the amount of information and thus allow to reconsider the use of more complex Riemann solvers, in order to improve the accuracy of the numerical solution.

Two scientific objectives have therefore been pursued in the present work:
\begin{itemize}
\item \textbf{the development of a promising discretization for solid mechanics problems}
\item \textbf{the identification of the response of two-dimensional elastic-plastic solids to dynamic step loading under small strains}
\end{itemize}

\subsection{The strategy adopted}
%In the present manuscript, a numerical method for applications in solid mechanics aiming at benefiting from the schemes mentioned in the previous section is developed.
%% Lagrangian
First, the numerical scheme developed here provides a material description of the motion so as to handle history-dependent constitutive models while avoiding the shortcomings of ALE and Eulerian methods.
%% Meshfree
In addition, in order not to suffer from mesh tangling instabilities, it seems better to turn to a mesh-free method.
%% DG approximation
This method is wanted to mimic the physics of hyperbolic problems by accounting for their intrinsic structure.

%% MPM + DG
It is therefore proposed here to mix the above features by extending the material point method to the Discontinuous Galerkin approximation.
%% Pourquoi pas SDG ou TDG
The Discontinuous Galerkin approximation provides an appealing framework for describing moving discontinuities such as waves propagating in solids, and the potential ability of increasing approximation order.
Moreover, the numerical fluxes naturally arising from the use of DG approximation enables to introduce the characteristic structure on the one hand, and to take advantage of the works done in the context of finite volumes on the other hand.
%% Pourquoi MPM
Furthermore, the use of an arbitrary grid motivates the choice of the MPM due to the convenience it allows in computing the numerical fluxes at finite elements interfaces by means of an approximate Riemann solver \cite{Toro}.
%% Réduire la diffusion de la pic grace à DG
A balance between diffusion and oscillations exists in PIC methods in such a way that there is some freedom in regard to the MPM setting used.
It is however thought that the use of DG approximation leads to a reduction of the influence domain of the particles and hence, to a reduction of the numerical diffusion. 
Therefore, it is preferred here to avoid oscillations at the risk of introducing diffusion, which should be limited by the DG approximation.
As a first development step of \textbf{the Discontinuous Galerkin Material Point Method}, we restrict our attention to space-DG.
At last, particular attention is paid here to discontinuous solutions, the extension of the method to higher-order approximation for regular ones will be the purpose of future works.

%% Travail sur la modélisation des écoulements plastiques en 2D
The approach described previously should be able to account for the characteristic structure of hyperbolic problems.
Indeed, the point of view adopted here is that a numerical scheme can properly mimic the solution providing that a sufficient amount of information about the model is available.
Nonetheless, it is not the case for all the constitutive models considered in this work.
More specifically, while the response of one-dimensional elastoplastic solids is well-known, there are some lacks in describing the behavior of such materials in more dimensions. 
As a consequence, the characteristic structure of the solution of hyperbolic problems in elastic-plastic solids is also investigated in the present work.







%%% Local Variables:
%%% mode: latex
%%% TeX-master: "../mainManuscript"
%%% End:



\section{Organization of the manuscript}
etc.




%%% Local Variables:
%%% mode: latex
%%% TeX-master: "../mainManuscript"
%%% End:



%%% MPM
% The Material Point Method (MPM) developed in the 90’s [1] can be viewed as an extension of the FEM in which quadrature points can move in an arbitrary computational grid. Those material points represent a continuum with a Lagrangian description and are used to store all the fields and internal variables of the problem. The underlying mesh is only used to compute gradients, apply boundary conditions and solve discrete equations. The method hence uses many transfer steps of fields between material points and nodes. Since the data are stored at material points, the grid can be discarded at each time step and re-constructed for computational convenience without additional projections, thus avoiding mesh entanglement.



%%% DG
% The discontinuous Galerkin (DG) approximation [2] enables to build numerical schemes that benefit from both finite element and finite volume methods. DG-methods are based on piece-wise polynomial shape functions and their combination with the finite element formulation (DGFEM) led to several development steps to reach Total Variation Diminishing in the Means (TVDM) and Total Variation Bounded (TVB) high order schemes that ensure convergence to entropy-satisfying solutions [14]. In addition, the same order of accuracy is reached for velocity and gradients within a finite element framework when the weak form is based on a conservation laws system. Although this approach can easily handle mesh-adaption strategies due to the relaxation of fields continuity, the mesh tangling problems do not vanish. Furthermore, the Courant condition that applies is restrictive [14] and prevents to accurately capture discontinuities.

%%% Purpose
%The purpose of this work is to develop a numerical method gathering the strengths of both material point and DG methods and avoiding their respective limits by means of the discontinuous Galerkin approximation. The computational grid of the MPM is used to build a piece-wise linear approximation of the solution of conservation laws, which weak form is written on each element. Boundary integrals arising from integration by parts involve numerical fluxes that connect the elements together and are computed with an approximate Riemann solver [3]. Projection steps between the underlying mesh, used to solve the discrete equations, and material points that store the fields of the problem, provide a Lagrangian description of the deformation in an arbitrary computational grid. The resulting Discontinuous Galerkin Material Point Method (DGMPM) hence aims to enable an accurate wave paths tracking in solid media undergoing large strains. A similar approach based on the Particle-In-Cell (PIC) method has already been developed in [15] for electromagnetism equations. This discontinuous Galerkin version of PIC, unlike the DGMPM, uses classical Gauss points to integrate the weak form. Moreover, electric and magnetic fields are stored at nodes whereas charge and current densities are stored at particles which prevent the arbitrariness of the grid.




%%%
%In what follows, after a brief review of continuum equations (section 2), the DGMPM formulation is derived within the large strains framework with a total Lagrangian approach for hyperelasticity in section 3. Section 4 is dedicated to the computation of interface fluxes arising in the weak form. First, Riemann problems are written at interfaces with a quasi-linear form which spectral analysis is performed. Second, an approximate Riemann solver that can take into
% account transverse corrections for multi-dimensional problems is derived. Finally, the enforcement of boundary conditions
% is discussed. Next, after a summary of the numerical procedure in section 5, the stability analysis of the one-dimensional
% scheme is carried out in section 6. This analysis shows that a reduced integration of the weak form provides an improve-
% ment of the stability condition that limits the DGFEM. At last, the method is illustrated in section 7 on one-dimensional
% cases in order to compare it to analytical solutions, and two-dimensional problems for which comparisons are performed
% with the Finite Element Method.