\section*{General conclusion}

%% Introduction:
The purpose of this work was the development of a numerical method allowing the accurate solution of hyperbolic problems in solid mechanics.
This class of mathematical problems governs the propagation of waves in finite deforming media.
Given the ability to deal with large deformations enabled by mesh-free methods and the possibility of describing discontinuous solutions provided by the DG approximation, it has been proposed to merge those two approaches.
Furthermore, particular attention has been paid throughout this manuscript to the characteristic structure of the solution of hyperbolic problems in order to: (i) provide qualitative and quantitative comparisons between numerical solutions and exact ones; (ii) allow numerical schemes to mimic the behavior of physical systems.

$\newline$
%% Chapter2: Hyperbolic partial differential equations for solid dynamics
In chapter \ref{chap:chap2} the balance laws and constitutive equations of solid mechanics have been derived and written as a system of conservation laws in conservative form.
The characteristic analysis of the associated quasi-linear form then leads to the formulation of a hyperbolicity condition of the system in terms of the eigenstructure of the acoustic tensor.

% Type of waves
Exact solutions of one-dimensional problems have then been presented.
First, the existing solutions of Riemann problems in elastic and elastoplastic solids with linear hardening under small strains emphasized that discontinuous waves may propagate.
Second, \textbf{the derivation of exact solutions of Picard problems in a one-dimensional Saint-Venant-Kirchhoff hyperelastic medium} highlighted the existence of shocks and rarefaction waves in solid dynamics.
Given the complexity hyperbolic systems may present, especially for more than one space dimension or other constitutive models, a well-known approximate-state Riemann solver has been presented.
Such solvers represent an ideal way to take into account the characteristic structure of the solutions within numerical schemes and are therefore widely used in finite volume.

$\newline$
%% Chapter3: The Discontinuous Galerkin Material Point Method
\textit{Eulerian} and Lagrangian formulations of the Material Point Method have been recalled in chapter \ref{chap:chap3}.
This method constitutes the starting point of the present work due to its ability to avoid the mesh instabilities caused by large deformations often met in solid mechanics.
Nevertheless, the oscillating solutions provided by this numerical scheme prevent accurately following the waves.
The aforementioned numerical noise can be eliminated by going back to the early version of the approach: the Particle-In-Cell method, at the cost of additional numerical diffusion.
Thus, the approach proposed consisted of: (i) removing the oscillations by re-introducing within the MPM the diffusive mapping between material points and grid nodes used in PIC; (ii) reducing the diffusion thus introduced by means of the discontinuous Galerkin approximation.

\textbf{The Discontinuous Galerkin Material Point Method has therefore been derived} with a total Lagrangian formulation.
The method inherits appealing features of the finite element, finite volume and material point methods.
%% Similarities with FVM
First, it is based on the weak form of a system of conservation laws so that the same order of accuracy is achieved for both velocity and gradients.
Second, numerical fluxes defined at element interfaces arising in the weak form are computed with the solution of Riemann problems by means of the approximate-state Riemann solver presented in chapter \ref{chap:chap2}.
This approach not only allows the introduction of the characteristic structure in the numerical solution, but it also enables \textbf{taking into account transverse corrections by adapting the Corner Transport Upwind method}.
Third, the DGMPM solution scheme can be combined with ODE integrators for source terms if fractional-step methods are considered. 
The last three points are common with FVM.
Then, the FEM approximation using polynomial shape functions is used, thus providing a convenient framework for high-order approximation.
In particular, as for other DG-methods, the order of approximation may be modified element-wise.
At last, the arbitrary grid of the MPM, in which the particles move while carrying fields and internal variables, enables the employment of mesh-adaption strategies without the need for additional diffusive projection steps.
Notice however that the last aspect is not considered in this thesis.

Then, the numerical analysis of DGMPM schemes showed that the stability properties of the method may be better than those of the original MPM depending on the space discretization used.
% 1D stability
Namely, if the DGMPM semi-discrete system is discretized in time with the forward Euler algorithm, the optimal Courant number can be achieved when one particle lies in every cell of a one-dimensional grid.
The same result holds if the second-order Runge-Kutta time discretization is used, and even for more than one material point per cell depending on their positions.
Even though the condition CFL=$1$ is not valid in general, \textbf{a formula for evaluating the critical Courant number allows the ensuring of the stability of the scheme}, whereas it is not possible for the MPM. 
% 2D stability
Analogously, the two-dimensional DGMPM scheme exhibits the optimal stability condition providing that one particle per cell is used along with the CTU algorithm for transverse corrections.
It has been also shown that the stability features of the two-dimensional scheme depend on the ratio of the characteristic speeds involved in the linear advection equation. 
% Convergence (numerically)
On the other hand, the convergence analysis of the method on a one-dimensional linear elasticity problem under small strains showed that by using first-order shape functions, only first-order accuracy can be achieved for velocity and gradients.
Although one could expect second-order convergence since the DGMPM approximation is similar to that of FEM, the PIC mapping of the updated nodal velocity from the grid to particles used yields first-order accuracy.

$\newline$
%% Chapter4: Numerical Results
Chapter \ref{chap:chap4} was devoted to the illustration of the DGMPM on simulations of solid dynamics problems.
Comparisons with other numerical methods and exact solutions showed very good behavior.
More specifically, since the optimal \text{CFL} number can be achieved, the DGMPM is able to capture the discontinuities arising in the Riemann problem in an elastic bar within the linearized geometrical framework.
On the other hand, it has been shown that when this condition no longer holds, that is $\text{CFL}<1$, the method suffers less from diffusion than the original MPM using the PIC mapping.
Furthermore, the same behavior has been observed for a two-dimensional problem in an elastic medium. 
% Hyperelasticity
The DGMPM moreover provides good results on problems involving large deformations in one and two-dimensional solids.
Hence, the method fulfills its objectives.

%% History-dependent constitutive models
Then, history-dependent constitutive models have been considered within the infinitesimal theory.
First, the solutions of a plane wave in an elasto-viscoplastic solid resulting from DGMPM, MPM, FVM and FEM have been compared.
In the stiff limit, that is when the viscosity is close to zero, the above solutions should tend to the elastic-plastic one.
The simulations based on fractional-step methods combined with the DGMPM highlighted that the use of suitable ODE integrators is required in order to compute the correct solution and the ability to achieve high-order accuracy.
Alternatively, dedicated approximate Riemann solvers can be employed within the DGMPM for elastoplasticity so as to take into account elastic as well as plastic characteristics and hence, to get more accurate solutions.
Nevertheless, the structure of the solutions of hyperbolic problems in elastic-plastic materials is only known for one-dimensional solids in such a way that approximate Riemann solvers do not exist for multi-dimensional cases.
%Those results then motivated further developments carried out in chapter \ref{chap:chap5}.

$\newline$
%% Chapter5: Contribution to the solution of elastic-plastic problems in two space dimensions
The simulations in history-dependent materials performed in chapter \ref{chap:chap4}, as well as the point of view adopted in this thesis, motivated the developments presented in chapter \ref{chap:chap5}.
Indeed, the need for a better understanding of the characteristic structure of elastic-plastic hyperbolic problems arises from the lack of approximate Riemann solvers in more than one-dimension.
%As a result, the spectral analysis of hyperbolic systems in elastoplastic solids under small strain has been proposed.
Although some references have proposed the spectral analysis of hyperbolic systems in elastoplastic solids under small strains since the 60s, the problems treated were limited to particular cases of plane stress and plane strain conditions.
The present approach consisted in \textbf{rewriting the existing formulations by means of the elastoplastic stiffnesses to provide a unified framework for all two-dimensional problems in the Cartesian coordinates system}.
It is well-known that the particularity of elastoplastic problems in more than one space dimension lies in the fact that the wave pattern of the solution depends on the external loads.
More precisely, one must be able to connect the stationary stress state of a Picard problem to the initial data through loading paths followed inside fast and slow simple waves of combined-stress.
Hence, \textbf{the mathematical study of the loading paths} resulting from the characteristic analysis of the hyperbolic system, has been performed.
% The characteristic analysis of the hyperbolic system thus written then enabled to highlight the existence of two families of simple waves propagating in elastoplastic media.
% The combined stress-nature of these waves has next be emphasized through \textbf{the mathematical study of loading paths followed inside fast and slow simple waves}.
This analysis resulted in the identification of typical behaviors of these stress paths. %which are of major importance since they allow to determine the wave pattern of the solution of Picard problems for given external loadings. % that may help to build an approximate Riemann solver.
Nevertheless, the complexity of the equations prevents the complete characterization of the stress evolution inside the simple waves.
As a consequence, a numerical study of the loading paths based on the implicit integration of the ODEs governing the evolution of the system within the waves has been carried out for plane strain and plane stress problems.
First, the particularization of the equations to those governing the thin-walled tube problem treated in the literature, allowed to validate the present approach.
Second, \textbf{the loading paths followed inside fast and slow waves for problems under plane strain and plane stress have been plotted} so that additional features have been highlighted.
Finally, some ideas in order to develop an iterative Riemann solver have been drawn based on the identified behaviors of the loading paths.

%% Ajouter un mot pour répondre aux objectifs fixés
This aspect of the present work aimed at building numerical schemes that are based on a robust discretization, and embed a sufficient amount of information in order to mimic the physical response.
Though interesting features of the loading paths have been emphasized, additional effort is required so as to introduce the characteristic structure by means of an approximate Riemann solver.


\subsection*{Future work}
This thesis has consisted of: (i) the foundation of a computational framework for the simulation of hyperbolic problems in finite deforming solids which enables several extensions; (ii) the study of the characteristic structure of the solution of elastoplastic hyperbolic problems in two space dimensions.
%Each of these aspects comes with more or less long-term perspectives.
Each of these aspects comes with medium/long term perspectives.

\subsubsection*{(i) Improving the DGMPM}
%% High-order approximation
Although the DGMPM has been developed in order to capture discontinuities so that its first-order accuracy is sufficient, the method should be adapted so as to achieve higher-order convergence in regions where the solution is smooth.
% p-adaption
Higher-order accuracy should be achieved by using shape functions based on high-order polynomials ($p$-adaption).
Notice furthermore that the polynomial order can be set arbitrarily within the background mesh due to the discontinuous Galerkin approximation.
Nevertheless, the particle-based quadrature of the DGMPM weak form may limit the polynomial order for a given space discretization in order to avoid problems due to reduced integration.
Indeed, it has been seen that the diagonal lumping of the mass matrix can balance the reduced integration with one particle and linear shape functions, but nothing can be stated for higher-order polynomials.
Hence, analogously to recent developments of the MPM \cite{MPM_BSpline1,MPM_BSpline2,IMPM}, functions reconstruction techniques such as moving least squares or spline interpolation may be employed in order to use Gauss quadrature even for few particles in a cell.
However, such an implementation would be equivalent to the DGFEM which suffers from a very restrictive \text{CFL} condition that prevents the capturing of discontinuities, which is prohibitive.
One solution to circumvent this issue would be to combine the $p$-adaption with $h$-adaption by refining the arbitrary grid in the vicinity of discontinuities.
The underlying idea is to use a thin layer of elements containing one particle along the discontinuity, for which the \text{CFL} number can be set to unity.
The thickness $h$ of that layer would be set so that the time step $\Delta t^s$ corresponding to the most restrictive Courant number in the smooth regions \text{CFL}$^s$, allows the capture of the discontinuity, that is: $\Delta t^s = \text{CFL}^s \frac{h}{c}$, where $c$ is the speed of the fastest wave.

% particle-adaption
% Remove material points
Alternatively, one can imagine a \textit{particle}-adaption, while ensuring the conservation of mass, so that the particle-based integration can be adapted to the regularity of the solution. 
That is: remove material points in such a way that one particle per cell is used near a discontinuity; add particles in elements in which the polynomial order requires so. 
Furthermore, one could benefit from Roe's theorem \ref{th:Roe} in order to adapt the distribution of particles so that a given order of accuracy is achieved for a given polynomial order, as suggested in section \ref{sec:DGMPM_analysis}.
Nonetheless, it is not guaranteed that such a configuration can be found.

% Waves or slope limiters
Slope or wave limiters have not been considered so far within the DGMPM due to the non-oscillatory solutions provided by the scheme.
Nevertheless, the extension of the method to higher-order approximation may require the introduction of such numerical tools that are well-known in DG-methods \cite{Cockburn}.

% RK2 en 2D
At last, the second-order Runge-Kutta time integrator has not been considered for two-dimensional problems, due to the lack of a formula to determine the critical \text{CFL} number for a given space discretization.
Such a two-stages time discretization could be embedded in the method in order to reduce the numerical diffusion when more than one particle lies in a cell, providing that a von-Neumann analysis of two-dimensional DGMPM schemes is performed.
%Therefore, providing that a von-Neumann analysis of DGMPM two-dimensional schemes is performed, the two-stages time discretization could be embedded in the method in order to reduce the numerical diffusion when more than one particle lie in a cell.


\subsubsection*{(ii) On the solution of hyperbolic systems in elastoplastic solids}
%%% Approx RP elastoplastic
The study of the characteristic structure of hyperbolic systems in elastic-plastic solids allows perspectives.

First, the iterative procedure inspired by that of \textsc{Lin} and \textsc{Ballman} \cite{Lin_et_Ballman} and proposed in chapter \ref{chap:chap5} has to be implemented.
Moreover, the analytical developments proposed here can be prosecuted in order to get additional information about the solution so that an approximate Riemann solver can be developed.

Second, the framework provided allows to take into account more general hardening laws that is, kinematic, combined isotropic and kinematic, and nonlinear hardenings.
Nevertheless, the resulting equations should be much more complex and prevent the building of iterative or approximate Riemann solvers.

At last, plastic shocks, which raised scientific questions that are still unanswered, have to be studied for they are involved in problems of dynamic impacts in elastic-plastic solids.

% Essaye le solveur itératif
% Développer un solveur approximé
% D'autres écrouissage
% Etude des chocs plastiques


%%% Local Variables:
%%% mode: latex
%%% ispell-local-dictionary: "american"
%%% TeX-master: "../mainManuscript"
%%% End:
