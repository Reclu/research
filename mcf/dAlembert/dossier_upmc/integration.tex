Comportement dynamique des structures hétérogènes (microstructures, milieux poreux ...) en grandes transformations.

\begin{itemize}
\item Prise en compte des ondes éventuellement discontinues
\item Grandes transformations (méthode numérique adaptée)
\item Plasticité
  \begin{itemize}
  \item influence ondes $\rightarrow$ matériau: calcul des états résiduels ; structure caractéristique du problème (slow, fast ; choc plastique ? même en HPP)
  \item influence matériau $\rightarrow$ ondes: modification des propriétés du milieu (vitesses d'ondes différentes (est-ce que ça ne constitute justement pas une manière de construire des procédure expérimentale?)); instabilités (localisation, PLC, etc.) et influences sur la nature mathématique du problème
  \end{itemize}
\item Aspect multi-physique : comportement thermomécanique $\rightarrow$ recristallisation + perte d'hyperbolicité du problème ; stratégie numérique innovante pour problèmes hyperbolique/parabolique ; envisager des couplages avec d'autres physiques (électromagnétisme par exemple)
\item exemple concret du lasat pour modifier l'état de surface qui est étudié à d'alembert ? avec ensta ou esam
\item Collaboration avec l'équipe fluide pour le développement de méthodes numériques car la dgmpm est inspirée des FVM
\item Collaboration avec GeM (Nantes) et IUSTI (Marseille) pour le dialogue fluide/solide en numérique (je ne sais pas si c'est bien de parler du GeM, ça montre que je suis encore un peu là-bas)
\item Collaboration avec MSSMat
\end{itemize}

Il y a la rupture aussi qui est traitée à d'alembert


%%% Local Variables:
%%% mode: latex
%%% TeX-master: "main"
%%% End:
