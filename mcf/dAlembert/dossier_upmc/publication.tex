\documentclass[10pt,a4paper]{moderncv}
\moderncvtheme[blue]{classic}                
\usepackage[utf8]{inputenc}
% \usepackage[top=1.1cm, bottom=1.1cm, left=2cm, right=2cm]{geometry}
\usepackage{geometry}
\geometry{hmargin=2.cm,vmargin=3.5cm}
% Largeur de la colonne pour les dates
\setlength{\hintscolumnwidth}{2.5cm}

\firstname{Adrien}
\familyname{Renaud}
\photo[1.4cm][0.cm]{photo.JPG}
\title{Docteur en mécanique des solides
  \vskip 0.1cm \footnotesize Méthodes numériques -- Problèmes hyperboliques -- Solides dissipatifs}              
\address{257 rue des Pyrénées}{75020 Paris}    
\email{adri.renaud@gmail.com}                      
\mobile{(+33)6 79 38 81 43} 
\extrainfo{Né le 12/10/1988 (31 ans)}
%\extrainfo{Nationalité Française}

\definecolor{Purple}{RGB}{120,28,129}
\definecolor{Blue}{RGB}{63,96,174}
\definecolor{Duck}{RGB}{83,158,182}
\definecolor{Green}{RGB}{109,179,136}
\definecolor{Yellow}{RGB}{202,184,67}
\definecolor{Orange}{RGB}{231,133,50}
\definecolor{Red}{RGB}{217,33,32}

\begin{document}
% \makecvtitle

%%%%% COMMUNICATIONS
\section{Publications}
          \cventry{2018}{Journal of Computational Physics}{A. Renaud,T. Heuzé and L. Stainier}{A Discontinuous Galerkin Material Point Method for the solution of impact problems in solid dynamics}{369,80-102}{doi: j.jcp.2018.05.001}
          \cventry{2020}{International Journal for Numerical Methods in Engineering}{A. Renaud,T. Heuzé and L. Stainier}{Stability properties of the Discontinuous Galerkin Material Point Method for hyperbolic problems in one and two space dimensions}{121,664-689}{doi: 10.1002/nme.6239}
          \cventry{2020}{Computer Methods in Applied Mechanics and Engineering}{A. Renaud,T. Heuzé and L. Stainier}{The Discontinuous Galerkin Material Point Method for variational hyperelastic–plastic solids}{365}{doi: 10.1016/j.cma.2020.112987}
          \cventry{2020}{Journal of the Mechanics and Physics of Solids}{A. Renaud,T. Heuzé and L. Stainier}{On loading paths followed inside plastic simple waves in two-dimensional elastic-plastic solids}{143}{doi: 10.1016/j.jmps.2020.104064}
          \cventry{2020}{Ultrasonics}{A. Renaud, B. Tie, J.H Schmitt and A.S. Mouronval}{Multi-parameter optimization of attenuation data for characterizing grain size distributions and application to bimodal microstructures}{}{Under review (minor revision requested on 02/16/21)}


\section{\color{Blue}Communications}

        \subsection{\color{Blue}Conférences à comité de lecture}
        \cventry{\textit{Juillet 2018}}{WCCM}{13th World Congress on Computationl Mechanics}{New York (USA)}{\textbf{A. Renaud}, T. Heuzé and L. Stainier: \textit{The Discontinuous Galerkin Material Point Method for Hyperbolic Problems in Solid Mechanics}}{}
        \cventry{\textit{Juin 2018}}{ECCM}{6th European Conference on Computational Mechanics}{Glasgow (Ecosse)}{\textbf{A. Renaud}, T. Heuzé and L. Stainier: \textit{The Discontinuous Galerkin Material Point Method for the simulation of hyperbolic problems in finite deforming solids }}{}
        \cventry{Juin 2017}{COMPDYN}{6th International Conference on Computational Methods in Structural Dynamics and Earthquake Engineering}{Ile de Rhodes (Grèce)}{\textbf{A. Renaud}, T. Heuzé and L. Stainier: \textit{A Discontinuous Galerkin Material Point Method
        (DGMPM) for the simulation of impact problems in solid mechanics
        }}{}
          \cventry{Mai 2017}{CSMA}{13ème colloque national en calcul des structures}{Giens (France)}{\textbf{A. Renaud}, T. Heuzé and L. Stainier: \textit{La Méthode des Points Matériels (MPM) en Galerkin Discontinue (DG) pour la simulation d’impact sur des solides }}{}
          \subsection{\color{Blue}Groupe de travail}
          \cventry{Septembre 2019}{LUS4Metals}{4th International Workshop on Laser-Ultrasonics for metals}{Linz (Austria)}{\textbf{A. Renaud}, A.S. Mouronval, J.H. Schmitt, D. Aubry and B. Tie: \textit{Numerical investigation of ultrasonic wave scattering in polycrystalline materials}}{}
          \cventry{Mars 2017}{Journées Mécadymat}{}{Cachan (France)}{\textbf{A. Renaud} et T. Heuzé : \textit{Extension de la méthode des points matériels à l’approximation de Galerkin discontinue pour la simulation d’impacts en mécanique des solides.}}{}
	




\end{document}
%%% Local Variables:
%%% mode: latex
%%% TeX-master: t
%%% End:
