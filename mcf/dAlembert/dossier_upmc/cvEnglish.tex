\documentclass[11pt,a4paper]{moderncv}
\usepackage{footmisc}
\definecolor{Purple}{RGB}{120,28,129}
\definecolor{Blue}{RGB}{63,96,174}
\definecolor{Duck}{RGB}{83,158,182}
\definecolor{Green}{RGB}{109,179,136}
\definecolor{Yellow}{RGB}{202,184,67}
\definecolor{Orange}{RGB}{231,133,50}
\definecolor{Red}{RGB}{217,33,32}

%\moderncvtheme[blue]{casual}  % style options are 'casual' (default), 'classic', 'banking', 'oldstyle' and 'fancy'               
\moderncvstyle{casual}                            
\moderncvcolor{blue}                              

\usepackage{ragged2e} % used for the command \justify
\usepackage[utf8]{inputenc}
%\usepackage[T1]{fontenc} 
\usepackage[scale=.8]{geometry}
% Largeur de la colonne pour les dates
\setlength{\hintscolumnwidth}{2.75cm}
\usepackage{academicons}

\AtBeginDocument{\recomputelengths}
\AtBeginDocument{\definecolor{sectionrectanglecolor}{RGB}{63,96,174}}
\AtBeginDocument{\definecolor{sectiontitlecolor}{RGB}{63,96,174}}
\makeatletter
\renewcommand{\section}[1]
{ \vspace*{2.5ex \@plus 1ex \@minus .2ex}%
\phantomsection{}% reset the anchor for hyperrefs
\addcontentsline{toc}{part}{#1}%
\parbox[m]{\hintscolumnwidth}{\raggedleft  \hintfont{\color{sectionrectanglecolor}\rule{\hintscolumnwidth}{2pt}}}%<- auf 11pt geaendert
\hspace{\separatorcolumnwidth}%
\parbox[m]{\maincolumnwidth}{\sectionstyle{#1}}\\[1ex]}
\makeatother

\firstname{Adrien}
\familyname{Renaud}
\photo[2.cm][0.cm]{photo.JPG}
%\title{PhD in solid mechanics}              
\title{Researcher in computational mechanics and applied mathematics} 
\address{257 rue des Pyrénées}{75020 Paris}    
\email{adri.renaud@gmail.com}                      
\mobile{(+33)6~79~38~81~43}

% \social[orcid][orcid.org/0000-0002-3075-4933]{Adrien Renaud}
% \social[researchgate][www.researchgate.net/profile/Adrien_Renaud2]{Adrien Renaud}

%\extrainfo{Naissance 12/10/1988}
%\extrainfo{Nationalité Française}



%% Add academicons logos
% hlink
\makeatletter
\newcommand*{\hlink}[2][]{%
  \ifthenelse{\equal{#1}{}}%
    {\href{#2}{#2}}%
    {\href{#2}{#1}}}
\makeatother

% social
% \makeatletter
% \newcommand*\orcidsocialsymbol{{\aiOrcid}}
% \newcommand*\researchgatesocialsymbol{{\aiResearchGate}}
% % \newcommand\faSkype{{\FA\symbol{"F17E}}}
% % \newcommand*{\skypesocialsymbol}{\faSkype~}
% \RenewDocumentCommand{\social}{O{}O{}m}{%
%   \ifthenelse{\equal{#2}{}}%
%     {%
%       \ifthenelse{\equal{#1}{linkedin}}{\collectionadd[linkedin]{socials}{\protect\httplink[#3]{www.linkedin.com/in/#3}}}{}%
%       \ifthenelse{\equal{#1}{github}}  {\collectionadd[github]{socials}  {\protect\httplink[#3]{www.github.com/#3}}}     {}%
%       \ifthenelse{\equal{#1}{orcid}}   {\collectionadd[orcid]{socials}   {\protect\hlink[#3]{#3}}}              {}%
%       \ifthenelse{\equal{#1}{researchgate}}   {\collectionadd[researchgate]{socials}   {\protect\hlink[#3]{#3}}}              {}%
%     }
%     {\collectionadd[#1]{socials}{\protect\httplink[#3]{#2}}}}
% \makeatother

\nopagenumbers 
\renewcommand\thefootnote{\textcolor{Blue}{\arabic{footnote}}}
\begin{document}


\makecvtitle



\section{\color{Blue}Research experiences}
% mettre ici le doctorat et le postdoc
\cventry{\textbf{2019-2020}}{Postdoctoral researcher}{CentraleSup{\'e}lec -- MSSMat (France)}{}{}{
  \begin{itemize}
%  \item[-] 3D numerical modeling of ultrasonic wave propagation during heat treating of superalloys
  \item[-] Three-dimensional numerical modeling of wave propagation in polycrystalline materials,
  \item[-] Investigation of the effect of the grain morphology on ultrasonic wave scattering.
  \end{itemize}
  \vskip 0.1cm Contributors: Bing Tie, Anne-Sophie Mouronval, Jean-Hubert Schmitt, Denis Aubry.}
\cventry{2015-2018}{PhD thesis}{Ecole centrale de Nantes -- GeM (France)}{}{}{
  \begin{itemize}
  \item[-] Development of a numerical method for solving hyperbolic problems in finite deforming dissipative solids: the Discontinuous Galerkin Material Point Method (DGMPM),
  \item[-] Numerical analysis of the DGMPM,
  \item[-] Identification of the response of two-dimensional elastic-plastic solids to dynamic loadings under small strains.
  \end{itemize}
  \vskip 0.1cm Supervisors: Laurent Stainier, Thomas Heuz{\'e}
}
\cventry{2013--2015}{Master research internship}{Ecole centrale de Nantes -- GeM}{}{}{%Thesis: \textit{Numerical simulation of the formation of banded structures in Friction Stir Welding by tool eccentricity}
  \begin{itemize}
  \item[-] Numerical simulation of the formation of banded structures in Friction Stir Welding,
  \item[-] Numerical modeling based on X-FEM technology and thermo-mechanical coupling
  \end{itemize}
  \vskip 0.1cm Supervisor: Thomas Heuz{\'e}
}
  
\section{\color{Blue}Education}
\cventry{2015-2018}{PhD of mechanical engineering}{Ecole centrale de Nantes -- GeM}{}{}{Thesis: \textit{The Discontinuous Galerkin Material Point Method: application to hyperbolic problems in solid mechanics}
  %\vskip 0.1cm Development of a numerical method for the solution of hyperbolic problems in finite deforming dissipative solids; Identification of the response of two-dimensional elastic-plastic solids to dynamic loadings under small strains
  %\vskip 0.1cm Supervisors: Laurent Stainier, Thomas Heuz{\'e}
}
\cventry{2013--2015}{Master of Computational mechanics (with honors)}{University of Nantes}{}{}{%Thesis: \textit{Numerical simulation of the formation of banded structures in Friction Stir Welding by tool eccentricity}
  %\vskip 0.1cm Modeling the eccentricity of the tool using X-FEM technology; Eulerian thermofluid simulation
  %\vskip 0.1cm Supervisor: Thomas Heuz{\'e}
}
\cventry{2012--2013}{Bachelor of Mechanics (with honors)}{University of Nantes}{}{}{
}


\section{\color{Blue}Teaching experiences}
\cventry{2015--2018}{Assistant teacher}{Bachelor students -- Ecole centrale de Nantes}{ Continuum Mechanics (192 hours)}{}{}

% \section{\color{Blue}Research interests}
% \cvitemwithcomment{Solid mechanics}{Hyperbolic problems -- (Hyper)Elastic-plastic solids}{}
% \cvitemwithcomment{Simulation}{Numerical analysis -- DG approximation -- FEM -- MPM -- FVM}{}


\section{\color{Blue}Academic experiences}
\cventry{2017--2018}{Elected member of the laboratory concil of GeM}{}{}{}{}

\newpage	
\section{\color{Blue}Skills}
\cvitemwithcomment{Languages}{French: native -- English: fluent \textsc{Toeic} 910}{}
\cvitemwithcomment{Programming}{Object oriented \& template programming: Python, C++}{}
\cvitemwithcomment{}{Procedural Programming: Python, Matlab, Fortran}{}
\cvitemwithcomment{}{Parallel computing: MPI, OpenMP}{}
\cvitemwithcomment{Post-processing}{Gmsh, Paraview}{}
\cvitemwithcomment{Miscellaneous}{\LaTeX, Emacs, Git, GNU/Linux}{}


\section{\color{Blue}Journal publications}
\cventry{2018}{Journal of Computational Physics}{A. Renaud,T. Heuz{\'e} and L. Stainier}{A Discontinuous Galerkin Material Point Method for the solution of impact problems in solid dynamics}{369,80-102}{doi: j.jcp.2018.05.001}
\cventry{2020}{International Journal for Numerical Methods in Engineering}{A. Renaud,T. Heuz{\'e} and L. Stainier}{Stability properties of the Discontinuous Galerkin Material Point Method for hyperbolic problems in one and two space dimensions}{121,664-689}{doi: 10.1002/nme.6239}
\cventry{2019}{Computer Methods in Applied Mechanics and Engineering}{A. Renaud,T. Heuz{\'e} and L. Stainier}{The Discontinuous Galerkin Material Point Method for variational hyperelastic-plastic solids}{}{Under review}
\cventry{2020}{Journal of the Mechanics and Physics of Solids}{A. Renaud,T. Heuz{\'e} and L. Stainier}{A unified framework for simple wave solutions in two-dimensional elastic-plastic solids}{}{Under review}

\section{\color{Blue}Communications}
\cventry{September 2019}{LUS4Metals}{4th International Workshop on Laser-Ultrasonics for metals}{Linz (Austria)}{\textbf{A. Renaud}, A.S. Mouronval, J.H. Schmitt, D. Aubry and B. Tie: \textit{Numerical investigation of ultrasonic wave scattering in polycrystalline materials}}{}
\cventry{July 2018}{WCCM}{13th World Congress on Computationl Mechanics}{New York (USA)}{\textbf{A. Renaud}, T. Heuz{\'e} and L. Stainier: \textit{The Discontinuous Galerkin Material Point Method for Hyperbolic Problems in Solid Mechanics}}{}
\cventry{June 2018}{ECCM}{6th European Conference on Computational Mechanics}{Glasgow (Scotland)}{\textbf{A. Renaud}, T. Heuz{\'e} and L. Stainier: \textit{The Discontinuous Galerkin Material Point Method for the simulation of hyperbolic problems in finite deforming solids }}{}
\cventry{June 2017}{COMPDYN}{6th International Conference on Computational Methods in Structural Dynamics and Earthquake Engineering}{Rhodes Island (Greece)}{\textbf{A. Renaud}, T. Heuz{\'e} and L. Stainier: \textit{A Discontinuous Galerkin Material Point Method
(DGMPM) for the simulation of impact problems in solid mechanics
}}{}
\cventry{May 2017}{CSMA}{13ème colloque national en calcul des structures}{Giens (France)}{\textbf{A. Renaud}, T. Heuz{\'e} and L. Stainier: \textit{La M{\'e}thode des Points Mat{\'e}riels (MPM) en Galerkin Discontinue (DG) pour la simulation d’impact sur des solides }}{}



\end{document}
%%% Local Variables:
%%% mode: latex
%%% TeX-master: t
%%% End:
