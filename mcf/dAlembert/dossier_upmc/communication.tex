% Exemple de CV utilisant la classe moderncv
% Style classic en bleu
% Article complet : http://blog.madrzejewski.com/creer-cv-elegant-latex-moderncv/

\documentclass[10pt,a4paper]{moderncv}
\moderncvtheme[blue]{classic}                
\usepackage[utf8]{inputenc}
\usepackage[top=3.5cm, bottom=3.5cm, left=2cm, right=2cm]{geometry}
\usepackage{geometry}
%\geometry{hmargin=2.cm,vmargin=3.5cm}
% Largeur de la colonne pour les dates
\setlength{\hintscolumnwidth}{2.5cm}

\firstname{\LARGE Adrien}
\familyname{Renaud}
\photo[1.4cm][0.cm]{photo.JPG}
\title{Docteur en mécanique des solides
  \vskip 0.1cm \footnotesize Méthodes numériques -- Problèmes hyperboliques -- Solides dissipatifs}              
\address{257 rue des Pyrénées}{75020 Paris}    
\email{adri.renaud@gmail.com}                      
\mobile{(+33)6 79 38 81 43} 
\extrainfo{Né le 12/10/1988 (31 ans)}
%\extrainfo{Nationalité Française}

\definecolor{Purple}{RGB}{120,28,129}
\definecolor{Blue}{RGB}{63,96,174}
\definecolor{Duck}{RGB}{83,158,182}
\definecolor{Green}{RGB}{109,179,136}
\definecolor{Yellow}{RGB}{202,184,67}
\definecolor{Orange}{RGB}{231,133,50}
\definecolor{Red}{RGB}{217,33,32}

\begin{document}
        %%%%% COMMUNICATIONS
        \section{Communications}
        \subsection{Conférences à comité de lecture}
        \cventry{\textit{Juillet 2018}}{WCCM}{13th World Congress on Computationl Mechanics}{New York (USA)}{\textbf{A.Renaud}, T.Heuzé and L.Stainier: \textit{The Discontinuous Galerkin Material Point Method for Hyperbolic Problems in Solid Mechanics}}{}
        \cventry{\textit{Juin 2018}}{ECCM}{6th European Conference on Computational Mechanics}{Glasgow (Ecosse)}{\textbf{A.Renaud}, T.Heuzé and L.Stainier: \textit{The Discontinuous Galerkin Material Point Method for the simulation of hyperbolic problems in finite deforming solids }}{}
        \cventry{Juin 2017}{COMPDYN}{6th International Conference on Computational Methods in Structural Dynamics and Earthquake Engineering}{Ile de Rhodes (Grèce)}{\textbf{A.Renaud}, T.Heuzé and L.Stainier: \textit{A Discontinuous Galerkin Material Point Method
        (DGMPM) for the simulation of impact problems in solid mechanics
        }}{}
          \cventry{Mai 2017}{CSMA}{13ème colloque national en calcul des structures}{Giens (France)}{\textbf{A.Renaud}, T.Heuzé and L.Stainier: \textit{La Méthode des Points Matériels (MPM) en Galerkin Discontinue (DG) pour la simulation d’impact sur des solides }}{}
          \subsection{Groupe de travail}
          \cventry{Mars 2017}{Journées Mécadymat}{}{Cachan (France)}{}{}
	
          \section{Publications}
          \cventry{2018}{Journal of Computational Physics}{A.Renaud,T.Heuzé and L.Stainier}{A Discontinuous Galerkin Material Point Method for the solution of impact problems in solid dynamics}{369,80-102}{doi: j.jcp.2018.05.001}
          \cventry{2018}{International Journal for Numerical Methods in Engineering}{A.Renaud,T.Heuzé and L.Stainier}{Stability properties of the Discontinuous Galerkin Material Point Method for hyperbolic problems in one and two space dimensions}{}{Under review}

      \end{document}
%%% Local Variables:
%%% mode: latex
%%% TeX-master: t
%%% End:

