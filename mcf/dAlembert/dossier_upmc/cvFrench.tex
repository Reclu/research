% Exemple de CV utilisant la classe moderncv
% Style classic en bleu
% Article complet : http://blog.madrzejewski.com/creer-cv-elegant-latex-moderncv/

\documentclass[10pt,a4paper]{moderncv}
\moderncvtheme[blue]{classic}                
\usepackage[utf8]{inputenc}
\usepackage[top=6.cm, bottom=3.5cm, left=2cm, right=2cm]{geometry}
\usepackage{geometry}
%\geometry{hmargin=2.cm,vmargin=3.5cm}
% Largeur de la colonne pour les dates
\setlength{\hintscolumnwidth}{2.5cm}

\firstname{\LARGE Adrien}
\familyname{Renaud}
\photo[1.4cm][0.cm]{photo.JPG}
\title{Docteur en mécanique des solides
  \vskip 0.1cm \footnotesize Méthodes numériques -- Problèmes hyperboliques -- Solides dissipatifs}              
\address{5 rue Clisson}{75013 Paris}    
\email{adri.renaud@gmail.com}                      
\mobile{(+33)6 79 38 81 43} 
\extrainfo{Né le 12/10/1988 (32 ans)}
%\extrainfo{Nationalité Française}

\definecolor{Purple}{RGB}{120,28,129}
\definecolor{Blue}{RGB}{63,96,174}
\definecolor{Duck}{RGB}{83,158,182}
\definecolor{Green}{RGB}{109,179,136}
\definecolor{Yellow}{RGB}{202,184,67}
\definecolor{Orange}{RGB}{231,133,50}
\definecolor{Red}{RGB}{217,33,32}

\begin{document}
%\vskip 1.cm
\makecvtitle
\vspace{ -.75cm}
	\section{\color{Blue}Formation}
	\cventry{2015-2018}{Doctorat en génie mécanique}{Ecole centrale de Nantes -- GeM}{Bourse MESRI}{}{\underline{Thèse} : \textit{The Discontinuous Galerkin Material Point Method: application to hyperbolic problems in solid mechanics.}} 
          % \\\noindent\rule[0.5ex]{\linewidth}{1pt}
          % \vskip 0.1cm Jury :
          % \vskip 0.1cm Deborah Sulsky, Université du Nouveau-Mexique, Rapporteur
          % \vskip 0.1cm Antonio J. Gil, Université de Swansea, Rapporteur
          % \vskip 0.1cm Nicolas Favrie, Université d'Aix Marseille, Examinateur
          % \vskip 0.1cm Anthony Gravouil, INSA Lyon, Président
          % \vskip 0.1cm Laurent Stainier, Centrale Nantes, Directeur
          % \vskip 0.1cm Thomas Heuzé, Centrale Nantes, Directeur
          % \\\noindent\rule[0.5ex]{\linewidth}{1pt}
          \cventry{2014--2015}{Master 2 : Mécanique Numérique}{Université de Nantes}{Mention Très Bien}{}{}
          %\underline{Mémoire} : \textit{Simulation de l'extrusion de matière lors du procédé de soudage par friction malaxage due à un défaut d'excentrement de l'outil.}
          % \vskip 0.1cm \textbf{Mots-clés }: Simulation thermo-mécanique, X-FEM, Fluide visco-plastique.
          % \vskip 0.1cm Encadrant : Thomas Heuz{\'e} (GeM)
        
	\cventry{2013--2014}{Master 1 : Physique-Mécanique}{Université de Nantes}{Mention Bien}{}{}
          % \underline{Stage} : \textit{Développement d'un logiciel de vérification aux états limites des serres de production horticoles.}
          % \vskip 0.1cm Encadrant : Jean-Vivien Heck (Centre Scientifique et Technique du Bâtiment -- Nantes)
        %}
        \cventry{2012--2013}{Licence 3 : Physique-Mécanique}{Université de Nantes}{Mention Très Bien}{}{}
        \cventry{2010--2012}{Licences 1 et 2 : Mathématiques-Physique-Mécanique}{Université de Nantes}{}{}{}
          %\underline{Projet étudiant} : \textit{Application de la théorie linéarisée des coques de Kirchhoff-Love au flambement des tubes cylindriques sous pression externe.}
          %\vskip 0.1cm Encadrant : Anh Le van (GeM)
        

        \section{\color{Blue}Expériences}
        \subsection{\color{Blue}Recherche}
        % mettre ici le doctorat et le postdoc
        \cventry{\textbf{Depuis 2020}}{Chercheur postdoctoral}{CEA Paris-Saclay -- LC2M (France)}{}{}{\textit{{\'E}tude numérique de la diffusion des ultrasons dans les matériaux polycristallins.}
          \vskip 0.1cm Collaborateurs : Laurent Dupuy, Lionel Gélébart.}
        \cventry{2019--2020}{Chercheur postdoctoral}{CentraleSup{\'e}lec -- MSSMat (France)}{}{}{\textit{{\'E}tude numérique de la diffusion des ultrasons dans les matériaux polycristallins : Développement d'une procédure de caractérisation des microstructures bimodales.}
          \vskip 0.1cm Collaborateurs : Bing Tie, Anne-Sophie Mouronval, Jean-Hubert Schmitt, Denis Aubry, Denis Solas.}
        \subsection{\color{Blue}Enseignement}
        \cventry{2020--2021}{CentraleSup{\'e}lec}{}{2ème année de cycle ingénieur}{}{
          \begin{itemize}
        \item[-] Travaux dirigés de \textit{Structural Vibrations and Acoustics} : 18h
          \end{itemize}
        }
        \cventry{2015--2018}{Ecole Centrale de Nantes}{}{1ère année de cycle ingénieur}{}{
          \begin{itemize}
          \item[-] Travaux dirigés de Modélisation et Mécanique des Milieux Continus : 94h
          \item[-] Travaux pratiques de Modélisation et Mécanique des Milieux Continus : 98h
          \end{itemize}
        }
        \subsection{\color{Blue}Académique}
        \cventry{2017--2018}{Conseil de laboratoire du GeM}{Membre élu doctorant}{}{}{}

        
	\section{\color{Blue}Compétences}
	\cvitemwithcomment{Anglais}{Courant -- \textsc{Toeic} 910 (2016)}{}
        \cvitemwithcomment{Programmation}{Orientée objet et programmation template : Python, C++}{}
        \cvitemwithcomment{}{Fonctionnelle : Python, Matlab, Fortran}{}
        \cvitemwithcomment{}{Calcul parallèle : MPI, OpenMP}{}
	\cvitemwithcomment{Simulation}{FEM, MPM, FVM, DGMPM, FFT, Discrete Dislocation Dynamics}{}
	\cvitemwithcomment{Visualisation}{\textsc{Neper}, \textsc{Gmsh}, Paraview}{}
	\cvitemwithcomment{Autre}{\LaTeX, Emacs, Git, GNU/Linux}{}

        %   \section{Publications}
        %   \cventry{2018}{Journal of Computational Physics}{A. Renaud,T. Heuzé and L. Stainier}{A Discontinuous Galerkin Material Point Method for the solution of impact problems in solid dynamics}{369,80-102}{doi: j.jcp.2018.05.001}
        %   \cventry{2020}{International Journal for Numerical Methods in Engineering}{A. Renaud,T. Heuzé and L. Stainier}{Stability properties of the Discontinuous Galerkin Material Point Method for hyperbolic problems in one and two space dimensions}{121,664-689}{doi: 10.1002/nme.6239}
        %   \cventry{2020}{Computer Methods in Applied Mechanics and Engineering}{A. Renaud,T. Heuzé and L. Stainier}{The Discontinuous Galerkin Material Point Method for variational hyperelastic–plastic solids}{365}{doi: 10.1016/j.cma.2020.112987}
        %   \cventry{2020}{Journal of the Mechanics and Physics of Solids}{A. Renaud,T. Heuzé and L. Stainier}{On loading paths followed inside plastic simple waves in two-dimensional elastic-plastic solids}{143}{doi: 10.1016/j.jmps.2020.104064}
        % %%%%% COMMUNICATIONS
        % \section{Communications}
        % \subsection{Conférences à comité de lecture}
        % \cventry{\textit{Juillet 2018}}{WCCM}{13th World Congress on Computationl Mechanics}{New York (USA)}{\textbf{A. Renaud}, T. Heuzé and L. Stainier: \textit{The Discontinuous Galerkin Material Point Method for Hyperbolic Problems in Solid Mechanics}}{}
        % \cventry{\textit{Juin 2018}}{ECCM}{6th European Conference on Computational Mechanics}{Glasgow (Ecosse)}{\textbf{A. Renaud}, T. Heuzé and L. Stainier: \textit{The Discontinuous Galerkin Material Point Method for the simulation of hyperbolic problems in finite deforming solids }}{}
        % \cventry{Juin 2017}{COMPDYN}{6th International Conference on Computational Methods in Structural Dynamics and Earthquake Engineering}{Ile de Rhodes (Grèce)}{\textbf{A. Renaud}, T. Heuzé and L. Stainier: \textit{A Discontinuous Galerkin Material Point Method
        % (DGMPM) for the simulation of impact problems in solid mechanics
        % }}{}
        %   \cventry{Mai 2017}{CSMA}{13ème colloque national en calcul des structures}{Giens (France)}{\textbf{A. Renaud}, T. Heuzé and L. Stainier: \textit{La Méthode des Points Matériels (MPM) en Galerkin Discontinue (DG) pour la simulation d’impact sur des solides }}{}
        %   \subsection{Groupe de travail}
        %   \cventry{Septembre 2019}{LUS4Metals}{4th International Workshop on Laser-Ultrasonics for metals}{Linz (Austria)}{\textbf{A. Renaud}, A.S. Mouronval, J.H. Schmitt, D. Aubry and B. Tie: \textit{Numerical investigation of ultrasonic wave scattering in polycrystalline materials}}{}
        %   \cventry{Mars 2017}{Journées Mécadymat}{}{Cachan (France)}{\textbf{A. Renaud} et T. Heuzé : \textit{Extension de la méthode des points matériels à l’approximation de Galerkin discontinue pour la simulation d’impacts en mécanique des solides.}}{}
	


      \end{document}
%%% Local Variables:
%%% mode: latex
%%% TeX-master: t
%%% End:
