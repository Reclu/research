\section{Activités d'enseignement}

J'ai bénéficié lors de mon doctorat d'un monitorat de trois ans à l'{\'E}cole Centrale de Nantes.
J'y ai assuré des Travaux Dirigés (TD) et des Travaux Pratiques (TP) du module de Mécanique des Milieux Continus et MOdélisation (MEMCO) en première année du cycle ingénieur, devant des groupes allant de vingt étudiants en TP à trente en TD.

\subsection*{Détail des activités}
\label{sec:detail-des-activites}

\begin{itemize}
\item \textbf{Travaux dirigés} de Mécanique des Milieux Continus et Modélisation (2015--2018)
\item[] \underline{Responsables}: Michel Coret, Laurent Stainier
\item[] \underline{Volume horaire}: 94 heures
\item[] \underline{Niveau}: Première année de cycle ingénieur
\item[] \underline{Description}: Les TDs présentaient en premier lieu les outils d'algèbres linéaire et tensorielle utilisés en mécanique des milieux continus. Ensuite, les notions de descriptions Eulérienne et Lagrangienne ainsi que les mesures de déformation associées étaient manipulées. Après linéarisation des déformations, les notions de contraintes et de critères de ruptures étaient illustrées en utilisant les cercles de Mohr. Les lois de comportement étaient abordées grâce à une approche orientée sur la caractérisation expérimentale pour des matériaux isotropes et orthotropes, ce qui permettait de fermer le problème d'élasticité 3D et de le résoudre. Le parallèle avec les milieux fluides était alors effectué. La fin du programme portait sur la résolution de problèmes d'élasticité par des méthodes énergétiques (principes du minimum de l'énergie potentielle et d'Hamilton) et l'introduction à des méthodes numériques (Ritz, Rayleigh-Ritz et {\'E}léments finis).

\item \textbf{Travaux pratiques} de Mécanique des Milieux Continus et Modélisation: Mesures de déformations sur un cylindre sous pression (2015--2018)
\item[] \underline{Responsables}: Michel Coret, Thomas Heuzé
\item[] \underline{Volume horaire}: 74 heures équivalent TD
\item[] \underline{Niveau}: Première année de cycle ingénieur
\item[] \underline{Description}: Le premier objectif du TP était de valider la solution analytique du problème de tube cylindrique ``épais'' soumis à une pression interne calculée en TD à l'aide de mesures de déformations par jauges. Le second objectif était de regarder l'influence de l'épaisseur du cylindre grâce à un second dispositif expérimental de cylindre ``mince'', capable par ailleurs de simuler des conditions aux limites de bords ouverts ou de bords fermés. Les résultats obtenus par passage à la limite de la solution cylindre épais étaient eux aussi comparés aux mesures afin de sensibiliser les étudiants à l'importance des hypothèses simplificatrices de modélisation. 
  
\item \textbf{Travaux pratiques} de Mécanique des Milieux Continus et Modélisation: Mesures de portance sur un patin hydrodynamique (2015)
\item[] \underline{Responsables}: Michel Coret, Thomas Heuzé
\item[] \underline{Volume horaire}: 24 heures équivalent TD
\item[] \underline{Niveau}: Première année de cycle ingénieur
\item[] \underline{Description}: Le TP visait à évaluer une charge portante par l'intermédiaire de mesures de pression sur un patin formant un espace convergeant avec un tapis roulant. Les équations du problème, ainsi que la solution exacte dans un cas idéalisé et une solution numérique du cas considéré, étaient présentées aux étudiants pour comparaison avec les mesures expérimentales. Le sujet permettait aux étudiants d'identifier les limites de la solution idéalisée et des cas de fonctionnement optimaux d'un système de guidage (\textit{i.e. portance maximale en fonction de l'inclinaison du patin et de la vitesse du tapis}). 
\end{itemize}



\section{Projet d'enseignement}
La fonction de transmission est pour moi essentielle au métier d'enseignant-chercheur.
D'une part, elle est nécessaire à la formation d'esprits critiques et, d'autre part, elle requiert une prise de recul sur les activités de recherche en vue d'une bonne restitution.
Ce second aspect, à travers les questionnements des étudiants que j'ai côtoyés, m'a poussé à envisager les concepts de la mécanique sous des angles différents.
Cela se traduit par une volonté d'enrichir ma culture avec des cas pratiques illustrant des phénomènes émanant des équations manipulées.

Mes cinq années d'études à l'université de Nantes me portent par ailleurs à me tourner en priorité vers un poste d'enseignant-chercheur en université.
Le milieu universitaire représente pour moi un espace privilégié pour la mise en place d'une pédagogie étalée sur plusieurs années, basée sur un cursus continu et cohérent tel que ceux proposés sur le campus Pierre et Marie Curie.

Ma formation ainsi que mes activités de recherche m'ont permis de développer un bagage solide en \textit{mécanique des milieux continus}, \textit{programmation} (Matlab, Python, Fortran, C++ et calcul parallèle) et \textit{mathématiques appliquées}.
Aussi, je peux intervenir sur l'ensemble du parcours Licence-Master en proposant une animation participative, nourrie par des illustrations qui, le cas échéant, peuvent résulter de simulations numériques.
De plus, l'écoute et le suivi des étudiants sont pour moi capitaux afin de proposer un discours adapté et d'éviter au maximum les situations de décrochage.  
C'est pourquoi il est envisageable que je contribue à la mise en place ou au maintien de supports pédagogiques (énoncés de cours ou TD/TP, moodle, \textit{etc.}) en adéquation avec les demandes des étudiants et la rigueur scientifique requise, et que je prenne des responsabilités au sein de l'équipe pédagogique (responsabilité de modules, mise à jour des plaquettes de parcours, \textit{etc.}).

Dans la suite, je détaille mon intégration dans les modules proposés par l'offre pédagogique de Sorbonne université.
\subsection*{Interventions en Licence}
\label{sec:interv-en-licence}
% D'après ce que j'ai compris, il s'agit surtout dans un premier temps de prendre en charge des TD et des TP, mais je suis prèt à m'investir que ce soit par un dialogue auprès des responsables de cours ou par la prise en charge de cours.

En ce qui concerne les interventions sous forme de TD et de TP, je pourrais intégrer l'équipe pédagogique dans les modules suivants:
\begin{itemize}
\item \text{Statique et dynamique des solides indéformables} (TD)
\item Bases de la thermodynamique (TD/TP)
\item Méthodes mathématiques pour l’ingénierie (TD)
\item Statique et Dynamique des fluides (TD/TP)
\item Programmation pour le calcul scientifique (TD/TP)
\item \text{Bases de la mécanique des milieux continus} (TD/TP)
\item \text{Méthodes numériques pour la mécanique} (TD/TP)
\item {\'E}quations aux dérivées partielles de la mécanique 1 et 2 (TD)
\item \text{Structures élastiques} (TD/TP) 
\end{itemize}
% Statique et dynamique des solides indéformables (CM/TD) ; Bases de la thermodynamique (TD/TP) ; Méthodes mathématiques pour l’ingénierie (TD) ; Statique et Dynamique des fluides (CM/TD/TP) ; Programmation pour le calcul scientifique (TD/TP/Projet) ; Bases de la mécanique des milieux continus (CM/TD/TP) ; Méthodes numériques pour la mécanique (CM/TD/TP/Projet) ; Equations aux dérivées partielles de la mécanique 1 et 2 (TD) ; Structures élastiques (CM/TD/TP) 

En fonction des besoins immédiats ou à venir, je pourrais également intervenir sur les cours de \textit{Statique et dynamique des solides indéformables}, \textit{Bases de la mécanique des milieux continus}, \textit{Méthodes numériques pour la mécanique} et \textit{Structures élastiques}.
% Les modules d'enseignements soulignés dans la liste précédentes sont ceux où je suis par ailleurs susceptible d'intervenir en cours en fonction des besoins immédiats ou à venir.
De plus, je peux encadrer des projets numériques dans le contexte des cours de Programmation pour le calcul scientifique et de Méthodes numériques pour la mécanique.
% Par ailleurs, en fonction des besoins immédiats ou à venir, il est possible que j'assure les cours 
% {\`A} terme, prise de responsabilité pour le maintien des cours de: Statique et dynamique des solides indéformables (CM/TD) ; Bases de la mécanique des milieux continus (CM/TD/TP) ; Méthodes numériques pour la mécanique (CM/TD/TP/Projet) ; Structures élastiques (CM/TD/TP) 


\subsection*{Interventions en Master}
\label{sec:interv-en-licence}

Je suis à même d'intervenir en première année de Master Mécanique des Solides au niveau des enseignements suivants:
\begin{itemize}
\item \text{MMC solide} (TD)
\item Vibrations et ondes (TD/TP)
\item Méthodes numériques (TP)
\item Analyse des structures par éléments finis (TD/TP)
\item \text{Comportement des matériaux solides} (TD)
\item structures élancées (TD) 
\item \text{Plasticité et analyse limite} (TD)  
\item \text{Dynamique des structures}
\end{itemize}

Je pourrais également prendre les responsabilités des cours de \textit{Comportement des matériaux solides} et de \textit{Plasticité et analyse limite} qui risqueraient d'être vacantes dans les années à venir. 
%Je peux également m'investir dans les cours de \textit{Comportement des matériaux solides} et de \textit{Plasticité et analyse limite}, Corrado Maurini m'ayant informé que leurs responsabilités allaient être vacantes dans les années à venir.
Il en va de même pour les cours de \textit{MMC solide} et de \textit{Dynamique des structures}, ce dernier étant assuré par des extérieurs à l'heure actuelle. 
% Encore une fois, les matières soulignées sont celles où je suis susceptible de prendre en charge des cours magistraux.

%En particulier ceux de \textit{Comportement des matériaux solides} et de \textit{Plasticité et analyse limite}, dont les responsables actuels sont amenés à quitter l'université dans les années à venir, et celui de \textit{Dynamique des structures} assuré en ce moment par des extérieurs.

$\newline$
En ce qui concerne la deuxième année de Master, j'envisage d'intégrer la thématique Modélisation et Simulation dans les modules suivants:
\begin{itemize}
\item Calcul numérique des solides et structures non-linéaires 
\item Pratiques de codes de calculs des structures et applications
\end{itemize}
% M1: TC [MMC solide (CM/TD); Vibrations et ondes (TD/TP) ; Méthodes numériques (TP) ]; Analyse des structures par éléments finis (TD/TP); comportement des matériaux solides (TD) ; structures élancées (TD) ; Plasticité et analyse limite (TD) ; cours de dynamique des structures

% M2 Modélisation et Simulation: Calcul numérique des solides et structures non-linéaires (CM/TP) ; Pratiques de codes de calculs des structures et applications



%%% Local Variables:
%%% mode: latex
%%% TeX-master: "main"
%%% End:
