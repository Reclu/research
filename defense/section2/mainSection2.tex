\subsection{Continuum equations}

\begin{frame}
  \begin{block}{Solid volume $\Omega_0$ bounded by $\partial \Omega$ in the reference configuration}
    \metroset{block=fill}
    % Cartesian coordinates system
    \begin{footnotesize}
      \begin{columns}
        \begin{column}{0.5\textwidth}
          \begin{block}{Finite strain}
            \begin{flalign*}
              \: \rho_0 & \drond{\vect{v}}{t} - \nablav_0 \cdot \tens{\Pi} = \rho_0\vect{b} &\\
              & \drond{\tens{F}}{t} - \nablav_0 \cdot (\vect{v} \otimes \tens{I})= \tens{0} \\
              & \Ucb = \matrice{\rho_0\vect{v} \\ \tens{F}} \: ; \: \Fcb=-\matrice{\tens{\Pi}\\ \vect{v} \otimes \tens{I}}\: ; \: \Scb=\matrice{\rho_0\vect{b} \\ \tens{0}}
            \end{flalign*} 
          \end{block}
        \end{column}
        \begin{column}{0.5\textwidth}
          \begin{block}{Linearized geometrical framework}
            \begin{flalign*}
              \: \rho_0 & \drond{\vect{v}}{t} - \nablav_0 \cdot \tens{\sigma} = \rho_0\vect{b} &\\
              & \drond{\tens{\eps}}{t} - \nablav_0 \cdot \frac{\vect{v} \otimes \tens{I} + \tens{I}\otimes \vect{v}}{2}= \tens{0} \\
              & \Ucb = \matrice{\rho_0\vect{v} \\ \tens{\eps}} \: ; \: \Fcb=-\matrice{\tens{\sigma}\\ \frac{\vect{v} \otimes \tens{I} + \tens{I}\otimes \vect{v}}{2}}\: ; \: \Scb=\matrice{\rho_0\vect{b} \\ \tens{0}}
            \end{flalign*}
          \end{block}
        \end{column}
      \end{columns}
      \pause
      \textbf{\normalsize System of Lagrangian conservation laws \cite{Plohr}:}
      \begin{equation*}
        \drond{\Ucb}{t} + \nablav_0 \cdot \Fcb = \Scb
      \end{equation*}
    \end{footnotesize}
    {\tiny
      \usebibitemtemplate{\color{structure}\insertbiblabel} 
      \usebibliographyblocktemplate{\color{structure}}{\color{black}}{\color{structure!75}}{\color{structure!75}} 
      \begin{thebibliography}{ThomasEM}
      \bibitem[1]{Plohr}
        B.J. Plohr, D.H. Sharp.
        \newblock A conservative Eulerian formulation of the equations for elastic flow
        \newblock {\em Advances in Applied Mathematics (1988)}.
      \end{thebibliography}}
  \end{block}
  
\end{frame}

\begin{frame}
  %% Alternatively by introducing the constitutive equations, one can write the quasi-linear form
  %% Required ??
  \begin{columns}
    \begin{column}{0.5\textwidth}
      \begin{block}{Constitutive models}
        \metroset{block=fill}
        \begin{block}{Finite strains}
          Hyperelasticity: $\dot{\tens{\Pi}} = \Hbb(\tens{F}) : \dot{\tens{F}}$
        \end{block}
        \begin{block}{Linearized geometrical framework}
          Linear elasticity: $\dot{\tens{\sigma}} = \Cbb : \dot{\tens{\eps}}$ \\
          Elasto-viscoplasticity: $\dot{\tens{\sigma}} = \Cbb : (\dot{\tens{\eps}}-\dot{\tens{\eps}^p})$\\
          Elastoplasticity: $\dot{\tens{\sigma}} = \Hbb(\tens{\sigma}) : \dot{\tens{\eps}}$
        \end{block}    
      \end{block}
    \end{column}
    \pause
    \begin{column}{0.5\textwidth}
      \begin{block}{Quasi-linear forms}
        \begin{block}{}
          $\Qcb=\matrice{\tens{\Pi}\\\vect{v}} \rightarrow \drond{\Qcb}{t} + \nablav_0 \cdot \Fcb = \Scb$
        \end{block}
        \begin{block}{}
          % Linear elasticity: $\dot{\tens{\sigma}} = \Cbb : \dot{\tens{\eps}}$ \\
          % Elasto-viscoplasticity: $\dot{\tens{\sigma}} = \Cbb : (\dot{\tens{\eps}}-\dot{\tens{\eps}^p})$\\
          % Elastoplasticity: $\dot{\tens{\sigma}} = \Hbb(\tens{\sigma}) : \dot{\tens{\eps}}$
        \end{block}    
      \end{block}
    \end{column}
  \end{columns}
  
  
\end{frame}

\subsection{Discrete system}
\subsection{Non-homogeneous systems}
\subsection{Interface fluxes}
%%% Local Variables:
%%% mode: latex
%%% TeX-master: "../presentation"
%%% End:
