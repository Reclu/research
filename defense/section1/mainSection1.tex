\begin{frame}{From physics to the mathematical model}
  \vspace{-0.6cm}
  \begin{overprint}
    \onslide<1>
    \begin{columns}
      \begin{column}{0.45\textwidth}
        \begin{block}{Solid dynamics problems}
          \begin{itemize}
          \item[] \textbf{Impact; Crash-proof design}
          \item[] High-speed forming
          \item[] Earthquake reliability of structures 
          \end{itemize}
        \end{block}
      \end{column}
      
      \begin{column}{0.55\textwidth}
      \end{column}
    \end{columns}
    
    \begin{figure}[ht]
      \centering
      \subcaptionbox{Bird strike on aircrafts}{\includegraphics[height=0.3\paperheight]{section1/pictures/birdstrike.jpg}}
      \subcaptionbox{Glasgow Museum of Transport}{\includegraphics[height=0.3\paperheight]{section1/pictures/crash2.jpg}}
    \end{figure}
    
    \onslide<2>
    \begin{columns}
      \begin{column}{0.45\textwidth}
        \begin{block}{Solid dynamics problems}
          \begin{itemize}
          \item[] Impact; Crash-proof design
          \item[] \textbf{High-speed forming}
          \item[] Earthquake reliability of structures 
          \end{itemize}
        \end{block}
      \end{column}
      
      \begin{column}{0.55\textwidth}
      \end{column}
    \end{columns}
    \centering
      \movie[height = 0.35\paperheight,width=0.25\linewidth,loop,poster,autostart]{}{%
      section1/animation/output3.mp4}\\
    \scriptsize Electromagnetic forming \cite{Guillaume}
    \footnoteCite{Guillaume}
    
    % \vfill
    % {\tiny
    %   \usebibitemtemplate{\color{structure}\insertbiblabel} 
    %   \usebibliographyblocktemplate{\color{structure}}{\color{black}}{\color{structure!75}}{\color{structure!75}} 
    %   \begin{thebibliography}{EMF}
    %     \tiny \bibitem[1]{Formage}
    %     Bon E.,Priem D.,Sow C., Heuzé T., Racineux G.
    %     \newblock Electromagnetic bending of an aluminum sheet
    %     \newblock {\em GeM, Ecole centrale de Nantes, 2015}.
    %   \end{thebibliography}}
    

    \onslide<3>
    \begin{columns}
      \begin{column}{0.45\textwidth}
        \begin{block}{Solid dynamics problems}
          \begin{itemize}
          \item[] Impact; Crash-proof design
          \item[] High-speed forming
          \item[] \textbf{Earthquake reliability of structures}
          \end{itemize}
        \end{block}
      \end{column}
      
      \begin{column}{0.55\textwidth}
      \end{column}
    \end{columns}

    \centering
    \includegraphics[scale=0.15]{section1/pictures/TaipeiTower.png} \quad
    \includegraphics[scale=0.08]{section1/pictures/MassDamper.jpg}\\
    \scriptsize Taipei 101 mass damper

    \onslide<4>
    \begin{columns}
      \begin{column}{0.45\textwidth}
        \begin{block}{Solid dynamics problems}
          \begin{itemize}
          \item[] Impact; Crash-proof design
          \item[] High-speed forming
          \item[] Earthquake reliability of structures 
          \end{itemize}
        \end{block}
      \end{column}
      
      \begin{column}{0.55\textwidth}
        \begin{block}{Partial differential equations}
          \begin{equation*}
            \Rightarrow \left\lvert
              \begin{aligned}
                & \text{Conservation laws} \\
                & \text{Constitutive equations} 
              \end{aligned}
            \right. = \textbf{Hyperbolic system}
            % Préciser les équations dans le dévelopement de la DGMPM
          \end{equation*}
        \end{block}
      \end{column}
    \end{columns}
    
    \begin{block}{Difficulties for the solution of hyperbolic equations:}
      \begin{itemize}
      \item complex geometries
      \item waves propagating/interacting in solids \cite{Wang}
      \item finite deformations
      \end{itemize}
    \end{block}
    \textbf{$\Rightarrow$ Resort to numerical simulation:} space and time discretization techniques
    \footnoteCite{Wang}
  \end{overprint}
  
\end{frame}

% Problems considered -- dynamics + applications + solid mechanics (equations)
% Soit (i) on inclue les équations du modèle dans les motivations soit (ii) on parle d'abord des difficultés qu'après on illustre ?

% (i) - Problèmes de dynamique des solides: lois de conservations + lois constitutives => système hyperbolique
% - EDP dont les solutions font intervenir des ondes
% - Equations complexes (non-linéaires + multi-dimensionnelle) + solutions compliquées car ondes qui interagissent
% - Intérêt de la simulation pour calculer des solutions approchées
% - Mais cependant, on a des limitations (grandes defs ; suivi des ondes [oscillations + diffusion] ; difficultées à assurer la convergence vers une solution physique ??? [un truc pour introduire la prise en compte de la structure caractéristique])
% - Exemples de limitations avec les méthodes existantes : FEM -- FV -- Meshfree 

% \begin{frame}{The numerical simulation}
%   \begin{block}{It allows to:}
%     \begin{itemize}
%     \item compute approximate solutions
%     \item highlight phenomena implicitly described by the model
%     \item perform virtual experiments
%     \end{itemize}
%   \end{block}
%   \begin{block}{While struggling with:}
%     \begin{itemize}
%     \item the space discretization (mesh-based or mesh-free techniques)
%     \item the regularity of the problem (discontinuous solutions)
%     \item 
%     \end{itemize}
%   \end{block}
% \end{frame}

\begin{frame}{Suitability of some Lagrangian explicit methods}
  \begin{columns}
    \begin{column}{0.3\textwidth}
      \begin{block}{\footnotesize Finite Element Method \cite{Belytschko}}
        \begin{tikzpicture}[scale=0.5]
          \tkzKiviatDiagram[lattice=4,
          label style/.append style={font=\scriptsize},radial  style/.style ={->},lattice style/.style ={white,opacity=0}]{CFL,Non-diffusive,Entanglement,Non-oscillating,High-order}
          \tkzKiviatLine[thick,color = black!50](4,4,4,4,4)
          
          \tkzKiviatLine[thick,color = Blue,fill= Blue,opacity=.7](4,4,1,1,3)
        \end{tikzpicture}
      \end{block}
    \end{column}
    \begin{column}{0.3\textwidth}
      \begin{block}{\footnotesize Finite Volume Method \cite{Leveque}}
        \begin{tikzpicture}[scale=0.5]
          \tkzKiviatDiagram[lattice=4,
          label style/.append style={font=\scriptsize},radial  style/.style ={->},lattice style/.style ={white,opacity=0}]{CFL,Non-diffusive,Entanglement,Non-oscillating,High-order}
          \tkzKiviatLine[thick,color = black!50](4,4,4,4,4)
          
          \tkzKiviatLine[thick,color = Red,fill= Red,opacity=.7](4,4,1,4,1)
        \end{tikzpicture}
      \end{block}
    \end{column}
    \begin{column}{0.33\textwidth}
      \begin{block}{\footnotesize Discontinuous Galerkin FEM \cite{Cockburn}}
        \begin{tikzpicture}[scale=0.5]
          \tkzKiviatDiagram[lattice=4,
          label style/.append style={font=\scriptsize},radial  style/.style ={->},lattice style/.style ={white,opacity=0}]{CFL,Non-diffusive,Entanglement,Non-oscillating,High-order}
          \tkzKiviatLine[thick,color = black!50](4,4,4,4,4)
          
          \tkzKiviatLine[thick,color = Blue!70](4,4,1,1,3)
          \tkzKiviatLine[thick,color = Red!70](4,4,1,4,1)
          \tkzKiviatLine[thick,color = Purple,fill= Purple,opacity=.7](2,4,1,4,4)
        \end{tikzpicture}
      \end{block}
    \end{column}
  \end{columns}
  \footnoteCite{Belytschko,Leveque,Cockburn}
\end{frame}

\begin{frame}{Mesh-free Lagrangian approaches: The Material Point Method \cite{Sulsky94}}
  \nocite{Sulsky94}
  % \begin{block}{Collection of particles in an arbitrary grid}
  %   \centering
  %   \begin{tikzpicture}[scale=0.8]
  % \draw[step=1.0,black,thin] (-3.,-1.) grid (3,4.);
  \draw[white] (-3,-1) -- (3,-1) -- (3,4) -- (-3,4) -- (-3,-1);
  \begin{scope}[scale=0.5]
    \draw[thick] (-3,0.6) .. controls +(1,0) and +(-1,0) .. (0,1.8)  
    .. controls +(1,0) and +(0,-3) .. (5,3.2) 
    .. controls +(0,2) and +(2,0)  .. (0,5.2) 
    .. controls +(-1,0) and +(0,3) .. (-4.5,2.2) 
    .. controls +(0,-1) and +(-1,0).. (-3,0.6) ;
    \begin{scope}  % pour limiter la portée du clip
      \clip (-3,0.6) .. controls +(1,0) and +(-1,0) .. (0,1.8) 
      .. controls +(1,0) and +(0,-3) .. (5,3.2)
      .. controls +(0,2) and +(2,0)  .. (0,5.2)
      .. controls +(-1,0) and +(0,3) .. (-4.5,2.2)
      .. controls +(0,-1) and +(-1,0).. (-3,0.6);
    \end{scope}
    \node[below] at (0,1) {$\Omega_t$};
  \end{scope}
  \draw [->,very thick,gray] (3.,1.5) -- (3.6,1.5);
\end{tikzpicture}
  %   %Projection of fields Particles $\Leftrightarrow$ Nodes
  % \end{block}
  \begin{columns}
    \begin{column}{0.4\textwidth}
      \begin{block}{\footnotesize FLuid Implicit Particle mapping \cite{PIC_Nishiguchi}}
        \begin{tikzpicture}[scale=0.5]
          \tkzKiviatDiagram[lattice=4,
          label style/.append style={font=\scriptsize},radial  style/.style ={->},lattice style/.style ={white,opacity=0}]{CFL,Non-diffusive,Entanglement,Non-oscillating,High-order}
          \tkzKiviatLine[thick,color = black!50](4,4,4,4,4)
          
          \tkzKiviatLine[thick,color = Yellow,fill= Yellow,opacity=.7](3,3,4,1,3)
        \end{tikzpicture}
      \end{block}
    \end{column}
    \begin{column}{0.4\textwidth}
      \begin{block}{\footnotesize Particle-in-cell mapping \cite{PIC}}
       \begin{tikzpicture}[scale=0.5]
          \tkzKiviatDiagram[lattice=4,
          label style/.append style={font=\scriptsize},radial  style/.style ={->},lattice style/.style ={white,opacity=0}]{CFL,Non-diffusive,Entanglement,Non-oscillating,High-order}
          \tkzKiviatLine[thick,color = black!50](4,4,4,4,4)
          
          \tkzKiviatLine[thick,color = Yellow,fill= Yellow,opacity=.7](3,1,4,4,3)
        \end{tikzpicture}
      \end{block}
    \end{column}
  \end{columns}
  \footnoteCite{Sulsky94,PIC_Nishiguchi,PIC}
\end{frame}

\begin{frame}
  \metroset{block=fill}
  \begin{block}{Objective 1}
    Merge the advantages of FEM, FVM and MPM by means of the DG approximation
  \end{block}
  \metroset{block=transparent}
  \begin{block}{The Discontinuous Galerkin Material Point Method}
    \begin{columns}
      \begin{column}{0.6\textwidth}
        \begin{tikzpicture}[scale=0.5]
          \tkzKiviatDiagram[lattice=4,
          label style/.append style={font=\scriptsize},radial  style/.style ={->},lattice style/.style ={white,opacity=0}]{CFL,Non-diffusive,Entanglement,Non-oscillating,High-order}
          \tkzKiviatLine[thick,color = black!50](4,4,4,4,4)
          
          \tkzKiviatLine[thick,color = Yellow!70](3,1,4,4,3)
          \tkzKiviatLine[thick,color = Purple!70](2,4,1,4,4)
          \tkzKiviatLine[thick,color = Orange,fill= Orange,opacity=.7](3,3,4,4,3)
        \end{tikzpicture}
      \end{column}
      \begin{column}{0.4\textwidth}
        \textbf{Ingredients:}
        \begin{itemize}
        \item MPM space discretization
        \item PIC projection of fields
        \item DG approximation
        \end{itemize}
      \end{column}
    \end{columns}
  \end{block}
        
      
\end{frame}

\begin{frame}{The simulation is bounded by the model}
  %%
  \begin{block}{Inheritance from fluid dynamics}
    Numerical tools to embed information about the solution in numerical approaches
  \end{block}
  \begin{block}{Point of view adopted}
    \begin{itemize}
    \item Such numerical tools + robust discretization techniques $\rightarrow$ accurate solutions
    \item Make a numerical approach able to mimic the physical response
    \end{itemize}
  \end{block}
  \begin{block}{Limitations}
    Gaps about some constitutive models (damage, plasticity, thermo-mechanical coupling etc.)
  \end{block}
  \metroset{block=fill}
  \begin{block}{Objective 2}
    Investigate the response of two-dimensional elastic-plastic solids to dynamic loading
  \end{block}
\end{frame}



%%% Local Variables:
%%% mode: latex
%%% TeX-master: "../presentation"
%%% End:
