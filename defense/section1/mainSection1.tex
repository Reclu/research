\begin{withoutheadline}
  \begin{frame}{From physics to the mathematical model}
  \vspace{-0.6cm}
  \begin{overprint}
    \onslide<1>
    
    \begin{columns}
      \begin{column}{0.45\textwidth}
        \begin{block}{Solid dynamics problems}
          \begin{itemize}
          \item[] Impact; Crash-proof design
          \item[] \textbf{High-speed forming}
          \end{itemize}
        \end{block}
      \end{column}
      
      \begin{column}{0.55\textwidth}
      \end{column}
    \end{columns}

    
    \begin{columns}
      \begin{column}{0.45\textwidth}
        \centering
        \movie[height = 0.3275\paperheight,width=0.535\linewidth,loop,poster,autostart]{}{%
      section1/animation/output3.mp4}\\
    \scriptsize Electromagnetic forming \cite{Guillaume}
      \end{column}
      \begin{column}{0.45\textwidth}
        \begin{block}{}
          \begin{footnotesize}
            \begin{itemize}
            \item High-velocity ($\approx \: 200\:m/s$) $\rightarrow$ waves
            \item Large deformations
            \item Irreversible strains
            \item Complex geometries
            \item[]
            \item[]
            \end{itemize}
          \end{footnotesize}
        \end{block}  
      \end{column}
    \end{columns}
    \footnoteCite{Guillaume}
    
    \onslide<2>
    \begin{columns}
      \begin{column}{0.45\textwidth}
        \begin{block}{Solid dynamics problems}
          \begin{itemize}
          \item[] Impact; Crash-proof design
          \item[] High-speed forming
          \end{itemize}
        \end{block}
      \end{column}
      
      
      \begin{column}{0.55\textwidth}
        \begin{block}{Mathematical model}
          \begin{itemize}
          \item[] Partial differential equations
          \item[] Hyperbolic system
          \end{itemize}
        \end{block}
      \end{column}
    \end{columns}

    \vspace{0.2cm}
    \begin{block}{Challenging task}
      \alert{Accurately track waves propagating/interacting in irreversibly finite deforming solids \cite{Wang}}
    \end{block}
    %% Il faut dire que suivre ces ondes est très important pour comprendre la physique et évaluer correctement les états résiduels

    \vspace{0.2cm}
    \begin{block}{Complex equations}
      \textbf{$\Rightarrow$ Resort to numerical simulation}
    \end{block}
    \footnoteCite{Wang}
  \end{overprint}
\end{frame}\end{withoutheadline}


\begin{withoutheadline}
  \begin{frame}{Some explicit numerical methods}
    \vspace{-0.5cm}
    \begin{overprint}
      \onslide<1>
      \begin{columns}
        \begin{column}{0.49\textwidth}
          \begin{block}{The Finite Element Method \cite{Belytschko}}
            \begin{footnotesize}
              \begin{block}{\footnotesize Pros}
                \vspace{-0.2cm}
                \begin{itemize}
                \item[] Constitutive equations
                \item[] High-order approximation
                \end{itemize}
              \end{block}
              \vspace{-0.2cm}
              \begin{block}{\footnotesize Cons}
                \vspace{-0.2cm}
                \begin{itemize}
                \item[] Oscillations 
                \item[] Lagrangian: mesh entanglement
                \item[] Eulerian: diffusion
                \end{itemize}
              \end{block}
            \end{footnotesize}
          \end{block}
        \end{column}
        \begin{column}{0.49\textwidth}
        \end{column}
      \end{columns}
      \vspace{-0.25cm}
      \footnoteCite{Belytschko}
      \onslide<2>
      \begin{columns}
        \begin{column}{0.49\textwidth}
          \vspace{-0.1cm}
          \begin{block}{The Finite Element Method \cite{Belytschko}}
            \begin{footnotesize}
              \begin{block}{\footnotesize Pros}
                \vspace{-0.2cm}
                \begin{itemize}
                \item[] Constitutive equations
                \item[] High-order approximation 
                \end{itemize}
              \end{block}
              \vspace{-0.2cm}
              \begin{block}{\footnotesize Cons}
                \vspace{-0.2cm}
                \begin{itemize}
                \item[] Oscillations
                \item[] Lagrangian: mesh entanglement 
                \item[] Eulerian: diffusion
                \end{itemize}
              \end{block}
            \end{footnotesize}
          \end{block}
        \end{column}
        \begin{column}{0.49\textwidth}
          \begin{block}{The Finite Volume Method \cite{Leveque}}
            \begin{footnotesize}
              \begin{block}{\footnotesize Pros}
                \vspace{-0.2cm}
                \begin{itemize}
                \item[] Conservation laws % donc même approx de v et F
                \item[] Use of numerical fluxes \cite{Godunov_method}
                \end{itemize}
              \end{block}
              \vspace{-0.2cm}
              \begin{block}{\footnotesize Cons}
                \vspace{-0.2cm}
                \begin{itemize}
                \item[] Essentially low-order methods
                \item[] Lagrangian \cite{Haider_FVM}: update geometries  
                \item[] Eulerian \cite{Gavrilyuk}: constitutive equations ?
                \end{itemize}
              \end{block}
            \end{footnotesize}
          \end{block}
        \end{column}
      \end{columns}
      \vspace{-0.3cm} 
     \footnoteCite{Belytschko,Leveque,Godunov_method,Haider_FVM,Gavrilyuk}
    \end{overprint}
  \end{frame}
\end{withoutheadline}

\begin{withoutheadline}
  \begin{frame}{Some explicit numerical methods}
    \begin{block}{The space-Discontinuous Galerkin Finite Element Method \cite{Cockburn}}
      \begin{footnotesize}
        \begin{block}{\footnotesize Merge FVM and FEM}
          \vspace{-.2cm}
          \begin{itemize}
          \item[] Local high-order approximation \cite{NeutronDG}
          \item[] Same approximation of field and gradient
          \item[] Numerical fluxes: no oscillations
          \end{itemize}
        \end{block}
        \vspace{-.2cm}
        \begin{block}{\footnotesize Limitations}
          \vspace{-.2cm}
          \begin{itemize}
          \item[] Restrictive time step
          \item[] \alert{Lagrangian: update of the geometry \cite{LagrangianDG_thesis}}
          \item[] Eulerian: diffusive advection terms -- constitutive equaions 
          \end{itemize}
        \end{block}
      \end{footnotesize}
    \end{block}
    \footnoteCite{Cockburn,NeutronDG,LagrangianDG_thesis}
  \end{frame}
\end{withoutheadline}


\begin{withoutheadline}
  \begin{frame}{Mesh-free Lagrangian approaches}
    
    \begin{block}{The Material Point Method \cite{Sulsky94}}
      \begin{footnotesize}
        \begin{itemize}
        \item Particle-based space discretization
        \item Arbitrary computational grid
        \end{itemize}
      \end{footnotesize}
      
      \centering
      \textbf{Projections of fields Particles $\leftrightarrow$ Nodes}
    \end{block}

    \begin{columns}
      \begin{column}{0.48\textwidth}
        \begin{block}{\footnotesize Particle-In-Cell mapping \cite{PIC}}
          \begin{footnotesize}
            \begin{itemize}
            \item[] No oscillations 
            \item[] Diffusion
            \end{itemize}
          \end{footnotesize}
        \end{block}
      \end{column}
      \begin{column}{0.48\textwidth}
        \begin{block}{\footnotesize FLuid Implicit Particle mapping \cite{PIC_Nishiguchi}}
          \begin{footnotesize}
            \begin{itemize}
            \item[] Reduced diffusion 
            \item[] Oscillations
            \end{itemize}
          \end{footnotesize}
        \end{block}
      \end{column}
    \end{columns}
    \vspace{-0.3cm}
    \footnoteCite{Sulsky94,PIC,PIC_Nishiguchi}
  \end{frame}
\end{withoutheadline}

\begin{withoutheadline}
  \begin{frame}{\text{  }}
    \begin{block}{Challenging task for numerical methods}
      \alert{Accurately track waves in irreversibly finite deforming solids}
      \begin{itemize}
      \item[1-] Handle large deformations -- No oscillation -- Low diffusion
      \item[2-] Embed sufficient amount of information about the solution (intercell fluxes)
      \end{itemize}
    \end{block}\pause 
    
    \metroset{block=fill}
    \begin{block}{Objective 1}
      Develop a numerical method that accurately capture waves in finite deforming solids \\
      \textbf{Merge FEM--FVM--MPM: DG approximation}
    \end{block}\pause
    \begin{block}{Objective 2}
      Identify the response of two-dimensional elastic-plastic solids to dynamic loading \\
      \textbf{Analysis of governing equations} % under small strains% Not only numerical
    \end{block}
  \end{frame}
\end{withoutheadline}


%%% Afin de calculer précisément les états résiduels et aussi de mieux comprendre les phénomènes physiques, on voudrait, à travers la simulation, suivre précisément (sans oscilation et diffusion) les ondes dans des solides subissant de grandes déformations.
%% Clairement afficher un objectif principal lié aux difficultés mentionées:
%% Difficultés:
%%% (Proposer)/(Améliorer la précision) des solutions approximées de problèmes de dynamique dans les solides élastoplastiques subissant de grandes transformations
%%% Objectif 1: rassembler les avantage des FEM, FVM et MPM grâce au DG
%%% Objectif 2: approfondir la connaissance que l'on a de la réponse physique etc. puisque ça permet d'améliorer la simuation


%%% Local Variables:
%%% mode: latex
%%% TeX-master: "../presentation"
%%% End:
