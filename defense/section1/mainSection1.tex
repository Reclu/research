\begin{frame}{From physics to the mathematical model}
  % Concerned with solid dynamics problems such as Impact or crash-proof design.
  % in the first case, one wants to ensure that a structure undegoing dynamic loadings remains usable while in the latter, it is expected to dissipate as much energy as possible so that the passengers are safe
  
  % On the other hand, high-speed forming techniques such as electromagnetic forming also involve dynamic loadings

  % At last, another challenging field of dynamics is the reliability of structures to earthquakes
  % Here is depicted the mass damper of the Taipei 101 tower which design was possible thanks to a good understanding of the physics.
  \vspace{-0.6cm}
  \begin{overprint}
    \onslide<1>
    \begin{columns}
      \begin{column}{0.45\textwidth}
        \begin{block}{Solid dynamics problems}
          \begin{itemize}
          \item[] \textbf{Impact; Crash-proof design}
          \item[] High-speed forming
          \item[] Earthquake reliability of structures 
          \end{itemize}
        \end{block}
      \end{column}
      
      \begin{column}{0.55\textwidth}
      \end{column}
    \end{columns}
    
    \begin{figure}[ht]
      \centering
      \subcaptionbox{Bird strike on aircrafts}{\includegraphics[height=0.3\paperheight]{section1/pictures/birdstrike.jpg}}
      \subcaptionbox{Glasgow Museum of Transport}{\includegraphics[height=0.3\paperheight]{section1/pictures/crash2.jpg}}
    \end{figure}
    
    \onslide<2>
    \begin{columns}
      \begin{column}{0.45\textwidth}
        \begin{block}{Solid dynamics problems}
          \begin{itemize}
          \item[] Impact; Crash-proof design
          \item[] \textbf{High-speed forming}
          \item[] Earthquake reliability of structures 
          \end{itemize}
        \end{block}
      \end{column}
      
      \begin{column}{0.55\textwidth}
      \end{column}
    \end{columns}
    \centering
      \movie[height = 0.35\paperheight,width=0.25\linewidth,loop,poster,autostart]{}{%
      section1/animation/output3.mp4}\\
    \scriptsize Electromagnetic forming \cite{Guillaume}
    \footnoteCite{Guillaume}
    
    \onslide<3>
    \begin{columns}
      \begin{column}{0.45\textwidth}
        \begin{block}{Solid dynamics problems}
          \begin{itemize}
          \item[] Impact; Crash-proof design
          \item[] High-speed forming
          \item[] \textbf{Earthquake reliability of structures}
          \end{itemize}
        \end{block}
      \end{column}
      
      \begin{column}{0.55\textwidth}
      \end{column}
    \end{columns}

    \centering
    \includegraphics[scale=0.15]{section1/pictures/TaipeiTower.png} \quad
    \includegraphics[scale=0.08]{section1/pictures/MassDamper.jpg}\\
    \scriptsize Taipei 101 mass damper

    \onslide<4>
    \begin{columns}
      \begin{column}{0.45\textwidth}
        \begin{block}{Solid dynamics problems}
          \begin{itemize}
          \item[] Impact; Crash-proof design
          \item[] High-speed forming
          \item[] Earthquake reliability of structures 
          \end{itemize}
        \end{block}
      \end{column}
      
      %% The governing equations allowing to mathematically model those physical phenomena are composed of Conservation laws and constitutive equations, which can be written as a hyperbolic system
      %% Even though models can be built, the complex geometries, waves propagating in solids and the possibly finite deformations make in general the exact solution not possible.
      %% As a result, the numerical simulation based on space and time approximations becomes helpfull
      \begin{column}{0.55\textwidth}
        \begin{block}{Partial differential equations}
          \begin{equation*}
            \Rightarrow \left\lvert
              \begin{aligned}
                & \text{Conservation laws} \\
                & \text{Constitutive equations} 
              \end{aligned}
            \right. = \textbf{Hyperbolic system}
            % Préciser les équations dans le dévelopement de la DGMPM
          \end{equation*}
        \end{block}
      \end{column}
    \end{columns}
    
    \begin{block}{Difficulties for the solution of hyperbolic equations:}
      \begin{itemize}
      \item complex geometries
      \item waves propagating/interacting in solids \cite{Wang}
      \item finite deformations
      \end{itemize}
    \end{block}
    \textbf{$\Rightarrow$ Resort to numerical simulation:} space and time discretization techniques
    \footnoteCite{Wang}
  \end{overprint}
  
\end{frame}

\begin{frame}{Suitability of some explicit methods}
  %% Among the big stars of existing numerical methods, the FEM
  \begin{block}{The Finite Element Method \cite{Belytschko}}
    \vskip 4pt
    \begin{overprint}
      \onslide<1>
      \begin{columns}
        \begin{footnotesize}
          \begin{column}{0.5\textwidth}
            \begin{itemize}
            \item[] Mesh-based space discretization
            \item[] Weak form of balance equation
            \end{itemize}
          \end{column}
          \begin{column}{0.5\textwidth} 
            \begin{itemize}
            \item[] Polynomial approximation
            \item[] Gauss points constitutive update
            \end{itemize}
          \end{column}
        \end{footnotesize}
      \end{columns}
      \onslide<2>
      \begin{columns}
        \begin{footnotesize}
          \begin{column}{0.5\textwidth}
            \begin{itemize}
            \item[] Mesh-based space discretization
            \item[] Weak form of balance equation
            \end{itemize}
          \end{column}
          \begin{column}{0.5\textwidth} 
            \begin{itemize}
            \item[] Polynomial approximation
            \item[] Gauss points constitutive update
            \end{itemize}
          \end{column}
        \end{footnotesize}
      \end{columns}
      \vskip -10pt
      \begin{columns}
        \begin{column}{0.48\textwidth}
          \begin{block}{\footnotesize Lagrangian formulation}
            \centering
            \begin{tikzpicture}[scale=0.4]
              \tkzKiviatDiagram[lattice=4,
              label style/.append style={font=\tiny},radial  style/.style ={->},lattice style/.style ={white,opacity=0}]{CFL,Non-diffusive,Mesh robustness,Non-oscillating,High-order}
              \tkzKiviatLine[thick,color = black!50](4,4,4,4,4)
              
              \tkzKiviatLine[thick,color = Blue,fill= Blue,opacity=.7](4,4,1,1,3)
            \end{tikzpicture}
          \end{block}
        \end{column}
        \begin{column}{0.48\textwidth}
          % \begin{block}{\footnotesize Eulerian formulation}
          %   \centering
          %   \begin{tikzpicture}[scale=0.4]
          %     \tkzKiviatDiagram[lattice=4,
          %     label style/.append style={font=\tiny},radial  style/.style ={->},lattice style/.style ={white,opacity=0}]{CFL,Non-diffusive,Distortion-free,Non-oscillating,High-order}
          %     \tkzKiviatLine[thick,color = black!50](4,4,4,4,4)
              
          %     \tkzKiviatLine[thick,color = Blue,fill= Blue,opacity=.7](4,2,4,1,3)
          %   \end{tikzpicture}
          % \end{block}
        \end{column}
      \end{columns}
    \end{overprint}
  \end{block}
\footnoteCite{Belytschko}
\end{frame}

%% Pas forcément mettre en avant le tangling
%% Approches Lagrangiennes FVM qui ont des di
%% Citer Maire le papier de 2006 sur la définition de la vitesse nodale (similaire au DG puisque v est discontinue) -> pas tant mesh entanglement que mesh update issues (le second étant valable même en total lagrangian)
%% Virer l'Eulérien; mettre mesh drawbacks; citer le cas échéant; et préciser à l'oral les mesh drawbacks: pour FVM, il s'agit de bouger le maillage avec un champ de vitesse discontinu
\begin{frame}{Suitability of some explicit methods}
  \begin{block}{The Finite Volume Method \cite{Leveque}}%\cite{Haider_FVM} for total Lagrangian
    \vspace{-0.2cm}
    \begin{overprint}
      \onslide<1>
      \vspace{-0.2cm}
      \begin{columns}
        \begin{footnotesize}
          \begin{column}{0.4\textwidth}
            \begin{itemize}
            \item[] Mesh-based space discretization
            \item[] Conservation laws
            \end{itemize}
        \end{column}
        \begin{column}{0.6\textwidth}
            \begin{itemize}
            \item[] Cell-wise approximation and constitutive update
            \item[] Intercell fluxes -- characteristic structure \cite{Godunov_method}
            \end{itemize}
          \end{column}
        \end{footnotesize}
      \end{columns}
      \vspace{3.65cm}
      \footnoteCite{Leveque,Godunov_method}
      \onslide<2>
      \vspace{-0.2cm}
      \begin{columns}
        \begin{footnotesize}
          \begin{column}{0.4\textwidth}
            \begin{itemize}
            \item[] Mesh-based space discretization
            \item[] Conservation laws
            \end{itemize}
          \end{column}
          \begin{column}{0.6\textwidth}
            \begin{itemize}
            \item[] Cell-wise approximation and constitutive update
            \item[] Intercell fluxes -- characteristic structure \cite{Godunov_method}
            \end{itemize}
          \end{column}
        \end{footnotesize}
      \end{columns}
      \vskip -10pt
      \begin{columns}
        \begin{column}{0.48\textwidth}
          \begin{block}{\footnotesize Lagrangian formulation \cite{Haider_FVM}} %[Haider]?}
            \centering
            \begin{tikzpicture}[scale=0.4]
              \tkzKiviatDiagram[lattice=4,
              label style/.append style={font=\tiny},radial  style/.style ={->},lattice style/.style ={white,opacity=0}]{CFL,Non-diffusive, Mesh robustness,Non-oscillating,High-order}
              \tkzKiviatLine[thick,color = black!50](4,4,4,4,4)
              
              \tkzKiviatLine[thick,color = Red,fill= Red,opacity=.7](4,4,1,4,1)
            \end{tikzpicture}
          \end{block}
        \end{column}
        \begin{column}{0.48\textwidth}
          % \begin{block}{\footnotesize Eulerian formulation}
          %   \centering
          %   \begin{tikzpicture}[scale=0.4]
          %     \tkzKiviatDiagram[lattice=4,
          %     label style/.append style={font=\tiny},radial  style/.style ={->},lattice style/.style ={white,opacity=0}]{CFL,Non-diffusive,Distortion-free,Non-oscillating,High-order}
          %     \tkzKiviatLine[thick,color = black!50](4,4,4,4,4)
              
          %     \tkzKiviatLine[thick,color = Red,fill= Red,opacity=.7](4,2,4,4,1)
          %   \end{tikzpicture}
          % \end{block}
        \end{column}
      \end{columns}
      \vspace{-0.2cm}
      \footnoteCite{Leveque,Godunov_method,Haider_FVM}
    \end{overprint}
  \end{block}
  %compatibility between the two configurations based on Eulerian and Lagrangian coordinates
  
  
\end{frame}


\begin{frame}{Suitability of some explicit methods}
  \begin{block}{The Discontinuous Galerkin Finite Element Method \cite{Cockburn}}
    \vspace{-0.2cm}
    \begin{overprint}
      \onslide<1>
      \vspace{-0.2cm}
      \begin{columns}
        \begin{footnotesize}
          \begin{column}{0.4\textwidth}
            \begin{itemize}
            \item[] Mesh-based space discretization
            \item[] Cell-wise weak form \cite{NeutronDG}
            \end{itemize}
          \end{column}
          \begin{column}{0.6\textwidth}
            \begin{itemize}
            \item[] Gauss points constitutive update
            \item[] Intercell fluxes -- characteristic structure
            \end{itemize}
          \end{column}
        \end{footnotesize}
      \end{columns}
      \vspace{3.65cm}
      \footnoteCite{Cockburn,NeutronDG}
      \onslide<2>
      \vspace{-0.2cm}
      \begin{columns}
        \begin{footnotesize}
          \begin{column}{0.4\textwidth}
            \begin{itemize}
            \item[] Mesh-based space discretization
            \item[] Cell-wise weak form \cite{NeutronDG}
            \end{itemize}
          \end{column}
          \begin{column}{0.6\textwidth}
            \begin{itemize}
            \item[] Gauss points constitutive update
            \item[] Intercell fluxes -- characteristic structure
            \end{itemize}
          \end{column}
        \end{footnotesize}
      \end{columns}
      \vskip -10pt
      \begin{columns}
        \begin{column}{0.48\textwidth}
          \begin{block}{\footnotesize Lagrangian formulation \cite{LagrangianDG_thesis}}
            \centering
            \begin{tikzpicture}[scale=0.4]
              \tkzKiviatDiagram[lattice=4,
              label style/.append style={font=\tiny},radial  style/.style ={->},lattice style/.style ={white,opacity=0}]{CFL,Non-diffusive,Mesh robustness,Non-oscillating,High-order}
              \tkzKiviatLine[thick,color = black!50](4,4,4,4,4)
              
              \tkzKiviatLine[thick,color = Purple,fill= Purple,opacity=.7](1,4,1,4,4)
            \end{tikzpicture}
          \end{block}
        \end{column}
        \begin{column}{0.48\textwidth}
          % \begin{block}{\footnotesize Eulerian formulation (check diffusion)}
          %   \centering
          %   \begin{tikzpicture}[scale=0.4]
          %     \tkzKiviatDiagram[lattice=4,
          %     label style/.append style={font=\tiny},radial  style/.style ={->},lattice style/.style ={white,opacity=0}]{CFL,Non-diffusive,Mesh robustness,Non-oscillating,High-order}
          %     \tkzKiviatLine[thick,color = black!50](4,4,4,4,4)
              
          %     \tkzKiviatLine[thick,color = Purple,fill= Purple,opacity=.7](1,2,4,4,4)
          %   \end{tikzpicture}
          % \end{block}
        \end{column}
      \end{columns}
      \vspace{-0.2cm}
      \footnoteCite{Cockburn,NeutronDG,LagrangianDG_thesis}
    \end{overprint}
  \end{block}
\end{frame}



%% OLD VERSION == ONE SLIDE
% \begin{frame}{Suitability of some Lagrangian explicit methods}
%   \begin{columns}
%     \begin{column}{0.3\textwidth}
%       \begin{block}{\footnotesize Finite Element Method \cite{Belytschko}}
%         \begin{tikzpicture}[scale=0.5]
%           \tkzKiviatDiagram[lattice=4,
%           label style/.append style={font=\tiny},radial  style/.style ={->},lattice style/.style ={white,opacity=0}]{CFL,Non-diffusive,Mesh robustness,Non-oscillating,High-order}
%           \tkzKiviatLine[thick,color = black!50](4,4,4,4,4)
          
%           \tkzKiviatLine[thick,color = Blue,fill= Blue,opacity=.7](4,4,1,1,3)
%         \end{tikzpicture}
%       \end{block}
%     \end{column}
%     \begin{column}{0.3\textwidth}
%       \begin{block}{\footnotesize Finite Volume Method \cite{Leveque}}
%         \begin{tikzpicture}[scale=0.5]
%           \tkzKiviatDiagram[lattice=4,
%           label style/.append style={font=\tiny},radial  style/.style ={->},lattice style/.style ={white,opacity=0}]{CFL,Non-diffusive,Mesh robustness,Non-oscillating,High-order}
%           \tkzKiviatLine[thick,color = black!50](4,4,4,4,4)
          
%           \tkzKiviatLine[thick,color = Red,fill= Red,opacity=.7](4,4,1,4,1)
%         \end{tikzpicture}
%       \end{block}
%     \end{column}
%     \begin{column}{0.33\textwidth}
%       \begin{block}{\footnotesize Discontinuous Galerkin FEM \cite{Cockburn}}
%         \begin{tikzpicture}[scale=0.5]
%           \tkzKiviatDiagram[lattice=4,
%           label style/.append style={font=\tiny},radial  style/.style ={->},lattice style/.style ={white,opacity=0}]{CFL,Non-diffusive,Mesh robustness,Non-oscillating,High-order}
%           \tkzKiviatLine[thick,color = black!50](4,4,4,4,4)
          
%           \tkzKiviatLine[thick,color = Blue!70](4,4,1,1,3)
%           \tkzKiviatLine[thick,color = Red!70](4,4,1,4,1)
%           \tkzKiviatLine[thick,color = Purple,fill= Purple,opacity=.7](2,4,1,4,4)
%         \end{tikzpicture}
%       \end{block}
%     \end{column}
%   \end{columns}
%   \footnoteCite{Belytschko,Leveque,Cockburn}
% \end{frame}

%% It then appears that some difficulties related to the mesh are encourted with all the previous methods. 
%% Mesh-free approaches allow however, to circumvent some of them.
%% In particular, the material point method
\begin{frame}{Mesh-free Lagrangian approaches: The Material Point Method \cite{Sulsky94}}
  \nocite{Sulsky94}
  \begin{columns}
    \begin{column}{0.4\textwidth}
      \begin{block}{\footnotesize Particle-in-cell mapping \cite{PIC}}
       \begin{tikzpicture}[scale=0.5]
          \tkzKiviatDiagram[lattice=4,
          label style/.append style={font=\tiny},radial  style/.style ={->},lattice style/.style ={white,opacity=0}]{CFL,Non-diffusive,Mesh robustness,Non-oscillating,High-order}
          \tkzKiviatLine[thick,color = black!50](4,4,4,4,4)
          
          \tkzKiviatLine[thick,color = Yellow,fill= Yellow,opacity=.7](3,1,4,4,3)
        \end{tikzpicture}
      \end{block}
    \end{column}
    \begin{column}{0.4\textwidth}
      \begin{block}{\footnotesize FLuid Implicit Particle mapping \cite{PIC_Nishiguchi}}
        \begin{tikzpicture}[scale=0.5]
          \tkzKiviatDiagram[lattice=4,
          label style/.append style={font=\tiny},radial  style/.style ={->},lattice style/.style ={white,opacity=0}]{CFL,Non-diffusive,Mesh robustness,Non-oscillating,High-order}
          \tkzKiviatLine[thick,color = black!50](4,4,4,4,4)
          
          \tkzKiviatLine[thick,color = Orange,fill= Orange,opacity=.7](3,3,4,1,3)
        \end{tikzpicture}
      \end{block}
    \end{column}
  \end{columns}
  \footnoteCite{Sulsky94,PIC_Nishiguchi,PIC}
\end{frame}

\begin{frame}
  \metroset{block=fill}
  \begin{block}{Objective 1}
    Capture waves with a Lagrangian description while avoiding mesh-related difficulties \\
    %% Solution proposed, though others exist
    \alert{$\Rightarrow$ Merge the advantages of FEM, FVM and MPM by means of the DG approximation}
  \end{block}
  \metroset{block=transparent}
  \begin{block}{The Discontinuous Galerkin Material Point Method}
    \begin{columns}
      \begin{column}{0.6\textwidth}
        \begin{tikzpicture}[scale=0.5]
          \tkzKiviatDiagram[lattice=4,
          label style/.append style={font=\tiny},radial  style/.style ={->},lattice style/.style ={white,opacity=0}]{CFL,Non-diffusive,Mesh robustness,Non-oscillating,High-order}
          \tkzKiviatLine[thick,color = black!50](4,4,4,4,4)
          
          \tkzKiviatLine[thick,color = Yellow!70](3,1,4,4,3)
          \tkzKiviatLine[thick,color = Orange!70](3,3,4,1,3)
          \tkzKiviatLine[thick,color = Purple!70](1,4,1,4,4)
          \tkzKiviatLine[thick,color = Green,fill= Green,opacity=.7](3,2,4,4,3)
        \end{tikzpicture}
      \end{column}
      \begin{column}{0.4\textwidth}
        \textbf{Ingredients:}
        \begin{itemize}
        \item MPM space discretization
        \item PIC projection of fields
        \item DG approximation
        \end{itemize}
      \end{column}
    \end{columns}
  \end{block}
\end{frame}

\begin{frame}{The simulation is bounded by the model}
  %%
  \begin{block}{Inheritance from fluid dynamics}
    Numerical tools to embed information about the solution in numerical approaches
  \end{block}
  \begin{block}{Point of view adopted: Such numerical tools + robust discretization techniques}
    \begin{itemize}
    \item Provide accurate solutions
    \item Enable a numerical approach to mimic the physical response
    \end{itemize}
  \end{block}
  \begin{block}{Limitations}
    Gaps about some constitutive models (damage, plasticity, thermo-mechanical coupling etc.)
  \end{block}
  \metroset{block=fill}
  \begin{block}{Objective 2}
    Identify the response of two-dimensional elastic-plastic solids to dynamic loading
  \end{block}
\end{frame}



%%% Local Variables:
%%% mode: latex
%%% TeX-master: "../presentation"
%%% End:
