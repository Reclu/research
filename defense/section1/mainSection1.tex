

\begin{frame}{From physics to the mathematical model}
  Problem $\rightarrow$ model for solid mechanics
  % Problems considered -- dynamics + applications + solid mechanics (equations)
  % Soit (i) on inclue les équations du modèle dans les motivations soit (ii) on parle d'abord des difficultés qu'après on illustre ?

  % (i) - Problèmes de dynamique des solides: lois de conservations + lois constitutives => système hyperbolique
  %     - EDP dont les solutions font intervenir des ondes
  %     - Equations complexes (non-linéaires + multi-dimensionnelle) + solutions compliquées car ondes qui interagissent
  %     - Intérêt de la simulation pour calculer des solutions approchées
  %     - Mais cependant, on a des limitations (grandes defs ; suivi des ondes [oscillations + diffusion] ; difficultées à assurer la convergence vers une solution physique ??? [un truc pour introduire la prise en compte de la structure caractéristique])
  %     - Exemples de limitations avec les méthodes existantes : FEM -- FV -- Meshfree 
      
  Model: hyperbolic problems + difficulties (finite deformations, waves that can interact $\rightarrow$ complex solution)
\end{frame}
\subsection*{Numerical simulation}
\begin{frame}
  Allows to compute approximate solutions

  Shortcomings of FEM -- FVM -- Meshfree

  Fail to mimic the physical behavior due to the lack of information about the structure (talk about dammage, softening, plasticity)
\end{frame}

\subsection*{Objectives and strategy}
\begin{frame}{Objectives}
  Numerical methods able to deal with finite deformation + waves

  Solution of elastoplasticity problems in 2D
\end{frame}

\begin{frame}{Strategy}
  PIC mapping in MPM + DG approx

  Characteristic structure + loading path etc.

  Il faut bien démarquer les deux parties dans le plan de la prez.
\end{frame}


%%% Local Variables:
%%% mode: latex
%%% TeX-master: "../aRenaud"
%%% End:
