\documentclass[11pt,aspectratio=169]{beamer}
\usetheme{default}
\usepackage[utf8]{inputenc}
\usepackage[T1]{fontenc}
\usepackage{hyperref}
\usepackage{multimedia}
\usepackage{media9}
\usepackage{subfig}
% \usepackage[font=small]{caption}
% \usepackage[font=small]{subcaption}

\usepackage[tensorialbold]{mescommandes}
\usepackage[babel=true,kerning=true]{microtype}
\usepackage{amsmath}
\usepackage{amsfonts}
\usepackage{amssymb}
\usepackage{mathrsfs}
\usepackage{graphicx}
\graphicspath{{figures/}}
\usepackage{fancybox}
\usepackage{textcomp}
\usepackage{multicol}
\usepackage{xcolor}
\usepackage{lmodern}
\RequirePackage{tikz}
\usetikzlibrary{patterns} 
%\usepackage[hang,tight,scriptsize]{subfigure}
\usetikzlibrary{shapes}
\usetikzlibrary{snakes}
\usepackage{pgfplots}
%\usepackage{pgfplotsthemetol}
\pgfplotsset{compat=newest,
	grid=both,
        tick label style={font=\normalsize},
	label style={font=\normalsize},
	legend style={font=\normalsize},
	legend cell align={left},
        yticklabel style={/pgf/number format/fixed},
        % define user colormap
	colormap={tol}{[1cm] rgb255(0cm)=(120,28,129) rgb255(1cm)=(63,96,174) rgb255(2cm)=(83,158,182) rgb255(3cm)=(109,179,136) rgb255(4cm)=(202,184,67) rgb255(5cm)=(231,133,50) rgb255(6cm)=(217,33,32)}
}

% define user color
\definecolor{col1}{RGB}{51,34,136}
\definecolor{col2}{RGB}{136,204,238}
\definecolor{col3}{RGB}{17,119,51}
\definecolor{col4}{RGB}{221,204,119}
\definecolor{col4}{RGB}{204,102,119}
\definecolor{col5}{RGB}{217,33,32}
\definecolor{col6}{RGB}{170,68,153}
\definecolor{col7}{RGB}{227,156,55}

%% User colors
\definecolor{Purple}{RGB}{120,28,129}
\definecolor{Blue}{RGB}{63,96,174}
\definecolor{Duck}{RGB}{83,158,182}
\definecolor{Green}{RGB}{109,179,136}
\definecolor{Yellow}{RGB}{202,184,67}
\definecolor{Orange}{RGB}{231,133,50}
\definecolor{Red}{RGB}{217,33,32}

%\setcounter{tocdepth}{1}
\usefonttheme{professionalfonts}
\usetheme[progressbar=foot,subsectionpage=progressbar,sectionpage=none]{metropolis}
\useoutertheme{Headinfoline}
\setbeamertemplate{section in toc}{{\inserttocsectionnumber.}~\inserttocsection    \vspace{-.1\baselineskip}}

\setbeamerfont{section in toc}{size=\normalsize,series=\bfseries}
\setbeamerfont{subsection in toc}{size=\footnotesize}
    
%% CHANGE COLOR SETTINGS
\definecolor{mDarkBrown}{HTML}{604c38}
\definecolor{mDarkTeal}{HTML}{23373b}
\definecolor{mLightBrown}{HTML}{EB811B}
\definecolor{mLightGreen}{HTML}{14B03D}
\definecolor{CNBlue}{RGB}{16,38,72}
\definecolor{CNYellow}{RGB}{250,182,0}

%% fg= ; bg= background 
\setbeamercolor{normal text}{ fg= CNBlue!90 , bg= black!2 }
%\setbeamercolor{alerted text}{ fg=mDarkTeal  }
%\setbeamercolor{exemple text}{ fg=mDarkTeal  }





\setbeamerfont{bibliography entry author}{size=\scriptsize,series=\normalfont}
\setbeamerfont{bibliography entry title}{size=\scriptsize,series=\bfseries}
\setbeamerfont{bibliography entry location}{size=\scriptsize, series=\normalfont}
\setbeamerfont{standout}{size=\Large,series=\bfseries}
%%%%%%%%%%caracterisation du document %---------------------------------------------------------------------
\hypersetup{
	pdftitle    = {Formulation of the DGMPM},
	pdfsubject  = {MS team meeting - March 2018},
	linkcolor    = red,
	pdfauthor   = {Adrien Renaud},
	pdfkeywords = {numerical simulation, hyperbolic problems, discontinuous Galerkin}
	colorlinks=true,
	linkcolor=black,
	citecolor=blue,
	urlcolor=blue
}



%%-------------- Construction de la page de presentation -------------------------------------------------------
\title[The Discontinuous Galerkin Material Point Method]
{\Large\bf  {The Discontinuous Galerkin Material Point Method: \\application to hyperbolic problems in solid mechanics}}

\date[]{
	\footnotesize{PhD defense} --
	December 14 2018 \\ \hspace*{7.cm}\includegraphics[trim = 0cm 4cm 0cm 0cm, clip,scale=0.1]{Logo_GEM.pdf} \hspace*{2.cm}\includegraphics[scale=0.25]{Logo_ECN.pdf}}%\logo{ \includegraphics[trim = 0cm 4cm 0cm 0cm, clip,scale=0.1]{Logo_GEM.pdf} \hspace*{2.cm}\includegraphics[scale=0.25]{Logo_ECN.pdf}}
\author{A. Renaud \\ Supervisors: T. Heuz\'e, L. Stainier} 


%------------------------------------------------------------------------

\setbeamertemplate{bibliography item}{\insertbiblabel}


%% Baptist's beamer clock
\newcommand{\myBeamerClock}[2]{
  % #1 is the radius of the clock
  % #2 is the vertical shift for inline placement
  \tikz[baseline=#2]{
    \filldraw (0,0) -- (0,#1) arc (90:(90-\insertframenumber/(\inserttotalframenumber)*360):#1);
    \draw (0,0) circle (#1);
  }
}


\begin{document}
\begin{frame}[plain]
  \maketitle
\end{frame}

\begin{frame}[plain]{Outline}
  \tableofcontents%[hideallsubsections]
\end{frame}
\begin{frame}[plain]{Outline}
  \begin{columns}
    \begin{column}{0.5\textwidth}
      \tableofcontents%[hideallsubsections]
    \end{column}
    \begin{column}{0.5\textwidth}
      \tableofcontents%[hideallsubsections]
    \end{column}
  \end{columns}
\end{frame}


\section{Motivations}
% \begin{frame}[plain]{Outline}
%   \tableofcontents[currentsection]
% \end{frame}

\subsection{Physics // Mathematics}
\begin{frame}
  Problem $\rightarrow$ model for solid mechanics
  % Problems considered -- dynamics + applications + solid mechanics (equations)
  % Soit (i) on inclue les équations du modèle dans les motivations soit (ii) on parle d'abord des difficultés qu'après on illustre ?

  % (i) - Problèmes de dynamique des solides: lois de conservations + lois constitutives => système hyperbolique
  %     - EDP dont les solutions font intervenir des ondes
  %     - Equations complexes (non-linéaires + multi-dimensionnelle) + solutions compliquées car ondes qui interagissent
  %     - Intérêt de la simulation pour calculer des solutions approchées
  %     - Mais cependant, on a des limitations (grandes defs ; suivi des ondes [oscillations + diffusion] ; difficultées à assurer la convergence vers une solution physique ??? [un truc pour introduire la prise en compte de la structure caractéristique])
  %     - Exemples de limitations avec les méthodes existantes : FEM -- FV -- Meshfree 
      
  Model: hyperbolic problems + difficulties (finite deformations, waves that can interact $\rightarrow$ complex solution)
\end{frame}
\subsection{Numerical simulation}
\begin{frame}
  Allows to compute approximate solutions

  Shortcomings of FEM -- FVM -- Meshfree

  Fail to mimic the physical behavior due to the lack of information about the structure (talk about dammage, softening, plasticity)
\end{frame}

\subsection{Objectives and strategy}
\begin{frame}{Objectives}
  Numerical methods able to deal with finite deformation + waves

  Solution of elastoplasticity problems in 2D
\end{frame}

\begin{frame}{Strategy}
  PIC mapping in MPM + DG approx

  Characteristic structure + loading path etc.

  Il faut bien démarquer les deux parties dans le plan de la prez.
\end{frame}

\section{Derivation of the DGMPM}
\subsection{Discrete system}
\subsection{Non-homogeneous systems}
\subsection{Interface fluxes}

\section{Numerical analysis}
\subsection{von-Neumann linear stability analysis}
\subsection{Convergence analysis}

\section{Numerical simulations}

\section{Elastoplastic hyperbolic problems in two space dimensions}

% \begin{frame}{Test video}
%   %\href{comparison.ogv}{\includegraphics[trim = 0cm 4cm 0cm 0cm, clip,scale=0.1]{Logo_GEM.pdf}}
%   \includemedia[width=0.6\linewidth,height=0.3375\linewidth, % 16:9
% activate=pageopen,
% addresource=comparison.ogv,
% flashvars={source=comparison.ogv}
% % flashvars={
% % modestbranding=1 % no YT logo in control bar
% % &autohide=1       % controlbar autohide
% % &showinfo=0       % no title and other info before start
% % &rel=0            % no related videos after end
% % }
% ]{}{VPlayer.swf}
% \end{frame}


\begin{frame}{Test2}
  % \movie[showcontrols]{\includegraphics[trim = 0cm 4cm 0cm 0cm, clip,scale=0.1]{Logo_GEM.pdf}}{compare.mp4}
  \begin{center}
    % \movie[width=.9\textwidth,height=0.9\textheight,showcontrols,loop]{\includegraphics[scale=0.2]{Logo_GEM.pdf}}{compare.mp4}
    \movie[width=.9\textwidth,height=0.9\textheight,showcontrols,loop]{\includegraphics[scale=0.2]{Logo_GEM.pdf}}{compare2.mp4}
  \end{center}
  add a hidden link to vlc player somewhere (frame title?)
\end{frame}



\end{document}

%%% Local Variables:
%%% mode: latex
%%% TeX-master: t
%%% End:
