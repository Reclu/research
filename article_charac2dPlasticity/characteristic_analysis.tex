In this section, the governing equations of general dynamic problems in elasto-plastic solids are first written in a unique and generic framework based on the fourth-order stiffnesses tensor.
Thus, the works mentioned in the introduction, which are formulated in terms of elastoplastic softnesses, appear as particular cases of the present investigation.

Second, the spectral analysis of the resulting hyperbolic system is performed by considering one arbitrary propagation direction of waves.
This approach is motivated by the resolution of Riemann problems in one space dimension in numerical schemes such as the Finite Volume Method (FVM) \cite{Leveque,Toro}.
It furthermore avoids the cumbersome search for bi-characteristics as \textsc{Clifton} proposed in order to build elastoplastic finite difference schemes \cite{Clifton_thesis}.
It must be emphasized that the present work aims at highlighting a sufficient amount of information so as to enable numerical schemes to mimic the analytical behavior, and does not requires the complete bi-characteristic structure.

\subsection{Governing equations}
We consider the isothermal deformation of a solid body of mass density $\rho$ in the linearized geometrical framework.
The balance equation of linear momentum with neglected body forces, and the geometrical balance equations \cite{Plohr,Gil_HE} are:  
\begin{equation}
  \label{eq:balance_equations}
  \left\lbrace \begin{aligned}
    & \rho \dot{\vect{v}} - \nablav \cdot \tens{\sigma} = \vect{0} \\
    &  \dot{\tens{\eps}} - \nablav \cdot \(\frac{\vect{v}\otimes \tens{I} + \tens{I} \otimes \vect{v}}{2}\) = \vect{0} 
  \end{aligned} \right.
\end{equation}
where $\vect{v}$, $\tens{I}$ , $\tens{\sigma}$ and $\tens{\eps}$ denote respectively the velocity vector, the second-order identity tensor, the Cauchy stress tensor and the strain tensor.
The latter additively decomposes into an elastic strain $\tens{\eps}^e$ and a plastic strain $\tens{\eps}^p$.
Assuming a Cartesian coordinate system, equations \eqref{eq:balance_equations} can also be written as:
\begin{equation}
  \label{eq:conservative_form}
  \drond{\Ucb}{t} + \drond{\Fcb\cdot \vect{e}_i}{x_i}=\tens{0}
\end{equation}
in which the vector of conserved quantities $\Ucb$ and the flux vectors $\Fcb_i=\Fcb\cdot \vect{e}_i$ are:
\begin{equation}
  \label{eq:vectors}
  \Ucb =\matrice{\rho\vect{v} \\ \tens{\eps}} \quad ; \quad \Fcb_i = \matrice{-\tens{\sigma}\cdot\vect{e}_i\\-\frac{\vect{v}\otimes\vect{e}_i +\vect{e}_i \otimes\vect{v} }{2} }
\end{equation}
Alternatively, introduction of an auxiliary vector of conserved quantities $\Qcb$ allows rewriting equation \eqref{eq:conservative_form} as a quasi-linear form by means of the chain rule:
\begin{equation}
  \label{eq:quasi-linear_form}
  \drond{\Qcb}{t} + \Absf^i \drond{\Qcb}{x_i}=\tens{0}
\end{equation}
In particular, setting $\Qcb=\matrice{\vect{v} \\ \tens{\sigma}}$, one write:
\begin{equation}
  \label{eq:matrix-quasi}
  \Absf^i = \(\drond{\Ucb}{\Qcb}\)^{-1}\drond{\Fcb_i}{\Qcb} = -\matrice{\tens{0}^2 & \frac{1}{\rho}\tens{I}\otimes\vect{e}_i\\ \Hbb\cdot \vect{e}_i & \tens{0}^4}
\end{equation}
$\tens{0}^p$ being a $p$th-order zero tensor.
It then appears that the characteristic structure of the hyperbolic problem, which is driven by the matrices $\Absf^i$, depends on the nature of the deformation (\textit{i.e. elastic or plastic}) through the fourth-order tangent modulus $\Hbb=\drond{\tens{\sigma}}{\tens{\eps}}$.

Reversible evolutions are governed the elastic law involving the elastic stiffness tensor $\Cbb$ and defined as: 
\begin{subequations}
  \begin{alignat}{1}
    \label{eq:elastic_law}
    & \tens{\sigma} = \Cbb: \tens{\eps}^e = 2\mu \tens{\eps}^e + \lambda \tr \tens{\eps}^e \: \tens{I} \\ 
    \label{eq:elastic_inverse}
    & \tens{\eps}^e = \Cbb^{-1}:\tens{\sigma} = \frac{1+\nu}{E} \tens{\sigma} - \frac{\nu}{E} \tr \tens{\sigma}  \: \tens{I}
  \end{alignat}
\end{subequations}
in which $(\lambda,\mu)$ are Lam{\'e}'s parameters, and $(E,\nu)$ are Young's modulus and Poisson's ratio respectively.
% We are concerned with linear isotropic hardening materials whose elastic domain is given by the von-Mises yield surface, under isothermal deformations in the linearized geometrical framework.
On the other hand, following the Generalized Standard Material framework \cite{GSM}, irreversible processes are described by a Helmholtz free-energy density potential and the von-Mises yield function.
Restricting ourselves to linear isotropic hardening, the elastic-plastic constitutive equations are: 
%Furthermore, materials falling within the Generalized Standard Material framework \cite{GSM} are considered so that, by postulating the existence of a Helmholtz free-energy density potential%that depends on a suitable set on internal variables
\begin{subequations}
  \label{eq:plasticity_equations}
  \begin{alignat}{1}
    \label{eq:von-Mises_yield}
    & f\(\tens{\sigma},R \)= \sqrt{\frac{3}{2}}\norm{\tens{s}} - \(R(p)+\sigma^y\) \leq 0 \\
    \label{eq:iso_hard}
    & R(p)=C \: p \\
    \label{eq:elastoplastic_tangent}
    & \tens{\dot{\sigma}}=\(\Cbb - \beta\:\tens{s}\otimes\tens{s} \):\tens{\dot{\eps}} = \Cbb^{ep}:\tens{\dot{\eps}} \\
    \label{eq:plastic_flow}
    & \beta = \frac{6\mu^2}{3\mu +C}\times\frac{1}{\tens{s}:\tens{s}}
    % \label{eq:EP_acoustic}
    % & A_{ij}^{ep}=n_k C^{ep}_{ikjl}n_l=  A_{ij}^{elast} -  \beta (n_k s_{ki})(s_{jl}n_l)
  \end{alignat}
\end{subequations}
Equation \eqref{eq:von-Mises_yield} describes the von-Mises yield surface in which $\tens{s}=\tens{\sigma}-\frac{1}{3}\tr \tens{\sigma} \tens{I}$ denotes the deviatoric part of the Cauchy stress, $\sigma^y$ denotes the yield stress in tension, and $R(p)$ is the thermodynamical force $R$ conjugated to the cumulated plastic strain $p$.
Linear isotropic hardening is then considered through equation \eqref{eq:iso_hard} by means of the hardening modulus $C$.
At last, it is seen in equation \eqref{eq:elastoplastic_tangent} that the elastoplastic tangent modulus $\Cbb^{ep}$ can be decomposed into the elasticity tensor and another part depending on the direction of the plastic flow through the coefficient $\beta$ \eqref{eq:plastic_flow}.

\subsection{Spectral analysis}
\label{sec:spectral-analysis}

Considering an arbitrary direction of space $\vect{n}$, the quasi-linear form \eqref{eq:quasi-linear_form} reads: 
% The quasi-linear form of the sets of equations \eqref{eq:balance_equations} and \eqref{eq:plasticity_equations} in a Cartesian coordinate system and an arbitrary direction $\vect{n}$ is:
\begin{equation}
  \label{eq:quasilinear_normal}
  \Qcb_t + \Jbsf \drond{\Qcb}{x_n} = \vect{0} 
\end{equation}
where $x_n=\vect{x}\cdot\vect{n}$ and $\Jbsf=n_i\Absf^i$ is the Jacobian matrix.
The characteristic structure of the problem is given by the $m$ eigenvalues $c_K$ and associated left eigenvectors $\Lcb^K= \[ \vect{v}^K \: , \: \tens{\sigma}^K \]$ of the Jacobian matrix satisfying:
\begin{equation}
  \label{eq:eigen_system}
  \vect{\Lc}^K \(\Jbsf - c_K \Ibsf\) = \vect{0}
\end{equation}
with $\Ibsf$, the $m\times m$ identity matrix.
Thus, for non-zero eigenvalues one gets:
\begin{subequations}
  \begin{alignat}{1}
    \label{eq:eigen_left_stress}
    & -\tens{\sigma}^K:\(\Hbb\cdot  \vect{n}\) - c_K  \vect{v}^K =\vect{0} \\
    \label{eq:eigen_left_velo}
    & -\frac{1}{\rho}\vect{v}^K\otimes\vect{n} - c_K \tens{\sigma}^K = \tens{0}
  \end{alignat}
\end{subequations}
Substitution of $\tens{\sigma}^K$ obtained from \eqref{eq:eigen_left_velo} in \eqref{eq:eigen_left_stress} leads to:
\begin{equation}
  \label{eq:acoustic_eigen}
 (\vect{v}^K\otimes\vect{n}):\(\Hbb\cdot  \vect{n}\) - \rho c_K^2 \vect{v}^K = \tens{0}
\end{equation}
which is the left eigensystem of the acoustic tensor $A_{ij}=n_k H_{ik j l}  n_l$.
Due to the symmetry of $\tens{A}$, system \eqref{eq:acoustic_eigen} is equivalent to the right eigensystem:
\begin{equation}
  \label{eq:acoustic_eigen_system_lambda}
  \(  n_k H_{ik j l}  n_l - \rho c_K^2 \delta_{ij} \) v_j^K =0
\end{equation}
or alternatively, with the eigenvalues $\omega_p$ and associated eigenvectors of the acoustic tensor $\vect{l}^p\: \: (p=1,2,3)$:
\begin{equation}
  \label{eq:acoustic_eigen_system}
   \( \tens{A} - \omega_p \tens{I} \) \cdot \vect{l}^p  = \vect{0}
\end{equation}
The condition for system \eqref{eq:quasilinear_normal} to be hyperbolic (real eigenvalues and independent eigenvectors) is thus ensured by the positive definiteness of the acoustic tensor, also known as the \textit{strong ellipticity} condition \cite{Foundation_of_elasticity}:
\begin{equation}
  \label{eq:strong_ellipticity}
  (\vect{m}\otimes \vect{n}): \Hbb: (\vect{n}\otimes \vect{N}) > 0 \quad \forall \vect{n},\vect{m} \in \Rbb^3 \: ; \: \vect{n},\vect{m} \ne \vect{0}
\end{equation}
If the condition holds, the acoustic tensor admits $3$ couples of eigenvalue--eigenvector $\{\omega_p,\vect{l}^p\}$ leading to $6$ couples $\{c_K,\Lcb^K\}$ for the Jacobian matrix, the $6$ other eigenvalues being null \cite{Kluth}.
The couples $\{c_K,\Lcb^K\}$ are referred to as \textit{left characteristic fields}.
Notice that since both the elastic stiffness tensor $\Cbb$ and the elastoplastic tangent modulus $\Cbb^{ep}$ may be involved in equation \eqref{eq:quasi-linear_form}, six left characteristic fields are obviously associated with elastic and plastic evolutions respectively.
The left eigenvectors associated with non-zero eigenvalues of the Jacobian matrix are obtained by using equation \eqref{eq:eigen_left_velo} so that the following $6$ eigenfields of the quasi-linear form \eqref{eq:quasilinear_normal} can be defined:
\begin{equation}
  \label{eq:left_eigenfields}
    \left\lbrace \pm \sqrt{\frac{\omega_p}{\rho_0}} ; \: \[ \pm \rho_0\sqrt{\frac{\omega_p}{\rho_0}} \vect{l}^p , -\vect{l}^p\otimes \vect{n} \]  \right\rbrace ,\quad p=1,2,3
\end{equation}
At last, six independent left eigenvectors associated with the null eigenvalue of multiplicity $6$ can be found by solving equation \eqref{eq:eigen_left_stress} for the null eigenvalue:
\begin{equation}
  \label{eq:left_null_eigenvectors}
  \tens{\sigma}^K:\(\Hbb\cdot  \vect{n}\) =\vect{0},\quad K=1,...,6
\end{equation}

%In what follows, the above equations are specified to plane strain and plane stress cases.



%%% Local Variables:
%%% mode: latex
%%% ispell-local-dictionary: "american"
%%% TeX-master: "manuscript"
%%% End:
