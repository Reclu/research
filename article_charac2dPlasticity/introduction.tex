%%%%% Problems considered and why: forming; simulations; simulation based on analytical works; but limited to particular problems.
% Dire de quel type de problèmes on parle. Ensuite, dire que des solutions analytiques existent mais que pour les problèmes de l'ingénieur, la simulation numérique permet de calculer des solutions approchées pour des géométries complexes, ou pour lesquelles les solutions impliquent beaucoup d'ondes intéragissant entre lles. Ensuite, l'arrivée des méthode de type volumes finis a permit de prendre en compte la conaissance que l'on avait des solutions exactes dans les schémas numériques afin de mieux évaluer les états résiduels. Cependant, ça vaut pour les problèmes simples (élasticité, plasticité 1D, ou onde plane plastique). Pour des problèmes plus complexes 2D et 3D, un gain dans la connaissance de la structure caractéristique pourrait permettre de grandement améliorer les résultats numérique à travers le développement d'outils dédiés.

% It has been shown throughout this manuscript that hyperbolic problems in solid mechanics are solved in a different manner depending on the numerical method employed. 
% In particular, irreversible deformations which are usually numerically computed based on well-known constitutive integrators, may greatly differ from one scheme to another even for one-dimensional problems.
% However, the accurate assessment of residual stresses and strains are of major importance for many industrial applications such as high-speed metal forming, crash-proof design or the study of the impact of earthquakes on structures.
% The simulations performed in chapter \ref{chap:chap4} emphasized the improvements enabled by the knowledge of the characteristic structure of the solutions of conservation laws, especially for elastoplastic solids.
% Nevertheless, the use of an elastic-plastic approximate Riemann solver is so far only possible for problems in one space dimension.

% The purpose of this chapter is to identify typical behavior of the solutions of two-dimensional elastoplasticity problems under small strains.
% It is believed that the knowledge of these solutions will allow, through the better understanding of their mathematical features, the building of approximate numerical solutions embedding a sufficient amount of information in order to mimic the analytical behavior.
% This will be possible at a low computational cost provided that some key-properties of the exact solutions are clearly identified. 

% This chapter is organized as follows.
% A brief historical review of the solution of dynamic problems in two-dimensional elastic-plastic solids is made in section \ref{sec:review}.
% Then, the equations of plasticity are recalled in section \ref{sec:charac_plast} so that the characteristic analysis, followed by the application of the method of characteristics, can be carried out.
% In section \ref{sec:stress_paths}, attention is paid to the evolution of stress components inside simple waves that might propagate by means of a mathematical study of the ODEs satisfied within these waves.
% Since the developments rapidly become cumbersome, the analysis is supplemented with numerical results in section \ref{sec:stress_paths_num}.
% At last, some identified trends are discussed at the end of the chapter in order to use them for building a dedicated Riemann solver. 


Until the 50s, research on dynamic problems in elastic-plastic solids were focused on uni-axial stress or strain, pure bending or pure torsion loading conditions \cite{Taylor,vonKarman}, and were carried out for materials characterization purposes.
The first references that brought some understanding about the response of linearly hardening solids to combined shear and pressure loads are those of \textsc{Rakhmatulin} \cite{Rakhmatulin} and \textsc{Cristescu} \cite{CRISTESCU19591605}.
These early analytical investigations on plane stress impacts in the plastic regime led to the conclusion that elastic waves, as well as plastic combined-stress simple waves, can propagate in two-dimensional solids. 
While the former were well-known, the latter were shown to fall into two families: the \textit{fast waves} and the \textit{slow waves}.

Later, \textsc{Bleich} and \textsc{Nelson} \cite{Bleich} considered superimposed plane and shear waves in an ideally elastic-plastic material submitted to step loads.
It has then been highlighted that different loading cases yield different characteristic structures of problems in a semi-infinite medium with prescribed traction forces and initial conditions (the so-called Picard problem).
The results of \textsc{Bleich} and \textsc{Nelson} thus revealed the complexity of plastic flows in multi-dimensional solids undergoing dynamic loadings.
The same conclusions have been drawn by \textsc{Clifton} \cite{Clifton} for hardening materials under tension-torsion, who furthermore studied the influence of plastic pre-loading on the solution.
This contribution established the existence of loading paths through the simple waves arising from the characteristic analysis of the hyperbolic system.
Indeed, the combined-stress wave nature lies in ODEs which govern the evolution of stress components within the simple waves.
% The integration of these equations of the form $d\sigma_{11}=\psi d\sigma_{12}$ allows the building of curves that connect the applied stress state of the Picard problem $(\sigma^d_{11},\sigma^d_{12})$ to the initial state of the medium.
The integration of these equations allows the building of curves that connect the applied stress state of the Picard problem to the initial state of the medium.
% It has been for instance shown that if a solid is acted upon by a traction force such that $\sigma^d_{11}=0$ and $\sigma^d_{12}$ lies outside the elastic domain, only an elastic shear discontinuity, followed by a slow simple wave, propagates.
It has been for instance shown that if a solid is acted upon by a pure shear traction beyond the elastic domain, only an elastic shear discontinuity, followed by a slow simple wave, propagates.
Conversely, other loading conditions may lead to the combination of an elastic pressure discontinuity and a fast wave, possibly followed by a slow wave.
Another notable conclusion is that the combined loading paths followed inside plastic simple waves are not necessarily radial even if a von-Mises flow rule is considered.

Experimental data collected on a thin-walled tube submitted to a dynamic tensile load \cite{Clifton_exp,Clifton_exp2} confirmed the existence of two distinct families of  simple waves, both involving combined stress paths.
These works nevertheless exhibited some discrepancies with the theory which have been attributed to the assumption made on the von-Mises yield surface.
As a matter of fact, a constant strain region lying between the fast and slow waves that is predicted by the theory \cite{Clifton} could not be seen in experimental results.
However, by following the endochronic theory of plasticity \cite{Valanis} which does not require the introduction of a yield surface, \textsc{Wu} and \textsc{Lin} \cite{Wu_experimental} obtained numerical results that better fit the experimental data provided by \textsc{Lipkin} and \textsc{Clifton} \cite{Clifton_exp2}.
The good agreement showed between numerical and experimental results \cite{Wu_experimental} thus confirmed the theory.

\textsc{Ting} and \textsc{Nan} \cite{Ting68} then generalized the work of \textsc{Bleich} and \textsc{Nelson} to hardening materials and \textsc{Ting} \cite{Ting69} widened that of \textsc{Clifton} to more complex loadings, that is, a superimposition of one plane wave and two shear waves.
Once again, the mathematical study of the ODE system governing the stress evolution inside fast and slow simple waves led to the construction of loading paths in stress space that depend on the external loads. A review of governing equations for all the cases depending on one space dimension considered above can be found in \cite{Nowacki}.

The information on characteristic structures thus provided has then been used by \textsc{Lin} and \textsc{Ballman} \cite{Lin_et_Ballman} for the development of an iterative Riemann solver.
This procedure is based on successive guesses of the stress state lying in the stationary region so that the loading paths predicted by the theory of \textsc{Clifton} \cite{Clifton} can be integrated numerically until convergence.
The implementation of this solver within a second-order Godunov scheme provided results that were in good agreement with the exact solutions.
Nevertheless, the theoretical investigations mentioned above restrict the development of such numerical tools to problems that depend on one space dimension.
%%
\textsc{Clifton} tackled the solution of plane strain problems in elastic-plastic solids by looking for bi-characteristics \cite{Clifton_thesis} in order to build finite difference schemes that account for plastic waves.
The point of view adopted here is that one can benefit from the simplifications introduced by the writing of Riemann problems in an arbitrary direction.
Indeed, the method of characteristics rather than the more complex method of bi-characteristics can be employed with the hyperbolic systems.

On the other hand, the existence of plastic shocks in solids under plane wave assumptions has been investigated by several authors. 
First, \textsc{Mandel} \cite{Mandel1} showed the existence of stable plastic shocks in three-dimensional elastoplastic media.
In this work, Hugoniot curves are built by assuming that the internal variables followed a radial loading path through a plastic shock.
\textsc{Lee} and \textsc{Germain} \cite{Germain_shock} considered that Hugoniot curves in elastic-plastic solids cannot be constructed without studying the internal structure of the shock.
Thus, an elastic-viscoplastic continuum problem is solved by magnifying the narrow region in the vicinity of the shock in which the fields vary sharply.
The shock solution was then taken as the limit when viscosity tends to zero.
A study of the internal structure of the shock has also been made by \textsc{Stolz} \cite{Claude}.
In the latter approach, the Hugoniot conditions across a shock moving at constant speed are derived by doing an asymptotic analysis. 
The author thus provided existence and uniqueness conditions for a shock in compression provided that elastic stiffening dominates the (concave) hardening saturation.
Nevertheless, according to \textsc{Mandel} \cite{Mandel2}, such an analysis of the internal structure of the shock is not required to build Hugoniot curves, provided one chooses $\eps^p_1$ as internal variable and not the specific work $w$.
However, the propagation of plastic shocks is still an open scientific issue and subject to debate.

In what follows, simple waves are considered in elastic-plastic solids with natural initial conditions by assuming a concave hardening law with no stiffening so that plastic shocks do not arise.




%%% Local Variables:
%%% mode: latex
%%% ispell-local-dictionary: "american"
%%% TeX-master: "manuscript"
%%% End:
