Appuyer sur le fait que la dgmpm est cool pour 1ppc mais que pour plus de points par cellule on a une diffusion qui nécessite encore un peu de travail.
On a vu sur un cas 1D que les ondes peuvent être correctement suivies et les états résiduels estimés de manière satisfaisente.


On peut envisager des stratégies d'adaptation de maillage et également d'autres projections des champs entre la grille et les particules plutot que celles utilisées dans les méthodes PIC.

Par ailleurs, lagrangien actualisé ou adaptation de grille pour des problèmes avec très grandes transformations.
%%% Local Variables:
%%% mode: latex
%%% TeX-master: "manuscript"
%%% End:
