\subsection{Conservation laws}
Consider a solid domain in the initial or reference configuration $\Omega_0 \subset \Rbb^3$, bounded by the surface $\partial \Omega$, in which material particles are located with Lagrange coordinates $\vect{X}=X_\alpha\vect{e}_\alpha \in \Omega_0$.
The motion of the particles within the time interval $t\in\tau= [0,T^{end}]$ is described by the deformation function $\vect{\varphi}(\vect{X},t)$ that also defines the current configuration as: $\Omega(t)=\{\vect{x} \in \Rbb^3 \: \lvert \: \vect{x}=\varphi(\vect{X},t), \:\vect{X},t \in \Omega_0\times\tau\}$.
%We consider within the time interval $\tau= [0,T^{end}]$ the isothermal deformation of a solid domain $\Omega \subset \Rbb^3$, bounded by the surface $\partial \Omega$, between the reference configuration $\Omega_0=\Omega(t=0)$ and subsequent ones $\Omega(t) \: (t\in\tau)$.
%The motion of material particles initially located with Lagrange coordinates $\vect{X} \in \Omega_0$ is described by the deformation function $\vect{\varphi}(\vect{X},t)$ that also defines the current configuration as: $\Omega(t)=\{\vect{x} \in \Rbb^3 \: : \: \vect{x}=\varphi(\vect{X},t), \:\vect{X},t \in \Omega_0\times\tau\}$.
The second-order deformation gradient tensor $\tens{F}$ and the velocity vector $\vect{v}$ are then defined as:
\begin{align}
  \label{eq:deformation_gradient}
  &\tens{F}(\vect{X},t) = \drond{\vect{\varphi}(\vect{X},t)}{\vect{X}} = \nablat_0 \vect{\varphi}(\vect{X},t) \\
  \label{eq:velocity}
  & \vect{v}(\vect{X},t)= \drond{\vect{\varphi}(\vect{X},t)}{t} = \vect{\dot{\varphi}}(\vect{X},t)
\end{align}
where $\nablat_{0}(\bullet)$ denotes the gradient operator in the reference configuration and the superposed dot refers to the material time derivative.
Geometrical balance laws \cite{Plohr,Gil_HE,Haider_FVM} are written by differentiating equation \eqref{eq:deformation_gradient} with respect to time and combining it with equation \eqref{eq:velocity}:
% The time derivative of equation \eqref{eq:deformation_gradient} combined with equation \eqref{eq:velocity} and the use of the divergence operator in the reference configuration $\nablav_0\cdot(\bullet)$, yields geometrical balance laws \cite{Plohr,Gil_HE,Haider_FVM}:
\begin{equation}
  \label{eq:kinematic_laws}
  \dot{\tens{F}} - \nablav_0 \cdot \( \vect{v} \otimes \tens{I} \) = \tens{0} \quad \forall \vect{X},t \in \Omega_0 \times \tau 
\end{equation}
in which $\nablav_0\cdot(\bullet)$ is the divergence operator in the reference configuration, and $\tens{I}$ is the second-order identity tensor.
Moreover, the balance equation of the Lagrangian linear momentum must be satisfied:
\begin{equation}
  \label{eq:linear_momentum}
  %\rho_0(\vect{X}) \: \dot{\vect{v}} - \nablav_0 \cdot \tens{\Pi}(\vect{X},t) = \rho_0(\vect{X}) \vect{b}(\vect{X},t)  \quad \forall \vect{X},t \in \Omega_0 \times \tau
  \rho_0 \: \dot{\vect{v}} - \nablav_0 \cdot \tens{\Pi}= \rho_0\vect{b} \quad \forall \vect{X},t \in \Omega_0 \times \tau
\end{equation}
where $\rho_0(\vect{X})$ is the reference mass density, $\vect{b}(\vect{X},t)$ is the body forces vector, and $\tens{\Pi}(\vect{X},t)$ is the first Piola-Kirchhoff stress tensor.
%In what follows, the body forces are neglected.
From now on, calligraphic symbols stand for column array and the body forces are neglected.

%From now on, calligraphic symbols stand for column array.
Introduction of the vector of conserved quantities $\Ucb=\[\rho_0 \vect{v} \: ,\:\tens{F}\]$ allows the writing of a system of conservation laws consisting of equations \eqref{eq:kinematic_laws} and \eqref{eq:linear_momentum}:
\begin{equation}
  \label{eq:conservative_form}
  \drond{\Ucb}{t} + \nablav_0 \cdot \Fcb = \vect{0} \quad \forall \vect{X},t \in \Omega_0 \times \tau 
\end{equation}
where $\Fcb=-\[\tens{\Pi} \: ,\:\vect{v}\otimes \tens{I}\]$ is the flux vector.
In Cartesian coordinates, system \eqref{eq:conservative_form} reads:
\begin{equation}
  \label{eq:conservative_cartesian}
  \drond{\Ucb}{t} + \drond{\Fcb\cdot\vect{e}_\alpha}{X_\alpha} = \vect{0} \quad \forall \vect{X},t \in \Omega_0 \times \tau 
\end{equation}
with $\Fcb\cdot\vect{e}_\alpha=-\[\tens{\Pi}\cdot\vect{e}_\alpha \: ,\:\vect{v}\otimes \vect{e}_\alpha\]$.
Alternatively, the use of an auxiliary vector $\Qcb=\[\vect{v}\:,\:\tens{\Pi}\]$ along with the chain rule leads to the following quasi-linear form \cite{Trangenstein91}:
\begin{align}
  \label{eq:quasi-linear}
  &\drond{\Qcb}{t} + \Jbsf_\Qc^\alpha \drond{\Qcb}{X_\alpha} = \vect{0} \quad \forall \vect{X},t \in \Omega_0 \times \tau \\
  & \text{with } \Jbsf_\Qc^\alpha= -\(\drond{\Ucb}{\Qcb}\)^{-1}\drond{\Fcb\cdot\vect{e}_\alpha}{\Qcb} = -\matrice{\tens{0}^2 & \frac{1}{\rho_0} \tens{I}\otimes \vect{e}_\alpha \\ \drond{\tens{\Pi}}{\tens{F}}\cdot\vect{e}_\alpha & \tens{0}^4}
\end{align}
in which $\tens{0}^k$ denotes a $k$th-order zero tensor.
The constitutive model of the material is then explicitly involved in the quasi-linear form through the fourth-order tangent modulus tensor $\Hbb=\drond{\tens{\Pi}}{\tens{F}}$.

\subsection{Constitutive model: Hyperelastic-plastic materials}
\label{subsec:cont_constitutive}
Following the multiplicative decomposition of the deformation gradient \cite{Lee_FeFp}: $\tens{F}=\tens{F}^e\tens{F}^p$, irreversible processes that account for the evolution of the micro-structure of the material are described by means of internal variables \cite{Lubliner}.
Denoting by $\Vcb$ the column array consisting of hardening variables, the complete set of internal variables is: $\{\tens{F}^p,\Vcb \}$.
It is assumed that the evolution of internal variables only depends on the local state, so that kinetic equations can be written:
\begin{equation}
  \label{eq:plastic_kinetic_laws}
  \dot{\Vcb}=f(\tens{F},\tens{F}^p,\Vcb)
\end{equation}
as well as a general plastic flow rule:
\begin{equation}
  \label{eq:flow_rule}
  \dot{\tens{F}^p}{\tens{F}^p}^{-1}=\dot{\Vcb}\tens{M}
\end{equation}
In equation \eqref{eq:flow_rule}, kinematic conditions for the plastic flow are prescribed through the second-order tensor $\tens{M}$.
More specifically, the von Mises flow rule results from the restriction $\tens{M} \in \{\tens{A} \in \Rbb^3\times \Rbb^3\: \lvert\: \tens{A}: \tens{A}=\frac{2}{3},\: \tens{A}:\tens{I}=0\}$. 


For materials in which internal processes do not influence the elastic response (\textit{e.g.  metals}), the thermodynamic state is described by the Helmholtz free energy density that is assumed to decompose additively as:
\begin{equation}
  \label{eq:helmholtz_decomposition}
  \psi(\tens{F},\tens{F}^p,\Vcb)= \psi^e(\tens{F}{\tens{F}^p}^{-1}) +  \psi^p(\tens{F}^p,\Vcb)
\end{equation}
where $\psi^e$ and $\psi^p$ respectively govern elastic and plastic evolutions.
The first Piola-Kirchhoff stress tensor as well as the thermodynamic forces conjugate to $\tens{F}^p$ and $\Vcb$ follow from the free energy density as \cite{Truesdell}:
\begin{equation}
  \label{eq:thermodynamic_forces}
  \tens{\Pi}=\drond{\psi}{\tens{F}} \quad ;\quad \tens{T}=-\drond{\psi}{\tens{F}^p} \quad ; \quad \Acb = -\drond{\psi}{\Vcb}
\end{equation}
By considering the additive decomposition \eqref{eq:helmholtz_decomposition} combined with the flow rule \eqref{eq:flow_rule}, the last relation reads:
\begin{equation}
  \label{eq:hardening_forces}
  \Acb = -\drond{\psi}{\tens{F}^p}\drond{\tens{F}^p}{\Vcb} - \drond{\psi^p}{\Vcb} = \tens{T}\(\tens{M}\tens{F}^p\) - \drond{\psi^p}{\Vcb}
\end{equation}

We then postulate the existence of a pseudo-dissipation potential $\phi$ from which the rate of internal variables can be derived:
\begin{equation}
  \label{eq:pseudo-dissipation_potential}
  \dot{\Vcb}= \drond{\phi(\Acb)}{\Acb}
\end{equation}
Alternatively, the dual pseudo-dissipation potential $\phi^*(\dot{\Vcb})$, that results from the Legendre transform of $\phi$, allows writing:
% \begin{equation}
%   \label{eq:dual_pseudo-dissipation_potential}
%   \phi^*(\dot{\Vcb}) = \Acb \cdot \dot{\Vcb} - \phi(\Acb)
% \end{equation}
% in such a way that one has
\begin{equation}
  \label{eq:dual_form_kinetic}
  \Acb = \drond{\phi^*(\dot{\Vcb})}{\dot{\Vcb}}
\end{equation}




%%% Local Variables:
%%% mode: latex
%%% TeX-master: "manuscript"
%%% End:
