%Hyperbolic systems of conservation laws that govern the propagation of mechanical waves are considered here in dissipative solids.
Systems of hyperbolic conservation laws governing the propagation of mechanical waves in dissipative solids are of great interest for the numerical simulation of high-speed forming techniques or crash-proof design for instance.
%This class of solid behaviors, depending on , is implicated in 
In these solids, whose behavior depends on history effects, the waves propagate while carrying the information about the loading, interacting with each other and reflecting on the boundaries so that complex responses arise.
The accurate assessment of irreversible deformations in dissipative solids therefore requires the correct description of those waves as well as the ability to account for their interactions.
Hyperbolic problems may be further complicated by possibly large displacements, rotations or strains, undergone by the solid.

% Requirements for the numerical methods
For the reasons mentioned above, the simulation of such problems is not a straightforward undertaking.
Indeed, the precise tracking of waves involves the use of a numerical method whose solutions are devoided of oscillations and too much numerical diffusion on the one hand, and that is able to manage large deformations on the other hand.
% FEM
The numerical simulation of hyperbolic initial boundary value problems in solids has been so far mostly addressed in indusrtial codes by using the Finite Element Method (FEM) \cite{Belytschko}.
FEM became attractive due to its ability to handle low or high-order approximations, and to easily deal with complex geometries and nonlinear constitutive models. % as those already mentioned.
Nevertheless, some difficulties related to the deformation of the mesh may be encountered when the formulation is built upon a material description of the motion (Lagrangian approach). 
In fact, the method is less efficient and accurate when the elements are highly distorted or entangled so that re-meshing techniques and projection steps must be employed.
These issues can be avoided by using a spatial description of the motion (Eulerian approach).
However interface tracking techniques and diffusing convection steps are required in order to follow the boundaries and transport internal variables, which is less convenient for solid mechanics.
Alternatively, Arbitrary Lagrangian Eulerian (ALE) methods aim at meeting advantages of both approaches while freeing themselves of their respective limits by distinguishing the motion of the mesh to these of material points.
This type of strategy nonetheless also requires re-meshing or re-zoning procedures that can be costly for fine meshes, as well as diffusive projection steps of internal variables for solid media.
In addition to problems implied by finite deformations, classical explicit time integrators used in solid dynamics with FEM introduce high frequency noise in the vicinity of discontinuities.
Such regions of high gradient may be caused by discontinuous waves that can occur in the solutions of hyperbolic problems.
The removal of these spurious numerical oscillations with additional viscosities is difficult to achieve without loss of accuracy, and can be troublesome for the wave tracking.

% FVM
On the other hand, the Finite Volume Method (FVM) \cite{Leveque}, initially developed for fluid dynamics, provided until the 90s piece-wise constant or piece-wise linear approximate solutions in cells that discretize a continuum.
The extension to very high-order has since been proposed by increasing the stencil of the scheme (see WENO \cite{WENO}).
This class of methods can embed tools based on the Total Variation Diminishing (TVD) stability condition \cite{Harten}, thus ensuring that no numerical spurious oscillations arise in the solutions. 
The formulation moreover lies on a conservative form leading to the same order of accuracy for all components of an unknown vector.
In particular, one shows that both velocity and gradients arise in that vector for solid mechanics \cite{Lee_FVM}.
This point makes a big difference with respect to methods that do not use a formulation written as a differential system of order one, namely FEM, for which the convergence rate of gradients is one order less than that of displacement.
%To some extent, the writing of solid mechanics equations in the form of conservation laws amounts to a mixed approach, well-known in FEM.
FVM formulations moreover rely on numerical fluxes that enable to account for the characteristic structure of hyperbolic equations, thus allowing an accurate tracking of the path of waves.
%Hence, finite volumes allow an accurate tracking of the path of waves although the most widely used approximations are second-order. 
Recent studies furthermore extended these approaches to solid mechanics for problems involving history-dependent models \cite{Gavrilyuk,Thomas_EP}, or finite deformations with a Lagrangian formulation \cite{Lee_FVM,Haider_FVM}.
Nevertheless, these methods are also grid-based techniques for which the numerical difficulties linked to large deformations occur.

% PIC-MPM
Alternatively, Particle-In-Cell methods (PIC) \cite{PIC} are based on particles that move in a computational mesh while carrying the fields of a problem.
The underlying grid is used in order to compute an approximate solution that is projected and stored at particles.
Hence, the background mesh can be discarded at each time step and re-constructed for computational convenience.
The application of PIC to solid mechanics led to the Material Point Method (MPM) \cite{Sulsky94} in which the constitutive equations are managed at particles.
As a result, the MPM can be seen as a mesh-free extension of FEM with quadrature points moving in elements.
Overcoming the storage of the approximate solutions based on elements, or cells, enables the removal of mesh entanglement instabilities.
Nevertheless, the projection required from the grid to the particles gives rise to some issues.
As originally proposed in PIC, a classical interpolation leads to significant numerical diffusion due to a spreading of the information over several cells.
An alternative procedure has then been introduced by the FLuid Implicit Particle method (FLIP) \cite{FLIP0} in order to reduce the diffusion at the cost of spurious oscillations \cite{XPIC}.
% DGMPM
More recently, the extension of the MPM to the Discontinuous Galerkin approximation in space (DG) \cite{NeutronDG} has been proposed in order to address the balance between numerical noise and diffusion.
The DGMPM \cite{DGMPM} thus makes use of an interpolation of updated nodal quantities to the particles (\textit{i.e. PIC projection}), based on shape functions whose supports reduce to one cell so that the numerical diffusion is limited \cite{Thesis}.
It then follows that the dual mesh consisting of an arbitrary grid and a set of particles allows to take advantage of both FEM and FVM.
Indeed, the element-wise weak form written on the background grid involves intercell fluxes and well-known constitutive integrators \cite{Simo} can be used at material points, the mapping between the nodes and the particles being based on piece-wise linear Lagrange polynomials.
The numerical solution of hyperbolic problems with the DGMPM in hyperelastic solids shows good agreements with exact solutions as well as significant improvements with respect to the original MPM \cite{DGMPM}.

The purpose of the present paper is to illustrate the method on problems involving rate-independent plastic solids undergoing finite deformations.
In particular, the focus is on hyperelastic-plastic solids owing to the thermodynamically consistent framework on which they are based, in contrast with hypoelastic-plastic models. 
Moreover, the thermodynamic framework enables the variational formulation of the equations of hyperelasto-plasticity that can be solved with dedicated numerical integrators based on incremental potentials \cite{Laurent99}.
The use of such variational constitutive integrators should facilitate the extension of the DGMPM to strongly thermo-mechanically coupled problems and provides a generic framework for the integration of rate-independent and rate-dependent plasticity.
% Furthermore, the error estimation allowed by the introduction of variational solvers might pave the way for the developement of mesh adaption strategies based on criteria given by the wave structure.
   

In what follows, the continuum equations of isothermal solid dynamics including balance laws and constitutive equations are recalled in section \ref{sec:continuum_problem}.
Next, section \ref{sec:discretization} is devoted to the derivation of the DGMPM discrete system that must be solved on the background grid at each time step. 
The discrete system is based on the calculation of intercell fluxes by means of a hyperelastic approximate-state Riemann solver which is presented in section \ref{sec:riemann_solver}.
The constitutive relations are restated in section \ref{sec:constitutive-update} in order to provide a variational constitutive update for the integration of plastic flow at material points.
At last, the method is illustrated on one and two-dimensional simulations in section \ref{sec:test_cases} for which comparisons with FEM and MPM solutions are made.


%%% Local Variables:
%%% mode: latex
%%% TeX-master: "manuscript"
%%% End:
